\documentclass[12pt,reqno]{amsart}
\setlength{\textwidth}{\paperwidth}
\addtolength{\textwidth}{-2in}
\calclayout
\usepackage{graphicx}
\usepackage{amssymb}
\usepackage{epstopdf}
\DeclareGraphicsRule{.tif}{png}{.png}{`convert #1 `dirname #1`/`basename #1 .tif`.png}
\usepackage{hyperref}
\usepackage{mathrsfs} % for script C

\title{Proposed Errata (4) for \\ \textit{Excursions in Multiplicative Number Theory}}
\author{Allen Stenger}
\address{Boulder, Colorado USA}
\email{StenBiz@gmail.com}
\date{\today}

%-------------------------------------------------------
% Proclamations
%-------------------------------------------------------

\theoremstyle{plain}
\newtheorem{theorem}{Theorem}
\newtheorem{lemma}{Lemma}
\newtheorem{proposition}{Proposition}

\theoremstyle{definition}
\newtheorem{definition}{Definition}
\newtheorem{example}{Example}
\newtheorem{exercise}{Exercise}
\newtheorem{conjecture}{Conjecture}

\theoremstyle{remark}
\newtheorem*{remark*}{Remark}
\newtheorem*{note*}{Note}
\newtheorem{case}{Case}

%-------------------------------------------------------
% local definitions
%-------------------------------------------------------

\newcommand{\Z}{{\mathbb Z}}
\newcommand{\R}{{\mathbb R}}
\newcommand{\Q}{{\mathbb Q}}
\newcommand{\C}{{\mathbb C}}
\newcommand{\N}{{\mathbb N}}

\providecommand{\abs}[1]{\left \lvert #1 \right \rvert}
\providecommand{\norm}[1]{\left \lVert #1 \right \rVert}

\newcommand{\divides}{\mid}
\newcommand{\notdivides}{\nmid}

% special for this document
\newcommand{\conj}[1]{\overline{#1}}

\begin{document}
\maketitle

These are additional proposed errata (4th set) for Olivier Ramar\'{e}, 
\textit{Excursions in Multiplicative Number Theory}.
Most are typographical errors.

\begin{itemize}

\item
(p.~56, line 3) ``there exists forcibly'': ``forcibly'' is the wrong word, it probably should be ``necessarily''
\item
(p.~87, line 3 from bottom) ``the problematic we address'': ``problematic'' is the wrong word, it should be probably be ``problem'' or perhaps ``challenge''

\item
(p.~147, last line of Theorem 13.4 statement) ``where \(C\) is as in Theorem 13.3.'' should be
``where \(C\) is as in Theorem 13.3, taking \(g(d) = f(d)/d\).''

\item
(p.~148, lines 6--7) \(Q\) should be \(D\) throughout

\item
(p.~235, Exercise 23-1) ``Let \(D(t) = \sum_{n \ge 1} a_n n^{i t}\) and \(D^*(t) = \sum_{n \ge 1} a^*_n n^{i t}\)
be two Dirichlet series, both absolutely convergent for \(\Re s \ge 0\).''
I was confused by this phrasing. First, \(D(t)\) and \(D^*(t)\) are not Dirichlet series because they do 
not have the correct form (although they can be considered as Dirichlet series restricted to the line
\(\sigma = 0\)). Also, \(s\) is not mentioned in their definition, so \(\Re s \ge 0\) is not clear;
this probably means that they are absolutely convergent for all real \(t\). 
We don't consider any \(\sigma > 0\).
Could we instead say 
``Let \(D(t) = \sum_{n \ge 1} a_n n^{i t}\) and \(D^*(t) = \sum_{n \ge 1} a^*_n n^{i t}\)
be two series, both absolutely convergent for all real \(t\).''?

\item
(p.~240, Exercise 23-9) \(\mathscr{C}_1\) should be \(\sqrt{2} \mathscr{C}_1\)

\item
(p.~243, line 6) ``due to the K.~Iseki'' should be``due to K.~Iseki''

\item
(p.~251, line 7 from bottom) ``to every integers'' should be ``to every integer''

\item
(p.~251, last displayed formula) \(\zeta(2 - a)\) should be \(C \zeta(2 - a)\)

\item
(p.~252, line 4) Exer. 3-4 should be Exer. 3-6

\item
(p.~252, Exercise 25-2 part 1: ``any nonnegative \(\sigma\)'' should be ``any positive \(\sigma\)''
(\(\sigma = 0\) causes a division by zero)

\item
(p.~255, line 13)
\[
\sum_{\genfrac{}{}{0 pt}{}{d \le D}{d \divides P(z)}} \mu(d)
\text{ should be }
\sum_{\genfrac{}{}{0 pt}{}{d \le D}{d \divides Q}} \mu(d)
\]

\item
(p.~258, line 1) ``The sieve problematic'' should be ``The sieve method''

\item
(p.~258, reference [5]) This reference is to a book and should be formatted as a book

\item
(p.~260, line 3) \((\rho - \rho^{-1})\rho^{-2k}\) should be \((\rho + \rho^{-1})\rho^{-2k}\) (wrong sign)

\item
(p.~260, line 8 from bottom) 
\(N_k \ge (1 + \sqrt{2})^k / 2\) should be \(N_k \le (1 + \sqrt{2})^k / 2\)

\item
(p.~260, line 2 from bottom) ``coprime positive integer'' should be ``coprime positive integers''

\item
(p.~261, line 5) ``the points \(h \alpha\) are distant from one another by at least \(1/(2q)\)''
should be 
``the points \(h \alpha\) modulo \(1\) are distant from one another by at least \(1/(2q)\)''
\
\item
(p.~262, line 9) ``loose a logarithmic factor'' should be ``lose a logarithmic factor''

\item
(p.~264, line 7 from bottom) ``forcibly larger than or equal to \(z\)'': The word ``forcibly'' is
wrong here; I don't know what word is intended.

\item
(pp.~264--265, proof of Theorem 26.2) The symbol \(Q\) is not defined but is used in two places.
I think it means the same as \(P(z)\).

\item
(p.~265, proof of Theorem 26.3) The function \(g\) to use in the application of Theorem 26.2
is not stated; it should be \(g(n) = \operatorname{e}(\rho m)\).

\item
(p.~266, line 13) \(Q \le X/n\) should be \(Q \le x/n\) (lower case \(x\))

\item
(p.~266, last displayed equation)
There seems to be a missing step in this proof: to pick the value of \(z\). At this point we are
almost at the end of the proof, but all we know is \(z \ge 2\) and \(\log z \le \sqrt{\log x}/2\).
I think we should pick \(\log z = \sqrt{\log x}/2\), which we can do at the beginning of the proof.
This gives the indicated bound in this last line. The bound is not true for all \(z\) in
the above range; for example, \(z = 2\) gives a bound of \(O(x \log x)\), which is worse than
the trivial bound \(O(x)\).

\item
(p.~268, line 6)
``if only finitely primes \(p\) where such that \(\{\rho p\} \notin [a,b]\)''
should be
``if there are only finitely primes \(p\) such that \(\{\rho p\} \notin [a,b]\)''

\item
(p.~273, lines 1--2)
``the sequence \((\cos 2 \pi \rho n)\) where \(n\) ranges the irregular integers dense in \([-1, 1]\)''
should be
``the sequence \((\cos 2 \pi \rho n)_n\) where \(n\) ranges over the irregular integers is dense in \([-1, 1]\)''

\item
(p.~273, reference [3]) Distjointness should be Disjointness

\item
(p.~274, reference [19] This reference is to a book and should be formatted as a book

\item
(p.~275, line 11) E.~Cohen should be H.~Cohen

\item
(p.~276, line 4) \(x \ge 2 \cdot 10^6\) should be \(x \le 2 \cdot 10^6\) 

\item
(p.~277, line 6) \(\sqrt{x} \ge 3500\) is not correct; from \(x \ge 2 \cdot 10^6\) we only get
\(\sqrt{x} \ge 1414\).

\item
(p.~278, line 4) \(\frac{2}{25} + \frac{27 \log x}{x^{1/3}} \le 4\)
should be
\(\frac{2\cdot 3}{25} + \frac{27 \log x}{x^{1/3}} \le 4\)
(add factor for \(D(\abs{h_6},0)\)).

\item
(p.~279, line 5) ``the \textit{large sieve}. and stems from'' should be 
``the \textit{large sieve}, and stems from'' (replace period with comma)

\item
(p.~281, Exercise 28-3) This result is incorrect as stated. As a counterexample, consider
\(f\) defined by \(f(1,2) = 1\) and \(f(i,j) = 0\) otherwise. Then the left-hand side is
\[
\abs{\xi_1 \conj{\xi_2}} = \abs{\xi_1} \abs{\xi_2}
\]
and the right-hand side is
\[
\abs{\xi_1}^2.
\]
The exercise then asserts that \(\abs{\xi_1} \abs{\xi_2} \le \abs{\xi_1}^2\), which implies
\(\abs{\xi_2} \le \abs{\xi_1}\).
One way to repair this would be to add the hypothesis that \(f(i,j) = f(j,i)\) for all
\(i\) and  \(j\).
In the application of this item to Exercise 29-2, this additional symmetry hypothesis is satisfied.

\item
(p.~285, p.~291) The ordering of parameters of \(S(q;b)\) is inconsistent between Chapters 28 and 29;
the \(S(q;b)\) of Chapter 28 is the \(S(b;q)\) of Chapter 29.

\item
(p.~285, line 10)
\(W^*(d) = q \sum_{\delta \divides q}\) should be \(W^*(q) = \sum_{\delta \divides q}\) 
(change argument of function, omit factor \(q\))

\item
(p.~285, line 13)
\(V^*(d) = q^2 \sum_{\delta \divides q}\) should be \(V^*(q) = q \sum_{\delta \divides q}\)
(change argument of function, omit one factor \(q\))

\item
(p.~285, last displayed equation)
\[
\frac{N \log^2 Q}{Q \sqrt{p}} \text{ should be } \frac{N \log Q}{Q \sqrt{p}}
\text{ (omit one factor of } \log Q)
\]

\item
(p.~285, line 6 from bottom) ``almost every pairs'' should be ``almost every pair''

\item
(p.~292, Exercise 29-2, part 2) This is not true as stated. For example, for \(t = T\)
we have \(\hat{K}(t) = 0\).

\item
(p.~293, Exercise 29-3) The statement of part 2 is not clear. I think it means,
``Enumerate the triples \((n, n+2, n+4)\) where \(n \le 3^{100}\) 
and each element of the ordered triple is a prime power.''

\item
(p.~295, reference [4]) \(2^k \ p\) should be \(2^k + p\)

\item
(p.~295, reference [8]) \(p \ 2^k\) should be \(p + 2^k\)

\item
(p.~295, references [9] and [12]) These references are to books and should be formatted as books

\item
(p.~295, reference [14]) ``14 O.~Ramar\'{e}'' should be ``O.~Ramar\'{e}'' (omit repeated 14)
\item
(p.~318, Hint for 23-3) Lemma 19.2 should be Lemma 19.1 (there is no Lemma 19.2)

\item
(p.~318, Hint for 23-8) Lemma 12.2.1 should be Lemma 12.2.

\item
(p.~320, line 13 from bottom) \verb+for(y = 1, x,+ should be \verb+for(y = begx, endx,+

\item
(p.~324, line 2) \(\hat{K}(t)\) should be \(\hat{K}(t) \pi^2 T\)

\item
(p.~325, line 1) ofenly should be often (``oftenly'' is archaic)


\end{itemize}
%-------------------------------------------------------
\bibliography{AllenMain}
\bibliographystyle{amsplain}
%-------------------------------------------------------

\end{document}  