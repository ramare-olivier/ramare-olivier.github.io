\input amstex
\magnification=\magstep1
\documentstyle {amsppt}
\NoBlackBoxes
\refstyle{B}

%%% Version du 13 Janvier 2000

\def\goes{\mathrel{\rightarrow}}
\def\Log{\operatorname{Log}}
\def\section#1{\goodbreak\smallskip\noindent{\bf #1}}
\def\fin{$\diamond\diamond\diamond$\enddemo}

\document

\topmatter
\title
Le th\'eor\`eme de Brun-Titchmarsh par le crible de Selberg.
\endtitle
\author
Olivier Ramar\'e
\endauthor
\abstract
Expos\'e introductif au crible sup\'erieur de Selberg. L'exemple
choisi est bien s\^ur le th\'eor\`eme de Brun-Titchmarsh.
Version du 17 Janvier 2000.
\endabstract
\endtopmatter

\proclaim{Th\'eor\`eme de Brun-Titchmarsh}

Si $2\le y<q$, $x\ge0$ et $a$ est premier \`a $q$, nous avons
$$
\pi(x+y;q,a)-\pi(x;q,a)\le\frac{2y}{\phi(q)\Log(y/q)}.
\leqno(\star)
$$
\endproclaim
Rappelons que $\pi(x;q,a)$ d\'esigne le nombre de nombres premiers
inf\'erieurs \`a $x$ et congrus \`a $a$ modulo $q$.



Titchmarsh a d\'emontr\'e un r\'esultat plus faible que le
pr\'ec\'edent dans les ann\'ees~30 en utilisant le crible de Brun. La
d\'enomination ``th\'eor\`eme de Brun-Titchmarsh'' est d\^u \`a Linnik
et date des ann\'ees~40.

Nous allons montrer une version un peu plus faible de $(\star)$ et
utiliser ce probl\`eme pour illustrer la fa\c con dont le crible de
Selberg fonctionne. Ce crible date des ann\'ees~47-50. Le lecteur en
trouvera une pr\'esentation classique dans le livre de Halberstam \&
Richert cit\'e ci-dessous.

Posons
$$
S=\pi(x+y;q,a)-\pi(x;q,a)
=\sum\Sb x<p\le x+y\\ p\equiv a[q]\endSb 1
\leqno(1)
$$
et consid\'erons
$$
\Sigma=
\sum\Sb x<n\le x+y\\ n\equiv a[q]\endSb
\bigg(\sum_{d|n}\lambda_d\bigg)^2
\leqno(2)
$$
o\`u les $(\lambda_d)$ sont des nombres r\'eels qui v\'erifient
$\lambda_1=1$ et $\lambda_d=0$ si $d>z$ o\`u $z$ est un
param\`etre. Si $p$ est un nombre premier dans $]x+z,x+y]$, $p$
n'admet pas d'autres diviseurs inf\'erieurs \`a $z$ que 1 et par
cons\'equent
$$
\bigg(\sum_{d|p}\lambda_d\bigg)^2=1.
$$
Il vient alors
$$
S\le \Sigma + z.\leqno(3)
$$
Il nous reste \`a \'etudier $\Sigma$ et en fait \`a obtenir le minimum
de cette forme quadratique des $(\lambda_d)$. Pour cela nous d\'eveloppons le
carr\'e et obtenons
$$
\Sigma=\sum_{d_1,d_2\le z}\lambda_{d_1}\lambda_{d_2}
\sum\Sb x<n\le x+y\\ [d_1,d_2]|n\\ n\equiv a[q]\endSb 1,
$$
o\`u $[r,s]$ d\'esigne le ppcm de $r$ et $s$ et $(r,s)$ leur pgcd.
Comme $(a,q)=1$, seuls les $d$ tels que $(d,q)=1$ interviennent, ce
qui fait que nous pouvons imposer $\lambda_d=0$ si $(d,q)\neq1$. En
utilisant maintenant
$$
\sum\Sb x<n\le x+y \\ n\equiv b[q]\endSb 1= \frac{y}{r}+\Cal O^*(1),
\leqno(4)
$$
nous obtenons
$$
\aligned
\Sigma
&=
\frac{y}{r}\sum_{d_1,d_2\le z}
\frac{\lambda_{d_1}\lambda_{d_2}}{[d_1,d_2]}
+\Cal O^*\bigg(\sum_{d_1,d_2}|\lambda_{d_1}||\lambda_{d_2}|\bigg)\\
&=\frac{y}{r}
\Sigma_0
+\Cal O^*\bigg(\bigg(\sum_{d}|\lambda_{d}|\bigg)^2\bigg)\quad\text{disons.}
\endaligned
\leqno(5)
$$
Nous diagonalisons alors $\Sigma_0$ par un proc\'ed\'e
mis au point par Selberg.
\'Ecrivons
$$
\Sigma_0=\sum_{d_1,d_2\le z}(d_1,d_2)
\frac{\lambda_{d_1}}{d_1}\frac{\lambda_{d_2}}{d_2}.
$$
Or $d=\sum_{\ell|d}\phi(\ell)$, d'o\`u
$$
\Sigma_0=\sum_{\ell\le z}\phi(\ell)\bigg(\sum\Sb \ell|d\le z\endSb
\frac{\lambda_d}{d}\bigg)^2,
\leqno(6)
$$
ce qui est la forme diagonale annonc\'ee. Posons
$$
y_{\ell}=\sum\Sb \ell|d\le z\endSb
\frac{\lambda_d}{d}.\leqno(7)
$$
La matrice de passage des $(\lambda_d)$ aux $(y_{\ell})$ est
triangulaire avec des \'el\'ements non nuls sur la diagonale et est
donc inversible. De fa\c con explicite, nous avons
$$
\lambda_d=d\sum_{d|\ell\le z} \mu(\ell/d) y_{\ell}\leqno(8)
$$
ce que l'on v\'erifie en introduisant cette expression dans
$(7)$. Cela nous permet notamment de traduire la condition
$\lambda_1=1$ en terme des $(y_{\ell})$. En d\'efinitive, notre
probl\`eme devient~: 
$$
\left\{
\aligned
&\text{minimiser}\quad \sum_{\ell}\phi(\ell)y_\ell^2,\\
&\text{sous}\quad
\left\{
\aligned
&\sum_{\ell\le z}\mu(\ell)y_{\ell}=1,\\
& y_{\ell}=0\quad\text{si}\quad (\ell,q)\neq1.
\endaligned
\right.
\endaligned
\right.
\leqno(9)
$$
Nous utilisons un multiplicateur ($\theta$) de Lagrange et obtenons
$$
\left\{
\aligned
&2\phi(\ell)y_\ell -\theta\mu(\ell)=0\quad(\ell,q)=1,\\
&\sum_{\ell\le z}\mu(\ell)y_{\ell}=1,
\endaligned
\right.
\leqno(10)
$$
ce qui donne
$$
y_\ell=\frac{\theta}2\frac{\mu(\ell)}{\phi(\ell)}\quad,
\quad
\frac{\theta}2\sum\Sb \ell\le z\\ (\ell,q)=1\endSb
\frac{\mu^2(\ell)}{\phi(\ell)}=1.
\leqno(11)
$$
Remarquons, ce qui est \'evident sur $(9)$, que $y_\ell=0$ si $\ell$
est divisible par un carr\'e, ce qui \'equivaut \`a la m\^eme
propri\'et\'e sur les $(\lambda_d)$.

Posons
$$
G_f(z)=\sum\Sb \ell\le z\\ (\ell,f)=1\endSb
\frac{\mu^2(\ell)}{\phi(\ell)}.\leqno(12)
$$\
Alors
$$
\left\{\aligned
y_\ell&=\frac{1}{G_q(z)}\frac{\mu(\ell)}{\phi(\ell)}\quad(\ell,q)=1,\\
\lambda_d&=\mu(d)\frac{d}{\phi(d)}\frac{G_{dq}(z/d)}{G_q(z)}\quad(d,q)=1,\\
\Sigma_0&=1/G_q(z).
\endaligned\right.
\leqno(13)
$$
Il nous faut \`a pr\'esent \'evaluer $G_q(z)$ et nous commen\c cons
par un lemme de van Lint \& Richert (voir les r\'ef\'erences)~:
\proclaim{Lemme}

Soit $f$ et $h$ deux entiers tels que $(f,h)=1$. Nous avons
$$
\frac{f}{\phi(f)}G_{fh}(z/f)\le G_h(z)\le \frac{f}{\phi(f)}G_{fh}(z).
$$
\endproclaim

\demo{Preuve}
Nous \'ecrivons
$$
G_h(z)
=\sum_{r|f}\sum\Sb \ell\le z/r\\
(\ell,fh)=1\endSb\frac{\mu^2(\ell r)}{\phi(\ell r)}
=\sum_{r|f}\frac{\mu^2(r)}{\phi(r)}G_{fh}(z/r)
$$
et il nous suffit alors d'utiliser $G_{fh}(z/f)\le G_{fh}(z/r)\le
G_{fh}(z)$ ainsi que
$$
\sum_{r|f}\frac{\mu^2(r)}{\phi(r)}=\frac{f}{\phi(f)}
$$
pour conclure.
\fin

Ce lemme nous donne notamment
$$
|\lambda_d|\le 1,\leqno(14)
$$
et ram\`ene l'\'evaluation de $G_q(z)$ \`a celle de $G_1(z)$. Il nous
suffit d'ailleurs de minorer $G_1(z)$, ce qui se fait tr\`es
facilement de la fa\c con suivante~:
$$
\aligned
G_1(z)
&=\sum_{\ell\le z}\frac{\mu^2(\ell)}{\phi(\ell)}
=\sum_{\ell\le z}\frac{\mu^2(\ell)}{\ell}\prod_{p|\ell}\frac{1}{1-1/p}
\\&=
\sum_{\scriptstyle k\atop
{\text{tel que le noyau sans}
\atop\text{facteurs carr\'es de $k$ soit $\le z$}}}
\frac1k
\endaligned
\leqno(15)
$$
o\`u l'on obtient la derni\`ere \'egalit\'e \`a partir de
$$
\frac{1}{1-1/p}=1+\frac1p+\frac1{p^2}+\dots
$$
Il vient alors
$$
G_1(z)\ge\sum_{k\le z}\frac1k\ge\Log z.\leqno(16)
$$
Le lecteur pourra consulter l'article et/ou le livre de Halberstam \&
Richert cit\'es ci-apr\`es pour une \'evaluation compl\`ete de
$G_1(z)$.

Rassemblant $(3)$, $(5)$, $(13)$ et $(14)$, nous obtenons
$$
S\le \frac{y}{\phi(q)}\frac{1}{G_1(z)}+z^2+z\leqno(17)
$$
et avec $(16)$~:
$$
S\le \frac{y}{\phi(q)}\frac{1}{\Log z}+z^2+z.\leqno(18)
$$
Il nous suffit \`a pr\'esent de choisir $z$. Nous prenons
$$
z=\frac{y}{q}(\Log (y/q))^{-1}\leqno(19)
$$
ce qui nous donne
$$
S\le (2+o(1))\frac{y}{\phi(q)}\frac{1}{\Log (y/q)}\qquad(y/q\goes\infty).\leqno(20)
$$

\bigskip
La d\'emonstration que nous avons donn\'ee est insuffisante pour
\'etablir $(\star)$ puisque nous n'avons que $(2+o(1))$ au lieu de
$(2)$. On trouvera une preuve de $(\star)$ dans l'article de
Montgomery \& Vaughan cit\'e ci-apr\'es.


\Refs

\ref
\paper Mean value theorems for a class of arithmetic functions
\by H. Halberstam \& H. E. Richert
\yr 1971
\jour Acta Arith.
\vol 43
\pages 243--256
\endref

\ref
\paper Sieves methods
\by H. Halberstam \& H. E. Richert
\yr 1974
\jour Academic Press (London)
\pages 364pp
\endref

\ref
\paper On primes in arithmetic progressions
\by J.E. van Lint \& H.E. Richert
\yr 1965
\jour Acta Arith.
\vol 11
\pages 209--216
\endref

\ref
\by H. Montgomery \& R. C. Vaughan
\paper The large sieve
\jour Mathematika
\vol 20 no 2
\yr 1973 
\pages 119--133
\endref

\endRefs


\enddocument



%%% Local Variables: 
%%% mode: plain-tex
%%% TeX-master: t
%%% End: 
