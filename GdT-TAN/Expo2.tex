\input amstex
\magnification=\magstep1
\documentstyle {amsppt}
\NoBlackBoxes
\refstyle{B}

%%% Version du 13 Janvier 2000

\def\goes{\mathrel{\rightarrow}}
\def\Log{\operatorname{Log}}
\def\section#1{\goodbreak\smallskip\noindent{\bf #1}}
\def\fin{$\diamond\diamond\diamond$\enddemo}
\def\about{\mathrel{{\smallsmile\atop\smallfrown}}}

\document

\topmatter
\title
L'in\'egalit\'e du grand crible par la fonction de Selberg
\endtitle
\author
Olivier Ramar\'e
\endauthor
\abstract
Expos\'e introductif sur l'in\'egalit\'e du grand crible que nous
d\'emontrons \`a l'aide d'un lissage. Nous \'etablissons aussi la
forme arithm\'etique du grand crible, due \`a Montgomery.
Version du 10 F\'evrier 2000.
\endabstract
\endtopmatter


Le lecteur trouvera deux autres approches classiques dans les
monographes respectivement de Bombieri et de Montgomery cit\'es
ci-apr\`es.

\section{I. Une formule de Poisson.}

Bien que la formule de Poisson soit d'un usage courant en th\'eorie
analytique des nombres, il n'est pas facile de trouver dans cette
litt\'erature une telle formule avec des hypoth\`eses assez
larges. Nous donnons ici une telle formule.

La transform\'ee de Fourier d'une fonction $f:\Bbb R\goes\Bbb R$ est
d\'efinie au point $x$ par
$$
\hat f(x)=\int_{-\infty}^\infty f(t)\,e(xt)dt=
\lim_{T\goes\infty}\int_{-T}^{T} f(t)\,e(xt)dt
$$
avec $e(\alpha)=\exp(2i\pi\alpha)$. Nous avons, sous l'hypoth\`ese
suppl\'ementaire que $f$ et $\hat f$
appartiennent \`a $\Cal L^1(\Bbb R)$,
$$
f(t)=\int_{-\infty}^\infty \hat f(x)\,e(-xt)dx.
$$
Dans cette partie uniquement, nous posons
$$
\mathop{\sum_{n\in\Bbb Z}}\nolimits^{\text{sym}}\varphi_n
=\lim_{N\goes\infty}\sum_{-N\le n\le N}\varphi_n
$$
si cette limite existe.

Nous avons alors le th\'eor\`eme suivant d\^u \`a Bochner~:
\proclaim{Th\'eor\`eme}
Soit $f:\Bbb R\goes\Bbb R$ et $\alpha$ r\'eel. Supposons que
\roster
\item la fonction $f$ est continue est $\Bbb R$,
\item la s\'erie
$\sum_{n\in\Bbb Z}^{\text{sym}}f(t+n)e(\alpha(t+n))$
converge uniform\'ement pour $t\in[-\tfrac12,\tfrac12]$,
\item la s\'erie
$\sum_{k\in\Bbb Z}^{\text{sym}}\hat f(k-\alpha)$
converge.
\endroster
Alors
$$
\mathop{\sum_{n\in\Bbb Z}}\nolimits^{\text{sym}}f(n)e(\alpha(n))
=
\mathop{\sum_{k\in\Bbb Z}}\nolimits^{\text{sym}}\hat f(k-\alpha).
$$
\endproclaim

Remarquons que sous les deux premi\`eres hypoth\`eses, la preuve
montre que $\hat f$ existe.

\demo{Preuve}
Voir le livre de Chandrasekharan, cit\'e ci-dessous,
chap\^\i tre II, th\'eor\`eme~1.
\fin

\section{II. Une bonne fonction de lissage.}

Voici notre fonction favorite~:

\proclaim{Th\'eor\`eme}
Soit $N\ge1$ et $\delta>0$ deux r\'eels. Il existe une fonction
$b(t)=b_{N,\delta}(t)$ de $\Bbb R$ dans $\Bbb R$ telle que
\roster
\item $b$ est $C^\infty$.
\item $b(t)=\Cal O_{N,\delta}(|t|^{-2})$ quand $|t|\goes\infty$.
\item $b(t)\ge0$ pour tout $t$.
\item $b(t)\ge1$ pour tout $t\in[1,N]$.
\item $\hat b(x)=0$ si $|x|\ge\delta$.
\item $\hat b(0)=N-1+\delta^{-1}$.
\endroster
\endproclaim

La fonction $b$ ressemble \`a~:

\vskip1cm

Nous avons aussi $b(t)\le 2$. Pour la construction, le lecteur
consultera l'article de Vaaler cit\'e ci-apr\`es.


\section{III. L'in\'egalit\'e du grand crible.}

Soit $(\varphi_n)$ une suite de complexes. Posons
$$
S(\alpha)=\sum_{1\le n\le N}\varphi_n e(n\alpha).
$$
Un ensemble fini $\Cal X\subset \Bbb R/\Bbb Z$ est dit $\delta$-bien
espac\'e si l'on a
$$
\min\big\{\|x-x'\|,\quad x,x'\in\Cal X,\ x\neq x'\big\}\ge\delta,
$$
o\`u $\|y\|$ est pris sur le cercle de circonf\'erence unit\'e
($\|y\|=\min_{k\in\Bbb Z}|y-k|$), et donc $\delta<1/2$. Dans le
cas (fortement inint\'eressant) o\`u $\Cal X$ est r\'eduit \`a un
point, on prend $\delta=1/2$. Nous avons alors
$$
\sum_{x\in\Cal X}|S(x)|^2\le \sum_{n\le N}|\varphi_n|^2\
(N-1+\delta^{-1})
\qquad(\Cal X\text{ $\delta$-bien espac\'e})
\leqno(3.1)
$$
\demo{Preuve}
\'Ecrivons
$$
\sum_{x\in\Cal X}|S(x)|^2
=
\sum_{1\le n\le N}\sum_{x\in\Cal X}\varphi_n\overline{S(x)}\,e(nx).
$$
Par cons\'equent,
$$
\aligned
\left(\sum_{x\in\Cal X}|S(x)|^2\right)^2
&\le
\sum_{n\le N}|\varphi_n|^2\sum_{n\le N}\bigg|\sum_{x\in \Cal
X}\overline{S(x)}\,e(nx)\bigg|^2\\
&\le
\sum_{n\le N}|\varphi_n|^2\sum_{n\in\Bbb Z}b(n)\bigg|\sum_{x\in \Cal
X}\overline{S(x)}\,e(nx)\bigg|^2\\
\endaligned
$$
o\`u $b=b_{N\delta}$ est la fonction donn\'ee par le th\'eor\`eme de
la seconde partie.
Par ailleurs
$$
\aligned
\sum_{n\le N}|\varphi_n|^2\sum_{n\in\Bbb Z}b(n)\bigg|\sum_{x\in \Cal
X}\overline{S(x)}\,e(nx)\bigg|^2
&=
\sum_{x,x'\in\Cal X}\overline{S(x)}S(x')\sum_{n\in\Bbb
Z}b(n)e(n(x-x'))\\
&=
\sum_{x,x'\in\Cal X}\overline{S(x)}S(x')\sum_{k\in\Bbb
Z}\hat b(k-(x-x'))\\
&=
\sum_{x\in\Cal X}|S(x)|^2(N-1+\delta^{-1})
\endaligned
$$
ce qui permet de conclure facilement.
\fin
Notons que
$$
|S(x_0)|\le \sum_{n\le N}|\varphi_n|^2 N \leqno(3.2)
$$
ce qui fait que, lorsque $\delta^{-1}=o(N)$, l'in\'equation $(3.1)$
donne essentiellement la m\^eme majoration que $(3.2)$, alors que l'on
majore simultan\'ement de nombreux termes.

\bigskip
Il existe \`a l'heure actuelle essentiellement quatre preuves de
l'in\'egalit\'e du grand crible. Celle que nous venons de donner
dont l'id\'ee se trouve dans un article de Davenport \& Halberstam de 1966,
une due \`a Gallagher et qui tente de comprendre le membre de
gauche de $(3.1)$ comme une somme de Riemann, une (due \`a Montgomery
\& Vaughan ?) qui passe par l'in\'egalit\'e de Hilbert et une derni\`ere, en fait
plus ancienne, qui utilisait notamment des disques de Gegenbauer. La
preuve que nous pr\'esentons est en quelque sorte la forme achev\'ee
de cette derni\`ere approche. Le lecteur trouvera aussi des
d\'emonstration hybrides, comme celle que pr\'esente Vaaler dans
l'article sus-cit\'e et qui combine la technique de Selberg et l'id\'ee
de Gallagher.


\section{IV. L'in\'egalit\'e du grand crible pour la suite de Farey.}

Presque toutes les utilisations de $(3.1)$ en th\'eorie des nombres se
font en prenant
$$
\Cal X=\{\tfrac aq,\quad q\le Q, (a,q)=1, 1\le a\le q\}.
$$
L'in\'egalit\'e du grand crible ($(3.1)$) s'\'ecrit alors~:
$$
\sum_{q\le Q}\sum_{a\mod^* q}|S(a/q)|^2\le
\sum_{n\le N}|\varphi_n|^2\ (N-1+\delta^{-1}).
\leqno(3.3)
$$
En effet, deux points distincts $a/q$ et $a'/q'$ de $\Cal X$
v\'erifient
$$
\big\|\frac aq-\frac{a'}{q'}\big\|
=
\big\|\frac {aq'-a'q}{qq'}\big\|\ge\frac1{Q^2}.
$$
En fait, si $a/q$ et $a'/q'$ sont cons\'ecutifs, il est classique que
$(q,q')=1$, et notamment $q\neq q'$ si $Q\ge2$. La minoration
pr\'ec\'edente peut donc \^etre remplac\'ee par $\frac{1}{Q(Q-1)}$,
qui est atteinte entre $1/Q$ et $1/(Q-1)$.

\bigskip
Notons que $|\Cal X\|=\sum_{q\le Q}\phi(q)\about Q^2$, ce qui fait que
les points de $\Cal X$ sont quasiment \'equi-distribu\'es sur le
cercle.

\bigskip
La remarque vitale ensuite consiste \`a dire que $S(a/q)$ ne d\'epend
que de la distribution de $(\varphi_n)_n$ modulo $q$ puisque nous
avons
$$
S(a/q)=\sum_{b\mod q}\bigg(\sum_{n\equiv b[q]}\varphi_n\bigg)e(ab/q).
$$
En particulier, si $q=p$ est un nombre premier, nous avons
$$
\sum_{a\mod^*p}|S(a/p)|^2=p\sum_{b\mod p}\bigg|\sum_{n\equiv b[p]}\varphi_n-\frac{S(0)}p\bigg|^2.
$$

\section{V. Le crible de Montgomery.}

Pour chaque nombre premier $p$, nous nous donnons un sous-ensemble
$\Cal K_p\subset\Bbb Z/p\Bbb Z$ (l'ensemble des classes que l'on
garde).

Nous dirons que la suite $(\varphi_n)_n$ est port\'ee par $(\Cal K_p)$
jusqu'au niveau $Q$ si nous avons
$$
\forall n\le N\qquad
\big[\varphi_n\neq0\implies\forall p\le Q,\quad n\in\Cal K_p\big].
\leqno(5.1)
$$
Par exemple, pour cribler la suite des nombres premiers compris entre
$\sqrt{N}$ et $N$, nous prendrons $\Cal K_p=\Bbb Z/p\Bbb
Z\setminus\{0\}$ et $Q=\sqrt{N}$. Notons que si $(5.1)$ est
v\'erifi\'e pour un certain $Q$, elle est encore triviallement vraie
pour tout $Q'$ inf\'erieur \`a $Q$. 
\bigskip
Posons
$$
G(\Cal K,Q)=\sum_{q\le Q}\mu^2(q)\prod_{p|q}
\big(\frac p{|\Cal K_p|}-1\big).\leqno(5.2)
$$
Alors, si $(\varphi_n)_n$ est port\'ee par $(\Cal K_p)$
jusqu'au niveau $Q$, nous avons
$$
\bigg|\sum_n \varphi_n\bigg|^2\le
\sum_n|\varphi_n|^2\frac{N-1+Q^2}{G(\Cal K,Q)}.
\leqno(5.3)
$$
\proclaim{Corollaire}
Soit $Z$ le nombre d'entiers $n$ de $[1,N]$ qui v\'erifient
$\forall p\le Q,\quad n\in\Cal K_p$. Nous avons
$$
Z\le\frac{N-1+Q^2}{G(\Cal K,Q)}.
$$
\endproclaim
\demo{Preuve du corollaire}
Nous prenons pour $(\varphi_n)$ la fonction caract\'eristique des
$n\in[1,N]$ qui sont tels que $\forall p\le Q,\quad n\in\Cal K_p$.
L'in\'egalit\'e $(5.3)$ nous donne alors le r\'esultat annonc\'e.
\fin

\demo{Preuve de $(5.3)$}
Nous proc\'edons \`a reculons en trois \'etapes.
Histoire se simplifier la typographie, nous posons~:
$$
g(q)=\mu^2(q)\prod_{p|q}
\big(\frac p{|\Cal K_p|}-1\big).
$$

Tout d'abord, il suffit de d\'emontrer l'in\'egalit\'e suivante (pour
$(\varphi_n)$ v\'erifiant $(5.1)$)~:
$$
g(q)\,|S(0)|^2\le
\sum_{a\mod^* q}|S(a/q)|^2.\leqno(5.4)
$$
En effet, combin\'e \`a $(3.3)$, cela nous donne bien l'in\'egalit\'e
attendue.

Secundo, pour prouver $(5.4)$ pour $qq'$, il suffit de le prouver pour
$q$ et pour $q'$ si $(q,q')=1$. En effet, dans ce cas, nous aurons~:
$$
\aligned
\sum_{b\mod^* qq'}\big|S\big(\frac{b}{qq'}\big)\big|^2
&=
\sum_{a'\mod^* q'}\sum_{a\mod^* q}
\big|S\big(\frac{a}{q}+\frac{a'}{q'}\big)\big|^2
\\&\ge
\sum_{a'\mod^* q'}g(q')\big|S\big(\frac{a'}{q'}\big)\big|^2
\endaligned
$$
obtenue en appliquant $(5.4)$ \`a $q$ et \`a la suite
$(\varphi_n e(na'/q'))$ qui v\'erifie bien $(5.1)$.
Nous r\'eappliquons $(5.4)$, mais cette fois-ci \`a $q'$ et \`a la
suite $(\varphi_n)$, ce qui nous donne
$$
\sum_{b\mod^* qq'}\big|S\big(\frac{b}{qq'}\big)\big|^2
\ge
g(q')g(q)|S(0)|^2=g(qq')|S(0)|^2,
$$
ce qui prouve notre assertion.

Finalement, il nous reste \`a prouver $(5.4)$ pour $q=p$, un nombre
premier. Dans ce cas, nous avons
$$
\aligned
\sum_{a\mod^* p}|S(a/p)|^2
&=\sum_{a\mod p}|S(a/p)|^2-|S(0)|^2
=p\sum_{b\mod p}\big|\sum_{n\equiv b[p]}\varphi_n\big|^2-|S(0)|^2
\\&=p\sum_{b\in \Cal K_p}\big|\sum_{n\equiv b[p]}\varphi_n\big|^2-|S(0)|^2.
\endaligned
\leqno(5.5)
$$
Mais l'in\'egalit\'e de Cauchy-Schwartz nous donne~:
$$
|S(0)|^2=\big|\sum_{b\in \Cal K_p}\sum_{n\equiv
b[p]}\varphi_n\big|^2
\le |\Cal K_p|\sum_{b\in \Cal K_p}\big|\sum_{n\equiv b[p]}\varphi_n\big|^2
$$
ce qui combin\'e \`a $(5.5)$ nous donne bien ce que nous souhaitions.
\fin

\Refs

\ref
\paper
Le grand crible dans la th\'eorie analytique des nombres
\by E. Bombieri
\yr 1974/1987
\jour Ast\'erisque
\vol 18
\pages 103pp
\endref

\ref
\paper Arithmetical Functions
\by K. Chandrasekharan
\jour Grundlehren der Mathematischen Wissenschaften (Springer-Verlag)
\vol 167
\yr 
\pages 
\endref

\ref
\paper Topics in Multiplicative Number Theory
\by H. Montgomery
\yr 1971
\jour Lecture Notes in Mathematics (Berlin)
\vol 227
\pages 178pp
\endref

\ref
\paper Somme Extremal Functions in Fourier Analysis
\by J. Vaaler
\yr 1985
\jour Bull. AMS
\vol 12
\pages 183--216.
\endref

\endRefs


\enddocument

%%% Local Variables: 
%%% mode: plain-tex
%%% TeX-master: t
%%% TeX-master: t
%%% End: 
