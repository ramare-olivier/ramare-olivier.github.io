\input amstex
\magnification=\magstep1
\documentstyle {amsppt}
\NoBlackBoxes
\refstyle{B}

%%% Version du 25 Janvier 2000

\def\goes{\mathrel{\rightarrow}}
\def\Log{\operatorname{Log}}
\def\section#1{\goodbreak\smallskip\noindent{\bf #1}}
\def\fin{$\diamond\diamond\diamond$\enddemo}
\def\Ker{\operatorname{Ker}}

\document

\topmatter
\title
Caract\`eres de Dirichlet, sommes de Gauss
et th\'eor\`eme de Bombieri-Davenport.
\endtitle
\rightheadtext{Caract\`eres de Dirichlet ... et
th\'eor\`eme de Bombieri-Davenport.}
\author
Olivier Ramar\'e
\endauthor
\abstract
Nous rappelons rapidement ce que sont les caract\`eres de Dirichlet et
nous les exprimons en termes des caract\`eres additifs de $\Bbb Z/q\Bbb
Z$, introduisant ainsi les sommes de Gauss.
Nous donnons alors l'extension grand crible de l'in\'egalit\'e de
Brun-Titchmarsh donn\'ee par Bombieri \& Davenport.
Version du 25 Janvier 2000.
\endabstract
\endtopmatter


\section{I. Les caract\`eres de Dirichlet.}

Nous supposons essentiellement que le lecteur est familier avec la
dualit\'e sur les groupes ab\'eliens finis.

Le groupe multiplicatif de $\Bbb Z/q\Bbb Z$ sera not\'e $\Cal U_q$ ou
$\big(\Bbb Z/q\Bbb Z\big)^*$~; il s'agit d'un groupe ab\'elien
fini. En ce qui concerne sa structure, rappelons que $\Cal U_{p^k}$
est cyclique si $p\neq2$ et $k\ge1$ et que $\Cal U_{2^{k+1}}$ est le
produit direct de $\Bbb Z/2\Bbb Z$ par un 2-groupe cyclique.

Nous consid\'erons alors $\widehat{\Cal U_q}$ le groupe des caract\`eres
de $\Cal U_q$, i.e. le groupe des morphismes de $(\Cal U_q,\cdot)$
dans $(\Bbb C\setminus\{0\},\cdot)$. Il est facile de voir qu'un tel
morphisme prend ses valeurs dans les racines de l'unit\'e et est donc
de module~1. C'est un tel morphisme que l'on appelle {\it un caract\`ere
de Dirichlet}. La th\'eorie g\'en\'erale nous apprend que
$\widehat{\Cal U_q}$ est isomorphe \`a $\Cal U_q$ et en particulier
est de cardinal $\phi(q)$.

Soit $\chi\in\widehat{\Cal U_d}$ o\`u $d$ divise $q$. Nous pouvons bien
s\^ur consid\'erer $\chi$ sur $\Cal U_q$ (en le composant avec la
projection canonique) et le probl\`eme est de d\'efinir un $d$ minimum
tel que $\chi$ provienne de $\widehat{\Cal U_d}$. Pour cela, notons le
r\'esultat suivant~: si $\chi$ provient de $\widehat{\Cal U_{d_1}}$ et de
$\widehat{\Cal U_{d_2}}$, alors $\chi$ provient de $\widehat{\Cal
U_{(d_1,d_2)}}$.
\demo{Preuve}
Remarquons que $\chi$ provient de $\widehat{\Cal U_d}$ si et seulement si
l'ensemble $E_d=\{1+kd, k\in\Bbb Z/q\Bbb Z\}$ est contenu dans
$\Ker\chi$, tout simplement parce que $E_d$ est le noyau de la
projection canonique de $\Cal U_q$ sur $\Cal U_d$. Notre hypoth\`ese
nous dit donc que
$E_{d_1}\subset\Ker\chi$ et similairement pour $E_{d_2}$.
Mais il est facile de v\'erifier que $E_{d_1}\cdot E_{d_2}=E_{(d_1,d_2)}$,
ce qui conclut la preuve.
\fin
Nous pouvons d\`es lors parler de {\it conducteur} d'un caract\`ere,
soit le plus petit $f$ tel ce caract\`ere provienne de $\widehat{\Cal U_f}$.
En particulier, le conducteur du caract\`ere identiquement \'egal
\`a~1, dit {\it principal}, est bien s\^ur~1.
Un caract\`ere modulo $q$ de
conducteur $q$ est dit {\it primitif modulo $q$}

Si $\chi$ est un caract\`ere de Dirichlet sur $\Cal U_q$, nous
l'\'etendons \`a $\Bbb Z/q\Bbb Z$ en posant $\chi(x)=0$ si
$(x,q)\neq1$. Il est alors imm\'ediat de v\'erifier que nous avons
encore $\chi(xy)=\chi(x)\chi(y)$ pour tout $x$, $y$ modulo $q$.
De plus, nous \'etendons aussi $\chi$ \`a $\Bbb Z$ en le composant
avec la surjection canonique. Il faut toutefois remarquer que m\^eme
si $\chi$ modulo $q$ provient de $\chi^*$ modulo $f$, en tant que
fonction sur $\Bbb Z$, $\chi$ et $\chi^*$ sont distincts, tout
simplement parce qu'ils ne prennent pas la m\^eme valeur sur les
entiers qui ne sont pas premiers \`a $q$. En particulier le
caract\`ere principal modulo $q$ n'est pas, en tant que fonction sur
$\Bbb Z$, la fonction identiquement \'egale \`a~1, ce qu'est pourtant
le caract\`ere principal modulo ~1.

Nous ne souhaitons pas entrer dans les d\'etails de la dualit\'e de
Pontrjagin dont le cadre naturel est la cat\'egorie des groupes
ab\'eliens localement compact, mais il nous faut tout de m\^eme faire
une remarque. Notre pr\'eoccupation ici est de d\'eterminer le groupe
des caract\`eres de $\widehat{\Cal U_q}$. Il se trouve qu'il est
possible d'identifier naturellement ce ``double-dual'' avec $\Cal U_q$
en remarquant que $\chi\mapsto\chi(x)$ est un caract\`ere sur
$\widehat{\Cal U_q}$ d\`es que $x\in\Cal U_q$. Un argument de comptage
permet de montrer que nous avons bel et bien exhib\'e un isomorphisme
entre $\widehat{\widehat{\Cal U_q}}$ et $\Cal U_q$.

\section{II. Analyse de Fourier des caract\`eres de Dirichlet.}

Les caract\`eres de Dirichlet pr\'esentent de remarquables
propri\'et\'es $L^2$ qui viennent essentiellement de l'\'egalit\'e
suivante ~:
$$
\sum_{x\mod^* q}\chi_1(x)\overline{\chi_2}(x)=
\cases
\phi(q)\quad&\text{si\quad}\chi_1=\chi_2,\\
0      \quad&\text{si\quad}\chi_1\neq\chi_2,\\
\endcases
$$
dont la duale s'\'ecrit~:
$$
\sum_{\chi\in\widehat{\Cal U_q}}\chi(x_1)\overline{\chi}(x_2)=
\cases
\phi(q)\quad&\text{si\quad}x_1=x_2,\\
0      \quad&\text{si\quad}x_1\neq x_2.
\endcases
$$
\demo{Preuve} Si $\chi_1\neq\chi_2$, alors il existe $x_0$ tel que
$\chi_1(x_0)\overline{\chi_2}(x_0)\neq1$. Comme le produit par $x_0$
est un isomorphisme de $\Cal U_q$, nous avons
$$
\sum_{x\mod^* q}\chi_1(x)\overline{\chi_2}(x)=
\sum_{x\mod^* q}\chi_1(x_0x)\overline{\chi_2}(x_0x)
$$
d'o\`u nous tirons
$$
(1-\chi_1(x_0)\overline{\chi_2}(x_0))
\sum_{x\mod^* q}\chi_1(x)\overline{\chi_2}(x)=0.
$$
Conclusion facile.
\fin
L'\'egalit\'e duale se comprend mieux en notant que
$\chi(x^{-1})=\overline{\chi}(x)$.

\bigskip
Une fois ceci d\^ument not\'e, nous nous tournons vers la comparaison
des structures additives et multiplicatives de $\Bbb Z/q\Bbb Z$. Les
caract\`eres de $(\Bbb Z/q\Bbb Z,+)$ sont les
$$
x\mapsto e(ax/q)
$$
et nous cherchons alors \`a exprimer les caract\`eres multiplicatifs
en termes de ces caract\`eres additifs. Soit donc
$\chi\in\widehat{\Cal U_q}$. Regardons
$$
\tau_q(\chi,a)=\sum_{x\mod^*q}\chi(x)e(ax/q).
$$
L'inversion classique de Fourier (que l'on peut prouver ici
directement) nous donne
$$
\chi(n)=\frac{1}{q}\sum_{a\mod q}\tau_q(\chi,a) e(-an/q)
\qquad(n\in\Bbb Z/q\Bbb Z).
$$
Il nous faut \`a pr\'esent \'evaluer ces coefficients
$\tau_q(\chi,a)$. L'\'egalit\'e de Parseval nous donne
$$
\sum_{a\mod q}|\tau_q(\chi,a)|^2=q\phi(q).
$$
\proclaim{Th\'eor\`eme 1} Soit $\chi$ un caract\`ere de Dirichlet
modulo $q$ et de conducteur $f$. Soit $d$ un diviseur de $q$
et soit $a'\in\Bbb Z/q\Bbb Z$ premier \`a $d$. Nous avons
$$
\tau_{q}(\chi,a'q/d)=\left\{
\aligned
&\mu(d/f)\overline{ \chi^*(a')}\chi^*(d/f)\frac{\phi(q)}{\phi(d)}
\tau_{f}(\chi,1)\quad
\text{si}\quad f|d\quad\text{et}\quad(d/f,f)=1,\\
&0 \qquad\text{sinon},
\endaligned
\right.
$$
o\`u $\chi^*$ est le caract\`ere induit par $\chi$ modulo $f$.
\endproclaim
\demo{Preuve}
Nous avons
$$
\tau_q(\chi,a'q/d)=\sum_{b\mod q}\,e(a'b/d)\chi(b)
=\sum_{c\mod d}e(a'c/d)\sum\Sb b\mod q\\ b\equiv c[d]\endSb\chi(b).
$$
Introduisons le sous-groupe $E_d=\{b\mod q, b\equiv 1[d]\}$. Si
$\Ker\chi\not\subset E_d$, alors les sommes int\'erieures sont nulles. Sinon,
i.e. si $f|d$,
$\chi$ est induit par un caract\`ere modulo $d$ que nous d\'enotons
aussi par $\chi$. Nous avons
$$
\tau_q(\chi,a'q/d)={\phi(q)\over\phi(d)}\sum_{c\mod d}\chi(c)\,e(a'c/d)
={\phi(q)\over\phi(d)}\tau_{d}(\chi,a').
$$
Nous avons donc r\'eduit notre probl\`eme initial au cas
$d=q$. Comme alors $b\mapsto a'b$ est une bijection, nous avons
$$
\tau_{d}(\chi,a')=\sum_{c\mod d}\,e(c/d)\chi(c)\overline{\chi(a')}
=\overline{\chi(a')}\tau_{d}(\chi,1).
$$
Le conducteur de $\chi$ \'etant $f$, nous avons
$$
\tau_{d}(\chi,1)=\sum_{b\mod f}\chi(b)\sum\Sb c\mod^* d\\ c\equiv
b[f]\endSb \,e(c/d)
$$
La somme int\'erieure vaut 0 si $(d/f,f)\neq1$ et $e(\nu b/f)\mu(d/f)$
sinon, o\`u $\nu$ est l'inverse de $d/f$ modulo $f$. Supposant d\`es
lors que $(d/f,f)=1$, nous constatons que l'expression ci-dessus \'egale
$\mu(d/f)\chi^*(d/f) \tau_{f}(\chi)$ o\`u $\chi^*$ est le caract\`ere modulo
$f$ induit par $\chi$.
\fin

Par cons\'equent seul $\tau_f(\chi,1)$ importe vraiment.
Pour ce qui est de son module, il nous suffit d'appliquer Parseval, et
pour cela, de nous restreindre au cas $f=q$.
Il vient alors
$$
|\tau_f(\chi,1)|=\sqrt{f}\qquad(\chi\text{\ de conducteur\ }f).
$$
Notons que lorsque $\chi$ est primitif, nous avons toujours
$$
\left\{\aligned
\tau_q(\chi,a)&=\overline{\chi}(a)\tau_q(\chi,1)
\qquad(\forall a\in\Bbb Z/q\Bbb Z,\quad\chi\text{\ primitif}),
\\
\tau_q(\chi,a)&=\overline{\chi}(a)\tau_q(\chi,1)
\qquad(\forall a\in\Cal U_q,\quad\chi\text{\ quelconque}).
\endaligned\right.
$$
ce qui montre clairement que nous n'avons pas vraiment calcul\'e la
transform\'ee de Fourier de $\chi$, mais bien plut\^ot \'etabli une
\'equation reliant cette transform\'ee \`a $\chi$.

\section{III. Le th\'eor\`eme de Bombieri-Davenport.}

Rappelons l'in\'egalit\'e du grand crible pour la suite de 
Farey~:
$$
\sum_{q\le Q}\sum_{a\mod^*q}|S(a/q)|^2\le \sum_{n\le N}|\varphi_n|^2
(N+Q^2)\leqno(3.1)
$$
o\`u
$$
S(\alpha)=\sum_{M<n\le M+N}\varphi_n \,e(n\alpha).\leqno(3.2)
$$
Ajoutons alors une hypoth\`ese $(H)$ sur la suite $(\varphi_n)$~: nous
supposons qu'elle est port\'ee par $(\Cal U_q)$ jusqu'au niveau $Q$,
ou, dit autrement,
$$
\forall n\in]M,M+ N]\quad\big[\varphi_n\neq0\implies \forall q\le Q
(n,q)=1\big].\leqno(H)
$$
Dans ce cadre, nous d\'efinissons
$$
G_q(Q/q)=\sum\Sb d\le Q/q\\ (d,q)=1\endSb\frac{\mu^2(q)}{\phi(q)}
\leqno(3.3)
$$
et rappelons que nous avons montr\'e que
$G_q(Q/q)\ge\frac{\phi(q)}{q}\Log(Q/q)$.
Nous posons enfin
$$
S(\chi)=\sum_{n\in]M,M+ N] }\varphi_n\,\chi(n)
\leqno(3.4)
$$
la distinction entre $(3.2)$ et $(3.4)$ \'etant clair d'apr\`es le
contexte. 
\proclaim{Th\'eor\`eme (Bombieri \& Davenport --1968)} Si
$(\varphi_n)$ v\'erifie $(H)$, alors
$$
\sum_{q\le Q}
\frac{q}{\phi(q)}G_q(Q/q)
\sum_{\chi\in\widehat{\Cal U_q}^*}|S(\chi)|^2\le
\sum_{n\in]M,M+ N]}|\varphi_n|^2 (N+Q^2)
$$
o\`u $\widehat{\Cal U_q}^*$ est l'ensemble des caract\`eres primitifs
modulo $q$.
\endproclaim

En restreignant le membre de gauche \`a $q=1$, nous obtenons~:
\proclaim{Corollaire}
Si $(\varphi_n)$ v\'erifie $(H)$, alors
$$
\big|S(0)\big|^2\le\sum_{n\in]M,M+ N]}|\varphi_n|^2 \frac{N+Q^2}{G_1(Q)}.
$$
\endproclaim
\demo{Preuve}
Soit $e_{\Cal U_d}(\cdot a/d)$ la fonction d\'efinie par
$$
e_{\Cal U_d}(n a/d)=\cases
e(na/d)\quad&\text{si}\quad n\in\Cal U_d,\\
0\quad&\text{sinon.}
\endcases
$$
Comme $\widehat{\Cal U_d}$ forme une base orthogonale de l'espace
vectoriels des fonctions sur $\Cal U_d$, nous pouvons exprimer
$e_{\Cal U_d}(\cdot a/d)$ en termes des caract\`eres odulo $d$ et plus
pr\'ecis\'ement~:
$$
e_{\Cal U_d}(n a/d)=
\sum_{\chi\in\widehat{\Cal U_d}}[e_{\Cal U_d}(\cdot a/d)|\chi]\,\chi(n)
\quad\text{avec}\quad
[e_{\Cal U_d}(\cdot a/d)|\chi]=\frac{1}{\phi(d)}\sum_{k\in\Cal U_d}e(ka/d)\overline{\chi}(k).
$$
Par ailleurs, par $(H)$, nous avons
$$
S(a/d)=\sum_{n\le N}\varphi_n\,e_{\Cal U_d}(n a/d)=
\sum_{\chi\in\widehat{\Cal U_d}}[e_{\Cal U_d}(\cdot a/d)|\chi]\, S(\chi).
$$
Il vient
$$
\sum_{a\mod d}|S(a/d)|^2=\sum_{\chi_1,\chi_2\in\widehat{\Cal U_d}}
S(\chi_1)\overline{S(\chi_2)}
\sum_{a\mod d}[e_{\Cal U_d}(\cdot a/d)|\chi_1]\,\overline{[a/d|\chi_2]}
$$
o\`u l'on reconna\^\i t en somme int\'erieure le produit scalaire (\`a
normalisation pr\`es) de
$\chi_1$ et $\chi_2$ exprim\'e dans la base $(e(\cdot a/d))$. Par
cons\'equent
$$
\sum_{a\mod d}|S(a/d)|^2=\frac{d}{\phi(d)}\sum_{\chi\in\widehat{\Cal U_d}}
|S(\chi)|^2.
$$
En posant
$$
W(f)=\sum_{\chi\in\widehat{\Cal U_q}^*}|S(\chi)|^2
$$
nous avons donc \'etabli
$$
\sum_{q|d}\sum_{a\mod^*q}|S(a/q)|^2=\frac{d}{\phi(d)}
\sum_{f|d}W(f).
$$
Nous utilisons alors la formule d'inversion de M\oe bius pour obtenir
$$
\aligned
\sum_{a\mod^*q}|S(a/q)|^2
&=\sum_{d|q}\mu(q/d)\frac{d}{\phi(d)}
\sum_{f|d}W(f)
\\&=
\sum_{f|q}\left(\sum_{f|d|q}\mu(q/d)\frac{d}{\phi(d)}\right)W(f)
\endaligned
$$
et il nous suffit \`a pr\'esent de calculer la somme interne. Par
multiplicativit\'e, nous v\'erifions que
$$
\sum_{f|d|q}\mu(q/d)\frac{d}{\phi(d)}=
\prod_p
\left(\sum_{p^{|v_p(f)}|p^a|p^{v_p(q)}}
\mu(p^{v_p(q)-a})\frac{p^a}{\phi(p^a)}\right).
$$
Il nous reste \`a \'evaluer les facteurs locaux. Or
\roster
\item Si $v_p(f)\ge1$ et $v_p(q)-v_p(f)\ge1$ alors ce facteur vaut 0.
\item Si $v_p(f)\ge1$ et $v_p(q)=v_p(f)$ alors ce facteur vaut
$\frac{p^{v_p(f)}}{\phi(p^{v_p(f)})}$.
\item Si $v_p(f)=0$ et $v_p(q)\ge2$ alors ce facteur vaut 0.
\item Si $v_p(f)=0$ et $v_p(q)=1$ alors ce facteur vaut $1/\phi(p)$.
\endroster
Nous en tirons une expression globale qui nous donne
$$
\sum_{a\mod^*q}|S(a/d)|^2=
\sum\Sb f|q\\ (f,q/f)=1\endSb
\frac{f}{\phi(f)}{\mu^2(q/f)}{\phi(q/f)}W(f)
$$
qu'il nous suffit alors d'ins\'erer dans $(3.1)$ pour obtenir
l'in\'egalit\'e annonc\'ee.
\fin

Remarquons que contrairement \`a la preuve usuelle (comme elle est par
exemple reprise dans le livre de Bombieri cit\'e ci-dessous), celle que
nous avons pr\'esent\'ee n'utilise pas la valeur des sommes de Gauss
et en particulier ignore le fait que les $\chi$ soient multiplicatifs.


\Refs

\ref
\paper
Le grand crible dans la th\'eorie analytique des nombres
\by E. Bombieri
\yr 1974/1987
\jour Ast\'erisque
\vol 18
\pages 103pp
\endref


\ref
\paper
On the large sieve method
\book
Abh. aus Zahlentheorie und Analysis zur Erinnerung an Edmund Landau,
\jour Deut. Verlag Wiss., Berlin
\by E. Bombieri \& H. Davenport
\yr 1968
\pages 11--22
\endref

\ref
\paper Sieving by prime powers
\by P.X. Gallagher
\yr 1974
\jour Acta Arith.
\vol 24
\pages 491--497
\endref

\endRefs

\enddocument

%%% Local Variables: 
%%% mode: plain-tex
%%% TeX-master: t
%%% End: 
