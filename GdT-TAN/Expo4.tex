\input amstex
\magnification=\magstep1
\documentstyle {amsppt}
\NoBlackBoxes
\refstyle{B}

%%% Version du 3 Fevrier 2000

\def\goes{\mathrel{\rightarrow}}
\def\Log{\operatorname{Log}}
\def\section#1{\goodbreak\smallskip\noindent{\bf #1}}
\def\fin{$\diamond\diamond\diamond$\enddemo}
\def\Ker{\operatorname{Ker}}

\document

\topmatter
\title
Une r\'egion sans z\'eros pour $\zeta$
\endtitle
\author
Olivier Ramar\'e
\endauthor
\abstract
Nous montrons comment obtenir une r\'egion sans z\'eros pour
$\zeta$. 
Version initiale : 3 F\'evrier 2000.
Version courante : 1 Avril 2013.
\endabstract
\endtopmatter


\section{Introduction.}

Nous donnons ici le sch\'ema classique d'une preuve donnant une
r\'egion sans z\'eros pour la fonction $\zeta$ de Riemann. Nous
d\'egageons trois \'etapes. Il est \`a noter que la premi\`ere
\'etape peut \^etre atteinte de multiples fa\c cons.

Dans cet expos\'e, $\rho=\beta+i\gamma$ d\'esigne un z\'ero de $\zeta$
v\'erifiant $\beta\in]0,1[$.

Nous d\'emontrons ici
\proclaim{Th\'eor\`eme}
Tous les z\'eros de la fonction $\zeta$ de Riemann v\'erifient
$$
(1-\beta)\Log|\gamma|\ge\frac{14-8\sqrt{3}}{5}=0.028718\dots
$$
\endproclaim

La meilleure constante obtenue \`a ce jour est due \`a Rosser et
vaut $1/R$ avec $R=9.645908801\dots$.

\section{I. Majorer $-\Re\zeta'/\zeta(s)$ en fonction des z\'eros
proches de $s$.}

\proclaim{\'Etape no~1}
Soit $s=\sigma+it$ avec $\sigma\in]1,2]$ et soit $r\ge0$. Pour $t>4$, nous
avons
$$
-\Re\frac{\zeta'}{\zeta}(s)\le
-\sum\Sb |\rho-s|\le r\endSb \frac{1}{s-\rho}+\tfrac12\Log (|t|/2.9).
$$
\endproclaim
Auquel il convient d'ajouter
$$
-\frac{\zeta'}{\zeta}(\sigma)\le\frac1{\sigma-1}\qquad
(1<\sigma\le\tfrac32),
$$
in\'egalit\'e dont le lecteur trouvera la d\'emonstration dans le
livre de Ellison/Mend\`es-France (p 184).

\demo{Preuve}

Nous avons, avec $b=-\Log(2\pi)+1+\gamma/2$, 
$$
-\frac{\zeta'}{\zeta}(s)=b+\frac{1}{s-1}+\frac12\frac{\Gamma'}{\Gamma}
(\frac s2+1)-\sum_{\rho}\left(\frac{1}{\rho}+\frac{1}{s-\rho}\right).
$$
Rappelons encore que (Davenport page 84)
$$
\sum_{\rho}\frac1\rho=-\gamma/2+1-\tfrac12\Log(2\pi).
$$
Il vient finalement
$$
-\Re\frac{\zeta'}{\zeta}(s)=-\tfrac12\Log(2\pi)+\Re\frac{1}{s-1}
+\frac12\Re\frac{\Gamma'}{\Gamma}
(\frac s2+1)-\sum_{\rho}\Re\frac{1}{s-\rho}.
$$

Rappelons aussi (voir l'\'egalit\'e $(9)$ de l'article de
de~la~Vall\'ee-Poussin cit\'e ci-apr\`es) que
$$
\Re\frac{\Gamma'}{\Gamma}(u+i\tau)=\Log|\tau|-\frac{u}{2(u^2+\tau^2)}
+\Cal O^*(\frac1{12 u|\tau|}+\frac{u^2}{2\tau^2})
\quad(\tau^2>u^2,\ u\ge0).
$$
En posant $s=\sigma+it$, nous obtenons alors
$$
\aligned
-\Re\frac{\zeta'}{\zeta}(s)
&\le
-\sum_{\rho}\Re\frac{1}{s-\rho}
+\frac12\Log|t|
\\&
+\frac{\sigma-1}{t^2}
-\frac{\sigma+2}{2((\sigma+2)^2+t^2)}
+\frac1{6 (\sigma+2)|t|}+\frac{(\sigma+2)^2}{4t^2}
-\tfrac12\Log(4\pi).
\endaligned
$$
Nous avons alors
$$
\aligned
\frac{\sigma-1}{t^2}
&-\frac{\sigma+2}{2((\sigma+2)^2+t^2)}
+\frac1{6 (\sigma+2)|t|}+\frac{(\sigma+2)^2}{4t^2}
-\tfrac12\Log(4\pi)
\\&\le\frac{5}{t^2}+\frac{1}{18|t|}-\tfrac12\Log(4\pi)
\le\frac{5}{16}+\frac{1}{72}-\tfrac12\Log(4\pi)\le
-\Log(2.9).
\endaligned
$$
La conclusion est facile si l'on remarque que
$$
\Re\frac{1}{\sigma-\rho}\ge0
$$
ce qui nous permet de ne garder dans la somme sur $\rho$ que ceux qui
nous int\'eressent.
\fin



\bigskip
Il existe d'autre fa\c cons de proc\'eder, notamment via le lemme
suivant, d\^u \`a Landau~:

\proclaim{Lemme}
Supposons connue une borne sup\'erieure $M$ pour la fonction $F$
holomorphe dans $|s-s_0|\le R$. Supposons encore que l'on connaisse
une borne inf\'erieure $m$ pour $|F(s_0)|$. Alors
$$
\frac{F'(s)}{F(s)}=\sum_{|\rho-s_0|\le R/2}\frac1{s-\rho}+\Cal O^*\bigg(
8\frac{\Log(M/m)}{R}\bigg)
$$
pour $|s-s_0|\le\tfrac14 R$.
\endproclaim\goodbreak

L'article de Heath-Brown cit\'e ci-apr\`es propose encore une autre
fa\c con de faire. D'un point de vue explicite, il semble que ce soit
une technique de Stechkin (dont le lecteur trouvera l'origine dans
l'article de de la Vall\'ee-Poussin cit\'e ci-dessous)
qui soit le plus efficace. Voir les articles de Stechkin, Rosser et
Ramar\'e \& Rumely cit\'es en fin d'exposition.

\section{II. Une in\'egalit\'e trigonom\'etrique.}


\proclaim{\'Etape no 2}
Soit $Q(\theta)=\sum_{k=0}^Ka_k\cos(k\theta)$ un polyn\^ome
trigonom\'etrique qui v\'erifie $a_k\ge0$ et $Q(\theta)\ge0$.
Nous avons, pour $\sigma>1$,
$$
\Re\sum_{k=0}^Ka_k
\sum_{n\ge1}\frac{\Lambda(n)}{n^\sigma}
\frac{1}{n^{ikt}}\ge0.
\leqno(\ddagger)
$$
\endproclaim

\demo{Preuve}
Il suffit d'\'ecrire
$$
\Re\sum_{k=0}^Ka_k
\sum_{n\ge1}\frac{\Lambda(n)}{n^\sigma}
\frac{1}{n^{ikt}}=
\sum_{n\ge1}\frac{\Lambda(n)}{n^\sigma}
Q(-t\Log n).
$$
\fin

Le polyn\^ome choisi par de la Vall\'ee-Poussin est
$$
2(1+\cos\theta)^2=3+4\cos\theta+\cos(2\theta).
$$
Il est en l'occurence plus efficace d'utiliser (voir Rosser)
$$
8(0.9126+\cos\theta)^2(0.2766+\cos\theta)^2.
$$
Voir les articles de Heath-Brown et de
Liu \& Wang cit\'es ci-dessous pour d'autres remarques sur ces polyn\^omes.


\section{III. La conclusion.}

\proclaim{\'Etape no 3}
Pour $\Re s>1$, nous avons
$$
\Re\frac{1}{s-\rho}\ge0.
$$
\endproclaim

Nous pouvons attaquer la preuve.
Des calculs num\'eriques que nous n'entreprendrons pas montre que nous
pouvons supposer $|\gamma|\ge4$.

Gr\^ace \`a l'\'etape no~3, nous pouvons \'eliminer autant de z\'eros
que nous le souhaitons dans la majoration donn\'ee \`a l'\'etape no~1.
Soit alors $\rho$ un z\'ero de $\zeta$ v\'erifiant $|\gamma|\ge4$.
En combinant cette remarque, l'\'etape no~1 et l'\'etape no~2, nous
obtenons
\def\tvi{\vrule height 11pt depth 8pt width 0pt}
\def\tv{\tvi\vrule}
\def\oh{\omit\hrulefill}
\def\oo{\omit}
\def\cca#1{\hbox to 15 pt{$#1$}}
\def\ccb#1{\hbox to 50 pt{\hfill$#1$}}
\def\ccc#1{\hbox to 75 pt{$#1$}}
\def\ccd#1{\hbox to 50 pt{$#1$}}
$$
\vbox{\offinterlineskip
\halign{%
& \cca{#} &\tv\ \ccb{#} & $\le$ \ccc{#}& \ccd{#} \cr
3\ \times\ & -\frac{\zeta'}{\zeta}(\sigma)&\frac1{\sigma-1}\cr
4\ \times\ & -\frac{\zeta'}{\zeta}(\sigma+it)&-\Re\frac1{\sigma+it-\rho}&+\tfrac12\Log|t|\cr
1\ \times\ & -\frac{\zeta'}{\zeta}(\sigma+2it)&&\mkern15mu\tfrac12\Log|t|\cr
\noalign{\hrule}\cr
\omit&0&\frac{3}{\sigma-1}-\Re\frac4{\sigma+it-\rho}&+\tfrac52\Log|t|\cr
}}
$$
Nous prenons $t=\gamma$. Il vient
$$
0\le\frac{3}{\sigma-1}-\frac4{\sigma-\beta}+\tfrac52\Log|\gamma|
$$
qu'il suffit alors de r\'esoudre. Pour cela, nous posons
$\lambda=(\sigma-1)\Log|\gamma|$ et $\eta=(1-\beta)\Log|\gamma|$. La
r\'esolution en $\eta$ se d\'eroule sans probl\`eme et donne
$$
\eta\ge\frac{2\lambda-5\lambda^2}{6+5\lambda}
$$
et la valeur optimale de $\lambda$ est $\lambda=(4\sqrt{3}-6)/5$, ce
qui donne le r\'esultat annonc\'e.

\bigskip
En utilisant un polyn\^ome trigonom\'etrique g\'en\'eral v\'erifiant
les conditions de l'\'etape no~2 ainsi que $a_1>a_0$, cette m\^eme
m\'ethode donne
$$
(1-\beta)\Log|\gamma|\ge2(1+o(1))\frac{a_1+a_0-2\sqrt{a_1a_0}}{a_1+a_2+\dots+a_K}
$$
o\`u le $o(1)$ est une fonction de $\gamma$ qui tend vers 0 quand
$|\gamma|$ tend vers l'infini.

\Refs

\ref
\paper
Le grand crible dans la th\'eorie analytique des nombres
\by E. Bombieri
\yr 1987
\jour Ast\'erisque
\vol 18
\pages 103pp
\endref

\ref
\paper Zero-free regions for Dirichlet L-functions and the least prime
in an arithmetic progression.
\by D.R.~Heath-Brown
\jour Proc. Lond. Math. Soc.
\vol 64
\yr 1991
\pages 265--338
\endref

\ref
\paper A numerical bound for small prime solutions of some ternary
linear equations
\by M.-C.~Liu \& T.~Wang
\jour Acta Arith.
\vol 86
\yr 1998
\pages 343--383
\endref

\ref
\paper Rational inequalities and zeros of the Riemann zeta-function.
\by S. B. Stechkin
\jour Trudy Mat. Inst. Steklov
\transl English transl. in Proc. Steklov Inst. Math. 189 (1990)
\vol 189
\yr 1989
\pages 127--134
\endref

\ref
\paper Zeros of Riemann zeta-function
\by S. B. Stechkin
\jour Math. Notes
\vol 8
\yr 1970
\pages 706--711
\endref

\ref
\paper Explicit bounds for some functions of prime numbers
\by J. B. Rosser
\jour Amer. J. Math.
\vol 63
\yr 1941
\pages 211--232
\endref

\ref
\paper Primes in Arithmetic Progressions
\by O. Ramar\'e \& R. Rumely
\jour Math. Comp.
\vol 65
\yr 1996
\pages 397--425
\endref

\ref
\paper Sur la fonction $\zeta(s)$ de Riemann et le nombre des nombres
premiers inf\'erieurs \`a une limite donn\'ee
\by C.-J.~de~la~Vall\'ee-Poussin
\jour CBRM : Colloque sur la Th\'eorie des Nombres
\vol Bruxelles, les 19, 20 et 21 D\'ecembre
\yr 1955
\pages 9--66
\endref

\endRefs

\enddocument

%%% Local Variables: 
%%% mode: plain-tex
%%% TeX-master: t
%%% End: 
