\input amstex
\magnification=\magstep1
\documentstyle {amsppt}
\NoBlackBoxes
\refstyle{B}

%%% Version du 4 Fevrier 2000

\def\goes{\mathrel{\rightarrow}}
\def\Log{\operatorname{Log}}
\def\section#1{\goodbreak\smallskip\noindent{\bf #1}}
\def\fin{$\diamond\diamond\diamond$\enddemo}
\def\Ker{\operatorname{Ker}}

\document

\topmatter
\title
Moyennes de fonctions multiplicatives positives~: la m\'ethode de convolution.
\endtitle
\rightheadtext{La m\'ethode de convolution}
\author
Olivier Ramar\'e
\endauthor
\abstract
Nous pr\'esentons une m\'ethode classique pour calculer l'ordre
moyen d'une fonction multiplicative non oscillante.
Version du 4 F\'evrier 2000.
\endabstract
\endtopmatter

\section{I. Premiers pas.}

Soit $f:\Bbb N\setminus\{0\}\goes\Bbb C$ une fonction multiplicative,
i.e. telle que
$$
\left\{
\aligned
&f(1)=1,\\
&f(mn)=f(m)f(n)\qquad\text{si $n$ et $m$ sont premiers entre eux.}
\endaligned
\right.
$$
Une telle fonction est d\'efinie par ses valeurs sur les puissances de
nombres premiers et, formellement, nous avons
$$
D(f,s)=\sum_{n\ge1}\frac{f(n)}{n^s}
=\prod_{p\ge2}\big(\sum_{k\ge0}\frac{f(p^k)}{p^{ks}}\big).
$$
Souvent, la fonction $f$ est assez ``proche'' d'une fonction
connue, et c'est cette id\'ee que nous mettons ici en pratique.

\noindent{\underbar{Exemples :}}

\roster
\item $f_1(n)=\prod_{p|n}(p-2)$. Il vient
$$
\aligned
D(f_1,s)
&=\prod_{p\ge2}\big(1+\frac{p-2}{p^s-1}\big)
=\prod_{p\ge2}\bigg(1-\frac{2p^{s-1}+p-3}{(p^s-1)p^{s-1}}\bigg)\frac{1}{1-1/p^{s-1}}
\\&=C_1(s)\zeta(s-1)
\endaligned
$$
o\`u $C_1(s)$ est holomorphe pour $\Re s>\tfrac32$. Cette \'ecriture
montre que $D(f_1,s)$ est m\'eromorphe pour $\Re s>\tfrac32$ et admet
un p\^ole simple en $s=2$.

\item  $f_2(n)=\mu^2(n)/\phi(n)$. Il vient
$$
\aligned
D(f_2,s)
&=\prod_{p\ge2}\big(1+\frac{1}{(p-1)p^s}\big)
\\&=\prod_{p\ge2}\bigg(1-\frac{1}{(p-1)p^{s+1}}-\frac{1}{(p-1)p^{2s+1}}\bigg)
\frac{1}{1-1/p^{s+1}}
\\&=C_2(s)\zeta(s+1)
\endaligned
$$
o\`u $C_2(s)$ est holomorphe pour $\Re s>-\tfrac12$. Cette \'ecriture
montre que $D(f_2,s)$ est m\'eromorphe pour $\Re s>-\tfrac12$ et admet
un p\^ole simple en $s=0$.

\item  $f_3(n)=2^{\Omega(n)}$. Il vient
$$
\aligned
D(f_3,s)
&=\prod_{p\ge2}\frac{1}{1-\frac{2}{p^s}}
=\prod_{p\ge2}\big(1+\frac{1}{p^{2s}-2p^s}\big)
\zeta^2(s)
\\&=C_3(s)\zeta^2(s)
\endaligned
$$
o\`u $C_3(s)$ est holomorphe pour $\Re s>\tfrac12$. Cette \'ecriture
montre que $D(f_3,s)$ est m\'eromorphe pour $\Re s>\tfrac12$ et admet
un p\^ole double en $s=1$.
\endroster

Nous d\'eveloppons alors les $C_i$ en s\'eries de Dirichlet ~:
$$
C_i(s)=\sum_{n\ge1}\frac{g_i(n)}{n^s}
$$
o\`u les fonctions $g_i$ sont bien s\^ur multiplicatives. Pour obtenir
leurs valeurs exactes, il suffit d'identifier les coefficients dans le
d\'eveloppement du facteur local en s\'erie de $p^-s$. Nous obtenons
ainsi
$$
\left\{
\aligned
g_1(p)&=-2\\
g_1(p^k)&=-(p^2-3p+2), \quad(k\ge2)
\endaligned
\right.
\qquad
\left\{
\aligned
g_2(p)&=g_2(p^2)=-\frac{1}{p(p-1)}\\
g_2(p^k)&=0, \quad(k\ge3)
\endaligned
\right.
$$
ainsi que
$$
\left\{
\aligned
g_3(p)&=0\\
g_3(p^k)&=2^{k-2}, \quad(k\ge2)
\endaligned
\right.
$$
Nous posons aussi
$$
\overline{C_i}(s)=\sum_n\frac{|g_i(n)|}{n^{s}}
$$
et il se trouve que ces s\'eries convergent encore l\`a o\`u nous
avons montr\'e que $C_i$ existait, c'est \`a dire respectivement pour
$\Re s\ge\tfrac32$, $\Re s\ge-\tfrac12$ et $\Re s\ge\tfrac12$.

\bigskip
Occupons-nous \`a pr\'esent des ordres moyens. La traduction sur les
coefficients de $D(f_1,s)=C_1(s)\zeta(s-1)$ nous donne
$$
f_1(n)=\sum_{\ell m=n}g_1(\ell)m
$$
et par cons\'equent
$$
\aligned
\sum_{n\le X}f_1(n)
&=\sum_{\ell m\le X}g_1(\ell)m
=\sum_{\ell\le X}g_1(\ell)
\bigg(\frac12\bigg(\frac X\ell\bigg)^2
+\Cal O\bigg(\frac X\ell\bigg)\bigg)
\\&=\frac{X^2}{2}
\sum_{\ell\le X}\frac{g_1(\ell)}{\ell^2}
+\Cal O\bigg(X \sum_{\ell\le X}\frac{|g_1(\ell)|}{\ell}\bigg)
\endaligned
$$
Nous utilisons alors
\let\ve=\varepsilon
$$
\aligned
\sum_{\ell\le X}\frac{g_1(\ell)}{\ell^2}
&=
\sum_{\ell\ge 1}\frac{g_1(\ell)}{\ell^2}
+\Cal O\bigg(
\sum_{\ell> X}\frac{|g_1(\ell)|}{\ell^2}
\bigg)
\\&=C_1(2)+\Cal O\bigg(
\sum_{\ell> X}\frac{|g_1(\ell)|}{\ell^{\frac32+\ve}}\frac{1}{X^{\frac12-\ve}}
\bigg)
\\&=C_1(2)+\Cal O_\ve\big(X^{-\frac12+\ve}\big)
\endaligned
$$
pour tout $\ve>0$,
et
$$
\sum_{\ell\le X}\frac{|g_1(\ell)|}{\ell}
\le
X^{\frac12+\ve}\sum_{\ell\le X}\frac{|g_1(\ell)|}{\ell^{\frac32+\ve}}
\ll_\ve X^{\frac12+\ve}\qquad(\ve>0).
$$
Bref
$$
\sum_{n\le X}f_1(n)=C_1(2)\frac{X^2}{2}+\Cal O_\ve\big(X^{\frac12+\ve}\big).
$$

\bigskip
Pour ce qui est de l'ordre moyen de $f_2$, nous proc\'edons comme
ci-dessus. Il vient
$$
\aligned
\sum_{n\le X}f_2(n)
&=\sum_{\ell m\le X}g_2(\ell)\frac{1}m
=\sum_{\ell\le X}g_2(\ell)
\bigg(\Log\frac X\ell+\gamma
+\Cal O\bigg(\frac \ell X\bigg)\bigg)
\\&=
\big(\Log X +\gamma\big)\sum_{\ell\le X}g_2(\ell)
-\sum_{\ell\le X}g_2(\ell)\Log\ell
+\Cal O\bigg(\frac1X\sum_{\ell\le X}|g_2(\ell)|\ell\bigg)
\endaligned
$$
o\`u $\gamma$ est la constante d'Euler. 
Le programme pr\'ec\'edent s'applique moyennant de rappeler que
$$
-\sum_{\ell\ge1}\frac{g_2(\ell)\Log\ell}{\ell^s}
\quad\left( \text{resp.}\ 
-\sum_{\ell\ge1}\frac{|g_2(\ell)|\Log\ell}{\ell^s} \right)
$$
est simplement la d\'eriv\'ee de $C_2(s)$ (resp. $\overline{C_2}(s)$)
et que cette s\'erie admet la m\^eme abscisse de convergence absolue
que la s\'erie initiale. Nous obtenons alors
$$
\sum_{n\le X}f_2(n)=
\big(\Log X +\gamma\big)\big(C_2(0)+\Cal O_\ve(X^{-\tfrac12+\ve})\big)
+C'_1(0)+\Cal O_\ve(X^{-\tfrac12+\ve})+\Cal O_\ve(X^{-\tfrac12+\ve}).
$$

\bigskip
Il nous reste \`a nous occuper de $C_3(s)$. Nous avons cette fois-ci
$$
\sum_{n\le X}f_2(n)
=\sum_{\ell m\le X}g_2(\ell)d(m)
$$
o\`u $d(m)$ est le nombre de diviseurs de $m$. Nous avons de fa\c con
classique ~:
$$
\sum_{m\le M}d(m)=M\Log M+(2\gamma-1)M+\Cal O(M^{1/2})
$$
et donc
$$
\aligned
\sum_{n\le X}f_2(n)
&=
\endaligned
$$

\section{II. Un th\'eor\`eme g\'en\'eral.}

Le lemme suivant est une g\'en\'eralisation d'un lemme de Riesel \&
Vaughan (le papier est cit\'e ci-dessous).
\def\O{\Cal O^*}

\proclaim{Lemme 1}
Soit $g$, $h$ et $k$ trois fonctions sur $\Bbb N\setminus\{0\}$ \`a
valeurs complexes.
Posons $H(s)=\sum_nh(n)n^{-s}$, et
$\overline{H}(s)=\sum_n|h(n)|n^{-s}$.
Supposons que $g=h\star k$, que $\overline{H}(s)$ soit convergente pour
$\Re(s)\ge-1/3$ et enfin qu'il existe quatre constantes
$A$, $B$, $C$ et $D$ telles que
$$
\sum_{n\le t}k(n)
=
A\Log^2t+B\Log t+C+\O(D t^{-1/3})
\text{\ \ pour\ \ }
t>0;
$$
Alors, pour tout $t>0$, nous avons :
$$
\sum_{n\le t}g(n)
=
u\Log^2t+v\Log t+w+\O(D t^{-1/3}\overline{H}(-1/3))
$$
avec
$u=AH(0)$, $v=2AH^{\prime}(0)+BH(0)$ and $w=AH^{\prime\prime}(0)+BH^{\prime}(0)+CH(0)$.
Nous avons aussi
$$
\sum_{n\le t}ng(n)
=
Ut\Log t+Vt+W+\O(2.5D t^{2/3}\overline{H}(-1/3))
$$
avec
$$
\left\{\aligned
U=&AH(0),\text{\ \ \ \ \ \ }V=-2AH(0)+2AH^{\prime}(0)+BH(0),\\
W=&A(H^{\prime\prime}(0)-2H^{\prime}(0)+2H(0))+B(H^{\prime}(0)-H(0))+CH(0).
\endaligned\right.
$$
\endproclaim

\demo{Preuve}
\'Ecrivons
$\sum_{\ell\le t}g({\ell})=\sum_{m}h(m)\sum_{n\le t/m}k(n)$, et toute
la r\'egularit\'e de nos expresions vient de ce qu'il n'est pas
n\'ecessaire d'imposer $m\le t$ dans $\sum_{m}h(m)$.
Nous compl\'etons alors la preuve facilement

Pour estimer $\sum_{\ell\le t}\ell g({\ell})$ for $t>0$, nous \'ecrivons
$$
\sum_{\ell\le t}\ell g({\ell})=
t\sum_{\ell\le t}g({\ell})-
\int_1^t \sum_{\ell\le u}g({\ell})du,
$$
et utilisons l'expression asymptotique de $\sum_{\ell\le u}g({\ell})$.
\fin


Pour appliquer le lemme pr\'ec\'edent, nous aurons besoin de

\proclaim{Lemme 2} Pour tout $t>0$, nous avons
$$
\sum_{n\le t}\frac 1n=\Log t+\gamma+\O\big(0.9105 t^{-1/3}\big).
$$
Soit $d(n)$ le nombre de diviseurs de $n$. Pour tout $t>0$, nous avons
$$
\sum_{n\le t}\frac{d(n)}n=
\frac12\Log^2t+2\gamma\Log t+\gamma^2-\gamma_1+\O\big(1.641 t^{-1/3}\big),
$$
avec
$$
\gamma_1=\lim_{n\rightarrow\infty}
\left(\shave{\sum_{m\le n}}\frac{\Log m}m-\frac{\Log^2n}2
\right).
$$
$(-0.072816<\gamma_1<-0.072815)$.
\endproclaim
\goodbreak
\demo{Preuve}
La preuve de la seconde partie de ce lemme se trouve dans le papier de
in Riesel \& Vaughan cit\'e ci-dessous
(Lemma 1).

Pour la premi\`ere partie, rappelons que
$$
|\sum_{n\le t}\frac1n-\Log t-\gamma|\le\frac{7}{12t}
\text{\ \ pour\ \ }t\ge1.
$$
Pour $0<t<1$, nous choisissons $a>0$ tel que $\Log t+\gamma+a\ t^{-1/3}\ge0$. Cette
fonction d\'ecro\^\i t de $0$ \`a $(a/3)^3$ et ensuite cro\^\i t. Cela
nous donne la valeur minimale $a=3\exp(-\gamma/3-1)\le0.9105$.
\fin

Dans la pratique, la fonction $g$ sera multiplicative et v\'erifiera
$g_p=b/p+o(1/p)$ avec $b=1$ or 2. Dans ce cas, nous prenons $\sum
k(n)n^{-s}=\zeta(s+1)^b$ et $h$ est la fonction multiplicative
d\'etermin\'ee par $\sum h(n)n^{-s}=\sum g(n)n^{-s}\zeta(s+1)^{-b}$.

Lorsque $h$ est multiplicative, nous avons
$$
H(0)=\prod_p(1+\sum_mh({p^m})),
$$
et
$$
\frac{H^{\prime}(0)}{H(0)}=
\sum_p
\frac{\sum_mmh({p^m})}{1+\sum_mh({p^m})}(-\Log p),
$$
ainsi que
$$
\frac{H^{\prime\prime}(0)}{H(0)}=
\bigg(
\frac{H^{\prime}(0)}{H(0)}
\bigg)^2+
\sum_p
\bigg\{
\frac{\sum_mm^2h({p^m})}{1+\sum_mh({p^m})}
-\bigg(\frac{\sum_mmh({p^m})}{1+\sum_mh({p^m})}\bigg)^2
\bigg\}
\Log^2p.
$$

Un exemple.

\proclaim{Lemma 3} Pour tout $X>0$ et tout entier $d\ge1$, nous avons
$$
\sum\Sb n\le X\\ (n,d)=1\endSb
\frac{\mu^2(n)}{\phi(n)}=
\frac{\phi(d)}d
\bigg\{\!
\Log X+
\gamma+\sum_{p\ge2}\frac{\Log p}{p(p-1)}
+\sum_{p|d}\frac{\Log p}{p}
\bigg\}
+\O(
7.284 X^{-1/3}f_1(d))
$$
avec
$$
f_1(d)=
\prod_{p|d}
(1+p^{-2/3})
\big(1+\frac{p^{1/3}+p^{2/3}}{p(p-1)}\big)^{-1}.
$$
\endproclaim

\underbar{Remarque} :
La somme de gauche est $G_d(X)$. Le cas $d=1$ a d\'ej\`a \'et\'e
\'etudi\'e plus haut. La s\'erie de Dirichlet associ\'ee est
$$
\sum_{n}\frac{\mu^2(n)}{\phi(n)n^{s-1}}=
\frac{\zeta(s)}{\zeta(2s)}
\prod_{p\ge2}
\bigg(
1+\frac1{(p-1)(p^s+1)}
\bigg)
$$
ce qui fait que le terme d'erreur $\Cal O(X^{-1/2})$ est admissible
(notre m\'ethode pourrait donner $\Cal O(X^{-1/2}\Log^2X)$), et que
nous ne pouvons esp\'erer mieux que $\Cal O(X^{-3/4})$.
\bigskip

Rosser \& Schoenfeld (voir ci-dessous, \'equation (2.11) du papier
cit\'e) nous donne
$$
\gamma+\sum_{p\ge2}\frac{\Log p}{p(p-1)}=
1.332\ 582\ 275\ 332\ 21...
$$
\demo{Preuve}
D\'efinissons la fonction multiplicative $h_d$ par
$$
h_d(p)=\frac1{p(p-1)},
\quad h_d(p^2)=\frac{-1}{p(p-1)},
\quad h_d(p^m)=0
\quad \text{si\ \ }m\ge3,
$$
si $p$ est un nombre premier qui ne divise pas $d$,
et par $h_d(p^m)=\frac{\mu(p^m)}{p^m}$ pour tout
$m\ge1$ si $p$ est un facteur premier de $d$.

Nous avons alors
$$
\sum\Sb n\ge1\endSb\frac{h_d(n)}{n^s}\zeta(s+1)=
\sum\Sb n\ge1\\ (n,d)=1\endSb\frac{\mu^2(n)}{\phi(n)n^s}
$$
ce qui nous permet d'appliquer le lemme 2. Nous v\'erifions que
$$
\prod_{p\ge2}
\bigg(1+\frac{p^{1/3}+p^{2/3}}{p(p-1)}\bigg)
\le8.
$$
\fin

\proclaim{Lemma 4}
\roster
\item Pour $z\ge1$, nous avons $G(z)\le\Log z+1.4709$.
\item Pour $z\ge6$, nous avons $1.07+\Log z\le G(z)$.
\item Pour $z\ge\exp(18)$ et $\alpha\ge1.1$, nous avons $G(z^\alpha)\le\alpha G(z)$.
\endroster
\endproclaim

\demo{Preuve}
La premi\`ere partie provient de notre expression asymptotique si $z\ge146\ 050$.
Finir par des calculs serait assez co\^uteux en temps, aussi modifions
nous le lemme~2 et prenons l'exposant $0.45$ au lieu de $1/3$. Nous
avons alors $G(z)-\Log z\le1.4708$ d\`es que
$$
\frac1{0.45}\Log\big(\overline{H}(-0.45)
\frac{\exp(-1-0.45\gamma)}{0.45(1.4708-1.332583)}\big)\le\Log z.
$$
Il nous faut calculer $\overline{H}(-0.45)$, et pour ce faire, un
contr\^ole du terme d'erreur est n\'ecessaire.
Nous avons
$$
\prod_{2\le p\le200\ 000}(1+\frac{p^{0.45}+p^{0.9}}{p(p-1)})\le20.26
$$
et, avec $F(t)=(t^{0.45}+t^{0.9})/[t(t-1)\Log t]$ et $X=200\ 000$,
nous obtenons
$$
\Log\prod_{p>X}(1+\frac{p^{0.45}+p^{0.9}}{p(p-1)})\le
1.001\ 093\big(XF(X)+\int_X^{\infty}F(t)dt\big)\le0.266\ 47
$$
en utilisant $\theta(t)\le1.001\ 093t$ si $t>0$ (cf l'article de Schoenfeld).
D'o\`u l'in\'egalit\'e voulue si $z\ge42\ 300$.
Un calcul direct donne alors
$$
\max_{z\ge1}(G(z)-\Log z)=G(7)-\Log 7\le1.4709.
$$
La seconde in\'egalit\'e est d\^ue \`a Montgomery \& Vaughan
(Lemma 7 du papier cit\'e ci-dessous). La troisi\`eme assertion est
une cons\'equence des deux premi\`eres,
\fin

\Refs


\ref
\by H.Montgomery \& R.C.Vaughan
\paper The large sieve
\jour Mathematika
\vol 20 no 2
\yr 1973 
\pages 119--133
\endref

\ref
\by H.Riesel \& R.C.Vaughan
\paper On sums of primes
\jour Arkiv f\"or mathematik 
\vol 21
\yr 1983
\pages 45--74
\endref

\ref
\by J.B.Rosser \& L.Schoenfeld 
\paper Approximate formulas for some functions of prime numbers
\jour Illinois J. Math.
\vol  6 
\yr 1962
\pages 64--94
\endref


\ref
\by L.Schoenfeld
\paper Sharper bounds for the Chebyshev functions $\psi$ and $\theta$. II
\jour Math. Comp.
\vol 30 no 134
\yr 1976 
\pages 337--360
\endref

\endRefs

\enddocument

%%% Local Variables: 
%%% mode: plain-tex
%%% TeX-master: t
%%% End: 
