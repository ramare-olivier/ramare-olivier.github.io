\documentclass[12pt]{article}

\usepackage{amsfonts} %\usepackage{eufrak} included
\usepackage{amssymb}
\usepackage{amsmath}
%\usepackage{amsthm} % \begin{proof}
\usepackage{mathrsfs} % \mathscr
\usepackage{latexsym}%utilisation des symboles LaTeX pour avoir un beau LaTeX
\usepackage{enumerate}
\usepackage{multicol}
\usepackage{verbatim}
\usepackage{latexsym}%utilisation des symboles LaTeX pour avoir un beau LaTeX ou TeX
\usepackage{url}

%\usepackage{epic}
%\usepackage[dvips]{graphicx}
%\usepackage[perpage,symbol*]{footmisc} %% Fancy footnotes symbols
%\usepackage{pstricks}
%\usepackage{pst-plot}

%\arraycolsep=1.4pt



\def\goes{\mathrel{\rightarrow}}

\def\mode{\,\mathrel{\Mode^*}}
\newcommand{\Log}{\mathop{{\rm Log}}\nolimits}
%\DeclareMathOperator{\Log}{Log}
%\DeclareMathOperator{\Id}{Id}
\newcommand{\lcm}{\mathop{{\rm lcm}}\nolimits}
\newcommand{\li}{\mathop{{\rm li}}\nolimits}
\newcommand{\Li}{\mathop{{\rm Li}}\nolimits}
\def\mff{\mathfrak{f}}
\def\mfg{\mathfrak{g}}
%\def\gM{\mathfrak{M}}
\def\Oc{\mathcal{O}}
\def\Xc{\Theta}
\def\Wc{\mathcal{W}}
\def\1{1\!\!1}
\def\Dis{\mathscr{D}}
\def\F{\mathscr{F}}
\def\G{\mathscr{G}}
\def\scrE{\mathscr{E}}
\def\mfG{\mathfrak{G}}
\def\mfH{\mathfrak{H}}
\def\mfB{\mathfrak{B}}
\def\mfL{\mathfrak{L}}
\newcommand{\MT}{\mathrm{M}}
\newcommand{\gM}{\mathfrak{M}}
\newcommand{\Pc}{\mathcal{P}}
\newcommand{\Qc}{\mathcal{Q}}
\newcommand{\Sc}{\mathcal{S}}
\newcommand{\Q}{\mathcal{Q}}
\newcommand{\Ucal}{\mathcal{U}}
\newcommand{\Kcal}{\mathcal{K}}
\newcommand{\Acal}{\mathcal{A}}
\newcommand{\Bcal}{\mathcal{B}}
\newcommand{\Ccal}{\mathcal{C}}
\newcommand{\Dcal}{\mathcal{D}}
\newcommand{\Ocal}{\mathcal{O}}
\newcommand{\Hcal}{\mathcal{H}}
\newcommand{\HHcal}{\mathscr{H}}
\newcommand{\Lcal}{\mathcal{L}}
%\newcommand{\L}{\Lambda}
\newcommand{\vp}{\varphi}
\newcommand{\ve}{\varepsilon}
\newcommand{\shift}{\mathfrak{s}}
\newcommand{\Z}[1]{\mathbb{Z}/#1\mathbb{Z}}
\def\sume{\mathop{\sum\mkern-3mu\hbox to 0pt{\raise3pt\hbox{${}^*$}\hss}\mkern3mu}}
\def\sumh{\mathop{\sum\mkern-3mu\hbox to 0pt{\raise3pt\hbox{${}^\star$}\hss}\mkern3mu}}
%Version envoyee a publication le 
%Revision le 
%Revision le 

\title{Additive energy of dense sets of primes and monochromatic sums}

\author{O. Ramar\'e}
%\date{\sl January, the 7th of 2004}

%\classno{11N35, 11N36 (primary), 42B10 (secondary)}

%\date{Received: date / Accepted: date}

\begin{document}
%\email{ramare@math.univ-lille1.fr}
%\keywords{Weighted sieve, Selberg sieve, Prime $\kappa$-tuple}

  When $K\geq 1$ is an integer and $S$ is a
       set of prime numbers in the interval $(\frac{N}{2},N]$ with $|S| \geq
       \pi^{*}(N)/K$, where $\pi^{*}(N)$ is the number of primes in this
       interval, we obtain an upper bound for the additive energy of $S$,
       which is the number of quadruples $(x_1,x_2,x_3,x_4)$ in $S^4$
       satisfying $x_1+x_2 = x_3+x_4$. We obtain this bound by modifying a
       method of Ramar{\'e} and Ruzsa. Taken together with an argument due to
       N. Hegyv{\'a}ri and F. Hennecart this bound implies that when the
       sequence of prime numbers is coloured with $K$ colours, every
       sufficiently large integer can be written as a sum of no more than $CK
       \log\log 4K$ prime numbers, all of the same colour, where $C$ is an
       absolute constant. This assertion is optimal in its dependence on $K$
       and answers a question of A. S{\'a}rk{\"o}zy.


\end{document}

%%% Local Variables: 
%%% mode: latex
%%% TeX-master: t
%%% End: 
