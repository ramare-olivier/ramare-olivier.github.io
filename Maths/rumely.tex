% **************************************************************************
% * The TeX source for AMS journal articles is the publisher's TeX code    *
% * which may contain special commands defined for the AMS production      *
% * environment.  Therefore, it may not be possible to process these files *
% * through TeX without errors.  To display a typeset version of a journal *
% * article easily, we suggest that you either view the HTML version or    *
% * retrieve the article in DVI, PostScript, or PDF format.                *
% **************************************************************************
%Author Package file for use with AMS-Tex 2.1
% Date: 26-JAN-1995
%   \pagebreak: 0   \newpage: 0   \displaybreak: 0
%   \eject: 19   \bye: 0   \break: 0   \allowbreak: 0
%   \allowdisplaybreak: 0   \allowdisplaybreaks: 0
%   \allowlinebreak: 0   \allowmathbreak: 0
%   \smallpagebreak: 0   \medpagebreak: 0   \bigpagebreak: 0
%   \smallbreak: 0   \medbreak: 0   \bigbreak: 0   \goodbreak: 0
%   \linebreak: 0   \mathbreak: 0   \newline: 0
%   \magnification: 1   \mag: 0
%   \baselineskip: 0   \normalbaselineskip: 0
%   \hsize: 0   \vsize: 0   \pagewidth: 0   \pageheight: 0
%   \hoffset: 0   \voffset: 0   \hcorrection: 0   \vcorrection: 0
%   \parindent: 0   \parskip: 0
%   \vfil: 0   \vfill: 19   \vskip: 4
%   \smallskip: 62   \medskip: 13   \bigskip: 11
%   \sl: 2   \def: 0   \let: 0   \redefine: 0   \predefine: 0
%   \tolerance: 0   \pretolerance: 0
%   \font: 0   \end: 0   \noindent: 29
%   ASCII 13 (Control-M Carriage return): 0
%   ASCII 10 (Control-J Linefeed): 0
%   ASCII 12 (Control-L Formfeed): 0
%   ASCII 0 (Control-@): 0
% Special characters: 0
%
%
\input amstex
\controldates{22-OCT-1995,22-OCT-1995,22-OCT-1995,20-NOV-1995}
\documentstyle{mcom} % {amsppt}
\issueinfo{65}{213}{JAN}{1996}
%\keyedby{ramare/xxx}
%\keyedby{mcom669e/lbd}

% Version of the 15th June 1994
%\NoBlackBoxes
\let\thm\proclaim
\let\ethm\endproclaim
\let\rem\remark
\let\endrem\endremark
\let\dfn\definition
\let\enddfn\enddefinition
\let\num\rom
\let\cal\Cal
\define\<{\!}
%\loadeurm
%\define\sans{\eurm}

\let\sl\it
\define\la{\Lambda}
\define\bn{\bigskip\noindent}
%\define\fin{\enddemo$\diamond\diamond\diamond$} 
\define\fin{\csname enddemo\endcsname$\diamond\diamond\diamond$} 
\define\dif{\partial}
\define\s{\sigma}
\define\k{\kappa}
\define\e{\varepsilon}

%\magnification=\magstep1
%\centerline {\bf 

\topmatter
\title Primes in arithmetic progressions\endtitle  \vskip.25in
\author %\centerline{
Olivier Ramar\'e and Robert Rumely\endauthor %}

\address %Olivier Ramar\'e,
D\'epartement de Math\'ematiques,
Universit\'e de Nancy I,
URA 750,
54506 Van-\linebreak
doeuvre Cedex,
France\endaddress%}

\address%{Robert Rumely
Department of Mathematics,
University of Georgia,
Athens, Georgia 30602
\endaddress%}
\date February 26, 1993 and, in revised form, January 24, 1994, June 27, 1994,
and January 10, 1995\enddate
\subjclass Primary 11N13, 11N56, 11M26; Secondary 11Y35, 11Y40,
11--04\endsubjclass
\abstract Strengthening work of Rosser, Schoenfeld, and McCurley, we
establish explicit Chebyshev-type estimates in the prime number
theorem for arithmetic progressions, for all moduli $k \le 72$ and
other small moduli.
\endabstract
\endtopmatter

%\vskip-.25in
\document

\head 1. Introduction\endhead

\smallskip In many applications it is useful to have explicit error
bounds in the prime number theorem for arithmetic progressions. 
Furthermore, in numerical work, such estimates for small moduli are
often the most critical.  Let us recall the usual notation, where 
$x \ge 0$  is real:
$$
\align
\theta(x;k,l) &= \sum\Sb p\equiv l(k)\\ p\le x\endSb \ln(p),\quad
\text{ where } p \text{ denotes a prime number};\\
\psi(x;k,l) &= \sum\Sb n \equiv l(k)\\ n\le x\endSb
\Lambda(n), \quad \text{ where } \Lambda(n) \text{ is Von Mangoldt's
function}.
\endalign
$$ 
Here we obtain estimates of the type
$$ 
(1 - \epsilon) \frac{x}{\varphi(k)} \le \theta (x; k, l) \le
\psi(x; k, l) \le (1 + \epsilon) \frac{x}{\varphi(k)}
$$ for all moduli $k \le 72$, all composite $k \le 112$, and 48
other moduli; $\varphi(k)$  denotes Euler's function.
\smallskip

To obtain such estimates for the progression with modulus 1, Rosser
and Schoenfeld (\cite{8, 9, 11}) developed an analytic
method which combines a numerical verification of the Riemann
Hypothesis to a given height together with an explicit asymptotic
zero-free region. McCurley (\cite{3, 4, 5, 6})
adapted this method to progressions with modulus  $k >
1$.  However, except for 
$k = 3$,  he got poor numerical results because of the paucity of
numerical work on the Generalized Riemann Hypothesis.  Recently, the
second author \cite{10} did such computations, and we get reasonably
good results for all moduli accessible from his list.  In fact, the
results are better than one would expect, owing to a smoothing process
implicit in the method, explained in\pagebreak\ \S4.4.
%\smallskip

Our main theoretical result is Theorem 4.3.2, which provides an easy
way to get applications like Theorem 1 below.  In Theorem 3.6.2 we
also establish a zero-free region for Dirichlet
$L$-functions, valid for heights  $|t| \ge 1000$, which improves
somewhat on the region found by McCurley \cite{4}.
%\smallskip

Finally, we mention that this paper is an important step in a
forthcoming paper by the first author, in which it is proved that
{\it every} even integer can be written as a sum of at most 
$6$  primes.
%\smallskip

Our main numerical result is the following:
%\vfill\eject

\proclaim{Theorem 1}  For any triple  $(k,\varepsilon,x_0)$  given
by Table $1$ \rom{(}see \rom{\S5)}, and any  $l$  prime to  $k$, we have
$$
\align &\max\limits_{1\le y\le x} |\theta(y;k,l) -
\frac{y}{\varphi(k)}|
\le
\varepsilon \frac{x}{\varphi(k)}\quad \text{for } x \ge x_0 , \\
&\max\limits_{1\le y\le x} |\psi(y;k,l) - \frac{y}{\varphi(k)}|
\le
\varepsilon \frac{x}{\varphi(k)}\quad \text{for } x \ge x_0 .
\endalign
$$ For example, if  $k = 1$,  
$$
\gather
\varepsilon = 0.000213  
\text{\ \  for\ \  }  x_0 = 10^{10} , \quad
\varepsilon = 0.000015 \text{\ \  for\ \  } x_0 = 10^{13} ,\\
\varepsilon = 0.000001 \text{\ \  for\ \  } x_0 = 10^{30} .
\endgather
$$ If $k \le 13$, one can take
$$
\align
\varepsilon &= 0.004560 \text{\ \  for\ \  } x_0 = 10^{10} , 
\quad \varepsilon = 0.002657 \text{\ \  for\ \  } x_0 = 10^{13} , \\
\varepsilon &= 0.002478 \text{\ \  for\ \  } x_0 = 10^{30} ,\quad  
\varepsilon = 0.002020 \text{\ \  for\ \  } x_0 = 10^{100} . 
\endalign
$$ If  $k \le 72$, one can take
$$
\align
\varepsilon &= 0.023269 \text{\ \  for\ \  } x_0 = 10^{10} ,\quad 
\varepsilon = 0.011310 \text{\ \ for\ \  } x_0 = 10^{13} , \\
\varepsilon &= 0.010484 \text{\ \  for\ \  } x_0 = 10^{30} , \quad
\varepsilon = 0.008672 \text{\ \  for\ \  } x_0 = 10^{100} .
\endalign
$$
\endproclaim
%\noindent 
We have supplemented these analytic results with a
tabulation up to  $10^{10}$, obtaining surprisingly uniform bounds:

\proclaim{Theorem 2}  For all moduli $k$ in Table $1$, and all $l$
prime to $k$, uniformly for 
$1 \le x \le 10^{10}$,
$$
\align
\max\limits_{1\le y\le x}\left| \theta(y;k,l) -
\frac{y}{\varphi(k)}\right| &\le 2.072
\sqrt{x} ,\\
\max\limits_{1\le y\le x}\left| \psi(y;k,l) -
\frac{y}{\varphi(k)}\right| &\le 1.745
\sqrt{x}  .
\endalign
$$ 
In particular, for a bound of the type in Theorem $1$ when  $x_0 \le x
\le 10^{10}$, one can take
$$
\varepsilon = 2.072 \varphi(k)/\sqrt{x}_0 .
$$ 
Sharper bounds for individual moduli are given in Table $2$
\rom{(}see \rom{\S5)}. 
Especially, for  $k=5$  or $k \ge 7$, the constant  $1.745$  in the
error  bound for  $\psi(x;k,l)$  can be replaced by  $1.000$, and
for  $k=1$  it can be replaced by  $\sqrt{2}$\pagebreak.
\endproclaim
%\medskip

%\vfill\eject

\head 2. Numerical results about the GRH\endhead
\smallskip

Throughout the paper,  $p$  always stands for a prime and the 
$\gcd$  of  $k$  and  $l$  is written  $(k,l)$. 

The letter $\rho$ always denotes a nontrivial zero of a Dirichlet 
$L$-function, i.e., a zero with $0<\Re \rho<1$. We will always write
$\rho=\beta+i\gamma$. 
%\smallskip 

Given a Dirichlet $L$-function $L(s,\chi)$, the set of its zeros
with  $0 < \beta < 1$  will be denoted by
$\Cal Z(\chi)$. Thus, if $\chi^{\prime}$ is induced by $\chi$, then
$\Cal Z(\chi^{\prime})=\Cal Z(\chi)$.
%\smallskip

Given a nonnegative real number $H$, we will say that $L(s,\chi)$ 
satisfies $\operatorname{GRH}(H)$, the Generalized Riemann Hypothesis to height
$H$,  if all its nontrivial zeros with
$|\gamma|\le H$ verify $\beta=\frac12$. 
%\smallskip

It is possible to prove or disprove  $\roman{GRH}(H)$ for any given $H$ and
$L(s, \chi)$ with only a finite amount of computation. Rumely [10]
did such computations, obtaining: 

\proclaim{Theorem 2.1.1}
%\smallskip

$\circ$ Every $L$-function associated with a Dirichlet character of 
conductor $k\le13$ satisfies $GRH(10000)$.
%\smallskip

$\circ$ Every $L$-function associated with a Dirichlet character of 
conductor in the sets
$$
\align &\{k\le72\},\\ &\{k\le112,k\text{\ not\ prime}\},\\
&\{116,117,120,121,124,125,128,132,140,143,144,\\ &\ \ \ \ \ 
156,163,169,180,216,243,256,360,420,432\} \endalign
$$ satisfies $GRH(2500)$.
\endproclaim

Sharper results for individual moduli, together with bounds for the
sums
$$
\sum\Sb |\text{Im}(\rho)|\le H \\ \rho \in \Cal Z(\chi) \endSb
\frac{1}{|\rho|}
$$ may be found in Tables 1 and 5 of \cite{10}; these results were
used in computing Table 1 below.  The following facts about the
Riemann zeta function were also used:
$$
\align &\zeta(s) \text{ satisfies GRH(545439823.215)} \\ &\hskip1in
\text{(van de Lune, te Riele, and Winter
\cite{2})};\\  &\text{for } H = 12030, \sum_{|\text{Im}(\rho)| \le H}
\frac{1}{|\rho|} \le 9.056 \\ &\hskip1in \text{(R. S. Lehman, cf.
Rosser and Schoenfeld [9])} .
\endalign
$$
%\vfill\eject
\head 3. Zero-free regions for $L$-functions\endhead

This part follows McCurley \cite{4} and some  evaluations are taken
directly from McCurley's paper without being re-established. We
employ a device due to Stechkin \cite{12} to widen the zero-free
region.
%\bigskip


Let $s=\s+it$ and $s_1=\s_1+it$ be complex numbers with 
$|t| \ge 1000$  satisfying
$$
\left\{\aligned
&1<\sigma\le\frac{1+\sqrt{2}}2\le\sigma_1\le\frac{1+\sqrt{5}}2,\\ 
&\sqrt{\sigma(\sigma-1)}+\sqrt{\sigma_1(\sigma_1-1)}=1
\endaligned\right.
\leqno(3.1)
$$ and put $\k(\s)=\frac{1+\frac45\sqrt{\s-1}}{\sqrt{5}}$. Note that
$1/\sqrt{5} \le \k(\s)\le0.61004$, and that $\s\ge\k(\s)\s_1$. 
%\smallskip 

These assumptions on $\sigma$ and $\sigma_1$ will hold throughout
the paper.

We define the auxiliary function
$$ f(t,\chi)=\Re\left(\sum_{n\ge1}\la(n)\left(\frac1{n^{\s}}-
\frac{\k(\s)}{n^{\s_1}}\right) \frac{\chi(n)}{n^{it}}\right)
\leqno(3.2)
$$ and introduce constants
$$
\left\{\aligned
\e_1&=0.0067,\\
\e_2&=10^{-6},
\endaligned\right. \qquad
\left\{\aligned &a_0=11.1859355312082048,\\ &a_1=19.073344004352,\\
&a_2=11.67618784,\\ &a_3=4.7568,\\ &a_4=1,\\ &A=a_1+a_2+a_3+a_4 =
36.506331844352 .
\endaligned\right.
\leqno(3.3)
$$
\subhead 3.1. A device of Stechkin\endsubhead
\proclaim{Lemma 3.1.1 \rom{(Stechkin)}} Let $z$ be a complex number with 
$0\le\Re z\le1$. Put 
$$ F(s,z)=\Re\big\{\frac1{s-z}+\frac1{s-(1-\bar{z})}\big\}. 
$$ Then
$$ F(s,z)\ge \kappa(\sigma)F(s_1,z).
$$ Moreover, if $\Im z= \Im s$ and $\frac12\le\Re z\le1$, and if  
$S(\sigma) = -\frac{1}{\sigma} +
\kappa(\sigma)\left(\frac{1}{\sigma_1} +
\frac{1}{\sigma_1-1}\right)$, then 
$$ -\Re\left(\frac{1}{s-(1-\bar{z})}\right) + \kappa(\sigma)F(s_1,z)
\le S(\sigma) .
$$
\endproclaim
%\vfill\eject

\demo{Proof}  The first inequality is implicit in Stechkin
\cite{12}:  see the top line on p.132 of the English translation. 
For the second inequality, note that when  $\Im z = \Im s = t$, then,
writing  $z = \beta + it$, the left side of the inequality becomes
$$ -\frac{1}{\sigma-1+\beta} +
\kappa(\sigma)\left(\frac{1}{\sigma_1-\beta} +
\frac{1}{\sigma_1-1+\beta}\right) .
$$
For  $\frac{1}{2} \le \beta \le 1$, both terms are increasing
with 
$\beta$.  Taking  $\beta = 1$  we obtain the result.
\qed\enddemo
%\fin
%\medskip

As will be seen later, we will only need to consider  $\sigma$  in
the interval  $[1,1.062]$.  Although we will not use this, it can be
shown that on  $[1,1.062]$
$$ S(\sigma) = 2\sqrt{\Delta} + \frac{86}{25}\Delta + \Cal
O(\Delta^{3/2}) ,
$$ where  $\Delta = \sigma - 1$  and the implied constant can be
taken as 18.  Furthermore, throughout the interval,
$$ 2\sqrt{\Delta} + 3\Delta \le S(\sigma) \le 2\sqrt{\Delta} +
5\Delta .
$$

\subhead 3.2. Handling the $\Gamma$-factor\endsubhead
\proclaim{Lemma 3.2.1} If $0\le a\le2$, $\e_1=0.0067$ and 
$\e_2=10^{-6}$, then for  $|t| \ge 1000$
$$
\frac12\Re\left\{
\frac{\Gamma^{\prime}}{\Gamma}\big(\frac{s+a}{2}\big) -
\k(\s)\frac{\Gamma^{\prime}}{\Gamma}\big(\frac{s_1+a}{2}\big)\right\}
\le\frac{1-\k(\s)}2\left(\ln\left(\frac{|t|}2\right)+\e_1-10\e_2
\right). 
$$
\endproclaim

\demo{Proof}  We follow Lemma 2 of McCurley (\cite{4}).  Since 
$\Re\left(\frac{\Gamma'}{\Gamma} (z)\right)$  is invariant under 
$z \rightarrow \bar{z}$, we have for  $z = x+iy$, with  $x > 0$:
$$
\Re \frac{\Gamma'}{\Gamma}\left(\frac{z+a}{2}\right) =
\ln\left(\frac{|y|}{2}\right) + \frac{1}{2} \ln\left| 1 +
\left(\frac{x+a}{y}\right)^2\right| - \frac{x+a}{(x+a)^2+y^2} +
\Theta(y) \leqno(3.2.1)
$$ where  $|\Theta(y)| \le \frac{1}{|y|} \cdot \left[\frac{\pi}{2} -
\arctan\left(\frac{x+a}{|y|}\right)\right] \le
\left|\frac{\pi}{2y}\right|$.  We apply this, taking  $z = s, s_1$. 
Using  $\sigma \ge \kappa(\sigma)\sigma_1$,  we find
$$
\left\{
\aligned &-\frac{\sigma+a}{(\sigma+a)^2+t^2} + \kappa(\sigma)
\frac{\sigma_1+a}{(\sigma_1+a)^2+t^2} \le 0, \\
&\frac{1}{4}\ln\left|1 + \left(\frac{\sigma+a}{t}\right)^2\right|
\le \frac{2.58}{t^2}, \\ &\frac{1}{2} (1+\kappa(\sigma)) \Theta(t)
\le \frac{1.26453}{|t|} .
\endaligned
\right.
$$ These facts, together with  $(1-\kappa(\sigma))/2 \ge 0.19498$,
yield the result.
\qed\enddemo%\fin

\subhead 3.3. Approximating $f(t,\chi)$\endsubhead
Here the character which is the argument of $f(t,\cdot)$
is assumed to  be primitive. 

\proclaim{Lemma 3.3.1} If $\chi_0$ is the trivial character and 
$$ E(\sigma)=1.08699 + 1.40018\sqrt{\sigma-1} + 1.86576(\sigma-1) +
2.32244(\sigma-1)^{3/2} , 
$$ then for  $1 < \sigma \le 1.062$,
$$  f(0,\chi_0)\le\frac1{\s-1}-E(\sigma) .
$$
\endproclaim

\demo{Proof} We follow McCurley's Lemma 3 ([4]). We have for any
complex number 
$z$
$$ -\frac{\zeta^{\prime}}{\zeta}(z)=\frac1{z-1}-\frac{\ln(\pi)}2 +
\frac{\Gamma^{\prime}}{2\Gamma}\big(\frac{z+2}2\big)
-\sum_{\rho\in\Cal Z(\chi_0)}\frac{1}{z-\rho},
\leqno(3.3.1)
$$ where $\sum_{\rho}$ is to be understood as
$\lim_{T\rightarrow\infty}\sum_{\rho,|\gamma|\le T}$. We use (3.3.1)
for $z=\s$. Since for any zero of $\zeta$, we have
$|\gamma|\ge10\ge\s-\beta$, we see that
$$
\frac{\s-\beta}{(\s-\beta)^2+\gamma^2}\ge \frac{1-\beta}{(1-\beta)^2
+\gamma^2}.
\leqno(3.3.2)
$$ It follows that
$$
\align f(0,\chi_0)
&=-\frac{\zeta^{\prime}}{\zeta}(\s)+\kappa(\s)\frac{\zeta^{\prime}}
{\zeta}(\s_1)
\\ &\le\frac1{\s-1}-\frac{\ln(\pi)}2+
{\Gamma^{\prime}\over2\Gamma}\big(\frac{\s+2}2\big)
-\Re\left(\sum_{\rho\in\Cal Z(\chi_0)}\frac{1}{1-\rho}\right) +
\kappa(\s)\frac{\zeta^{\prime}} {\zeta}(\s_1).
\endalign
$$ We recall McCurley's estimate (taken from the proof of Lemma 3 of
\cite{4})
$$ {\Gamma^{\prime}\over2\Gamma}\big(\frac{\s+2}2\big)
\le 1-{C_0\over2}-\ln (2)+({\pi^2\over8}-1)(\s-1),
$$ where $C_0 \cong 0.5772156649$ denotes Euler's constant. Further,
by (Landau \cite{1, \S76, pp. 316\--317}), 
$$
\sum_{\rho}
\frac{1}{1-\rho} = \sum_{\rho}\frac1{\rho}=-\frac12\ln(\pi)-\ln
(2)+1+\frac{C_0}{2}. 
\leqno(3.3.3)
$$ Thus, we can take
$$ E(\sigma) = C_0 - \left(\frac{\pi^2}{8} - 1\right)(\sigma-1) +
\kappa(\sigma)\left(-\frac{\zeta'}{\zeta}(\sigma_1)\right) .
\leqno(3.3.4)
$$

We now estimate  $-(\zeta'/\zeta)(\sigma_1)$, which is a decreasing,
nonnegative function of  $\sigma_1$.  As 
$\sigma\rightarrow 1^+$, $\sigma_1$  approaches  $(1+\sqrt{5})/2$ 
from below.  Writing  $\Delta = \sigma - 1$, we have
$$
\sigma_1 = \frac{1+\sqrt{5}\sqrt{1-(\frac{8}{5})\Delta^{1/2} +
(\frac{4}{5})\Delta -
\left(\frac{4}{5}\right)\Delta^{3/2} + ((\frac{4}{5})\Delta^2 +
(\frac{1}{5})\Delta^{5/2}) -
\dots}}{2} .
$$ For  $0 \le \Delta \le 0.062$  the series inside the radical is
alternating, with decreasing terms.  Thus
$$
\sigma_1 \le \frac{1 + \sqrt{5}\sqrt{1 - (\frac{8}{5})\Delta^{1/2} +
(\frac{4}{5})\Delta}}{2} .
$$ Upon expanding and using  $\Delta \le 0.062$,  we obtain
$$
\align
\sigma_1 &\le \frac{1 + \sqrt{5}\left(1 -
\frac{1}{2}\left(\frac{8}{5}\Delta^{1/2} - \frac{4}{5}\Delta\right)
- \frac{1}{8}\left(\frac{8}{5}\Delta^{1/2} -
\frac{4}{5}\Delta\right)^2\right)}{2} \\ &\le \frac{1 + \sqrt{5} (1 -
(\frac{4}{5})\Delta^{1/2}+(\frac{4}{25})\Delta)}{2} .
\endalign
$$ It follows that
$$ (1+\sqrt{5})/2 - \sigma_1 \ge
\left(\frac{2}{\sqrt{5}}\right)\Delta^{1/2} -
\left(\frac{2}{5\sqrt{5}}\right)\Delta .
$$ When  $-(\zeta'/\zeta)(\sigma_1)$  is developed as a Taylor series
about  $(1+\sqrt{5})/2$, its coefficients alternate in sign.  The
first four coefficients satisfy
$$ B_0 \ge 1.13991, \quad -B_1 \ge 2.48089, \quad B_2 \ge 4.20397, 
\quad -B_3 \ge 6.84664; 
$$ these values were obtained by expressing the coefficients in
terms of the derivatives of the zeta function, which were computed
using Euler-Maclaurin summation.  Thus,
$$
\aligned -(\zeta'/\zeta)(\sigma_1) &\ge 1.13991 + 2.48089
\left(\frac{2\sqrt{\Delta}}{\sqrt{5}} -
\frac{2}{5\sqrt{5}}\Delta\right) \\ &\qquad\qquad\ \  + 4.20397
\left(\frac{2\sqrt{\Delta}}{\sqrt{5}} -
\frac{2}{5\sqrt{5}}\Delta\right)^2 +
6.84664(0.84988\sqrt{\Delta})^3\\   &\ge 1.13991 +
2.21897\sqrt{\Delta} + 2.91938\Delta + 2.85764\Delta^{3/2} .
\endaligned
$$ (Here we have again used that  $0 \le \Delta \le 0.062$.) 
Inserting this  in (3.3.4) yields
$$ E(\sigma) \ge 1.08699 + 1.40018\Delta^{1/2} + 1.86576\Delta +
2.32244\Delta^{3/2}
$$ and for the purposes of the lemma we can replace  $E(\sigma)$  by
its bound.
\qed\enddemo%\fin

\proclaim{Lemma 3.3.2} If $\chi$ is a character of conductor 
$k$, then for  $|t| \ge 1000$ and $\varepsilon_1 = 0.0067$,
%\medskip
$$\align
f(t,\chi)\le&\frac{1-\k(\s)}{2}(\ln\left(\frac{k|t|}{2\pi}\right)
+\e_1)\\
&-\Sigma^{\prime}\
\Re\big\{\frac1{s-\rho}+\frac1{s-(1-\bar{\rho})}
-\frac{\k(\s)}{s_1-\rho}-\frac{\k(\s)}{s_1-(1-\bar{\rho})}\big\},
\endalign
$$
\medskip
\noindent where $\Sigma^{\prime}$ denotes a sum over the zeros of
$L(s,\chi)$ with
$\beta\ge1/2$ and where the zeros with $\beta=1/2$ have a weight
$\frac12$. 
\endproclaim
%\medskip

\demo{Proof} We consider separately the cases $k=1$ and $k>1$.

First suppose  $k = 1$. Using (3.3.1) with $z=s$, $s_1$, we get
$$
\aligned f(t,\chi_0) =&\  \Re\left\{{1\over s-1}-{\k(\s)\over
s_1-1}\right\} +
{1\over2}\Re\left\{{\Gamma^{\prime}\over\Gamma}\left({s+2\over2}
\right) -\k(\s){\Gamma^{\prime}\over\Gamma}\left({s_1+2\over2}\right)
\right\}\\  &\quad\quad -{1-\k(\s)\over2}\ln(\pi)
-\frac{1}{2}\sum_{\rho\in\Cal
Z(\chi_0)}\left(F(s,\rho)-\kappa(\sigma)F(s_1,\rho)\right).
\endaligned\leqno(3.3.5)
$$ By the functional equation of  $\zeta(s)$, the last sum can be
rewritten
$$
\Sigma^{\prime} \ \Re\left(\frac{1}{s-\rho} +
\frac{1}{s-(1-\bar{\rho})} -
\frac{\kappa(\sigma)}{s_1-\rho} -
\frac{\kappa(\sigma)}{s_1-(1-\bar{\rho})}\right),
$$ where the notation is as in the statement of the lemma.

Lemma 3.2.1, together with
$$
\Re\left\{{1\over s-1}-{\k(\s)\over s_1-1}\right\}\le{\s-1\over t^2}
\le\varepsilon_2
$$ and  $(1 - \kappa(\sigma))/2 \ge 0.19498$,  yields the result.

Now suppose  $k > 1$. The proof is similar to the previous one,
apart from the absence of the pole at  $s=1$. Following McCurley's
Lemma 5 (\cite{4}), we have
%\smallskip
$$
\align f(t,\chi)&=\frac{1-\k(\s)}{2}\ln\left(\frac{k}{\pi}\right)
+{1\over2}\Re\left\{{\Gamma^{\prime}\over\Gamma}\left({s+a\over2}\right)
-\k(\s){\Gamma^{\prime}\over\Gamma}\left({s_1+a\over2}\right)\right\}\\
&\ \ -\Sigma^{\prime}\left(F(s,\rho)-\k(\s)F(s_1,\rho)\right)
\endalign
$$
%\medskip
\noindent with $a=(1-\chi(-1))/2$, and Lemma 3.2.1 gives the
required estimate.\qed\enddemo
%\fin
%\medskip
%\vfill\eject

Using Lemmas 3.1.1 and 3.3.2, we get
%\medskip

\proclaim{Lemma 3.3.3} If $\chi$ is a character of conductor $k$, 
then for  $|t| \ge 1000$ and $\varepsilon_1 = 0.0067$
$$  f(t,\chi)\le\frac{1-\k(\s)}{2}(\ln\left(\frac{k|t|}{2\pi}\right)
+\e_1) 
$$ and if $t=\gamma$ is the ordinate of a zero $\rho=\beta+i\gamma$
with  $\beta > \frac{1}{2}$,  then 
$$  f(t,\chi)\le\frac{1-\k(\s)}{2}(\ln\left(\frac{k|t|}{2\pi}\right)
+\e_1) -\frac{1}{\s-\beta} + S(\sigma) .
$$
\endproclaim
%\medskip
\subhead 3.4. The default to primitivity\endsubhead
Let $\chi$ be a primitive character of conductor $k$  and let
$\chi_m$ be the primitive character associated with $\chi^m$, of
conductor $k_m$. For a prime  $p$  dividing  $k$  we put
%\medskip
$$
\left\{
\aligned &c_p(\s)=\frac1{p^{\s}-1}-\frac{\k(\s)}{p^{\s_1}-1},\\ 
&D_p(k,\s,\chi)=a_0c_p(\s)+\sum\Sb 1\le m\le4\\ p\nmid k_m\endSb
a_m\left(\frac{1-\k(\s)}2v_p(\frac{k}{k_m})-c_p(\s)\right),
\endaligned\right.
\leqno(3.4.1)
$$
%\medskip\noindent 
where $v_p(x)$ is the $p$-adic valuation of $x$  with 
$v_p(p) = 1$.  We seek a  lower bound for
%\medskip
$$ D(k,\s,\chi)=\sum_{p|k}\ln (p) D_p(k,\s,\chi). 
$$ This bound will be  $D^*(k,\sigma)$, defined below:  put
$$
\cases D^*_2(\s)=5.99\ c_2(\s),\\ D^*_3(\s)=5.467\ c_3(\s),\\
D^*_p(\s)=a_0\ c_p(\s),\text{\ \ for\ \ }p\ge5,
\endcases \leqno(3.4.2)
$$ and 
$$  D^*(k,\s)=\sum_{p|k}\ln (p) \ D^*_p(\s). \leqno(3.4.3)
$$

\proclaim{Lemma 3.4.1} The quantity $c_p(\s)$ is positive, and decreases as $p$
increases or as $\s$ increases.
\endproclaim

\demo{Proof} Easy.\qed
\enddemo
%\fin

\proclaim{Lemma 3.4.2} We have
$$
\left\{\aligned &\frac{1-\k(\s)}{c_2(\s)}-1\ge-0.298,\\
&\frac{1-\k(\s)}{2c_3(\s)}-1\ge-0.328,\\ &\frac{1-\k(\s)}2-c_5(\s)
\ge 0 .
\endaligned\right.
$$
\endproclaim

\demo{Proof} If $\s^-\le\s\le\s^+$, then
$$
\frac{1-\k(\s^+)}2-c_5(\s^-)\le\frac{1-\k(\s)}2-c_5(\s)\le
\frac{1-\k(\s^-)}2-c_5(\s^+). 
$$ Cutting the interval  $[1,(1+\sqrt{2})/2]$ into 10000 parts yields
the third inequality in the lemma.  The two other inequalities may be
obtained similarly.\qed
\enddemo
%\fin

\proclaim{Lemma 3.4.3} For each prime  $p \mid k$, we have
$D_p(k,\s,\chi)\ge D^*_p(\s)$. 
\endproclaim

\demo{Proof} If  $p \ge 5$, then  $c_5(\sigma) \ge c_p(\sigma)$, so
by Lemma 3.4.2 and the fact that  $k_1 = k$, we see that for each
prime dividing  $k$,
$$
\align D_p(\kappa,\sigma,\chi) &= a_0c_p(\sigma) + \sum^4\Sb m=2 \\
p\nmid k_m\endSb a_m \left(\frac{1-\kappa(\sigma)}{2}
v_p\left(\frac{k}{k_m}\right) - c_p(\sigma)\right) \\ &\ge
a_0c_p(\sigma) = D^*_p(\sigma) .
\endalign
$$ The proof is analogous for  $p = 2$  and  $p = 3$.  Note that
when 
$p = 2$, if  $p \nmid k_m$,  then  $v_p\left(\frac{k}{k_m}\right)
\ge 2$.  The constant in  $D^*_2(\sigma)$  is  $a_0 -
0.298(a_2+a_3+a_4)$, and the one in  $D^*_3(\sigma)$  is  $a_0 -
0.328(a_2+a_3+a_4)$.\qed
\enddemo
%\fin

Combining Lemmas 3.4.1 and 3.4.2 gives the desired result:

\proclaim{Corollary 3.4.4}  For  $1 \le \sigma \le (1 +
\sqrt{2})/2$, we have  $D(k,\sigma,\chi) \ge D^*(k,\sigma)$. 
Furthermore, on this interval  $D^*(k,\sigma)$  is a decreasing
function of 
$\sigma$, with  
$$ a_0 \sum_{p|k} \frac{\ln(p)}{p-1} \ \ge\  D^*(k,\sigma) \ \ge\  0
.
$$
\endproclaim

We finally need to go from an imprimitive character to a primitive
one.
%\bigskip

\proclaim{Lemma 3.4.5} The notations being as above, we have
$$ f(0,\chi_0)=f(0,\chi^0)+\sum_{p|k}\ln(p) c_p(\s),
$$ and
$$ f(t,\chi_m)\ge f(t,\chi^m)-\sum_{p|k\atop p\nmid k_m}\ln(p)
c_p(\s).
$$
\endproclaim

\demo{Proof} We have for any real number $t$,
$$ f(t,\chi_m)=f(t,\chi^m)+
\sum_{p|k\atop p\nmid k_m}\ln (p)\sum_{n\ge1}
\Re\left({\chi_m(p^n)\over p^{int}}\right)
\left({1\over p^{n\s}}-{\k(\s)\over p^{n\s_1}}\right).
$$ If $\chi_m=\chi_0$ and $t=0$, then $\chi_m(p^n)/p^{int}=1$,
otherwise its real part is  $\ge -1$.\qed
\enddemo
%\fin
\subhead 3.5. A positivity argument\endsubhead
The trigonometric polynomial
$$P(\theta)=\sum_{m=0}^4a_m\cos(m\theta)$$
is also given by
$$P(\theta)=8(0.9126+\cos\theta)^2(0.2766+\cos\theta)^2
$$  and thus is nonnegative. We have
$$
\sum_{m=0}^{4}a_m f(mt,\chi^m)=
\sum_{n\ge1}\Lambda(n)\left({1\over n^{\s}}-{\k(\s)\over
n^{\s_1}}\right)P(\arg( \chi(n)n^{-it}))\ge0.
\leqno(3.5.1)
$$ Recalling Lemma 3.4.5, we then get
$$
\sum_{m=0}^{4}a_m f(mt,\chi_m)\ge a_0\sum_{p|k}\ln (p) c_p(\s)
-\sum_{m=1}^4a_m\sum\Sb p\mid k\\ p\nmid k_m\endSb \ln (p)c_p(\s).
\leqno(3.5.2)
$$

\proclaim{Lemma 3.5.1}  We have
$$ a_0f(0,\chi_0)+\sum_{m=1}^{4}a_m
\left(f(mt,\chi_m)+\frac{1-\k(\s)}{2}\ln\left(\frac{k}{k_m}\right)
\right) \ge D^*(k,\s).
$$
\endproclaim

\demo{Proof} By the definition of  $D^*(k,\sigma)$, the
inequality in the lemma follows from (3.5.2) if we have
$$
\aligned a_0 \sum_{p\mid k} &\ln(p)c_p(\sigma) - \sum^4_{m=1} a_m
\sum\Sb p\mid k \\ p\nmid k_m\endSb \ln(p)c_p(\sigma) \\ &\ge
\sum_{p\mid k} \ln(p)D^*_p(\sigma) - \sum^4_{m=1}
a_m\left(\frac{1-\kappa(\sigma)}{2}\right)
\ln\left(\frac{k}{k_m}\right) .
\endaligned \leqno(3.5.3)
$$ However, by Lemma 3.4.3 we have, prime by prime,
$$ a_0c_p(\sigma) + \sum\Sb 1\le m \le 4 \\ p\mid k, p\nmid k_m\endSb
a_m\left(\frac{1-\kappa(\sigma)}{2}v_p
\left(\frac{k}{k_m}\right) - c_p(\sigma)\right) \ge D^*_p(\sigma)
$$ which certainly implies (3.5.3).
\qed
\enddemo
\subhead 3.6. The zero-free region\endsubhead
Inserting the bounds from Lemmas 3.3.1 and 3.3.3 in the expression
from Lemma 3.5.1, if  $t$  is taken to be the ordinate of a zero 
$\beta + i\gamma$  with  $\beta > 1/2$, and $|\gamma| \ge 1000$,
we find that
$$
\aligned
&a_0\left[\frac{1}{\sigma-1} - E(\sigma)\right] +
a_1\left[\left(\frac{1-\kappa(\sigma)}{2}\right)
(\ln\left(\frac{k|\gamma|}{2\pi}\right) + \varepsilon_1) -
\frac{1}{\sigma-\beta} + S(\sigma)\right] \\
&\qquad\quad+ \sum^4_{m=2}
a_m\left[\left(\frac{1-\kappa(\sigma)}{2}\right)
(\ln\left(\frac{k_m|m\gamma|}{2\pi}\right) + \varepsilon_1) +
\left(\frac{1-\kappa(\sigma)}{2}\right)
\ln\left(\frac{k}{k_m}\right)\right]  \\
&\qquad \ge
D^*(k,\sigma) .\endaligned\tag3.6.1 $$
 Recalling that  $\kappa(\sigma) =
\frac{1+\frac{4}{5}\sqrt{\s-1}}{\sqrt{5}}$, we can rewrite this as
$$ u - \alpha u\sqrt{\sigma-1} - \frac{w}{\sigma-\beta} +
\frac{1}{\sigma-1} - G \ge 0,\leqno(3.6.2)
$$ where
$$
\left\{
\aligned u &= \frac{1}{a_0} \frac{5-\sqrt{5}}{10} \sum^4_{m=1}
a_m(\ln
\left(\frac{km|\gamma|}{2\pi}\right) + \varepsilon_1),\\
\alpha &= (\sqrt{5} + 1)/5 \cong 0.647213595,\\ w &= a_1/a_0 \cong
1.705118356,\\ G &= E(\sigma)  - wS(\sigma) +
\frac{D^*(k,\sigma)}{a_0} .
\endaligned  \right.  \leqno(3.6.3)
$$ Put  $x = \sqrt{w} - 1 \cong 0.305801806$.  A near-optimal value
for  $\sigma$   in (3.6.2) is  $\sigma = 1 + x/u$.  Using this, and
the fact that 
$|\gamma| \ge 1000$, we find that  $1 \le \sigma \le 1.062$,
validating the bounds used in our estimates for  $E(\sigma)$  and 
$S(\sigma)$.  Inserting $\sigma = 1 + x/u$ in (3.6.2) and replacing 
$G$  by any lower bound  $\overline{G} = \overline{G}(\sigma)$, we
find
$$ u - \alpha u\sqrt{x/u} - \frac{w}{1-\beta+x/u} + \frac{u}{x} -
\overline{G} \ge 0 . 
$$ Solving for  $1 - \beta$  and using  $(1 + 1/x) = \sqrt{w}/x$ 
gives
$$ 1 - \beta \ge \frac{w}{\frac{u\sqrt{w}}{x} - \alpha\sqrt{xu} -
\overline{G}} - \frac{x}{u} .
$$ Using  $w - \sqrt{w} = x\sqrt{w}$  and cross-multiplying by 
$u/x^2$  yields
$$
\frac{u}{x^2} (1-\beta) \ge
\frac{1+\frac{\alpha\sqrt{x}}{\sqrt{wu}} + \frac{1}{u\sqrt{w}}
\overline{G}}{1 - \frac{\alpha x\sqrt{x}}{\sqrt{wu}} -
\frac{x}{u\sqrt{w}} \overline{G}} .
$$ Finally, using  $x = \sqrt{w} - 1$, we obtain
$$ 1 - \beta \ge \frac{1}{\frac{u}{x^2} - \frac{(\alpha/x)\sqrt{u/x}
+ (1/x^2)\overline{G}}{1 + (\alpha/\sqrt{w})\sqrt{x/u} +
(1/(x\sqrt{w}))(x/u)\overline{G}}} . \leqno(3.6.4)
$$ Note that
$$
\frac{u}{x^2} = R \ln(k|\gamma|/C_1),\leqno(3.6.5)
$$ where
$$
\left\{
\aligned R  &= \frac{1}{x^2} \frac{A}{a_0} \frac{5-\sqrt{5}}{10}
\cong 9.645908801,\\ C_1 &= 2\pi \exp\left(-\varepsilon_1 - (1/A)
\sum^4_{m=2} a_m\ln(m)\right) \cong 4.171838431 .
\endaligned \right. \leqno(3.6.6) 
$$

An especially useful choice for the lower bound  $\overline{G}$  in
(3.6.4) occurs when  $D^*(k,\sigma)$  is replaced by a constant
lower bound  $\overline{D}$, giving
$$
\overline{G} = \overline{G}(\sigma) = E(\sigma) - wS(\sigma) +
\frac{\overline{D}}{a_0} . \leqno(3.6.7)
$$ In particular, we can always take  $\overline{D} = 0$, or, if for
some  $H \ge 1000$ we are interested only in zeros  $\beta +
i\gamma$  with  $|\gamma| \ge H$, then by Corollary 3.4.4 we can
take  $\overline{D}$  to be the value of  $D^*(k,\sigma)$ 
corresponding to  $H$.  As will be seen,  $\overline{G} \ge 0$   
on  $[1, 1.062]$.

Writing  $\Delta = x/u$, and taking  $\overline{G}$  as in (3.6.7),
put
$$
\align G_1(\Delta) &= \frac{\alpha}{x} \sqrt{\frac{1}{\Delta}} +
\frac{1}{x^2} \overline{G}(1 + \Delta) , \\ G_2(\Delta) &= 1 +
\frac{\alpha}{\sqrt{w}} \sqrt{\Delta} +
\frac{1}{x\sqrt{w}} \cdot \Delta \cdot \overline{G}(1 + \Delta) .
\endalign
$$ Then  $G_1(\Delta)/G_2(\Delta)$  is the fraction in the
denominator of (3.6.4).
\proclaim{Lemma 3.6.1}  For each  $\overline{D} \ge 0$, the
functions  $G_1(\Delta)/G_2(\Delta)$ and 
$\overline{G}(1+\Delta)$  as given by $(3.6.7)$ are nonnegative and
decreasing on   $[0,0.062]$.
\endproclaim

\demo{Proof}  First suppose  $\overline{D} = 0$.  Note that
$$ E(\sigma) = 1.08699 + 1.40018\sqrt{\Delta} + 1.86576\Delta +
2.32244\Delta^{3/2}
$$ and that (after some manipulations)
$$ S(\sigma) = -\frac{1}{1+\Delta} + \frac{\left(1 +
\frac{4}{5}\sqrt{\Delta}\right)\sqrt{1 -
\frac{8}{5}\sqrt{\Delta^2+\Delta} +
\frac{4}{5}\left(\Delta^2+\Delta\right)}}{\left(1 -
\sqrt{\Delta^2+\Delta}\right)^2}.
$$ It follows that the derivatives of 
$\overline{G}(1+\Delta)$, $G_1(\Delta)$  and 
$G_2(\Delta)$  are sums and quotients of various terms monotonic on 
$[0,0.062]$.  Thus, on any given subinterval, rigorous upper bounds
for  $\overline{G}^{\prime}(1+\Delta)$
 and  $G^{\prime}_1(\Delta)$, and a rigorous lower bound for 
$G^{\prime}_2(\Delta)$, can be obtained.  Dividing  $[0,0.062]$ 
into subintervals of length $0.001$, it was found that 
$\overline{G}^{\prime}(1+\Delta) < 0$ and  
$G^{\prime}_1(\Delta) < 0$  throughout the interval, and 
$G^{\prime}_2(\Delta) > 0$  on  $[0,0.057]$.

Graphically it is clear that  $G_2(\Delta)$  has a maximum at about 
$0.0590$  and then slowly decreases.  To deal with the interval 
$[0.057, 0.062]$, multiply through by 
$\sqrt{\Delta}$.  The derivatives of 
$\sqrt{\Delta}
\cdot G_1(\Delta)$  and  $\sqrt{\Delta} \cdot G_2(\Delta)$  are sums
and quotients of mono-\linebreak
tonic terms.  Dividing  $[0,0.062]$  into
subintervals of length  $0.001$, it was found that
\linebreak 
  $(\sqrt{\Delta}
\cdot G_1(\Delta))' < 0$  on  $[0.047,0.062]$  and  $(\sqrt{\Delta}
\cdot G_2(\Delta))' > 0$  throughout  $[0,0.062]$.

Combining these facts yields the monotonicity of 
$G_1(\Delta)/G_2(\Delta)$.  Its non-\linebreak
negativity and that of 
$\overline{G}(1+\Delta)$ follow by evaluation at  $\Delta = 0.062$.

Now let  $\overline{D} \ge 0$  be arbitrary; let  $g_1(\Delta)$,
$g_2(\Delta)$  denote  $G_1(\Delta)$  and  $G_2(\Delta)$  when 
$\overline{D} = 0$.  By the discussion above, it follows that  $g_2
\cdot g'_1 - g_1\cdot g'_2 < 0$ and  $g'_1 < 0$  on  $[0,0.062]$. 
Applying the quotient rule to  $G_1(\Delta)/G_2(\Delta)$, it is
easily seen that for  $G_1(\Delta)/G_2(\Delta)$  to be decreasing, it
suffices to have
$$ g'_2 + \frac{x}{\sqrt{w}} g_1 > 0 .
$$ This was checked by the same means as before.\qed
\enddemo
%\fin

\proclaim{Theorem 3.6.2}  If  $\chi$  is a character of conductor 
$k$, and  $R = 9.645908801$, then any zero  $\beta + i\gamma$  of 
$L(s,\chi)$  with 
$|\gamma| \ge 1000$  satisfies
$$ 1 - \beta \ge \frac{1}{R \ln\left(\frac{k|\gamma|}{4.1718}\right)
- 3.2356 \sqrt{\ln\left(\frac{k|\gamma|}{4.1718}\right)}} \ .
$$
\endproclaim

\demo{Proof}  When  $\beta > \frac{1}{2}$, the result follows from
(3.6.4), taking  $\overline{G} = 0$  (which is permissible by Lemma
3.6.1); here  $3.2356 \le
\alpha\sqrt{R/x}/(1+(\alpha/\sqrt{w})\sqrt{0.062})$.  It is easy to
check that the bound is decreasing in  $k$  and  $|\gamma|$; when 
$k = 1$  and  $|\gamma| = 1000$, it is less than  $0.02194$.  Thus
the inequality holds when  $\beta \le \frac{1}{2}$  as well.
\qed
\enddemo
%\fin

%\vfill\eject 
\par
Taking  $\overline{D} = D^*(k,\sigma)$  in (3.6.7), we
obtain a result better suited to the needs of this paper: recall that
$\alpha = (\sqrt{5} + 1)/5$, $w = a_1/a_0$ and $x = \sqrt{w} - 1$.

\proclaim{Theorem 3.6.3}  If  $\chi$  is a character of conductor 
$k$, and  $H \ge 1000$, then any zero  $\beta + i\gamma$  of 
$L(s,\chi)$  with  $|\gamma| \ge H$  satisfies
$$ 1 - \beta \ge \frac{1}{R\ln(k|\gamma|/C_1(\chi))},
$$ where
$$ C_1(\chi) = C_1(\chi,H) = C_1 \cdot \exp\left(\frac{1}{R} \cdot
\frac{\frac{\alpha}{x}\sqrt{\frac{1}{\Delta}} + \frac{1}{x^2}
\overline{G}(1+\Delta)}{1 + \frac{\alpha}{\sqrt{w}} \sqrt{\Delta} +
\frac{1}{x\sqrt{w}} \cdot \Delta \cdot
\overline{G}(1+\Delta)}\right)
$$ with  $\Delta = 1/(xR \cdot \ln(kH/C_1))$, $\overline{G}$  given
by $(3.6.7)$, and 
$\overline{D} = D^*(k,1+\Delta)$  by $(3.4.3)$.
\endproclaim

\demo{Proof}  When  $\beta > \frac{1}{2}$, the result follows from
(3.6.4) and Lemma 3.6.1.  To see that it holds for  $\beta \le
\frac{1}{2}$, one can use the upper bound  $11.624 \ge
\frac{1}{x^2}\left(E(\sigma) - wS(\sigma)\right)$, the upper and
lower bounds for  $D^*(k,\sigma)$  from Corollary 3.4.4, and the
fact that  $|\gamma| \ge 1000$  to show that
$$ R\ln(k|\gamma|/C_1(\chi)) \ge 15.\quad\qed$$
\enddemo
%\fin

\bn {\bf Some examples :}
$$
\alignat4  &\text{For } k=1, \quad &|\gamma|&\ge545000000,  \quad
&1-\beta\ge 1/(R\cdot\ln(k|\gamma|/38.31)).&&&\\   &\text{For } k=3,
\quad &|\gamma|&\ge10000,  \quad &1-\beta\ge
1/(R\cdot\ln(k|\gamma|/20.92)).&&& \\   &\text{For } k=12, \quad
&|\gamma|&\ge10000, \quad &1-\beta\ge
1/(R\cdot\ln(k|\gamma|/29.68)).&&& \\   &\text{For } k=17, \quad
&|\gamma|&\ge2500,  \quad &1-\beta\ge
1/(R\cdot\ln(k|\gamma|/20.90)).&&&\\   &\text{For } k=420, \quad
&|\gamma|&\ge2500, \quad  &1-\beta\ge
1/(R\cdot\ln(k|\gamma|/56.59)).&&&
\endalignat 
$$
\head 4. $L$-functions, $\psi(X;k,l)$ and 
$\theta(X;k,l)$\endhead

\smallskip Before drawing consequences from Theorem 3.6.3, we need
some  lemmas. 
%\smallskip 

{\sl In the following, we assume that the
$L$-function under examination has no zero satisfying}
$$
\left\{\aligned &|\gamma|\ge H\ge1000,\\ &\beta > 1 - \frac{1}{R
\ln(k|\gamma|/C_1(\chi))}
\endaligned\right.
\leqno(4.1)
$$ {\sl for some positive constants $R$ and $C_1(\chi)$  such that}
$$  R\cdot\ln\left(\frac{kH}{C_1(\chi)}\right)\ge2.
\leqno(4.2)
$$
When  $R = 9.645908801$, then taking  $k=1$  and  $\gamma
= 1000$  in Theorem 3.6.2 shows we can always use  $C_1(\chi) =
9.14$; Theorem 3.6.3 gives sharper results.

\subhead 4.1. Preparatory lemmas\endsubhead
We restate McCurley's results before using them. 

First\pagebreak\ of all, let us note the following consequence of McCurley  [5]
:

\proclaim {Lemma 4.1.1 \rom{(McCurley)}} If $\chi$ is a 
Dirichlet character of conductor $k$, if  $T \ge 1$ is a real
number, and if $N(T,\chi)$ denotes the number of zeros
$\beta+i\gamma$ of
$L(s,\chi)$ in the rectangle $0<\beta<1$, $|\gamma| \le T$, then
$$ |N(T,\chi)-\frac T{\pi}\ln\left(\frac {kT}{2 \pi e}\right)|\le C_2
\ln
\ (kT)\  +C_3 
$$  with $C_2=0.9185$ and $C_3=5.512$.
\endproclaim

Our values of $C_2$ and $C_3$ correspond to the choice 
$\eta=\frac12$ in McCurley's notation and are undoubtedly not
optimal.  (Note that our  $C_2$  is his  $C_1$, and our  $C_3$  is
his  $C_2$.)  If 
$k > 1$, Lemma 4.1.1 follows immediately from (\cite{5, Theorem
2.1}).  When 
$k = 1$, it follows from the theorem of Rosser cited in (\cite{5,
formula 2.17}). 
%\bigskip

\proclaim {Lemma 4.1.2} If  $k$  is the conductor of  $\chi$, and  
$L(s,\chi)$ satisfies $GRH(H)$ for some  $H \ge 1$, then
$$
\sum\Sb |\gamma |\le H\\ \rho\in \Cal Z(\chi)\endSb \frac 1{|\rho|} 
\le \tilde E(H),
$$ with
$$
\tilde E(H)=
\frac 1{2\pi}
\ln ^2 (H) +\frac {\ln \left(\frac k{2\pi}\right)}{\pi}\ln (H) + C_2
+ 2\left( 
\frac {\ln \left(\frac k{2\pi e}\right)}{\pi} + C_2 \ln (k) + C_3
\right).
\leqno(4.1.1)
$$
\endproclaim

\demo{Proof} We consider what happens for $|\gamma|\le 1$ and for
$|\gamma| >1$.
%\smallskip 

For $|\gamma|\le 1$, using GRH(1), we have 
$$
\sum\Sb |\gamma |\le1\\ \rho\in\Cal Z(\chi)\endSb \frac 1{|\rho|} 
\le 2 N(1,\chi). \leqno(4.1.2)
$$

For $|\gamma|>1$, one gets
$$
\align
\sum\Sb 1<|\gamma |\le H\\ \rho\in\Cal Z(\chi)\endSb \frac 1{|\rho|} 
&\le
\sum\Sb 1<|\gamma |\le H\\ \rho\in\Cal Z(\chi)\endSb \bigg ( 
\int_{|\gamma|}^{H} \frac {dt}{t^2} + \frac 1H \bigg ) \\  &=
\int_{1}^{H} \frac {N(t,\chi)-N(1,\chi)}{t^2} dt + \frac
{N(H,\chi)-N(1,\chi)}H,
\endalign
$$ thus
$$
\sum\Sb 1<|\gamma |\le H\\ \rho\in\Cal Z(\chi)\endSb  \frac
1{|\rho|} \le \int_{1}^{H} \frac {N(t,\chi)}{t^2} dt +\frac
{N(H,\chi)}H -\frac {N(1,\chi)}1. \leqno(4.1.3)
$$
%\indent 
Now we easily finish the proof on using Lemma 4.1.1.\qed
\enddemo
%\fin
%\smallskip

\proclaim{Lemma 4.1.3} Let $\chi$ be a character with conductor 
$k$, let $m\ge 1$ be an integer and $x$, $x_0$ be two real numbers
such that $x\ge x_0\ge1$ and $2R\ln^2(kH/C_1(\chi))\ge\operatorname{Log}(x_0)$.
Then
$$
\sum\Sb | \gamma | \ge H \\ \rho \in \Cal Z(\chi) \endSb 
\frac {x^{\beta}}{|\gamma|^{m+1}} +\sum\Sb | \gamma|\ge H \\ \rho
\in \Cal Z(\bar\chi) \endSb 
\frac {x^{\beta}}{|\gamma|^{m+1}}
\le x(\tilde A+\tilde B)+\sqrt x (\tilde C+\tilde D),
$$ where, with  $C_2 = 0.9185$ and $C_3 = 5.512$, we have
$$
\left\{\aligned
\tilde A &=\dsize \int_{H}^{\infty}
\bigg \{\frac {\ln \left(\frac{kt}{2\pi}\right)}{\pi}+\frac{C_2}t
\bigg
\} 
\frac {\exp \big \{- \frac {\ln (x_0)}{R\ln(kt/C_1(\chi))} \big \}} 
{t^{m+1}}dt, \\
\tilde B &=\frac1{H^{m+1}}\exp\big\{- \frac{\ln
(x_0)}{R\ln(kH/C_1(\chi))}\big\} 2(C_2\ln (kH)+C_3),
\\
\tilde C &=\frac1{\pi mH^m}(\ln \left(\frac {kH}{2\pi}\right)+\frac
1m ),
\\
\tilde D &=\frac 1{H^{m+1}} (2C_2 \ln (kH) +2C_3+\frac{C_2}{m+1}).
\endaligned\right.
\leqno(4.1.4)
$$
\endproclaim

%\smallskip
\demo{Proof} The functional equation of the $L$-functions enables us
to  write 
$$
\sum\Sb | \gamma | \ge H \\ \rho \in \Cal Z(\chi) \endSb 
\frac {x^{\beta}}{|\gamma|^{m+1}} +\sum\Sb | \gamma|\ge H \\ \rho
\in \Cal Z(\bar\chi) \endSb 
\frac {x^{\beta}}{|\gamma|^{m+1}} =\sum\Sb | \gamma | \ge H \\ \rho
\in \Cal Z(\chi) \endSb \left(\frac
{x^{1-(1-\beta)}}{|\gamma|^{m+1}}  +\frac
{x^{1-\beta}}{|\gamma|^{m+1}}\right).
\leqno(4.1.5)
$$  One of $\beta$ and $1-\beta$ is at most $\frac 12$, the other is
less than 
$$  1-\frac 1{R\ln(k|\gamma|/C_1(\chi))}.
$$ We can split our upper bound in two pieces. For the first,
$$
\sum\Sb|\gamma|\ge H\\ \rho\in\Cal
Z(\chi)\endSb\frac1{|\gamma|^{m+1}},\leqno\text{\rm (i)}
$$
an integration by parts assures that this sum is bounded by
$\tilde C+\tilde D$. For the second,
$$ 
\tilde{S}(x) = \sum\Sb | \gamma|\ge H \\ \rho \in \Cal Z(\chi)
\endSb \frac x{|\gamma|^{m+1}}
\exp \big \{-\frac{\ln (x)}{R\ln(k|\gamma|/C_1(\chi))} \big \},
\leqno\text{\rm (ii)}
$$ we have $\tilde{S}(x)/x \le \tilde{S}(x_0)/x_0$, hence it remains
to evaluate $\tilde{S}(x_0)/x_0$. Let us define
$$
\varphi_m(t)=\frac 1{t^{m+1}}
\exp \big\{-\frac{\ln (x_0)}{R\ln(kt/C_1(\chi))}\big \} $$  and
integrate it by parts:
$$
\frac{\tilde{S}(x_0)}{x_0} =
\sum\Sb | \gamma|\ge H \\ \rho \in \Cal Z(\chi) \endSb
\dsize\int_{|\gamma|}^{\infty}-\varphi_m^\prime(t)dt
=\dsize\int_{H}^{\infty}
(N(t,\chi)-N(H,\chi))\left(-\varphi_m^\prime(t)\right)dt.
\leqno(4.1.6)
$$  But
$$
\frac{\varphi_m^\prime(t)}{\varphi_m(t)}=
\frac{m+1}{t\ln^2(kt/C_1(\chi))}
\left[ -\ln^2(kt/C_1(\chi)) +\frac {\ln (x_0)}{(m+1)R} \right] 
$$  and our hypothesis gives us $-\varphi_m^\prime(t)\ge0$ for $t\ge
H$.

Lemma 4.1.1 gives us an upper bound for $N(t,\chi)$ and we integrate
back by\linebreak
parts.\qed
\enddemo
%\fin

\proclaim {Lemma 4.1.4} If $k$ is the conductor of $\chi$, and
$L(s,\chi)$ satisfies $GRH(1)$, we have
$$
\sum\Sb \rho\in \Cal Z(\chi)\endSb \frac 2{|\rho(2-\rho)|} \le
3.94\ln (k)+12.7.
$$
\endproclaim

\demo{Proof} The contribution to the sum of the zeros with
$|\gamma|\le1$ is at most $\frac83 N(1,\chi)$. The contribution of
the remaining ones is not more than
$$ 4\int_{1}^{\infty}\frac{N(t,\chi)-N(1,\chi)}{t^3}dt
=-2N(1,\chi)+4\int_{1}^{\infty}\frac{N(t,\chi)}{t^3}dt, $$ and on
using Lemma 4.1.1, we obtain the result.\qed
\enddemo
\subhead 4.2. Computing $\tilde A$\endsubhead
The constant  $\tilde{A}$  is the dominant term in Lemma $4.1.3$. 
However, the integral defining 
$\tilde{A}$  converges slowly, and care is needed in evaluating it. 
This was forcefully brought home to us when we initially sought to
compute  $\tilde{A}$  using the pre-programmed numerical integration
of PARI.  Lionel Reboul of the University of Lyon kindly verified
the values by independent computations with MAPLE.  To our surprise,
the values were the same for 
$\ln(x_0) \ge 40$, but they were not close for small 
$x_0$  (for instance  $0.032$  instead of  $0.034$).

We will first give a simple lemma (Lemma 4.2.1) which gives a rather
good upper bound for  $\tilde{A}$.  Then we will describe a way for
accurately computing  $\tilde{A}$  advocated by Rosser, Schoenfeld,
and McCurley, which involves transforming the given integral to a
sum of rapidly convergent incomplete Bessel integrals.  
%\smallskip

For  $m \ge 2$  let us define $h_m(t)$ by
$$ h_m(t)=
\frac{1}{\pi(m-1)t^{m-1}}(\ln\big(\frac{kt}{2\pi}\big)+\frac1{m-1})+
\frac{C_2}{mt^m}.
\leqno(4.2.1)
$$
%\indent

\proclaim{Lemma 4.2.1} We have, under the hypothesis of Lemma $4.1.3$,
but with in addition $R\ln^2(kH/C_1(\chi))\ge\ln (x_0)$ and
$m\ge2$, 
$$
\left\{\aligned &\tilde A\le \frac{h_m(H)}{H}\exp(-\frac{\ln
(x_0)}{R\ln(kH/C_1(\chi))}),\\  &\tilde A\ge
h_{m+1}(H)\exp(-\frac{\ln (x_0)}{R\ln(kH/C_1(\chi))}).
\endaligned
\right.
$$
\endproclaim

\demo{Proof} Let us define
$$  g_m(t)=\frac{\ln(\frac{kt}{2\pi})}{\pi t^m}+\frac{C_2}{t^{m+1}}.
\leqno(4.2.2)
$$

Then, we readily see that $h^{\prime}_m(t)=-g_m(t)$. We also have 
$$
\aligned \tilde A=&\int_H^{\infty}g_m(t)\exp(-\frac{\ln
(x_0)}{R\ln(kt/C_1(\chi))})\frac{dt}t\\
 =&
\frac{h_m(H)}{H}\exp(-\frac{\ln (x_0)}{R\ln(kH/C_1(\chi))})\\
&-\int_H^{\infty}
\left(\ln^2(kt/C_1(\chi))-\frac{\ln (x_0)}{R}\right) \exp(-\frac{\ln
(x_0)}{R\ln(kt/C_1(\chi))})\frac{h_m(t)dt}{t^2\ln^2(kt/C_1(\chi))}\\
\le&\frac{h_m(H)}H\exp(-\frac{\ln (x_0)}{R\ln(kH/C_1(\chi))}); 
\endaligned
$$  the last inequality holds because our hypothesis ensures that 
$$
\left(\ln^2(kH/C_1(\chi))-\frac{\ln (x_0)}{R}\right)\ge0. 
$$
%\smallskip 

For the other inequality, we write
$$
\tilde A=\int_H^{\infty}g_{m+1}(t)\exp(-\frac{\ln
(x_0)}{R\ln(kt/C_1(\chi))})dt \leqno(4.2.3)
$$  and an integration by parts yields the result. 
\qed
\enddemo

We now turn to the accurate computation of  $\tilde{A}$.  Define
$$
\left\{\aligned &K_n(z,w)=\frac12\int_w^{\infty}u^{n-1}\exp[-\frac
z2(u+\frac1u)]du,\\  &U_{m}=\sqrt{\frac{Rm}{\ln
(x_0)}}\ln\big(\frac{kH}{C_1(\chi)}\big),\\  &z_m=2\sqrt{\frac{m\ln
(x_0)}{R}}.
\endaligned\right.
\leqno(4.2.4)
$$  Then simple algebraic manipulations yield

\proclaim{Lemma 4.2.2} There holds
$$
\align
\tilde A=&\frac{2}{\pi}\frac{\ln
(x_0)}{Rm}\left(\frac{k}{C_1(\chi)}\right)^m K_2(z_m,U_m)\\ 
&+\frac{2}{\pi}\ln\left(\frac{C_1(\chi)}{2\pi}\right)
\sqrt{\frac{\ln (x_0)}{Rm}}\left(\frac{k}{C_1(\chi)}\right)^m
K_1(z_m,U_m)\\  &+2C_2\sqrt{\frac{\ln (x_0)}{R(m+1)}}
\left(\frac{k}{C_1(\chi)}\right)^{m+1} K_1(z_{m+1},U_{m+1}). 
\endalign
$$
\endproclaim

The tails in the integrals  $K_1$  and  $K_2$  can be estimated as
follows.  Write
$$  \operatorname{erfc}(x)=\frac2{\sqrt{\pi}}\int_{x}^{\infty}e^{-t^2}dt.
\leqno(4.2.5)
$$ 

\proclaim{Lemma 4.2.3 \rom{(Rosser and Schoenfeld)}} Put
$y=\frac{\sqrt{w}-1/\sqrt{w}}{\sqrt{2}}$ for $w\ge1$. Then 
$$  K_1(z,w)<\frac{e^{-z}}{2z}
\big\{(1+\frac{3\sqrt{2}}8y)e^{-zy^2}
+(\frac{3}{8\sqrt{z}}+\sqrt{z})\sqrt{\frac{\pi}{2}}
\operatorname{erfc}(y\sqrt{z})\big\},
$$  and
$$
\align  K_2(z,w)<\frac{e^{-z}}{2z}
\big\{&[\frac{35\sqrt{2}}{64}y^3+2y^2+(\frac{105}{128z}+
\frac{15}{8})\sqrt{2}y+2+\frac2z]  e^{-zy^2}\\
&\qquad\qquad\qquad+(\frac{105}{128z}+\frac{15}{8}+z)\sqrt{\frac{\pi}{2z}}
\operatorname{erfc}(y\sqrt{z})\big\}.
\endalign
$$
\endproclaim

\proclaim{Lemma 4.2.4}  Assume $z > 0$.
\smallskip The integrand in $K_1(z,w)$ is decreasing in
$u$ for $u\ge1$. 
%\smallskip 

The integrand in $K_2(z,w)$ is increasing for $1\le
u\le\frac{1+\sqrt{1+z^2}}{z}$ and decreasing in $u$ afterwards.
\endproclaim

Lemma 4.2.3 is accurate if  $w$  is large.  Thus,  $K_1$  and 
$K_2$  can be computed by integrating numerically from  $w$  to some
finite bound and then using Lemma 4.2.3.  In computing Table 1, this
method was used in conjunction with Simpson's rule to get a sharp
rigorous upper bound for  $\tilde{A}$  (within 1\%).  It should also
be noted that R. Terras \cite{13} has given a rapid continued
fraction algorithm for computing integrals like  $K_1$ and $K_2$. 
Although this algorithm does not produce a rigorous error bound, we
have used it to give an independent check on the numerical
integrations.
\subhead 4.3. Estimating $\psi$ and $\theta$ through
$L$-functions\endsubhead
We are now in a position to prove Theorem 1. To do so we want to use
Theorem 3.6 of McCurley [5]. But in that theorem, the notation $\Cal
Z(\chi)$ denotes the set of zeros of
$L(s,\chi)$ with $0\le\beta<1$, so it is necessary to pay careful
attention to the distinction between conductor and modulus. In
\cite{5}, the zeros with $\beta=0$ are removed with some cumbersome
arguments (cf. \cite{5, (3.25), (3.26), (3.27) and (3.38)}). The
troubles caused by the use of imprimitive characters can also be
seen in (\cite{5, (3.15)}).
%\smallskip

However, it is possible to work from the beginning with primitive
characters only. If $\chi$ is a character modulo $k$, let $\chi_1$
be its associated primitive character. Given $l$ prime to
$k$, consider further
$$ w_k(l,n)=\frac1{\varphi(k)}\sum_{\chi\mod
k}\chi_1(n)\bar{\chi}(l).
$$ If $K$ is the largest divisor of $k$ coprime to $n$, we have
$$ w_k(l,n)=\cases
\frac{\varphi(K)}{\varphi(k)}&\text{if}\ n\equiv l\pmod{K},\\
0&\text{otherwise,}\endcases$$
which can be proved as follows:$$
\align
\varphi(k)w_k(l,n)=&\sum_{d|k}\ \ \sum_{\chi\!\mod\!^*
d}\chi(n)\bar{\chi}(l) \\  =&\sum_{d|K}\ \ \sum_{\chi\!\mod\!^*
d}\chi(n)\bar{\chi}(l)=\sum_{\chi\!\mod\! K}\chi(n)\bar{\chi}(l),
\endalign
$$ where $\sum_{\chi\!\mod\!^*d}$ is a summation over the primitive
characters modulo $d$. We now consider
$$
\psi^*(x;k,l)=\sum_{n\le x}w_k(l,n)\Lambda(n).
$$ We have \ \ $\psi(x;k,l)\le\psi^*(x;k,l)\le\psi(x;k,l)+f(k)\ln
(x)$
\ for
$x\ge1$\ \  with
$$ f(k)=\sum_{p|k}{1\over p-1}.\leqno(4.3.1)
$$ We remark that $f(k)\le3.5$ if $k$ has less than 12000 prime
factors, which is true if $k\le\exp(127000)$, and in any case
$f(k) = \Cal O(\ln(\ln(\ln(k))))$. Thus, we can work with 
$\psi^*$ instead of $\psi$ with a small loss. Now McCurley's
analysis (\cite{5, \S 3}) goes through with characters replaced by
primitive characters since $\psi^*$ is also an increasing function.
More precisely, McCurley's $m(\chi)$  now becomes the order of the
zero of $L(s,\chi_1)$ at $s = 0$,  i.e. $m(\chi) = 0$ if
$\chi(-1)=-1$ or $\chi$ is the principal character, and $m(\chi) =
1$ otherwise.  In McCurley's notation, this yields
$|d_2|\le1/2$. Following the proof of Lemma 3.5 of \cite{5}, we find
further that if $L(s,\chi)$ satisfies GRH(1) and
$\chi$ is not principal, then
$$  |b(\chi)|\le1.57+\sum_{\rho\in\Cal
Z(\chi)}\frac{2}{|\rho(2-\rho)|},
$$  which, with Lemma 4.1.4 together with $|b(\chi_0)|=\ln (2\pi)$
yields\ ``{\sl If every L-function of modulus k satisfies $\roman{GRH}(1)$,
then $|d_1+d_2|\le3.94\ln(k) +12.7$}" instead of Lemma 3.5 of
\cite{5}. Finally, the right-hand side of (\cite{5, (3.21)}) becomes
$$  (\delta x)^m\left[\frac{\ln(2)}2+\frac{\ln (2x)}{2}+3.94\ln
(k)+12.7\right].
$$  An additional error term comes from replacing $\psi^*$ by $\psi$.
The upper bound we get this way is an increasing function of $x$, so
we can introduce the maximum  $\max\limits_{1\le y\le x}$  which is
useful in practice.  Thus we get

\proclaim{Theorem 4.3.1 \rom{(McCurley)}}
Let $x \ge 1$ be a real number, $k \ge 1$ an integer,
$m$ a positive integer, $\delta$ a real number with 
$0<\delta<\frac{x-2}{mx}$, and $H$ a positive real number. Put
$$A(m,\delta)=\frac 1{\delta^m}\sum_{j=0}^{m}\binom mj 
(1+j\delta)^{m+1}.$$
Suppose that for each  $\chi$  with modulus 
$k$,  $L(s,\chi)$  satisfies $GRH(1)$. Then
$$
\align
\frac {\varphi(k)}x \max_{1\le y \le x}|\psi(y;k,l)-
\frac y{\varphi(k)}| <&  A(m,\delta)
    \sum_{\chi}
    \sum\Sb \rho \in \Cal Z(\chi) \\ |\gamma|> H \endSb
    \frac {x^{\beta-1}}{|\rho(\rho+1)...(\rho+m)|} \\
\ &+  (1+\frac {m\delta}2)
    \sum_{\chi}
    \sum\Sb \rho \in \Cal Z(\chi) \\ |\gamma|\le H \endSb
    \frac {x^{\beta-1}}{|\rho|} + \frac{m\delta}{2}+\tilde R,
\endalign
$$ where $\sum_{\chi}$ denotes the summation over all characters
modulo
$k$,
$$
\tilde R=
\frac {\varphi(k)}x
      \big[(f(k)+0.5)\ln (x)+4\ln (k)+13.4\big],
\leqno(4.3.2)
$$ and $f(k)$ is given by $(4.3.1)$\pagebreak.
\endproclaim
\bn

We also wish to allow
$\theta$ instead of $\psi$ which can be done by recalling Theorem
$6$ of Rosser and Schoenfeld ([9]) :
$$ 0\le\psi(x;k,l)-\theta(x;k,l)\le\psi(x)-\theta(x)\le 1.0012\sqrt
x+3x^{\frac13}
\text{\ \ for\ \ }x>0.
\leqno(4.3.3)
$$ Collecting our results, we finally get
\bigskip
%\vfill\eject

\proclaim{Theorem 4.3.2}
Let  $k \ge 1$  be an integer, and let  $H \ge 1000$  be a real
number such that every  $L$-function with modulus  $k$  satisfies 
$GRH(H)$.  Put  $C = C_1(\chi_0,H)$, where  $\chi_0$  is the trivial
character, and let 
$x_0$  be a real number such that  $R \cdot \ln^2(H/C) \ge
\ln(x_0)$.  Let 
$m
\ge 2$  be an integer, and let  $\delta > 0$  be a real number such
that  $0 < \delta < (x_0-2)/(mx_0)$.
%\smallskip 

Then for any  $x \ge x_0$  we have
$$
\align
\frac {\varphi(k)}x \max_{1\le y \le x}|\psi(y;k,l)-\frac
y{\varphi(k)}&| \le \frac{(1+(1+\delta)^{m+1})^m}{\delta^m} 
\frac{\varphi(k)}{2}((\tilde A+\tilde B)+\frac{1}{\sqrt{x_0}}(\tilde
C+\tilde D))\\ &\quad + (1+\frac
{m\delta}2)\frac{\varphi(k)}{\sqrt{x_0}}\tilde
E(H)+\frac{m\delta}2+\tilde R,
\endalign
$$  where $\tilde A$, $\tilde B$, $\tilde C$, $\tilde D$ are those
of Lemma $4.1.3$ with  $C_1(\chi)$  replaced by  $C$  and with 
$\tilde A$ estimated by Lemma $4.2.1$, 
$\tilde{E}(H)$  given by Lemma $4.1.2$, and  $\tilde R$  given by
$(4.3.2)$ with  $x = x_0$.
%\smallskip 

If  $\tilde{\Delta} = \varphi(k)(1.0012x^{-1/2}_0 + 3x^{-2/3}_0)$,
and 
$\tilde{Z}$  denotes the error bound above, then  $\tilde{Z} +
\tilde{\Delta}$  is an upper bound for
$$
\frac{\varphi(k)}{x} \max\limits_{1\le y\le x} \left|
\theta(y;k,l) - \frac{y}{\varphi(k)}\right| .
$$
\endproclaim
%\smallskip

\demo{Proof} We use Theorem 4.3.1 and evaluate the sums over the
zeros by appealing to Lemmas 4.1.2 and 4.1.3. We also remark (cf.
[8]) that 
$$  A(m,\delta)\le\frac{(1+(1+\delta)^{m+1})^m}{\delta^m}.\quad\qed
\leqno(4.3.4)
$$
\enddemo

Theorem 4.3.2 has been stated in such a way as to make its
implementation as simple as possible, and to minimize the numerical
input required.  For a stronger version which yields slightly better
results, see Theorem 5.1.1 below.

\subhead 4.4. Heuristic control of the values of $m$, $\delta$ and
$\varepsilon$\endsubhead
Let  $\varepsilon_k(m,\delta)$  denote the expression  $\tilde{Z} +
\tilde{\Delta}$ in Theorem 4.3.2, and let  $\varepsilon_k$  be its
minimum value as 
$m$  and  $\delta$  vary.  This optimization is achieved in practice
by first minimizing over  $\delta$ (with  $m$  held fixed) using any
standard minimization algorithm, and then minimizing over  $m$.

In this section we aim to heuristically estimate the optimal values
of  $m$  and  $\delta$ and the order of magnitude of 
$\varepsilon_k$, together with their dependence on  $k$  and  $H$. 
Under the hypothesis of Theorem 4.3.2, we note that in the formula
for  $\tilde{Z} + \tilde{\Delta}$:
\roster
\item"(1)"  $(1 + (1 + \delta)^{m+1})^m/\delta^m$  is equivalent to 
$2^m/\delta^m$\pagebreak\  if  $m\delta$  tends to  $0$. 
\item"(2)"  $(\tilde{A} +
\tilde{B} + (\tilde{C} + \tilde{D})/\sqrt{x_0})$  can be
approximated by  $\ln\left(\frac{kH}{2\pi}\right)/(\pi mH^m)$  for
large  $H$  and  $x_0$.
\item"(3)" $(1+m\delta/2)\tilde E(H) x_0^{-1/2}$ is asymptotic to
$x_0^{-1/2}\frac{\ln^2(kH)}{2\pi}$ as $H$ goes to infinity and
$\delta$ to 0. For large $x_0$ it is negligible.
\item"(4)" $\tilde R$ is $\Cal O(k\ln(kx_0)/x_0)$ when $x_0$ goes to
infinity. For $x_0$ large it is negligible.
\item"(5)"  For large  $x_0$, $\tilde{\Delta}$  is negligible.
\endroster

Hence our $\varepsilon_k$ can be approximated by 
$$ f(m,\delta)=\frac{m\delta}{2}
+\frac{\varphi(k)}{2}\left(\frac2{\delta H}\right)^m
\frac{\ln\left(\frac{kH}{2\pi}\right)}{\pi m}.
\leqno(4.4.1)
$$

We reduce this expression once more; numerically, it happens that
the values of $m$ are almost always near $7$. We replace the term
$\left(\frac2{\delta H}\right)^m\frac 1m$ by
$\left(\frac2{\delta H}\right)^m\frac 1{\bar m}$, where $\bar m$ is
an estimation of the value of $m$ and we finally consider 
$$
\bar f(m,\delta)=\frac{a}{(\delta B)^m}+\frac{m\delta}{2}
\leqno(4.4.2)
$$  with
$$ 
a=\frac{\varphi(k)}2\frac{\ln\left(\frac{kH}{2\pi}\right)}{\pi\bar
m} \text{\ \ \ and\ \ \ } B=\frac H2.
\leqno(4.4.3)
$$

We now let $m$ vary continuously. Then $$
\frac{\dif \bar f}{\dif\delta}=\frac{\dif\bar f}{\dif m}=0 $$ yields
$$
\left\{\aligned &\frac{a}{(\delta B)^m}=\frac{\delta}2,\\
&\frac{a}{(\delta B)^m}\ln(\delta B)=\frac{\delta}2,
\endaligned\right.
$$  hence
$$
\gather
\delta=\frac{2e}{H}
\text{,\ \ }\qquad m=\ln(aH/e)
\text{,\ \ }\\
\bar f(m,\delta)=(m+1)\frac{\delta}2 =
\ln\left(\frac{H\varphi(k)}{2\pi \bar m}
\ln\left(\frac{kH}{2\pi}\right)\right)\cdot \frac{e}{H} .
\tag4.4.4\endgather
$$

It is striking that in this approximation, $\delta$ depends neither
on $m$ nor on $k$. We have verified this numerically ($\delta$ is
almost constant). The other striking thing is that
$\bar f(m,\delta)$ (our $\varepsilon_k$) depends on $k$ mainly
through
$\ln(\varphi(k))$; we find numerically that our
$\varepsilon_k$'s increase very slowly when $k$ increases. This is
rather surprising since there are $\varphi(k)$ sums to evaluate, so
we might have expected $\varepsilon_k$ to depend on $k$ by a
multiplicative factor $\varphi(k)$.
%\vfill\eject

\head 5. Numerical estimates for $\theta$ and $\psi$\endhead

In this section, we present the numerical estimates one can deduce
from the results in \S\S2, 3, and 4.  Since the analytic
bounds only become useful for
$x_0\ge10^{10}$, we have supplemented them with a tabulation for low
heights. 
%\smallskip 

The following notation is used throughout:
$$
\cases
\aligned &\varepsilon(\theta,x,k) \,=\max_{(l,k) = 1}
\frac{\varphi(k)}{x}
\max\limits_{1\le y\le x}|\theta(y;k,l)-y/\varphi(k)|,
\\  &\varepsilon(\psi,x,k)=\max_{(l, k)=1}
\frac{\varphi(k)}{x}
\max\limits_{1\le y\le x}|\psi(y;k,l)-y/\varphi(k)| .
\endaligned\endcases\leqno(5.1)
$$ 
%\smallskip 

The computations for the tables were performed on PCs using the
Intel 80486 processor. The 80486-DX was used in Extended Precision
mode, where it calculates with an 80-bit word having a 1-bit sign, a
15-bit exponent, and a 64-bit mantissa. It carries approximately
19.2 decimal digits of accuracy, and can represent numbers over the
range $10^{-4392}$ to $10^{4392}$. In Tables 1 and 2, the results
are reported to 6 digits of accuracy, and were rounded up in their
last digit. 
%\smallskip

As a verification, independent computations were done using the
calculator GP of PARI, and some randomly chosen values were checked,
with satisfying agreement.

\subhead 5.1. Asymptotic results for  $x_0 \ge 10^{10}$\endsubhead
\proclaim{Theorem 1} For any triple $(k,\varepsilon,x_0)$ given by
Table $1$, 
$$
\left\{\aligned &\varepsilon(\theta,x,k)\le\varepsilon
\text{\ \ for\ \ }x\ge x_0,\\ &\varepsilon(\psi,x,k)\le\varepsilon
\text{\ \ for\ \ }x\ge x_0.
\endaligned\right.
$$
\endproclaim

Table 1 contains the best known bounds for all the moduli covered by
Rumely's $L$-series calculations.  The bounds were computed using a
strong form of Theorem 4.3.2, which broke the error term into its
contributions from each individual $L$-series, and replaced the
estimates for  $\tilde{E}(H)$  by tabulated values:
%\par
%\bigskip

\proclaim{Theorem 5.1.1}  Let  $k \ge 1$  be an integer.  For each
character  $\chi$  modulo  $k$, let  $H_{\chi} \ge 1000$  be such
that  $L(s,\chi)$  satisfies GRH$(H_{\chi})$; we suppose  $H_{\chi}
= H_{\bar{\chi}}$  for all  $\chi$, and put  $C_1(\chi) =
C_1(\chi,H_{\chi})$.  Let  $x_0 \ge 10$  be a real number such that
for each  $d$  dividing  $k$, and each  $\chi$  with conductor 
$d$,  $2R\cdot \ln^2(dH_{\chi}/C_1(\chi)) \ge \ln(x_0)$.  Let 
$m$  be a positive integer, and let  $\delta > 0$  be a real number
such that  $0 < \delta < (x_0 - 2)/(mx_0)$.  For each  $\chi$  with
conductor  $d$, let  $\tilde{A_{\chi}}$, $\tilde{B_{\chi}}$,
$\tilde{C_{\chi}}$, $\tilde{D_{\chi}}$  be the constants from Lemma
$4.1.3$ computed using  $d$  and  $H_{\chi}$; let 
$\tilde{E_{\chi}}$  be a tabulated bound for 
$\mathop{\sum}\Sb \rho\in Z(\chi) \\ |\beta|\le H_{\chi} \endSb
1/|\rho|$.  Let  $\tilde{R}$  be as in Theorem $4.3.1$.
%\vfill\eject

Then for  $x \ge x_0$ 
$$
\align
\varepsilon(\psi,x,k) \le \frac{1}{2} A(m,\delta) &\cdot
\sum_{\chi} \left[\tilde{A}_{\chi} + \tilde{B}_{\chi} +
(\tilde{C}_{\chi} + \tilde{D}_{\chi})/\sqrt{x}_0\right] \\ &+
\frac{(1 + m\delta/2)}{\sqrt{x_0}} \cdot \left(\sum_{\chi}
\tilde{E}_{\chi}\right) + m\delta/2 + \tilde{R} .
\endalign
$$

{\pageinsert
\centerline{{\smc Table 1.} \rom{Analytic epsilons for $x \ge
x_0$}}
\par 
\rom{The values reported are  $\varepsilon =
\max(\varepsilon(\theta,x_0,k), \varepsilon(\psi,x_0,k))$, and have
been rounded up in the last decimal place (our notation is as in
(5.1) of the paper).  }
\smallskip
%%\bigskip
{\hskip-.3in {\sevenpoint\vbox{\offinterlineskip
\halign{\strut \hfil \quad$#$ & 
 \quad$#$\hfil &
 \quad$#$\hfil &
\quad$#$\hfil &\quad$#$\hfil&
%%\hskip.7truein $#$ &
%%\quad$#$\hfil &
%xxxxxxx
\hskip.2truein $#$ &
\hfil \quad $#$ &
\quad $#$ \hfil & 
\quad$#$\hfil &
\quad$#$\hfil & \quad $#$ \hfil \cr 
\underline{k} & \hfil\underline{\quad 10^{10}\quad}\hfil&
\hfil\underline{\quad 10^{13}\quad}\hfil &
\hfil\underline{\quad 10^{30}\quad}\hfil & \hfil\underline{\quad
10^{100}\ }\hfil & %%&
\underline{k} &
\hfil\underline{\quad 10^{10}\quad} \hfil&
\hfil\underline{\quad 10^{13}\quad} \hfil&
\hfil\underline{\quad 10^{30}\quad}\hfil & \hfil\underline{\quad
10^{100}\ }\hfil\cr
\noalign{\smallskip}      1 & 0.000213 & 0.000015 & 0.000001 &
0.000001 & %%& 
46 & 0.012682 & 0.009592 & 0.009012 & 0.007046 \cr 2 &
0.000213 & 0.000015 & 0.000001 & 0.000001 & %%& 
47 & 0.018044 &
0.010720 & 0.010012 & 0.008148 \cr 3 & 0.002238 & 0.001951 &
0.001813 & 0.001310 & %%& 
48 & 0.011080 & 0.008876 & 0.008319 &
0.006373 \cr 4 & 0.002238 & 0.001943 & 0.001809 & 0.001324 & %%& 
49 &
0.017015 & 0.010461 & 0.009773 & 0.007891 \cr 5 & 0.002785 &
0.002250 & 0.002105 & 0.001606 & %%& 
50 & 0.012214 & 0.009447 &
0.008851 & 0.006861 \cr 6 & 0.002238 & 0.001951 & 0.001813 &
0.001310 & %%& 
51 & 0.014479 & 0.009803 & 0.009191 & 0.007312 \cr 7 &
0.003248 & 0.002406 & 0.002258 & 0.001770 & %%& 
52 & 0.012671 &
0.009058 & 0.008475 & 0.006656 \cr 8 & 0.002811 & 0.002257 &
0.002116 & 0.001634 & %%& 
53 & 0.019206 & 0.010734 & 0.010011 &
0.008188 \cr 9 & 0.003228 & 0.002380 & 0.002235 & 0.001759 & %%& 
54 &
0.011579 & 0.009061 & 0.008492 & 0.006572 \cr 10 & 0.002785 &
0.002250 & 0.002105 & 0.001606 & %%& 
55 & 0.016306 & 0.010159 &
0.009493 & 0.007591 \cr 11 & 0.004125 & 0.002590 & 0.002421 &
0.001954 & %%& 
56 & 0.012961 & 0.009462 & 0.008865 & 0.006969 \cr 12 &
0.002781 & 0.002241 & 0.002099 & 0.001610 & %%& 
57 & 0.015636 &
0.010300 & 0.009640 & 0.007691 \cr 13 & 0.004560 & 0.002657 &
0.002478 & 0.002020 & %%& 
58 & 0.014102 & 0.010007 & 0.009398 &
0.007440 \cr 14 & 0.003248 & 0.002406 & 0.002258 & 0.001770 & %%& 
59 &
0.020659 & 0.011057 & 0.010299 & 0.008451 \cr 15 & 0.008634 &
0.007628 & 0.007088 & 0.005045 & %%& 
60 & 0.010879 & 0.008740 &
0.008182 & 0.006204 \cr 16 & 0.008994 & 0.007938 & 0.007392 &
0.005393 & %%& 
61 & 0.021073 & 0.011080 & 0.010315 & 0.008475 \cr 17 &
0.010746 & 0.008587 & 0.008047 & 0.006203 & %%& 
 62 & 0.014535 &
0.010088 & 0.009468 & 0.007527 \cr 18 & 0.003228 & 0.002380 &
0.002235 & 0.001759 & %%& 
 63 & 0.015473 & 0.010048 & 0.009409 &
0.007510 \cr 19 & 0.011892 & 0.009442 & 0.008852 & 0.006838 & %%&
  64 &
0.014938 & 0.010019 & 0.009400 & 0.007530 \cr 20 & 0.008501 &
0.007465 & 0.006950 & 0.005024 & %%&
  65 & 0.018025 & 0.010412 &
0.009697 & 0.007837 \cr 21 & 0.009708 & 0.008143 & 0.007582 &
0.005625 & %%& 
 66 & 0.011685 & 0.008844 & 0.008276 & 0.006357 \cr 22 &
0.004125 & 0.002590 & 0.002421 & 0.001954 & %%&
  67 & 0.022414 &
0.011252 & 0.010451 & 0.008620 \cr 23 & 0.012682 & 0.009592 &
0.009012 & 0.007046 & %%& 
68 & 0.014691 & 0.009884 & 0.009272 &
0.007412 \cr 24 & 0.008173 & 0.007107 & 0.006615 & 0.004775 & %%&
  69 &
0.017285 & 0.010536 & 0.009835 & 0.007944 \cr 25 & 0.012214 &
0.009447 & 0.008851 & 0.006861 & %%&
  70 & 0.012809 & 0.009431 &
0.008821 & 0.006862 \cr 26 & 0.004560 & 0.002657 & 0.002478 &
0.002020 & %%&
  71 & 0.023269 & 0.011310 & 0.010484 & 0.008672 \cr 27 &
0.011579 & 0.009061 & 0.008492 & 0.006572 & %%&
  72 & 0.012665 &
0.009118 & 0.008546 & 0.006759 \cr 28 & 0.009908 & 0.008298 &
0.007741 & 0.005800 & %%&
  74 & 0.015830 & 0.010323 & 0.009668 &
0.007768 \cr 29 & 0.014102 & 0.010007 & 0.009398 & 0.007440 & %%&
  75 &
0.016346 & 0.010341 & 0.009659 & 0.007724 \cr 30 & 0.008634 &
0.007628 & 0.007088 & 0.005045 & %%&
  76 & 0.015819 & 0.010331 &
0.009671 & 0.007751 \cr 31 & 0.014535 & 0.010088 & 0.009468 &
0.007527 & %%& 
 77 & 0.020600 & 0.010854 & 0.010076 & 0.008235 \cr 32 &
0.011103 & 0.008857 & 0.008300 & 0.006399 & %%& 
 78 & 0.012621 &
0.009108 & 0.008514 & 0.006625 \cr 33 & 0.011685 & 0.008844 &
0.008276 & 0.006357 & %%&
 80 & 0.014527 & 0.009700 & 0.009085 &
0.007218 \cr 34 & 0.010746 & 0.008587 & 0.008047 & 0.006203 & %%&
  81 &
0.019611 & 0.010834 & 0.010094 & 0.008264 \cr 35 & 0.012809 &
0.009431 & 0.008821 & 0.006862 & %%&
 82 & 0.016744 & 0.010524 &
0.009843 & 0.007953 \cr 36 & 0.009544 & 0.007900 & 0.007372 &
0.005563 & %%&
  84 & 0.012721 & 0.009304 & 0.008703 & 0.006792 \cr 37 &
0.015830 & 0.010323 & 0.009668 & 0.007768 & %%&
 85 & 0.021395 &
0.010990 & 0.010205 & 0.008353 \cr 38 & 0.011892 & 0.009442 &
0.008852 & 0.006838 & %%&
  86 & 0.017180 & 0.010598 & 0.009904 &
0.008025 \cr 39 & 0.012621 & 0.009108 & 0.008514 & 0.006625 & %%&
  87 &
0.019865 & 0.010903 & 0.010152 & 0.008278 \cr 40 & 0.010833 &
0.008627 & 0.008084 & 0.006183 & %%&
  88 & 0.016484 & 0.010131 &
0.009474 & 0.007646 \cr 41 & 0.016744 & 0.010524 & 0.009843 &
0.007953 & %%&
  90 & 0.012817 & 0.009413 & 0.008809 & 0.006884 \cr 42 &
0.009708 & 0.008143 & 0.007582 & 0.005625 & %%&
  91 & 0.023159 &
0.011148 & 0.010314 & 0.008484 \cr 43 & 0.017180 & 0.010598 &
0.009904 & 0.008025 & %%&
  92 & 0.017508 & 0.010574 & 0.009872 &
0.008009 \cr 44 & 0.011798 & 0.008871 & 0.008310 & 0.006440 & %%&
 93 &
0.020727 & 0.011013 & 0.010245 & 0.008375 \cr 45 & 0.012817 &
0.009413 & 0.008809 & 0.006884 & %%&
  94 & 0.018044 & 0.010720 &
0.010012 & 0.008148 \cr}}}
}
\bigskip
\bigskip
\endinsert}


{\topinsert
\centerline{{\smc Table 1} \rom{(continued)}}
%\vfill\eject
\vskip2pc
{\hskip-.3in {\sevenpoint\vbox{\offinterlineskip
\halign{\strut \hfil \quad$#$ & 
 \quad$#$\hfil &
 \quad$#$\hfil &
\quad$#$\hfil &\quad$#$\hfil&
%\hskip.7truein $#$ &
\hskip.2truein $#$ &
%\quad$#$\hfil &
\hfil \quad $#$ &
\quad $#$ \hfil & 
\quad$#$\hfil &
\quad$#$\hfil & \quad $#$ \hfil \cr 
\underline{k} & \hfil\underline{\quad 10^{10}\quad}\hfil&
\hfil\underline{\quad 10^{13}\quad}\hfil &
\hfil\underline{\quad 10^{30}\quad}\hfil & \hfil\underline{\quad
10^{100}\ }\hfil & %%&
\underline{k} &
\hfil\underline{\quad 10^{10}\quad} \hfil&
\hfil\underline{\quad 10^{13}\quad} \hfil&
\hfil\underline{\quad 10^{30}\quad}\hfil & \hfil\underline{\quad
10^{100}\ }\hfil\cr
\noalign{\smallskip}          95 & 0.023180 & 0.011282 & 0.010444 &
0.008570 &%%&  
144 & 0.018241 & 0.010486 & 0.009772 & 0.007976 \cr 96
& 0.014710 & 0.009886 & 0.009271 & 0.007395 & %%& 
150 & 0.016346 &
0.010341 & 0.009659 & 0.007724 \cr 98 & 0.017015 & 0.010461 &
0.009773 & 0.007891 & %%& 
 154 & 0.020600 & 0.010854 & 0.010076 &
0.008235 \cr 
99 & 0.020644 & 0.010823 & 0.010051 & 0.008238 & %%&  
156
& 0.018025 & 0.010363 & 0.009653 & 0.007827 \cr 
100 & 0.016476 &
0.010310 & 0.009631 & 0.007737 & %%&  
162 & 0.019611 & 0.010834 &
0.010094 & 0.008264 \cr 
102 & 0.014479 & 0.009803 & 0.009191 &
0.007312 & %%&  
163 & 0.044431 & 0.012935 & 0.011505 & 0.009799 \cr 104
& 0.018292 & 0.010424 & 0.009715 & 0.007922 & %%& 
 168 & 0.017995 &
0.010459 & 0.009740 & 0.007882 \cr 105 & 0.017707 & 0.010434 &
0.009708 & 0.007787 & %%&
  169 & 0.042497 & 0.012769 & 0.011383 &
0.009670 \cr 106 & 0.019206 & 0.010734 & 0.010011 & 0.008188 & %%& 
170
& 0.021395 & 0.010990 & 0.010205 & 0.008353 \cr 108 & 0.015635 &
0.010079 & 0.009441 & 0.007589 & %%&
  174 & 0.019865 & 0.010903 &
0.010152 & 0.008278 \cr 110 & 0.016306 & 0.010159 & 0.009493 &
0.007591 & %%&
  180 & 0.018004 & 0.010482 & 0.009763 & 0.007903 \cr 111
& 0.023287 & 0.011261 & 0.010425 & 0.008589 & %%&
  182 & 0.023159 &
0.011148 & 0.010314 & 0.008484 \cr 112 & 0.018184 & 0.010498 &
0.009777 & 0.007942 & %%&
  186 & 0.020727 & 0.011013 & 0.010245 &
0.008375 \cr 114 & 0.015636 & 0.010300 & 0.009640 & 0.007691 & %%&
  190
& 0.023180 & 0.011282 & 0.010444 & 0.008570 \cr 116 & 0.020143 &
0.010962 & 0.010209 & 0.008353 & %%&
  198 & 0.020644 & 0.010823 &
0.010051 & 0.008238 \cr 117 & 0.023222 & 0.011135 & 0.010306 &
0.008499 & %%&
  210 & 0.017707 & 0.010434 & 0.009708 & 0.007787 \cr 118
& 0.020659 & 0.011057 & 0.010299 & 0.008451 & %%&
  216 & 0.023480 &
0.011154 & 0.010324 & 0.008542 \cr 120 & 0.014495 & 0.009785 &
0.009169 & 0.007257 & %%&
  222 & 0.023287 & 0.011261 & 0.010425 &
0.008589 \cr 121 & 0.031939 & 0.012053 & 0.010963 & 0.009223 & %%&
 234
& 0.023222 & 0.011135 & 0.010306 & 0.008499 \cr 122 & 0.021073 &
0.011080 & 0.010315 & 0.008475 & %%&
  242 & 0.031939 & 0.012053 &
0.010963 & 0.009223 \cr 124 & 0.021008 & 0.011055 & 0.010285 &
0.008436 & %%&
 243 & 0.043836 & 0.012879 & 0.011458 & 0.009740 \cr 125
& 0.029581 & 0.011885 & 0.010860 & 0.009072 & %%&
 250 & 0.029581 &
0.011885 & 0.010860 & 0.009072 \cr 126 & 0.015473 & 0.010048 &
0.009409 & 0.007510 &%%& 
256 & 0.036558 & 0.012394 & 0.011179 &
0.009461 \cr 128 & 0.022035 & 0.011123 & 0.010336 & 0.008526 & %%&
 286
& 0.033703 & 0.012101 & 0.010952 & 0.009218 \cr 130 & 0.018025 &
0.010412 & 0.009697 & 0.007837 & %%&
 326 & 0.044431 & 0.012935 &
0.011505 & 0.009799 \cr 132 & 0.016320 & 0.010123 & 0.009461 &
0.007593 & %%&
 338 & 0.042497 & 0.012769 & 0.011383 & 0.009670 \cr 134
& 0.022414 & 0.011252 & 0.010451 & 0.008620 & %%&
 360 & 0.028267 &
0.011631 & 0.010644 & 0.008854 \cr 138 & 0.017285 & 0.010536 &
0.009835 & 0.007944 & %%&
 420 & 0.027757 & 0.011602 & 0.010617 &
0.008766 \cr 140 & 0.017989 & 0.010523 & 0.009796 & 0.007898 & %%&
 432
& 0.039660 & 0.012542 & 0.011234 & 0.009519 \cr 142 & 0.023269 &
0.011310 & 0.010484 & 0.008672 & %%&
 486 & 0.043836 & 0.012879 &
0.011458 & 0.009740 \cr 143 & 0.033703 & 0.012101 & 0.010952 &
0.009218 \cr}}}
}
%\vfill\eject
\endinsert}

If   $\tilde{\Delta} = \varphi(k)(1.0012 x^{-1/2}_0 + 3x^{-2/3}_0)$,
and  $\tilde{Z}$  denotes the above bound for 
$\varepsilon(\psi,x,k)$, then
$$
\varepsilon(\theta,x,k) \le \tilde{Z} + \tilde{\Delta} .
$$
\endproclaim

If  $k$  is not of the form  $k = 2k'$  with  $k'$  odd, and  $k
\ne 1$, then the values reported in Table 1 are the minimal values
of the expression 
$\tilde{Z} +
\tilde{\Delta}$  in Theorem 5.1.1, when  $2
\le m
\le 14$  and 
$\delta$  ranges over its allowable values.  The minima were found
using Brent's golden section/parabolic interpolation algorithm
(\cite{7, p. 283}).  It should be noted that as  $x_0$  and the 
$H_{\chi}$  vary, different terms in the expression become dominant. 
Heuristically the best estimate is obtained using GRH data to
height  $H \cong 4e\pi\cdot\sqrt{x_0}/(\varphi(k)\cdot\ln(x_0))$. 
The values in Table 1 are minima over the following data sets:
%\smallskip

\roster
\item"{(A)}" Rumely's $L$-series and zeta data, using $T = 2500$ data
for
$k
\le 13$,
\item"{(B)}" Rumely's $L$-series and zeta data, using $T = 5000$ data
for $k
\le 13$,
\item"{(C)}" Rumely's $L$-series and zeta data,
\item"{(D)}" Rumely's $L$-series data $+$ Lehman's zeta data, with $T
= 12030$\pagebreak, 
\item"{(E)}" Rumely's $L$-series data $+$ vdL, teR, Winter zeta data
to
$T = 10^5$, 
\item"{(F)}" Rumely's $L$-series data $+$ vdL, teR, Winter zeta data
to
$T = 10^6$, 
\item"{(G)}" Rumely's $L$-series data $+$ vdL, teR, Winter zeta data
to
$T = 10^7$, 
\item"{(H)}" Rumely's $L$-series data $+$ vdL, teR, Winter zeta data
to
$T = 545439823$.
\endroster
%\bigskip

If  $k = 2k'$  with  $k'$  odd, and if  $N$  is the order of 2 in 
$(\Bbb Z/k'\Bbb Z)^{\times}$, then for  $(k,l) = 1$
$$
\gather |\psi(x;k,l) - \psi(x;k',l)| \le (1/N)\ln(x) +
\ln(2) , \\ |\theta(x;k,l) - \theta(x;k',l)| \le \ln(2) .
\endgather
$$ For the values of  $x_0$  considered in Table 1, these
inequalities yield estimates for 
$\varepsilon(\theta,x,k)$, $\varepsilon(\psi,x,k)$  better than
those provided by Theorem 5.1.1 directly, and are the ones reported. 
It will be observed that the bounds for 
$k$  and  $k'$  in Table 1 are always the same:  the additional term
in the error estimate was swallowed up when the values were rounded
up to the nearest  $10^{-6}$.

Finally, when  $k = 1, 2$ the  $x_0 = 10^{10}$  entry in Table 1 is
derived from the main table of Rosser and Schoenfeld (\cite{9}),
which gives 
$\varepsilon(\psi,e^{23},1) = 0.00020211$; this improves on the
value  $0.000272$  given by Theorem 5.1.1.

\subhead 5.2. Estimates over the range  $0 - 10^{10}$\endsubhead
In Table 2 we compare  $|\psi(x;k,l) - x/\varphi(k)|$  and 
$|\theta(x;k,l) - x/\varphi(k)|$  with  $\sqrt{x}$, reporting the
maximum ratio for  $0 < x \le 10^{10}$  for each  $k$.  Tabulations
over the subintervals 
$[10^n,10^{n+1}]$  show that the results are remarkably stable and
uniform over the entire range, so that Table 2 gives a good
representation of the true error.  The only exception is for very
small moduli  $(k \le 6)$  where the initial primes distort the
error estimates; more accurate asymptotic values are given at the
end of the table for these moduli.  Let 
$N(k,l)$  be the number of solutions to  $a^2 \equiv l
\pmod{k}$ with $0 \le a < k$.  The function 
$(x-N(k,l)\sqrt{x})/\varphi(k)$  usually gives a better
approximation to  $\theta(x;k,l)$  than 
$x/\varphi(k)$; we have also tabulated the maximal ratios 
$|\theta(x;k,l) - (x-N(k,l)\sqrt{x})/\varphi(k)|\sqrt{x}$.

We can summarize the main results of Table 2 as follows:

\proclaim{Theorem 5.2.1}  For all the moduli in Table $1$, uniformly
over the range  $0 \le x \le 10^{10}$, the following bounds hold:
$$
\align |\theta(x;k,l) - x/\varphi(k)| &\le 2.072\sqrt{x}, \\
\left|\theta(x;k,l) -
\frac{x-N(k,l)\sqrt{x}}{\varphi(k)}\right| &\le 1.174\sqrt{x},\\
|\psi(x;k,l) - x/\varphi(k)| &\le 1.745\sqrt{x} .
\endalign
$$ 
If  $k=5$, $k \ge 7$  or  $x \ge 224$, then
$$ 
|\psi(x;k,l) - x/\varphi(k)| \le \sqrt{x} ;
$$ 
for  $k = 1, 3, 4$  this holds if  $x \ge 14$.
\endproclaim

\proclaim{Corollary 5.2.2}  For these moduli, over the range  $0 < x
\le 10^{10}$,
$$
\varepsilon(\theta,x,k) \le 2.072 \frac{\varphi(k)}{\sqrt{x}} ,
\quad \varepsilon(\psi,x,k) \le 1.745 \frac{\varphi(k)}{\sqrt{x}} ;
$$ for  $k =5$ or\pagebreak\ $k \ge 7$, $\varepsilon(\psi,x,k) \le
\varphi(k)/\sqrt{x}$.
\endproclaim
%\bn

%\vfill\eject
\centerline{{\smc Table 2.} \rom{Tabulations $0-10^{10}$}}
\smallskip
\roster
%\widestnumber\itemitem{\quad ASYMPTOTIC :}
\item"{}"  $k$  :  \ \ The modulus. 
\item"{}"  $\theta  : \ \ \,\,
\operatornamewithlimits{\text{max}}\limits_{(l,k)=1} 
\operatornamewithlimits{\text{max}}\limits_{(0,10^{10}]}
|\theta(x;k,l) - x/\varphi(k)|/\sqrt{x}$
\item"{}"  $\theta\# : \ \,
\operatornamewithlimits{\text{max}}\limits_{(l,k)=1} 
\operatornamewithlimits{\text{max}}\limits_{(0,10^{10}]}
|\theta(x;k,l) - (x-N(k,l)\sqrt{x})/\varphi(k)|/\sqrt{x}$, 
\itemitem{}  \hskip.5in where
$N(k,l)$  is the number of solutions to $a^2 \equiv l
\pmod{k}$ 
\item"{}"  $\psi :\ \  
\operatornamewithlimits{\text{max}}\limits_{(l,k)=1} 
\operatornamewithlimits{\text{max}}\limits_{(0,10^{10}]}
|\psi(x;k,l) -x/\varphi(k)|/\sqrt{x}$ 
\endroster
\smallskip

\noindent Except for the $\psi$-values for a few small moduli, the
bounds for
$\theta$,
$\theta\#$ and
$\psi$ are remarkably good throughout the range
$0-10^{10}$. All values have been rounded up in the last decimal
place.
\smallskip

\hskip.25in{\sevenpoint\vbox{\offinterlineskip
\halign{\strut \hfil \quad$#$ & 
\quad$#$\hfil &
\quad$#$\hfil & 
\quad$#$\hfil &
%\hskip.7truein$#$ &
\hskip.2truein$#$ &
%\quad$#$\hfil &
\quad\hfil$#$ &
\quad$#$ \hfil & 
\quad$#$\hfil & \hfil \quad$#$  \cr 
\underline{k} & \hfil\underline{\quad\theta\quad}\hfil&
\hfil\underline{\quad\theta\#\quad} \hfil&
\hfil\underline{\quad\psi\quad}\hfil  & &
\underline{k} &
\hfil\underline{\quad\theta\quad} \hfil&
\hfil\underline{\quad\theta\#\quad} \hfil&
\hfil\underline{\quad\psi\quad}\hfil\cr
\noalign{\smallskip}  1 & 2.052818 & 1.052818 & 1.414214^* && 39 &
0.769446 & 0.666830 & 0.668572 \cr 2 & 2.071193 & 1.071193 &
1.744754^* && 40 & 1.076203 & 0.813546 & 0.816561 \cr 3 & 1.798158 &
1.053542 & 1.070833^* && 41 & 0.818620 & 0.818620 & 0.817634 \cr 4 &
1.780719 & 1.034832 & 1.118034^* && 42 & 1.130693 & 0.863011 &
0.813182 \cr 5 & 1.412480 & 0.912480 & 0.886346^* && 43 & 0.832936 &
0.785317 & 0.784932 \cr 6 & 1.798158 & 1.173049 & 1.322876^* && 44 &
0.873277 & 0.873277 & 0.871473 \cr 7 & 1.105822 & 0.829249 &
0.779283 && 45 & 0.844820 & 0.844820 & 0.844864 \cr 8 & 1.817557 &
1.000000 & 0.926535 && 46 & 0.973114 & 0.882205 & 0.881655 \cr 9 &
1.108042 & 0.899812 & 0.788900 && 47 & 0.744386 & 0.787865 &
0.790079 \cr 10 & 1.412480 & 0.912480 & 0.961267^* && 48 & 1.096688
& 0.870037 & 0.876180 \cr 11 & 0.976421 & 0.885771 & 0.878823 && 49
& 0.744132 & 0.696513 & 0.697158 \cr 12 & 1.735502 & 1.000000 &
0.906786^* && 50 & 0.821890 & 0.832182 & 0.762408 \cr 13 & 0.892444
& 0.741007 & 0.737610 && 51 & 0.817323 & 0.770458 & 0.771552 \cr 14
& 1.105822 & 0.829249 & 0.897528 && 52 & 0.884117 & 0.837232 &
0.837318 \cr 15 & 1.097307 & 0.769689 & 0.760264 && 53 & 0.829958 &
0.791497 & 0.725979 \cr 16 & 1.253606 & 0.771116 & 0.773805 && 54 &
0.881762 & 0.789677 & 0.794864 \cr 17 & 1.001057 & 0.876057 &
0.873548 && 55 & 0.795133 & 0.795133 & 0.798504 \cr 18 & 1.108042 &
0.899812 & 0.824180 && 56 & 0.981635 & 0.828677 & 0.833854 \cr 19 &
1.001556 & 0.911763 & 0.911536 && 57 & 0.834201 & 0.834201 &
0.833807 \cr 20 & 1.276501 & 0.776501 & 0.800391 && 58 & 0.793283 &
0.744371 & 0.745533 \cr 21 & 1.130693 & 0.863011 & 0.805421 && 59 &
0.710444 & 0.724352 & 0.710444 \cr 22 & 0.976421 & 0.885771 &
0.879312 && 60 & 1.056320 & 0.749787 & 0.748270 \cr 23 & 0.973114 &
0.882205 & 0.880877 && 61 & 0.719386 & 0.715829 & 0.717979 \cr 24 &
1.703144 & 1.000000 & 0.744935 && 62 & 0.771883 & 0.838550 &
0.845276 \cr 25 & 0.821890 & 0.832182 & 0.764098 && 63 & 0.847851 &
0.736979 & 0.750188 \cr 26 & 0.892444 & 0.744403 & 0.744403 && 64 &
0.842522 & 0.781042 & 0.779800 \cr 27 & 0.881762 & 0.789677 &
0.795451 && 65 & 0.753091 & 0.753091 & 0.753171 \cr 28 & 1.039662 &
0.764535 & 0.775919 && 66 & 0.797466 & 0.779343 & 0.788725 \cr 29 &
0.793283 & 0.744371 & 0.745590 && 67 & 0.708427 & 0.738730 &
0.731833 \cr 30 & 1.097307 & 0.769689 & 0.760264 && 68 & 0.803940 &
0.803940 & 0.805077 \cr 31 & 0.771883 & 0.838550 & 0.845276 && 69 &
0.740573 & 0.732708 & 0.733806 \cr 32 & 1.015064 & 0.866076 &
0.860459 && 70 & 0.865028 & 0.782493 & 0.779932 \cr 33 & 0.797466 &
0.779343 & 0.788725 && 71 & 0.750488 & 0.750488 & 0.750488 \cr 34 &
1.001057 & 0.876057 & 0.874371 && 72 & 1.062100 & 0.754145 &
0.765297 \cr 35 & 0.865028 & 0.782493 & 0.779932 && 74 & 0.867916 &
0.816491 & 0.815422 \cr 36 & 1.163674 & 0.830341 & 0.829482 && 75 &
0.729046 & 0.684061 & 0.685155 \cr 37 & 0.867916 & 0.816491 &
0.815422 && 76 & 0.771532 & 0.768761 & 0.702507 \cr 38 & 1.001556 &
0.911763 & 0.911624 && 77 & 0.690677 & 0.690677 & 0.690677 \cr }}}
%\vfill\eject
\pagebreak

{\pageinsert
\centerline{{\smc Table 2} \rom{(continued)}}
\hskip.25in{\eightpoint\vbox{\offinterlineskip
\halign{\strut \hfil \quad$#$ & 
\quad$#$\hfil &
\quad$#$\hfil & 
\quad$#$\hfil &
%\hskip.7truein$#$ &
\hskip.2truein$#$ &
\quad\hfil$#$ &
\quad$#$ \hfil & 
\quad$#$\hfil & \hfil \quad$#$  \cr 
\underline{k} & \hfil\underline{\quad\theta\quad}\hfil&
\hfil\underline{\quad\theta\#\quad} \hfil&
\hfil\underline{\quad\psi\quad}\hfil  & &
\underline{k} &
\hfil\underline{\quad\theta\quad} \hfil&
\hfil\underline{\quad\theta\#\quad} \hfil&
\hfil\underline{\quad\psi\quad}\hfil\cr
\noalign{\smallskip}  78 & 0.769446 & 0.666830 & 0.668572 && 130 &
0.753091 & 0.753091 & 0.753171 \cr 80 & 0.806030 & 0.774123 &
0.773105 && 132 & 0.834519 & 0.762402 & 0.763278 \cr 81 & 0.757028 &
0.757028 & 0.757028 && 134 & 0.708427 & 0.738730 & 0.731833 \cr 82 &
0.818620 & 0.818620 & 0.817634 && 138 & 0.740573 & 0.732708 &
0.733806 \cr 84 & 0.811175 & 0.805972 & 0.805906 && 140 & 0.737551 &
0.764606 & 0.769112 \cr 85 & 0.694146 & 0.694146 & 0.694146 && 142 &
0.750488 & 0.750488 & 0.750488 \cr 86 & 0.832936 & 0.785317 &
0.784932 && 143 & 0.713437 & 0.713437 & 0.713437 \cr 87 & 0.688240 &
0.759668 & 0.688240 && 144 & 0.753220 & 0.680366 & 0.680366 \cr 88 &
0.780273 & 0.742456 & 0.741766 && 150 & 0.729046 & 0.684061 &
0.685155 \cr 90 & 0.844820 & 0.844820 & 0.844864 && 154 & 0.690677 &
0.690677 & 0.690677 \cr 91 & 0.688707 & 0.688707 & 0.688707 && 156 &
0.789471 & 0.789471 & 0.790208 \cr 92 & 0.728679 & 0.720354 &
0.719766 && 162 & 0.757028 & 0.757028 & 0.757028 \cr 93 & 0.691390 &
0.758056 & 0.691390 && 163 & 0.719154 & 0.719154 & 0.719154 \cr 94 &
0.744386 & 0.787865 & 0.790048 && 168 & 0.699889 & 0.674579 &
0.674416 \cr 95 & 0.698739 & 0.732484 & 0.698739 && 169 & 0.718525 &
0.718525 & 0.718525 \cr 96 & 0.825985 & 0.744111 & 0.742971 && 170 &
0.694146 & 0.694146 & 0.694146 \cr 98 & 0.744132 & 0.696513 &
0.697303 && 174 & 0.688240 & 0.759668 & 0.688240 \cr 99 & 0.740587 &
0.691390 & 0.691390 && 180 & 0.723937 & 0.723937 & 0.723937 \cr 100
& 0.813770 & 0.813770 & 0.812489 && 182 & 0.688707 & 0.688707 &
0.688707 \cr 102 & 0.817323 & 0.770458 & 0.771552 && 186 & 0.691390
& 0.758056 & 0.691390 \cr 104 & 0.723333 & 0.767924 & 0.765581 &&
190 & 0.698739 & 0.732484 & 0.698739 \cr 105 & 0.653897 & 0.653897 &
0.653897 && 198 & 0.740587 & 0.691390 & 0.691390 \cr 106 & 0.829958
& 0.791497 & 0.725979 && 210 & 0.653897 & 0.653897 & 0.653897 \cr
108 & 0.764890 & 0.722346 & 0.661992 && 216 & 0.698739 & 0.698739 &
0.698739 \cr 110 & 0.795133 & 0.795133 & 0.795133 && 222 & 0.698739
& 0.754294 & 0.698739 \cr 111 & 0.698739 & 0.754294 & 0.698739 &&
234 & 0.698739 & 0.698739 & 0.698739 \cr 112 & 0.740472 & 0.740472 &
0.742195 && 242 & 0.711433 & 0.717617 & 0.711433 \cr 114 & 0.834201
& 0.834201 & 0.834201 && 243 & 0.719154 & 0.731499 & 0.719154 \cr
116 & 0.764606 & 0.751262 & 0.700136 && 250 & 0.709028 & 0.709827 &
0.709028 \cr 117 & 0.698739 & 0.698739 & 0.698739 && 256 & 0.714815
& 0.714815 & 0.714815 \cr 118 & 0.710444 & 0.724352 & 0.710444 &&
286 & 0.713437 & 0.713437 & 0.713437 \cr 120 & 1.032367 & 0.672746 &
0.672746 && 326 & 0.719154 & 0.719154 & 0.719154 \cr 121 & 0.711433
& 0.717617 & 0.711433 && 338 & 0.718525 & 0.718525 & 0.718525 \cr
122 & 0.719386 & 0.715829 & 0.716087 && 360 & 0.707926 & 0.707926 &
0.707926 \cr 124 & 0.691390 & 0.749162 & 0.691390 && 420 & 0.688445
& 0.688445 & 0.688445 \cr 125 & 0.709028 & 0.709827 & 0.709028 &&
432 & 0.717112 & 0.717112 & 0.717112 \cr 126 & 0.847851 & 0.736979 &
0.750188 && 486 & 0.719154 & 0.731499 & 0.719154 \cr 128 & 0.754709
& 0.817209 & 0.817799 \cr}}} 
\vskip.15in

\noindent(*) The $\psi$-bounds for small $k$ and $x$ are distorted 
by powers of 
$p = 2, 3$. If the range $0-10^{10}$ is replaced by $1500-10^{10}$,
the bounds are much improved, as shown below. Alternatively, a
$\psi$-bound of 1.0 holds for $X_0 \le x \le 10^{10}$, with $X_0$ as below: 
\bigskip

{\eightpoint\settabs 11\columns
\+ &&&&\underbar{ $k$ } & \underbar{$\psi : 1500 - 10^{10}$} &&
\underbar{$X_0$}\cr
\+ &&&& 1 & \quad\  0.790594 & &\ \  8 &&\cr
\+ &&&& 2 & \quad\  0.805589 & & \! 224 &&\cr
\+ &&&& 3 & \quad\  0.764104 & &\  14 &&\cr
\+ &&&& 4 & \quad\  0.879669 & &\ 14 &&\cr
\+ &&&& 5 & \quad\  0.806086 & &\ \ \, 0 &&\cr
\+ &&&& 6 & \quad\  0.785064 & &\  62 &&\cr
\+ &&&& 10 & \quad\  0.792735 & &\ \ \, 0 &&\cr
\+ &&&& 12 & \quad\  0.862547 & &\ \  \, 0 &&\cr}
%\vfill\eject
\endinsert}

Table 2 was computed by using a sieve.  First,  $\theta$  and 
$\psi$  were computed for each arithmetic progression.  All three
functions
$$
\gather [\theta(x,k,l) - x/\varphi(k)]/\sqrt{x} , \quad [\psi(x,k,l)
- x/\varphi(k)]/\sqrt{x} , \\ [\theta(x;k,l) -
(x-N(k,l)\sqrt{x})/\varphi(k)]/\sqrt{x}
\endgather
$$ are monotone decreasing between their jumps at primes or prime
powers.  Thus, to determine their maximal absolute values over a
given interval, it suffices to check their values at the endpoints
of the interval, and their left and right limits at primes and prime
powers within the interval.

Considerable roundoff error can accumulate in summing  $10^9$  or
more floating point numbers.  To assure the accuracy of the
computations, upper and lower bounds for  $\theta$  and  $\psi$ 
were computed by first rounding all the logarithms up and down to
the nearest  $2^{-30}$.  After this truncation, the upper and lower
bounds could essentially be computed by integer arithmetic, e.g.
without roundoff error, and  $\theta$  and  $\psi$  were replaced by
their bounds in such a way as to assure that Table 2 gave rigorous
upper bounds.  (This choice of the truncation would have allowed the
computation to continue as high as  $1.7 \times 10^{10}$,
considerably beyond its actual limit of  $10^{10}$.)  Finally, the
values were rounded up in their last digit.

\head Acknowledgements\endhead
\par
Ramar\'e wishes first of all
to thank his advisor J.-M. Deshouillers who, after having suggested
this problem, gave good directions for the research.  Then we warmly
thank L. Reboul who verified many of the computations, and
Professors M. Balazard and A. Odlyzko for their useful comments.  We
also wish to thank the referee, who found some errors in an earlier
version of the paper.

\Refs
\refstyle{c}
\ref
\no 1
\by E. Landau
\book Handbuch der Lehre von der Verteilung der
Primzahlen
\publ with an appendix by P. Bateman, 3rd edition
\publaddr Chelsea, New York
\yr 1974
\endref

\ref
\no 2
\by J. van de Lune, H. J. J. te Riele, and D. T. Winter
\paper On the
zeros of the Riemann zeta function in the critical strip. \rom{IV}
\jour Math. Comp.
\vol 46
\yr1986
\pages 667--681
\endref

\ref
\no 3
\by K. S. McCurley 
\paper Explicit estimates for functions of primes in
arithmetic progressions
\jour Ph.D. thesis, University of Illinois at
Urbana-Champagne, 1981
\endref

\ref
\no 4
\bysame
\paper Explicit zero-free regions for Dirichlet
$L$-functions
\jour J. Number Theory
\vol 19
\yr 1984
\pages 7--32
\endref

\ref
\no 5
\bysame
\paper Explicit estimates for the error term in the
prime number theorem for arithmetic progressions
\jour Math. Comp. \vol 42
\yr 1984
\pages  265--285
\endref

\ref
\no 6
\bysame
\paper Explicit estimates for $\theta(X;3,l)$ and
$\psi(X;3,l)$
\jour Math. Comp. 
\vol 42
\yr 1984
\pages 287--296
\endref

\ref
\no 7
\by W. Press, B. Flannery, S. Teukolsky, and W. Vetterling
\book Numerical recipes
\publ Cambridge Univ. Press
\publaddr Cambridge
\yr 1986
\endref

\ref
\no 8
\by J. B. Rosser
\paper Explicit bounds for some functions of prime
numbers
\jour Amer. J. Math.
\vol 63\yr 1941
\pages 211--232
\endref

\ref
\no 9
\by  J. B. Rosser and L. Schoenfeld
\paper Sharper bounds for the
Chebyshev functions $\theta(X)$ and $\psi(X)$
\jour Math. Comp.
\vol29
\yr1975
\pages 243--269
\endref

\ref
\no 10
\by R. Rumely
\paper Numerical computations concerning the ERH
\jour Math.
Comp.
\vol 62
\yr1993
\pages 415--440
\endref

\ref
\no 11
\by  L. Schoenfeld
\paper Sharper bounds for the Chebyshev functions
$\theta(X)$ and $\psi(X)$. \rom{II}
\jour Math. Comp.
\vol30
\yr1976
\pages 337--360
\endref
\pagebreak

\ref
\no 12
\by S. B. Stechkin
\paper Rational inequalities and zeros of the
Riemann zeta-function
\jour Trudy Mat. Inst. Steklov.
\vol 189
\yr1989
\pages110--116
\moreref English transl. in
Proc. Steklov Inst. Math.
\vol 189
\yr 1990
\pages 127--134
\endref

\ref
\no 13
\by R. Terras
\paper A Miller algorithm for an incomplete Bessel
function
\jour J. Comput. Phys. 
\vol39
\yr1981
\pages 233--240
\endref
\endRefs
%\vskip.15in
%\noindent Olivier Ramar\'e
%\smallskip
%\noindent D\'epartement de Math\'ematiques, \smallskip
%\noindent Universit\'e de Nancy I,
%\smallskip
%\noindent URA 750
%\smallskip
%\noindent 54506 Vandoeuvre CEDEX
%\smallskip FRANCE
%\bigskip%%

%\noindent Robert Rumely
%\smallskip
%\bheadversity of Georgia,
%\smallsk\endsubhead
%\noindent Athens, Georgia 30602\smallskip US%A%%%%

\enddocument
\endinput
25-Jan-95 14:17:14-EST,85066;000000000000
Return-path: <rr@alpha.math.uga.edu>
Received: from dns1.uga.edu by MATH.AMS.ORG (PMDF #7286 ) id
 <01HM9LNKWS80GC3MWR@MATH.AMS.ORG>; Wed, 25 Jan 1995 14:16:53 EST
Received: from alpha.math.uga.edu by dns1.uga.edu with SMTP id AA07592
 (5.67b/IDA-1.5 for <pub-submit@math.ams.org>); Wed, 25 Jan 1995 14:16:23 -0500
Received: by alpha.math.uga.edu id AA00207 (5.67b/IDA-1.5 for
 pub-submit@math.ams.org); Wed, 25 Jan 1995 14:15:22 -0500
Date: 25 Jan 1995 14:15:22 -0500
From: Robert Rumely <rr@alpha.math.uga.edu>
Subject: AMS-TEX paper for Mathematics of Computation
To: pub-submit@MATH.AMS.ORG
Message-id: <199501251915.AA00207@alpha.math.uga.edu>
Content-transfer-encoding: 7BIT
Content-Length: 82188

Dear Sirs,
      Below is an AMS-TEX file containing the paper "Primes in Arithmetic
Progressions", by Olivier Ramare and Robert Rumely, which has been accepted
for publication by Mathematics of Computation.  The reference number is 
P-6806.  

Sincerely,
Robert Rumely




