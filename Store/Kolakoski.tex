\def\widebar{\overline}
\def\order{\asymp}
\def\lcm{\text{lcm}}
\def\equivalent{\Leftrightarrow}
\def\Log{\operatorname{Log}}
\def\fin{$\diamond\diamond\diamond$\enddemo}
\def\dif{\partial}
\def\la{\Lambda}
\def\RamI{[R]}
\def\RamII{[RR]}
\magnification=\magstep1
\input amstex
\documentstyle {amsppt}
\NoBlackBoxes


\def\section#1{\goodbreak\noindent{\bf #1}}
\topmatter
\title
La suite de Kolakoski
\endtitle
\author
Olivier Ramar\'e
\endauthor
\abstract
Suite  A000002 chez Sloane.
\endabstract
\endtopmatter

\document
Les premiers termes~:
$$
\aligned
&   1,2,2,1,1,2,1,2,2,1,2,2,1,1,2,1,1,2,2,1,2,1,1,2,1,2,2,1,1,2,
\\& 1,1,2,1,2,2,1,2,2,1,1,2,1,2,2,1,2,1,1,2,1,1,2,2,1,2,2,1,1,2,
\\& 1,2,2,1,2,2,1,1,2,1,1,2,1,2,2,1,2,1,1,2,2,1,2,2,1,1,2,1,2,2,
\\& 1,2,2,1,1,2,1,1,2,2,1,2,1,1,2,1,2,2
\endaligned
$$

Le programme qui les g\'en\`ere~:
{\tt
\obeylines

a=[1,2,2];
for(n=3,80,
\quad  for(i=1,a[n],
\qquad    a=concat(a,1+((n-1)\%2))));
a
}


\def\tvi{\vrule height 9.5pt depth 3.5pt width 0pt}
\def\tv{\tvi\vrule}
\def\oh{\omit\hrulefill}
\def\oo{\omit}
\def\cc#1{\hbox to 12 pt{\hfill\kern2pt#1}}
\let\ve\varepsilon

La suite~:
$$
\vbox{\offinterlineskip
\halign{%
& #   & \cc{#} & # & \cc{#} & # & \cc{#} & # & \cc{#}
& #   & \cc{#} & # & \cc{#} & # & \cc{#} & # & \cc{#}
& #   & \cc{#} & # & \cc{#} & # & \cc{#} & # & \cc{#}
& #   & \cc{#} & # & \cc{#} & # & \cc{#} & # & \cc{#}
& #   & \cc{#} & # & \cc{#} & # & \cc{#} & # & \cc{#}
& #   & \cc{#} & # & \cc{#} & # & \cc{#} & # & \cc{#}
& #   & \cc{#} & # & \cc{#} & # & \cc{#} & # & \cc{#}
& #   & \cc{#} & # & \cc{#} & # & \cc{#} & # & \cc{#}
& #   & \cc{#} & # & \cc{#} & # & \cc{#} & # & \cc{#} & #\cr
\tv &  1  & \tv &  2  & \tv &  3  & \tv &  4  & 
\tv &  5  & \tv &  6  & \tv &  7  & \tv &  8  & 
\tv &  9  & \tv & 10  & \tv & 11  & \tv & 12  & 
\tv & 13  & \tv & 14  & \tv & 15  & \tv & 16  & 
\tv & 17  & \tv & 18  & \tv & 19  & \tv & 20  & 
\tv & 21  & \tv & 22  & \tv & 23  & \tv & 24  & \tv \cr
\oh & \oh & \oh & \oh & \oh & \oh & \oh & \oh &
\oh & \oh & \oh & \oh & \oh & \oh & \oh & \oh &
\oh & \oh & \oh & \oh & \oh & \oh & \oh & \oh &
\oh & \oh & \oh & \oh & \oh & \oh & \oh & \oh &
\oh & \oh & \oh & \oh & \oh & \oh & \oh & \oh &
\oh & \oh & \oh & \oh & \oh & \oh & \oh & \oh & \oh \cr
\tv &  1  & \tv &  2  & \tv &  2  & \tv &  3  & 
\tv &  3  & \tv &  4  & \tv &  5  & \tv &  6  & 
\tv &  6  & \tv &  7  & \tv &  8  & \tv &  8  & 
\tv &  9  & \tv &  9  & \tv & 10  & \tv & 11  & 
\tv & 11  & \tv & 12  & \tv & 12  & \tv & 13  & 
\tv & 14  & \tv & 15  & \tv & 15  & \tv & 16  & \tv \cr
\oh & \oh & \oh & \oh & \oh & \oh & \oh & \oh &
\oh & \oh & \oh & \oh & \oh & \oh & \oh & \oh &
\oh & \oh & \oh & \oh & \oh & \oh & \oh & \oh &
\oh & \oh & \oh & \oh & \oh & \oh & \oh & \oh &
\oh & \oh & \oh & \oh & \oh & \oh & \oh & \oh &
\oh & \oh & \oh & \oh & \oh & \oh & \oh & \oh & \oh \cr
\tv &  1  & \tv &  2  & \tv &  2  & \tv &  1  & 
\tv &  1  & \tv &  2  & \tv &  1  & \tv &  2  & 
\tv &  2  & \tv &  1  & \tv &  2  & \tv &  2  & 
\tv &  1  & \tv &  1  & \tv &  2  & \tv &  1  & 
\tv &  1  & \tv &  2  & \tv &  2  & \tv &  1  & 
\tv &  2  & \tv &  1  & \tv &  1  & \tv &  2  & \tv \cr
}}
$$

Bref
$$
\ve(F(n))=
\#\{m/F(m)=n\}=
\cases
1&\text{\ si $F(n)$ est impair}\\
2&\text{\ si $F(n)$ est pair}\\
\endcases
$$
Posons
$$
G(\ell)=\max\{n/F(n)=\ell\}.
$$
Il vient
$$
m+1 \ge G(F(m))\ge m\qquad,\qquad
F(G(m))=m.
$$
On a aussi
$$
G(\ell)=\sum_{n\le\ell}\ve(F(n))
$$

\vfill\eject

R\'ef\'erences~:

W. Kolakoski, Problem 5304, Amer. Math. Monthly, 73 (1966), 681-682.

I. Vardi, Computational Recreations in Mathematica.
           Addison-Wesley, Redwood City, CA, 1991, p. 233.

J. C. Lagarias, Number Theory and Dynamical Systems, pp. 35-72 of S. A.
           Burr, ed., The Unreasonable Effectiveness of Number Theory, Proc.
           Sympos. Appl. Math., 46 (1992). Amer. Math. Soc.

\enddocument

%%% Local Variables: 
%%% mode: plain-tex
%%% TeX-master: t
%%% TeX-master: t
%%% End: 
