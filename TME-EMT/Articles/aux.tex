\documentclass[12pt,a4paper,twoside]{article}
\usepackage{amsfonts} %\usepackage{eufrak} included
\usepackage{amssymb}
\usepackage{amsmath}
\usepackage{stmaryrd}

\usepackage{latexsym}%utilisation des symboles LaTeX pour avoir un beau LaTeX
\usepackage{enumerate}
\usepackage{multicol}
%\usepackage[active]{srcltx}

\usepackage{authordate1-4}
\usepackage[french]{babel}
%%%%%%\usepackage{epic}
\usepackage{epsfig}
%\usepackage{eepic}
%\usepackage[dvips]{graphics}
%\usepackage{floatflt}
%\setlength{\hoffset}{-0.6cm}
%\setlength{\oddsidemargin}{0cm}
%\setlength{\textwidth}{16.8cm}
%\setlength{\marginparsep}{0cm}
%\setlength{\marginparwidth}{0.5cm}

%\usepackage[active,displaymath,textmath,sections,auctex]{preview}
%\setlength{\voffset}{0cm}
%\setlength{\topmargin}{0cm}
%\setlength{\headheight}{0cm}
%\setlength{\headsep}{0cm}
%\setlength{\textheight}{25.4cm}
%\setlength{\topskip}{0cm}
%\setlength{\footskip}{1.2cm}
%\pagestyle{empty}

\renewcommand{\sectionmark}[1]{\markright{\thesection\ #1}}%rappel du titre de la section


\newtheorem{thm}{Theorem}[section]
\newtheorem{lem}{Lemma}[section]
\newtheorem{cor}{Corollary}[section]
\newtheorem{defi}{Definition}[section]

\newenvironment{dem}{\medbreak\noindent{\sc Preuve : }}%
{\hfill $\diamond\diamond\diamond$\par\medbreak}

\let\ve=\varepsilon
\def\Acal{\mathcal{A}}
\def\Bcal{\mathcal{B}}
\def\Gcal{\mathcal{G}}
\def\Hcal{\mathcal{H}}
\def\Kcal{\mathcal{K}}
\def\Ocal{\mathcal{O}}
\def\Pcal{\mathcal{P}}
\def\Ucal{\mathcal{U}}
\let\vp=\varphi
\let\about=\simeq
\def\goes{\mathrel{\rightarrow}}
\def\order{\asymp}%{\smile\atop\frown}
\DeclareMathOperator{\Log}{Log}
\DeclareMathOperator{\sgn}{sgn}
\def\1{1\!\!1}


%%% Enlever le commentaire pour avoir un espacement double :
\usepackage{setspace}
%\setstretch{1.5}   % version de travail
%\setstretch{2.0}  % version correcteur

\title{Moyennes de fonctions multiplicatives positives~: la m\'ethode de convolution}
\author{O. Ramar\'e}
\date{\sl Le 2 octobre 2004}
 
\usepackage{verbatim}

\begin{document}

\maketitle
\section{I. Premiers pas.}

Soit $f:\Bbb N\setminus\{0\}\goes\Bbb C$ une fonction multiplicative,
i.e. telle que
$$
\left\{
\aligned
&f(1)=1,\\
&f(mn)=f(m)f(n)\qquad\text{si $n$ et $m$ sont premiers entre eux.}
\endaligned
\right.
$$
Une telle fonction est d\'efinie par ses valeurs sur les puissances de
nombres premiers et, formellement, nous avons
$$
D(f,s)=\sum_{n\ge1}\frac{f(n)}{n^s}
=\prod_{p\ge2}\big(\sum_{k\ge0}\frac{f(p^k)}{p^{ks}}\big).
$$
Souvent, la fonction $f$ est assez ``proche'' d'une fonction
connue, et c'est cette id\'ee que nous mettons ici en pratique.

\noindent{\underbar{Exemples :}}

\begin{enumerate}
\item $f_1(n)=\prod_{p|n}(p-2)$. Il vient
$$
\aligned
D(f_1,s)
&=\prod_{p\ge2}\big(1+\frac{p-2}{p^s-1}\big)
=\prod_{p\ge2}\bigg(1-\frac{2p^{s-1}+p-3}{(p^s-1)p^{s-1}}\bigg)\frac{1}{1-1/p^{s-1}}
\\&=C_1(s)\zeta(s-1)
\endaligned
$$
o\`u $C_1(s)$ est holomorphe pour $\Re s>\tfrac32$. Cette \'ecriture
montre que $D(f_1,s)$ est m\'eromorphe pour $\Re s>\tfrac32$ et admet
un p\^ole simple en $s=2$.

\item  $f_2(n)=\mu^2(n)/\phi(n)$. Il vient
$$
\aligned
D(f_2,s)
&=\prod_{p\ge2}\big(1+\frac{1}{(p-1)p^s}\big)
\\&=\prod_{p\ge2}\bigg(1-\frac{1}{(p-1)p^{s+1}}-\frac{1}{(p-1)p^{2s+1}}\bigg)
\frac{1}{1-1/p^{s+1}}
\\&=C_2(s)\zeta(s+1)
\endaligned
$$
o\`u $C_2(s)$ est holomorphe pour $\Re s>-\tfrac12$. Cette \'ecriture
montre que $D(f_2,s)$ est m\'eromorphe pour $\Re s>-\tfrac12$ et admet
un p\^ole simple en $s=0$.

\item  $f_3(n)=2^{\Omega(n)}$. Il vient
$$
\aligned
D(f_3,s)
&=\prod_{p\ge2}\frac{1}{1-\frac{2}{p^s}}
=\prod_{p\ge2}\big(1+\frac{1}{p^{2s}-2p^s}\big)
\zeta^2(s)
\\&=C_3(s)\zeta^2(s)
\endaligned
$$
o\`u $C_3(s)$ est holomorphe pour $\Re s>\tfrac12$. Cette \'ecriture
montre que $D(f_3,s)$ est m\'eromorphe pour $\Re s>\tfrac12$ et admet
un p\^ole double en $s=1$.
\end{enumerate}

Nous d\'eveloppons alors les $C_i$ en s\'eries de Dirichlet ~:
$$
C_i(s)=\sum_{n\ge1}\frac{g_i(n)}{n^s}
$$
o\`u les fonctions $g_i$ sont bien s\^ur multiplicatives. Pour obtenir
leurs valeurs exactes, il suffit d'identifier les coefficients dans le
d\'eveloppement du facteur local en s\'erie de $p^-s$. Nous obtenons
ainsi
$$
\left\{
\aligned
g_1(p)&=-2\\
g_1(p^k)&=-(p^2-3p+2), \quad(k\ge2)
\endaligned
\right.
\qquad
\left\{
\aligned
g_2(p)&=g_2(p^2)=-\frac{1}{p(p-1)}\\
g_2(p^k)&=0, \quad(k\ge3)
\endaligned
\right.
$$

ainsi que
$$
\left\{
\aligned
g_3(p)&=0\\
g_3(p^k)&=2^{k-2}, \quad(k\ge2)
\endaligned
\right.
$$
Nous posons aussi
$$
\overline{C_i}(s)=\sum_n\frac{|g_i(n)|}{n^{s}}
$$
et il se trouve que ces s\'eries convergent encore l\`a o\`u nous
avons montr\'e que $C_i$ existait, c'est \`a dire respectivement pour
$\Re s\ge\tfrac32$, $\Re s\ge-\tfrac12$ et $\Re s\ge\tfrac12$.

\bigskip
Occupons-nous \`a pr\'esent des ordres moyens. La traduction sur les
coefficients de $D(f_1,s)=C_1(s)\zeta(s-1)$ nous donne
$$
f_1(n)=\sum_{\ell m=n}g_1(\ell)m
$$
et par cons\'equent
$$
\aligned
\sum_{n\le X}f_1(n)
&=\sum_{\ell m\le X}g_1(\ell)m
=\sum_{\ell\le X}g_1(\ell)
\bigg(\frac12\bigg(\frac X\ell\bigg)^2
+\Cal O\bigg(\frac X\ell\bigg)\bigg)
\\&=\frac{X^2}{2}
\sum_{\ell\le X}\frac{g_1(\ell)}{\ell^2}
+\Cal O\bigg(X \sum_{\ell\le X}\frac{|g_1(\ell)|}{\ell}\bigg)
\endaligned
$$
Nous utilisons alors
\let\ve=\varepsilon
$$
\aligned
\sum_{\ell\le X}\frac{g_1(\ell)}{\ell^2}
&=
\sum_{\ell\ge 1}\frac{g_1(\ell)}{\ell^2}
+\Cal O\bigg(
\sum_{\ell> X}\frac{|g_1(\ell)|}{\ell^2}
\bigg)
\\&=C_1(2)+\Cal O\bigg(
\sum_{\ell> X}\frac{|g_1(\ell)|}{\ell^{\frac32+\ve}}\frac{1}{X^{\frac12-\ve}}
\bigg)
\\&=C_1(2)+\Cal O_\ve\big(X^{-\frac12+\ve}\big)
\endaligned
$$
pour tout $\ve>0$,
et
$$
\sum_{\ell\le X}\frac{|g_1(\ell)|}{\ell}
\le
X^{\frac12+\ve}\sum_{\ell\le X}\frac{|g_1(\ell)|}{\ell^{\frac32+\ve}}
\ll_\ve X^{\frac12+\ve}\qquad(\ve>0).
$$
Bref
$$
\sum_{n\le X}f_1(n)=C_1(2)\frac{X^2}{2}+\Cal O_\ve\big(X^{\frac12+\ve}\big).
$$

\bigskip
Pour ce qui est de l'ordre moyen de $f_2$, nous proc\'edons comme
ci-dessus. Il vient
$$
\aligned
\sum_{n\le X}f_2(n)
&=\sum_{\ell m\le X}g_2(\ell)\frac{1}m
=\sum_{\ell\le X}g_2(\ell)
\bigg(\Log\frac X\ell+\gamma
+\Cal O\bigg(\frac \ell X\bigg)\bigg)
\\&=
\big(\Log X +\gamma\big)\sum_{\ell\le X}g_2(\ell)
-\sum_{\ell\le X}g_2(\ell)\Log\ell
+\Cal O\bigg(\frac1X\sum_{\ell\le X}|g_2(\ell)|\ell\bigg)
\endaligned
$$
o\`u $\gamma$ est la constante d'Euler. 
Le programme pr\'ec\'edent s'applique moyennant de rappeler que
$$
-\sum_{\ell\ge1}\frac{g_2(\ell)\Log\ell}{\ell^s}
\quad\left( \text{resp.}\ 
-\sum_{\ell\ge1}\frac{|g_2(\ell)|\Log\ell}{\ell^s} \right)
$$
est simplement la d\'eriv\'ee de $C_2(s)$ (resp. $\overline{C_2}(s)$)
et que cette s\'erie admet la m\^eme abscisse de convergence absolue
que la s\'erie initiale. Nous obtenons alors
$$
\sum_{n\le X}f_2(n)=
\big(\Log X +\gamma\big)\big(C_2(0)+\Cal O_\ve(X^{-\tfrac12+\ve})\big)
+C'_1(0)+\Cal O_\ve(X^{-\tfrac12+\ve})+\Cal O_\ve(X^{-\tfrac12+\ve}).
$$

\bigskip
Il nous reste \`a nous occuper de $C_3(s)$. Nous avons cette fois-ci
$$
\sum_{n\le X}f_2(n)
=\sum_{\ell m\le X}g_2(\ell)d(m)
$$
o\`u $d(m)$ est le nombre de diviseurs de $m$. Nous avons de fa\c con
classique ~:
$$
\sum_{m\le M}d(m)=M\Log M+(2\gamma-1)M+\Cal O(M^{1/2})
$$
et donc
$$
\aligned
\sum_{n\le X}f_2(n)
&=
\endaligned
$$

\section{Un th\'eor\`eme g\'en\'eral.}

Le lemme suivant est une g\'en\'eralisation d'un lemme de Riesel \&
Vaughan (le papier est cit\'e ci-dessous).
\def\O{\Cal O^*}

\proclaim{Lemme 1}
Soit $g$, $h$ et $k$ trois fonctions sur $\Bbb N\setminus\{0\}$ \`a
valeurs complexes.
Posons $H(s)=\sum_nh(n)n^{-s}$, et
$\overline{H}(s)=\sum_n|h(n)|n^{-s}$.
Supposons que $g=h\star k$, que $\overline{H}(s)$ soit convergente pour
$\Re(s)\ge-1/3$ et enfin qu'il existe quatre constantes
$A$, $B$, $C$ et $D$ telles que
$$
\sum_{n\le t}k(n)
=
A\Log^2t+B\Log t+C+\O(D t^{-1/3})
\text{\ \ pour\ \ }
t>0;
$$
Alors, pour tout $t>0$, nous avons :
$$
\sum_{n\le t}g(n)
=
u\Log^2t+v\Log t+w+\O(D t^{-1/3}\overline{H}(-1/3))
$$
avec
$u=AH(0)$, $v=2AH^{\prime}(0)+BH(0)$ and $w=AH^{\prime\prime}(0)+BH^{\prime}(0)+CH(0)$.
Nous avons aussi
$$
\sum_{n\le t}ng(n)
=
Ut\Log t+Vt+W+\O(2.5D t^{2/3}\overline{H}(-1/3))
$$
avec
$$
\left\{\aligned
U=&AH(0),\text{\ \ \ \ \ \ }V=-2AH(0)+2AH^{\prime}(0)+BH(0),\\
W=&A(H^{\prime\prime}(0)-2H^{\prime}(0)+2H(0))+B(H^{\prime}(0)-H(0))+CH(0).
\endaligned\right.
$$
\endproclaim

\demo{Preuve}
\'Ecrivons
$\sum_{\ell\le t}g({\ell})=\sum_{m}h(m)\sum_{n\le t/m}k(n)$, et toute
la r\'egularit\'e de nos expresions vient de ce qu'il n'est pas
n\'ecessaire d'imposer $m\le t$ dans $\sum_{m}h(m)$.
Nous compl\'etons alors la preuve facilement

Pour estimer $\sum_{\ell\le t}\ell g({\ell})$ for $t>0$, nous \'ecrivons
$$
\sum_{\ell\le t}\ell g({\ell})=
t\sum_{\ell\le t}g({\ell})-
\int_1^t \sum_{\ell\le u}g({\ell})du,
$$
et utilisons l'expression asymptotique de $\sum_{\ell\le u}g({\ell})$.
\fin


Pour appliquer le lemme pr\'ec\'edent, nous aurons besoin de

\proclaim{Lemme 2} Pour tout $t>0$, nous avons
$$
\sum_{n\le t}\frac 1n=\Log t+\gamma+\O\big(0.9105 t^{-1/3}\big).
$$
Soit $d(n)$ le nombre de diviseurs de $n$. Pour tout $t>0$, nous avons
$$
\sum_{n\le t}\frac{d(n)}n=
\frac12\Log^2t+2\gamma\Log t+\gamma^2-\gamma_1+\O\big(1.641 t^{-1/3}\big),
$$
avec
$$
\gamma_1=\lim_{n\rightarrow\infty}
\left(\shave{\sum_{m\le n}}\frac{\Log m}m-\frac{\Log^2n}2
\right).
$$
$(-0.072816<\gamma_1<-0.072815)$.
\endproclaim
\goodbreak
\demo{Preuve}
La preuve de la seconde partie de ce lemme se trouve dans le papier de
in Riesel \& Vaughan cit\'e ci-dessous
(Lemma 1).

Pour la premi\`ere partie, rappelons que
$$
|\sum_{n\le t}\frac1n-\Log t-\gamma|\le\frac{7}{12t}
\text{\ \ pour\ \ }t\ge1.
$$
Pour $0<t<1$, nous choisissons $a>0$ tel que $\Log t+\gamma+a\ t^{-1/3}\ge0$. Cette
fonction d\'ecro\^\i t de $0$ \`a $(a/3)^3$ et ensuite cro\^\i t. Cela
nous donne la valeur minimale $a=3\exp(-\gamma/3-1)\le0.9105$.
\fin

Dans la pratique, la fonction $g$ sera multiplicative et v\'erifiera
$g_p=b/p+o(1/p)$ avec $b=1$ or 2. Dans ce cas, nous prenons $\sum
k(n)n^{-s}=\zeta(s+1)^b$ et $h$ est la fonction multiplicative
d\'etermin\'ee par $\sum h(n)n^{-s}=\sum g(n)n^{-s}\zeta(s+1)^{-b}$.

Lorsque $h$ est multiplicative, nous avons
$$
H(0)=\prod_p(1+\sum_mh({p^m})),
$$
et
$$
\frac{H^{\prime}(0)}{H(0)}=
\sum_p
\frac{\sum_mmh({p^m})}{1+\sum_mh({p^m})}(-\Log p),
$$
ainsi que
$$
\frac{H^{\prime\prime}(0)}{H(0)}=
\bigg(
\frac{H^{\prime}(0)}{H(0)}
\bigg)^2+
\sum_p
\bigg\{
\frac{\sum_mm^2h({p^m})}{1+\sum_mh({p^m})}
-\bigg(\frac{\sum_mmh({p^m})}{1+\sum_mh({p^m})}\bigg)^2
\bigg\}
\Log^2p.
$$

Un exemple.

\proclaim{Lemma 3} Pour tout $X>0$ et tout entier $d\ge1$, nous avons
$$
\sum\Sb n\le X\\ (n,d)=1\endSb
\frac{\mu^2(n)}{\phi(n)}=
\frac{\phi(d)}d
\bigg\{\!
\Log X+
\gamma+\sum_{p\ge2}\frac{\Log p}{p(p-1)}
+\sum_{p|d}\frac{\Log p}{p}
\bigg\}
+\O(
7.284 X^{-1/3}f_1(d))
$$
avec
$$
f_1(d)=
\prod_{p|d}
(1+p^{-2/3})
\big(1+\frac{p^{1/3}+p^{2/3}}{p(p-1)}\big)^{-1}.
$$
\endproclaim

\underbar{Remarque} :
La somme de gauche est $G_d(X)$. Le cas $d=1$ a d\'ej\`a \'et\'e
\'etudi\'e plus haut. La s\'erie de Dirichlet associ\'ee est
$$
\sum_{n}\frac{\mu^2(n)}{\phi(n)n^{s-1}}=
\frac{\zeta(s)}{\zeta(2s)}
\prod_{p\ge2}
\bigg(
1+\frac1{(p-1)(p^s+1)}
\bigg)
$$
ce qui fait que le terme d'erreur $\Cal O(X^{-1/2})$ est admissible
(notre m\'ethode pourrait donner $\Cal O(X^{-1/2}\Log^2X)$), et que
nous ne pouvons esp\'erer mieux que $\Cal O(X^{-3/4})$.
\bigskip

Rosser \& Schoenfeld (voir ci-dessous, \'equation (2.11) du papier
cit\'e) nous donne
$$
\gamma+\sum_{p\ge2}\frac{\Log p}{p(p-1)}=
1.332\ 582\ 275\ 332\ 21...
$$
\demo{Preuve}
D\'efinissons la fonction multiplicative $h_d$ par
$$
h_d(p)=\frac1{p(p-1)},
\quad h_d(p^2)=\frac{-1}{p(p-1)},
\quad h_d(p^m)=0
\quad \text{si\ \ }m\ge3,
$$
si $p$ est un nombre premier qui ne divise pas $d$,
et par $h_d(p^m)=\frac{\mu(p^m)}{p^m}$ pour tout
$m\ge1$ si $p$ est un facteur premier de $d$.

Nous avons alors
$$
\sum\Sb n\ge1\endSb\frac{h_d(n)}{n^s}\zeta(s+1)=
\sum\Sb n\ge1\\ (n,d)=1\endSb\frac{\mu^2(n)}{\phi(n)n^s}
$$
ce qui nous permet d'appliquer le lemme 2. Nous v\'erifions que
$$
\prod_{p\ge2}
\bigg(1+\frac{p^{1/3}+p^{2/3}}{p(p-1)}\bigg)
\le8.
$$
\fin

\proclaim{Lemma 4}
\roster
\item Pour $z\ge1$, nous avons $G(z)\le\Log z+1.4709$.
\item Pour $z\ge6$, nous avons $1.07+\Log z\le G(z)$.
\item Pour $z\ge\exp(18)$ et $\alpha\ge1.1$, nous avons $G(z^\alpha)\le\alpha G(z)$.
\endroster
\endproclaim

\demo{Preuve}
La premi\`ere partie provient de notre expression asymptotique si $z\ge146\ 050$.
Finir par des calculs serait assez co\^uteux en temps, aussi modifions
nous le lemme~2 et prenons l'exposant $0.45$ au lieu de $1/3$. Nous
avons alors $G(z)-\Log z\le1.4708$ d\`es que
$$
\frac1{0.45}\Log\big(\overline{H}(-0.45)
\frac{\exp(-1-0.45\gamma)}{0.45(1.4708-1.332583)}\big)\le\Log z.
$$
Il nous faut calculer $\overline{H}(-0.45)$, et pour ce faire, un
contr\^ole du terme d'erreur est n\'ecessaire.
Nous avons
$$
\prod_{2\le p\le200\ 000}(1+\frac{p^{0.45}+p^{0.9}}{p(p-1)})\le20.26
$$
et, avec $F(t)=(t^{0.45}+t^{0.9})/[t(t-1)\Log t]$ et $X=200\ 000$,
nous obtenons
$$
\Log\prod_{p>X}(1+\frac{p^{0.45}+p^{0.9}}{p(p-1)})\le
1.001\ 093\big(XF(X)+\int_X^{\infty}F(t)dt\big)\le0.266\ 47
$$
en utilisant $\theta(t)\le1.001\ 093t$ si $t>0$ (cf l'article de Schoenfeld).
D'o\`u l'in\'egalit\'e voulue si $z\ge42\ 300$.
Un calcul direct donne alors
$$
\max_{z\ge1}(G(z)-\Log z)=G(7)-\Log 7\le1.4709.
$$
La seconde in\'egalit\'e est d\^ue \`a Montgomery \& Vaughan
(Lemma 7 du papier cit\'e ci-dessous). La troisi\`eme assertion est
une cons\'equence des deux premi\`eres,
\fin



\subsubsection{Estimations de fonctions classiques}

\begin{lem}
  Nous avons pour $X\ge1$
  \begin{equation*}
    \left\{\begin{array}{l}
        \displaystyle
        \sum_{n\le X}1=X+\tfrac12+\Ocal^*(\tfrac12) \\[1em]
         \displaystyle
       \sum_{n\le X}1/n=\Log X+\gamma+\Ocal^*(7/(12X)) \\
        \displaystyle
        \sum_{n\le X}1/\sqrt{n}=2\sqrt{X}+c_{12}+\Ocal^*(3/(2\sqrt{X})) 
      \end{array}
      \right.
    \end{equation*}
  o\`u $c_{12}=0.539\,645\,49\cdots$ est d\'efinie en~\eqref{defc12} ci-dessous.
  \begin{equation*}
    \sum_{n\le X}\Log m/m=\tfrac12\Log^2X+\gamma_1+\Ocal^*(\Log X/X) \quad(X\ge 1.7)
  \end{equation*}
  $\gamma_1=-0.072\,815\,845\,483\,676\,724\,860\,586\,375\,874\,901\,3\cdots$
\end{lem}%gamma1=-0.0728158454836767248605863758749013
La valeur de $\gamma_1$ est tir\'ee de \cite{Keiper*92}.
\begin{dem}
  Les deux premi\`eres estimations sont classiques. Pour la troisi\`eme, nous
  \'ecrivons
  \begin{eqnarray*}
    \sum_{n\le X}1/\sqrt{n}
    &=&\int_1^X\sum_{n\le t}1\frac{dt}{2t^{3/2}}+\sum_{n\le X}1/\sqrt{X}
    \\&=&2\sqrt{X}-\tfrac12+
    \int_1^\infty\bigl([t]-t+\tfrac12\bigr)\frac{dt}{2t^{3/2}}
    +\Ocal^*(1/\sqrt{X})+\tfrac12/\sqrt{X}.
  \end{eqnarray*}
  Pour expliciter un peu plus la constante $c_{12}$, nous remarquons que
  $t\mapsto [t]-t+\tfrac12$ est p\'eriodique de p\'eriode~1 et obtenons
  \begin{equation}\label{defc12}
    c_{12}=\tfrac12+\sum_{n\ge1}
    \frac1{4\sqrt{n(n+1)}(\sqrt{n}+\sqrt{n+1})(n+\tfrac12+\sqrt{n(n+1)})}.
   \end{equation}
   Pour la quatri\`eme estimations, nous \'ecrivons pour $X\ge e$
   \begin{eqnarray*}
     \sum_{n\le X}\Log m/m
     &=&\int_1^X[t]\frac{\Log t -1}{t^2}dt
     +\frac{(X+\tfrac12+\Ocal^*(\tfrac12))\Log X}{X}
     \\&=&
     \tfrac12\Log^2X+\gamma_1
     +\Ocal^*(\Log X/X).
   \end{eqnarray*}
   Nous \'etendons cette estimation \`a $X\ge1.7$ par calculs directs.
\end{dem}

\begin{lem}
  Pour $X\ge1$
  \begin{equation*}
    \sum_{n\le X}\tau_2(n)=X\Log X+(2\gamma-1) X+\Ocal^*(1.51\sqrt{X})
  \end{equation*}
  Pour $X\ge1$
  \begin{equation*}
    \sum_{n\le X}\tau_2(n)/n=\tfrac12\Log^2 X+2\gamma \Log X+
    \gamma^2-2\gamma_1+\Ocal^*(4.53/\sqrt{X}).
  \end{equation*}
\end{lem}

Il faut signaler ici une erreur dans le lemme3.3 de \cite{Ramare*95}, o\`u la
constante $\gamma^2-\gamma_1$ au lieu de $\gamma^2-2\gamma_1$.

\begin{dem}
  Nous utilisons la m\'ethode de l'hyperbole de Dirichlet.
  \begin{eqnarray*}
    \sum_{n\le X}\tau_2(n)
    &=&2\sum_{m\le \sqrt{X}}(X/m+\tfrac12+\Ocal^*(\tfrac12))
    -\bigl(\sqrt{X}+\tfrac12+\Ocal^*(\tfrac12)\bigr)^2
    \\&=&
    X\Log X+2\gamma X+\Ocal^*(7/6)+\sqrt{X}+\Ocal^*(1+\tfrac12\sqrt{X})
    \\&&-X-(1+\Ocal^*(1))\sqrt{X}+\tfrac14+\Ocal^*(3/4)
    \\&=&
    X\Log X+(2\gamma-1) X+\Ocal^*(\tfrac{19}{6}+\tfrac32\sqrt{X})
  \end{eqnarray*}
  ce qui donne notre estimation pour $X\ge32$ que nous \'etendons par un
  calcul direct. Pour l'autre, nous sommons la premi\`ere par parties~:
  \begin{eqnarray*}
    \sum_{n\le X}\tau_2(n)/n
    &=&\int_1^X\sum_{n\le t}\tau_2(n)\frac{dt}{t^2}+{\sum_{n\le X}\tau_2(n)}/{X}
    \\&=&\int_1^X(t\Log t+(2\gamma-1)t)\frac{dt}{t^2}
    +\int_1^\infty(\sum_{n\le t}\tau_2(n)- t\Log t-(2\gamma-1)t)\frac{dt}{t^2}
    \\&&+\int_X^\infty\Ocal^*(1.51)\frac{dt}{t^{3/2}}
    +{\sum_{n\le X}\tau_2(n)}/{X}
    \\&=&\tfrac12\Log^2X+2\gamma\Log X
    +c+
    \Ocal^*(4.53/\sqrt{X})
  \end{eqnarray*}
\end{dem}

\begin{lem}
  Pour $X\ge2$, nous avons
  \begin{equation*}
    \sum_{n\le X}\tau_3(n)=\frac X2\Log^2 X
    +(3\gamma-1)X\Log X +(3\gamma^2-3\gamma+1 -2\gamma_1)X+
    \Ocal^*(8.5 X^{2/3}\Log^{1/3}X)
  \end{equation*}
  Pour $X\ge0$, nous avons
  \begin{equation*}
    \sum_{n\le X}\frac{\tau_3(n)}n=
    \tfrac16\Log^3 X+ \tfrac{3\gamma}2\Log^2 X + 3(\gamma^2-\gamma_1 )\Log X
   +\gamma^3  - 6\gamma_1\gamma+ \tfrac32\gamma_2+\Ocal^*(6.2X^{-0.28}).
  \end{equation*}
  $\gamma_2=-0.009\,690\,363\,192\,872\,318\,484\,530\,386\,035\cdots$
\end{lem}%gamma2=-0.009690363192872318484530386035
O\`u la valeur de $\gamma_2$ est encore tir\'ee de \cite{Keiper*92}.
\begin{dem}
  Nous utilisons le principe de l'hyperbole de Dirichlet.
  \begin{multline*}
    S(X)=\sum_{n\le X}\tau_3(n)
    =
    \sum_{m\le M}\bigg(\frac{X}{m}\Log\frac{X}{m}
    +(2\gamma-1) \frac{X}m+\Ocal^*(1.51\sqrt{X/m})\bigg)
    \\+\sum_{\ell\le L}\tau_2(\ell)(X/\ell+\tfrac12+\Ocal^*(\tfrac12))
    -\sum_{m\le M}1\sum_{\ell\le L}\tau_2(\ell).
  \end{multline*}
  Ce qui nous donne
  \begin{multline*}
    S(X)=
    (X\Log X+(2\gamma-1)X)\bigl(\Log M+\gamma+\Ocal^*(7/(12X))\bigr)
    \\-X(\tfrac12\Log^2M+\gamma_1+\Ocal^*(\Log M /M))
    \\+\Ocal^*(1.51\sqrt{X}\bigl(2\sqrt{M}+c_{12}+3/(2\sqrt{M}\bigr))
    \\+X\bigl(\tfrac12\Log^2 L+2\gamma \Log L+
    \gamma^2-\gamma_1+\Ocal^*(4.53/\sqrt{L})\bigr)
    \\+(\tfrac12+\Ocal^*(\tfrac12))\bigl(L\Log L+(2\gamma-1)
    L+\Ocal^*(1.51\sqrt{L})\bigr)
    \\-(M+\tfrac12+\Ocal^*(\tfrac12))(L\Log L+(2\gamma-1)
    L+\Ocal^*(1.51\sqrt{L}))
  \end{multline*}
  \begin{multline*}
    S(X)=
    X\Log X\Log M+\gamma X\Log X+\Ocal^*(\tfrac7{12}\Log X)
    +(2\gamma-1)X\Log M\\+\gamma(2\gamma-1)X
    +\Ocal^*(\tfrac7{12}(2\gamma-1)+3.02\sqrt{XM}+1.51c_{12}\sqrt{X}+1.51\tfrac32\sqrt{L})
    \\-X\tfrac12\Log^2M-\gamma_1X+\Ocal^*(L\Log M)
    \\+\tfrac12X\Log^2 L+2\gamma X\Log L+
    (\gamma^2-2\gamma_1)X+\Ocal^*(4.53 X/\sqrt{L})
    \\+\tfrac12L\Log L+\frac{2\gamma-1}{2} L
    +\Ocal^*\bigl(\tfrac12L\Log L+\frac{2\gamma-1}{2}L+1.51\sqrt{L}\bigr)
    \\-X\Log L-(2\gamma-1) X+\Ocal^*(1.51 M\sqrt{L})
    -\tfrac12L\Log L-\frac{2\gamma-1}{2}L+\Ocal^*(\tfrac121.51\sqrt{L})
    \\+\Ocal^*(\tfrac12L\Log L+\frac{2\gamma-1}{2}L+\tfrac121.51\sqrt{L}).
  \end{multline*}
  Bon, un peu de nettoyage s'impose~!
    \begin{multline*}
    S(X)=
    \tfrac12X\Log^2 X
    +(3\gamma-1)X\Log X +(3\gamma^2-3\gamma+1 -2\gamma_1)X
    \\+\Ocal^*(9.06 X/\sqrt{L} +L\Log X+(2\gamma-1) L)
    \\+\Ocal^*(\tfrac7{12}(2\gamma-1)+1.51c_{12}\sqrt{X}
    +\tfrac7{12}\Log X+\tfrac721.51\sqrt{L}).
  \end{multline*}
  Nous prenons alors 
  \begin{equation}
    L=(4.53 X/\Log X)^{2/3}
  \end{equation}
  ce qui fait un $\Ocal^*$ de
\begin{verbatim}
   c12=0.53964549;
    {f(X)=3*(4.53)^(2/3)+(2*Euler-1)*(4.53)^(2/3)/log(X)
            +7/12*(2*Euler-1)/(X^(2/3)*(log(X))^(1/3))+15*c12*X^(-1/6)/log(X)^(2/3)
            +7/12/X^(2/3)*(log(X))^(2/3)+7/2*1.51*(4.53)^(1/3)/X^(1/3)/(log(X))^(2/3)}
\end{verbatim}
 mutipli\'e par $X^{2/3}(\Log X)^{1/3}$, soit
 $\Ocal^*(8.5X^{2/3}(\Log X)^{1/3})$ si $X\ge 200\,000$. Par ailleurs, cetreme
 d'erreur est inf\'erieur \`a $\Ocal^*(1.31X^{2/3}(\Log X)^{1/3})$ si
 $2\le X\le 200\,000$ par calculs directs.

 Nous prolongeons
 cette estimation par calcul direct. Ce terme d'erreur est aussi 
 $\Ocal^*(1.5X^{2/3}\Log X)$ si $X\ge 600\,000$ et par calculs directs, nous \'etendons
 la validit\'e de cette estimation \`a $X\ge2$.

 Pour $\tau_3(n)/n$, nous proc\'edons par sommations par parties pour avoir le terme d'erreur,
 mais pour l'expression des constantes, il vaut mieux raisonner en terme de s\'eries de
 Dirichlet. Nous \'ecrivons
 \begin{equation*}
   \zeta(1+s)=\frac1s+\gamma-\gamma_1 s+\tfrac12\gamma_2 s^2 +s^3f(s)
 \end{equation*}
 o\`u $f(s)$ est enti\`ere. La formule de sommation de Mellin nous
 garantit que la moyenne des $\tau_3(n)/n$ est donn\'ee par le
 r\'esidu en 0 de $\zeta(1+s)^3X^s/s$ soit~:
 \begin{equation*}
   \tfrac16\Log^3 X+ \tfrac32\gamma\Log^2 X + 3(\gamma^2-\gamma_1 )\Log X
   +\gamma^3  - 6\gamma_1\gamma+ \tfrac32\gamma_2. 
 \end{equation*}
 Concernant le terme d'erreur, notre somme $S'(X)$ s\'ecrit
 \begin{equation*}
   S'(X)=\int_{1}^X(\sum_{n\le t}\tau_3(n)-EMT)dt+\int_{1}^XEMTdt
   +\frac{\sum_{n\le X}\tau_3(n)}{X}
 \end{equation*}
 ce qui fait un terme d'erreur de
 \begin{equation*}
   8.5\times\Ocal^*\left(\int_1^X(\Log t)^{1/3}\frac{dt}{t^{4/3}}+\frac{(\Log X)^{1/3}}{X^{1/3}}
   \right)
 \end{equation*}
 qu'une int\'egration par parties nous permet de majorer facilement par
 \begin{equation*}
   \Ocal^*\left(8.5\frac{4(\Log X)^{1/3}+3/(\Log X)^{2/3}}{X^{1/3}}\right).
 \end{equation*}
 Il s'agit \`a pr\'esent de simplifier cette expression pour obtenir
 un terme d'erreur en $\Ocal^*(\xi X^{-0.28})$ pour $X>0$. Lorsque $X$
 est entre $0$ et $1$ exclus, il nous faut $\xi\ge6.2$.
 Notre terme ci-dessus est $\le 6.2/X^{0.28}$ si $X\ge8\,000\,000$.
\end{dem}

\subsubsection{Valeur moyenne de fonctions proches de fonctions classiques}

Nous \'etendons le lemme~3.2 de \cite{Ramare}, lui m\^eme \'etendant un lemme
de \cite{Riesel-Vaughan*83}.

\begin{lem}
  Soit $(g_n)_{n\ge1}$, $(h_n)_{n\ge1}$ et $(k_n)_{n\ge1}$ trois suites de
  complexes. Nous consid\'erons $H(s)=\sum_{n\ge1}h_n/n^s$ et
  $\overline{H}(s)=\sum_{n\ge1}|h_n|/n^s$ et nous supposons que cette
  derni\`ere s\'erie est convergente pour $\Re s\ge \sigma$ pour un certain
  $\sigma\in]0,1[$. Nous supposons aussi  que $g=h\star k$ et que $k$ admet une
  valeur moyenne de la forme
  \begin{equation*}
    \sum_{n\le t}k_n=A_{3}\Log^3t+A_{2}\Log^2t+A_{1}\Log t+A_0+\Ocal^*(D
    t^{-\sigma})
    \quad(t>0)
  \end{equation*}
  pour certaines constantes $A_0$, $A_1$, $A_2$, $A_3$ et $D$. Alors
   \begin{equation*}
    \sum_{n\le t}g_n=u_{3}\Log^3t+u_{2}\Log^2t+u_{1}\Log t+u_0+\Ocal^*(D
    t^{-\sigma}\overline{H}(\sigma))
    \quad(t>0)
  \end{equation*}
  avec
  \begin{equation*}
    \left\{
      \begin{array}{r}
        u_3=A_3H(0),\\
        u_2=3A_3H'(0)+A_2H(0),\\
        u_1=3A_3H''(0)+2A_2H'(0)+A_1H(0),\\
        u_0=A_3H'''(0)+A_2H''(0)+A_1H'(0)+A_0H(0).\\
     \end{array}
    \right.
  \end{equation*}
  Nous avons aussi
     \begin{equation*}
    \sum_{n\le t}ng_n=U_{2}t\Log^2t+U_{1}t\Log t+U_{0}t+\Ocal^*(\frac{2-\sigma}{1-\sigma}D
    t^{-\sigma}\overline{H}(\sigma))
    \quad(t>0)
  \end{equation*}
  avec
  \begin{equation*}
    \left\{
      \begin{array}{r}
        U_2=3A_3H(0),\\
        U_1=(-6A_3+2A_2)H(0)+6A_3H'(0),\\
        U_0=(A_3-2A_2+A_1)H(0)+(-6A_3+2A_2)H'(0)+3A_3H''(0)
     \end{array}
    \right.
  \end{equation*}
\end{lem}

\begin{dem}
  Nous \'ecrivons 
  \begin{eqnarray*}
    \sum_{n\le t}g_n
    &=&\sum_{\ell\ge1}h_{\ell}\sum_{n\le t/\ell}k_n
    \\&=&\sum_{\ell\ge1}h_{\ell}\sum_{0\le j\le J}A_j\Log^j(t/\ell)+
    \Ocal^*(D\overline{H}(\sigma)/t^{\sigma}).
  \end{eqnarray*}
  Il nous faut expliciter un peu plus le terme principal~:
  \begin{equation*}
    \sum_{0\le i\le J}\Log^{i}t\sum_{i\le j\le J}\binom{j}{i}A_jH^{(j-i)}(0)
  \end{equation*}
 et l'expression que nous donnons est la sp\'ecialisation en $J=3$. Pour la
 seconde expression, nous avons besoin d'int\'egrer $\Log^it$, ce pourquoi
 nous remarquons que le d\'eriv\'ee de
 \begin{equation*}
   t\sum_{0\le \ell\le i}(-1)^{\ell-i}\frac{i!}{\ell!}\Log^\ell t
 \end{equation*}
  est $\Log^it$. Il vient alors
  \begin{eqnarray*}
    \sum_{n\le t}n g_n
    &=&t\sum_{n\le t}g_n-\int_0^X\sum_{n\le t}g_ndt
    \\&=&
    t\sum_{0\le i\le J}\Log^{i}t\sum_{i\le j\le J}\binom{j}{i}A_jH^{(j-i)}(0)
    \\&&-t\sum_{0\le i\le J}\sum_{i\le j\le J}\binom{j}{i}A_jH^{(j-i)}(0)
    \sum_{0\le \ell\le i}(-1)^{\ell-i}\frac{i!}{\ell!}\Log^\ell t
    \\&&+
    \Ocal^*(\frac{2-\sigma}{1-\sigma}D
    t^{-\sigma}\overline{H}(\sigma))
  \end{eqnarray*}
  et nous pouvons r\'e\'ecrire le terme principal en
  \begin{equation*}
    t\sum_{0\le \ell\le J}\Log^{\ell}t\sum_{\ell\le j\le J}
    \left(\binom{j}{\ell}A_jH^{(j-\ell)}(0)
    -(-1)^{\ell-j}\sum_{j\le k\le J}\frac{k!}{\ell!(k-j)!}A_k H^{(k-j)}(0)
    \right)
  \end{equation*}
  Pour $\ell=J$ le coefficient s'annule.
\end{dem}

Il nous faut une expression de $H'''(0)$.
Or
\begin{equation}
  \frac{H'}{H}(0)=\sum_{p}\frac{\sum_{m\ge0}mh_{p^m}}{\sum_{m\ge0}h_{p^m}}(-\Log p)
\end{equation}
puis
\begin{equation}
  \frac{H''}{H}(0)-\left( \frac{H'}{H}(0)\right)^2
  =\sum_{p}\frac{\sum_{m\ge0}m^2h_{p^m}}{\sum_{m\ge0}h_{p^m}}(-\Log p)^2\dots
\end{equation}
et
\begin{equation}
  \frac{H'''}{H}(0)-3\frac{H''H'}{H^2}(0)+2\left( \frac{H'}{H}(0)\right)^3
  =\sum_{p}\frac{\sum_{m\ge0}m^3h_{p^m}}{\sum_{m\ge0}h_{p^m}}(-\Log p)^3\dots
\end{equation}


\end{document}
%%% Local Variables: 
%%% mode: latex
%%% TeX-master: t
%%% End: 

