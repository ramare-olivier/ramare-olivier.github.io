\chapter{   Explicit bounds on the Moebius function}

Corresponding html file: \texttt{../Articles/Art02.html}










 
 


Collecting references:
\cite{Diamond-Erdos*80},
\cite{Deleglise-Rivat*96-2},
\cite{Borwein-Ferguson-Mossinghoff*08}.


\section{Bounds on $M(D)=\sum_{d\le D}\mu(d)$}


The first explicit estimate for $M(D)$ is due to
\cite{VonSterneck*98}
where the author proved that $|M(D)|\le \tfrac19 D+8$ for any $D\ge0$. 
A popular estimate is the one of
\cite{MacLeod*69}.

\begin{thm}{Theorem (1967)}

When $D\ge 0$, we have $|M(D)|\le \tfrac1{80} D+5$. When $D\ge 1119$, we have $|M(D)|\le D/80$.
\end{thm}

We mention at this level the annoted bibliography contained at the end of 
\cite{Dress*83}.
\cite{CostaPereira*89} shows that

\begin{thm}{Theorem (1993)}

When $D\ge 120\,727$, we have $|M(D)|\le D/1036$.
\end{thm}


On elaborating on this method,
\cite{Dress-ElMarraki*93} showed that

\begin{thm}{Theorem (1993)}

When $D\ge 617\,973$, we have $|M(D)|\le D/2360$.
\end{thm}


One of the argument is the estimate from 
\cite{Dress*93}

\begin{thm}{Theorem (1993)}

When $33\le D\le 10^{12}$, we have $|M(D)|\le 0.571\sqrt{D}$.
\end{thm}


This has been extended by
\cite{Kotnik-VanDeLune*04}
to $10^{14}$ and recently in
\cite{Hurst*18}
to $10^{16}$, i.e.

\begin{thm}{Theorem (2018)}

When $33\le D\le 10^{16}$, we have $|M(D)|\le 0.571\sqrt{D}$.
\end{thm}



Another tool is 
\cite{Cohen-Dress*88}
where refined explicit estimates for the remainder term of the counting
functions of the squarefree numbers in intervals are obtained.

\par 
The latest best estimate of this shape comes from
\cite{Cohen-Dress-ElMarraki*96}.
This preprint being difficult to get, it has been republished in
\cite{Cohen-Dress-ElMarraki*07}.
\begin{thm}{Theorem (1996)}

When $D\ge 2\,160\,535$, we have $|M(D)|\le D/4345$.
\end{thm}

These results are used in
\cite{Dress*99}
to study the discrepancy of the Farey series.

\par \par 
Concerning upper bounds that tend to $0$, 
\cite{Schoenfeld*69} is the pioneer
and shows among other estimates that 
\begin{thm}{Theorem (1969)}

When $D>0$, we have $|M(D)|/D\le 2.9/\log D$.
\end{thm}

\cite{ElMarraki*95} improves that
into
\begin{thm}{Theorem (1995)}

When $D\ge 685$, we have $|M(D)|/D\le 0.10917/\log D$.
\end{thm}

The latest bound coming from
\cite{Ramare*12-2} improves that:
\begin{thm}{Theorem (2012)}

When $D\ge 1\,100\,000$, we have $|M(D)|/D\le 0.013/\log D$.
\end{thm}


In
\cite{Ramare*12-5},
bounds including coprimality conditions are proved and here is a
typical example.
\begin{thm}{Theorem (2013)}

  When $1\le q < D$, we have
                $\Bigl|\sum_{\substack{ d\le D, \\
                (d,q)=1}}\mu(d)\Bigr|/D\le
                \frac{q}{\varphi(q)}/(1+\log (D/q))$.
\end{thm}








\section{Bounds on $m(D)=\sum_{d\le D}\mu(d)/d$}


\cite{MacLeod*69} shows that the sum
$m(D)$ takes its minimal value at $D=13$.
A folklore result is generalized in 
\cite{Granville-Ramare*96} and reads
\begin{thm}{Theorem (1996)}

When $D\ge 0$ and for any integer $r\ge1$, we have $\Bigl|\sum_{\substack{d\le D,\\
(d,r)=1}}\mu(d)/d\Bigr|\le 1$.
\end{thm}

In fact, Lemma 1 of \cite{Davenport*37-1} already
contains the requisite material.

The next result is proved in
\cite{Ramare*12-5}.
\begin{thm}{Theorem (2013)}

When $D\ge 7$, we have $|\sum_{d\le D}\mu(d)/d|\le 1/10$. We can
replace the couple (7, 1/10) by (41, 1/20) or (694, 1/100).
\end{thm}


This is further extended in
\cite{Ramare*12-4} where it is shown
that
\begin{thm}{Theorem (2012)}

When $D\ge 0$ and for any integer $r\ge1$ and any real number $\varepsilon\ge0$, we have $\Bigl|\sum_{\substack{d\le D,\\
(d,r)=1}}\mu(d)/d^{1+\varepsilon}\Bigr|\le 1+\varepsilon$.
\end{thm}



Concerning upper bounds that tend to $0$, 
\cite{ElMarraki*96}
is the first to get such an estimate.
\begin{thm}{Theorem (1996)}

When $D\ge33$ we have $|m(D)|\le 0.2185/\log D$.

\par 
When $D > 1$ we have $|m(D)|\le 726/(\log D)^2$.
\end{thm}


This second bound is improved in
\cite{Bordelles*15}.
\begin{thm}{Theorem (2015)}

When $D > 1$ we have $|m(D)|\le 546/(\log D)^2$.
\end{thm}



\cite{Ramare*12-2} proves several
bounds of the shape $m(D)\ll 1/\log D$.
This is improved in
\cite{Ramare*12-5} by using
\cite{Balazard*12}.
which provides us with a better manner to convert bounds on $M(D)$
into bounds for $m(D)$. Here is one result obtained.

\begin{thm}{Theorem (2015)}

When $D\ge 463\,421$ we have $|m(D)|\le 0.0144/\log D$.
\par 
  We can for instance replace the couple  (463 421, 0.0144)by
any of  (96 955, 1/69), (60 298, 1/65), (1426, 1/20)
or (687, 1/12).
\end{thm}



In
\cite{Ramare*12-3} and
\cite{Ramare*12-5}, the
problem of adding coprimality conditions is further addressed.
Here is one of the results obtained.
\begin{thm}{Theorem (2015)}

  When $1\le q < D$ we have
  $\Bigl|\sum_{\substack{d\le D,\\ (d,q)=1}}\mu(d)/d\Bigr|\le
   \frac{q}{\varphi(q)}0.78/\log(D/q)$. When $D/q\ge 24233$, we can
  replace 0.78 by 17/125. 
\end{thm}




\section{Bounds on $\check{m}(D)=\sum_{d\le D}\mu(d)\log(D/d)/d$}


The initial investigations on this function go back to 
 \cite{vonSterneck*02}.
In \cite{Ramare*12-5} it is
proved that
\begin{thm}{Theorem (2015)}

  When $3846 \le D$ we have
  $|\check{m}(D)-1|\le 0.00257/\log D$.
  When $D > 1$, we have
    $|\check{m}(D)-1|\le 0.213/\log D$.
\end{thm}


This implies in particular that
\begin{thm}{Theorem (2015)}

  When $222 \le D$ we have
  $|\check{m}(D)-1|\le 1/1250$.
  When $D > 1$, the optimal bound 1 holds.
\end{thm}


These bounds are a consequence of the identity:
$$
|\check{m}(D)-1|\le \frac{\frac74-\gamma}{D^2}\int_1^D|M(t)|dt+\frac{2}{D}.
$$
It is also proved that, for any $D\ge1$, we have
$$
0\le \sum_{\substack{d\le D,\\ (d,q)=1}}\mu(d)\log(D/d)/d
\le 1.00303 q/\varphi(q).
$$


\section{Miscellanae}


Here is an elegant wide ranging estimate, taken from Claim 3.1 of
\cite{Trevino*15}.

\begin{thm}{Theorem (2015)}

When $D\ge1$ we have $|\sum_{d>D}\mu(d)/d^2|\le 1/D$.
\end{thm}



\section{The Moebius function and arithmetic progressions}


The results in this section are scarce. We mention a Theorem of
\cite{Bordelles*15}.

\begin{thm}{Theorem (2015)}

  Let $\chi$\, be a non-principal Dirichlet character mmodulo
  $q\ge37$ and let $k\ge1$ be some integer. Then
  $$
  \biggl|\sum_{\substack{n\le x,\\ (n,k)=1}}\frac{\mu(n)\chi(n)}{n}\biggr|
  \le\frac{k}{\varphi(k)}\frac{2\sqrt{q}\log q}{L(1,\chi)}.
  $$
\end{thm}






  
\begin{flushright}\small\sl{}   Last updated on February 18th, 2018, by Olivier Ramar\'e
 \end{flushright}















