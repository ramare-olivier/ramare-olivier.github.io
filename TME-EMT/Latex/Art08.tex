\chapter{  Explicit zero-free regions for the $\zeta$ and $L$ functions}

Corresponding html file: \texttt{../Articles/Art08.html}









\section{Numerical verifications of the Generalized Riemann Hypothesis}


Numerical verifications of the Riemann hypothesis for the Riemann
$\zeta$-function have been pushed extremely far. B. Riemann himself computed the
first zeros. Concerning more recent published papers, we mention
\cite{Lune-Riele-Winter*86}
who proved that
\begin{thm}{Theorem (1986)}

  Every zero $\rho$ of $\zeta$ that have a real part between 0 and 1 and
  an imaginary part not more, in absolute value, than $\le T_0=545\,439\,823$
  are in fact on the critical line, i.e. satisfy $\Re \rho=1/2$.
\end{thm}

The bound $545\,439\,823$ is increased to $1\,000\,000\,000$ in
\cite{Platt*11}.
In
\cite{Platt*17},
this bound is further increased to
$30\,610\,046\,000$.
Between these results, 
\cite{Wedeniwski*02}
announced that, he and many collaborators proved, using a network method:
\begin{thm}{Theorem (2002)}

  $T_0=29\,538\,618\,432$ is admissible in the theorem above.
\end{thm}


\cite{Gourdon-Demichel*04}
went one step further
\par 
\begin{thm}{Theorem (2004)}

  $T_0=2.445\cdot 10^{12}$ is admissible in the theorem above.
\end{thm}

These two last announcements have not been subject to any academic papers.

We now have
\cite{Platt-Trudgian*21a}
\par 
\begin{thm}{Theorem (2021)}

  $T_0=3\cdot 10^{12}$ is admissible in the theorem above.
\end{thm}





One of the key ingredient is an explicit Riemann-Siegel formula due to
\cite{Gabcke*79}
(the preprint of Gourdon mentionned above gives a version of Gabcke's result)
and such a formula is missing in the case of Dirichlet $L$-function.

Let us introduce some terminology. We say that a modulus $q\ge1$ (i.e. an
integer!) satisfies $GRH(H)$ for some numerical value $H$ when
every zero $\rho$ of the $L$-function associated to a primitive Dirichlet
character of conductor $q$ and whose real part lies within the critical line (i.e. has a
real part lying inside the open interval $(0,1)$) and whose imaginary part is
below, in absolute value, $H$, in fact satisfies $\Re\rho=1/2$.

By employing an Euler-McLaurin formula,
\cite{Rumely*93}
has proved that

\begin{thm}{Theorem (1993)}

  \begin{itemize}
  \item Every $q\le 13$ satisfies $GRH(10\,000)$.

  \item Every $q$ belonging to one of the sets
  \begin{itemize}
    \item \,\,$\{k\le 72\}$

    \item \,\,$\{k\le 112, \text{$k$ non premier}\}$

    \item \,$\begin{aligned}\{116, 117, &120, 121, 124, 125, 128, 132, 140,
     143, 144, 156, 163, \\ &169, 180, 216, 243, 256, 360, 420, 432\}\end{aligned}$

    \end{itemize}
    satisfies $GRH(2\,500)$.
    

    \end{itemize}
\end{thm}

These computations have been extended by 
\cite{Bennett*01}
by using Rumely's programm. All these computations have been
superseded by the work of D. Platt.
\cite{Platt*11} and
\cite{Platt*13}
use two fast Fourier transforms, one in the $t$-aspect and one in the
$q$-aspect, as well as an approximate functionnal equation to prove via
extremely rigorous computations that
\begin{thm}{Theorem (2011-2013)}

  Every modulus $q\le 400\,000$ satisfies
    $GRH(100\,000\,000/q)$.
\end{thm}


We mention here the algorithm of
\cite{Omar*01}
that enables one to prove efficiently that some $L$-functions have no zero
    within the rectangle
$1/2\le \sigma\le1$ et $2\sigma-|t|=1$ though this algorithm has not been put
    in practice.

There are much better results concerning real zeros of Dirichlet $L$-functions
    associated to real characters.


\section{Asymptotical zero-free regions}


The first fully explicit zero free region for the Riemann zeta-function is due
to \cite{Rosser*38} in Lemma 19 (essentially
with $R_0=19$ in the notations below). This is next imporved upon in Theorem 1
of \cite{Rosser-Schoenfeld*75}
by using a device of
\cite{Stechkin*70} (getting
essentially $R_0=9.646$).
The next step is in
\cite{Ramare-Rumely*96} 
where the second order term is improved upon, relying on
\cite{Stechkin*89}.

Next, in
\cite{Kadiri*02}
and later in
\cite{Kadiri*05},
the following result is proven.

\begin{thm}{Theorem (2002)}

  The Riemann $\zeta$-function has no zeros in the region
  $$
    \Re s \ge 1- \frac1{R_0 \log (| \Im s|+2)}\quad\text{with}\  R_0=5.70175.
  $$
\end{thm}



\cite{Jang-Kwon*14}
improved the value of $R_0$ by showing that $R_0=5.68371$ is admissible.
By plugging a better trigonometric polynomial in the same method,
it is proved in
\cite{Mossinghoff-Trudgian*15}
that

\begin{thm}{Theorem (2015)}

  The Riemann $\zeta$-function has no zeros in the region
  $$
    \Re s \ge 1- \frac1{R_0 \log (| \Im s|+2)}\quad\text{with}\  R_0=5.573412.
  $$
\end{thm}


Concerning Dirichlet $L$-function, the first explicit zero-free region has been obtained in
\cite{McCurley*84-1} by adaptating
\cite{Rosser-Schoenfeld*75}.
\cite{Kadiri*02} (cf also
\cite{Kadiri*02-2})
improves that into:

\begin{thm}{Theorem (2002)}

  The Dirichlet $L$-functions associated to a character of conductor $q$ has
  no zero in the region:
  $$
    \Re s \ge 1- \frac1{R_1 \log(q \max(1,| \Im s|))}  \quad\text{with}\
    R_1=6.4355, 
  $$
  to the exception of at most one of them which would hence be attached to a
  real-valued character. This exceptional one would have at most one zero
  inside the forbidden region (and which is loosely called a "Siegel zero").
\end{thm}

In
\cite{Kadiri*18}, the next
theorem is proved.
\begin{thm}{Theorem (2016)}

  The Dirichlet $L$-functions associated to a character of conductor $q\in[3,400\,000]$ has
  no zero in the region:
  $$
    \Re s \ge 1- \frac1{R_2 \log(q \max(1,| \Im s|))}  \quad\text{with}\
    R_1=5.60. 
  $$
\end{thm}



Concerning the Vinogradov-Korobov zero-free region,
\cite{Ford*01}
shows that

\begin{thm}{Theorem (2001)}

  The Riemann $\zeta$-function has no zeros in the region
  $$
    \Re s\ge  1-\frac{1}{58(\log |\Im s|)^{2/3}(\log\log |\Im s|)^{1/3}}
    \quad(|\Im s|\ge 3).
  $$
\end{thm}


Concerning the Dedekind $\zeta$-function, see
\cite{Kadiri*12}.



\section{Real zeros}


\cite{Rosser*49},
\cite{Rosser*50},
\cite{Chua*05},
\cite{Watkins*00-1},

\section{Density estimates}


After initial work of
\cite{Chen-Wang*89-2}
and
\cite{Liu-Wang*02-1},
here are the latest two best results. We first define
$$
  N(\sigma,T,\chi)=\sum_{\substack{\rho=\beta+i\gamma,\\ L(\rho,\chi)=0,\\
      \sigma\le \beta, |\gamma|\le T}}1
$$
which thus counts the number of zeroes $\rho$ of $L(s,\chi)$, zeroes
whose real part is denoted by $\beta$ (and assumed to be larger than
$\sigma$), and whose imaginary part in absolute value $\gamma$ is assumed to be
not more than $T$. For the Riemann $\zeta$-function (i.e. when
$\chi=\chi_0$ the principal character modulo~1), it is customary
to count only the zeroes with positive imaginary part. The relevant
number is usually denoted by $N(\sigma,T)$. We have $2N(\sigma,T)=N(\sigma,T,\chi_0)$.

For low values, we start with the main Theorem of
\cite{Kadiri-Ng*12}.
We reproduce only a special case.

\par 
\begin{thm}{Theorem (2013)}

  Let $T\ge3.061\cdot10^{10}$. We have
  $
    2N(17/20,T,\chi_0)\le 0.5561T+0.7586\log T-268 658
  $
  where $\chi_0$ is the principal character modulo 1.
\end{thm}


See also
\cite{Kadiri*13}.
Otherwise, here is the result of
\cite{Ramare*13d}.
\par 

\begin{thm}{Theorem (2016)}

  For $T\ge2\,000$ and $T\ge Q\ge10$, as well as $\sigma\ge0.52$, we have 
  $$
    \sum_{q\le Q}\frac{q}{\varphi(q)}
    \sum_{\chi\mod^* q}N(\sigma,T,\chi)
    \le 
    20\bigl(56\,Q^{5}T^3\bigr)^{1-\sigma}\log^{5-2\sigma}(Q^2T)
    +32\,Q^2\log^2(Q^2T)
  $$
  where $\chi\mod^* q$ denotes a sum over all primitive Dirichlet character
  $\chi$ to the modulus $q$. Furthermore, we have
  $$
    N(\sigma,T,\chi_0)\le 6T\log T
    \log\biggl(1+\frac{6.87}{2T}(3T)^{8(1-\sigma)/{3}}\log^{4-2\sigma}(T)\biggr)
    +103(\log T)^2
  $$
  where $\chi_0$ is the principal character modulo 1.
\end{thm}

In
\cite{Kadiri-Lumley-Ng*18}.
this result is improved upon, we refer to their paper for their result
by quote a corollary.


  For $T\ge1$, we have 
  $
    N(0.9,T)
    \le 
    11.5\, T^{4/14}\log^{16/5}(T)
    +3.2\,\log^2(T)
  $
  where $N(\sigma,T)=N(\sigma,T,\chi_0)$ and $\chi_0$ is the principal character modulo 1.


\section{Miscellanae}



The LMFDB\footnote{\url{http://www.lmfdb.org}} database contains the first zeros
of many $L$-functions. A part of Andrew Odlyzko's 
website\footnote{\url{http://www.dtc.umn.edu/~odlyzko/zeta_tables/index.html}}
contains extensive tables concerning zeroes of the Riemann zeta function.




 
 







  
\begin{flushright}\small\sl{}   Last updated on September 19th, 2021, by Olivier Ramar\'e
 \end{flushright}














