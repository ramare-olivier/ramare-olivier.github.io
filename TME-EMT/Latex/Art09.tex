\chapter{   Short intervals containing primes}

Corresponding html file: \texttt{../Articles/Art09.html}










 
 







\section{Interval with primes, without any congruence condition}




The story seems to start in 1845 when Bertrand conjectured after
numerical trials
that the interval $]n,2n-3]$ contains a prime as soon as $n\ge4$. This was proved
by \v Ceby\v sev in 1852 in a famous work where he got the first good
quantitative estimates for the number of primes less than a given bound,
say $x$. By now, analytical means combined with sieve methods
(see
\cite{Baker-Harman-Pintz*01}
)
ensures us that each of the
intervals $[x,x+x^{0.525}]$ for $x \geq x_0$
contains at least one prime.
This statement concerns only for the (very) large integers.

It falls very close to what we can get under the assumption
of the Riemann
Hypothesis: the interval $[x-K\sqrt{x}\log x,x]$ contains a prime, where
$K$ is an effective large constant and $x$ is sufficiently large
(cf
\cite{Wolke*83}
for an account on this subject). A theorem
of Schoenfeld
\cite{Schoenfeld*76}
also tells us that the interval
\begin{equation*}
  [x-\sqrt{x}\log^2x/(4\pi),x]
\end{equation*}
contains a prime for $x\geq 599$ under the Riemann Hypothesis. These results
are still far from the conjecture in
\cite{Cramer*36}
on
probabilistic grounds: the interval $[x-K\log^2x,x]$ contains a prime for any
$K > 1$ and $x\geq x_0(K)$. Note that this statement has been proved for almost
all intervals in a quadratic average sense in
\cite{Selberg*43}
assuming the Riemann Hypothesis and replacing $K$ by a function $K(x)$ tending
arbitrarily slowly to infinity.


\cite{Schoenfeld*76}
proved the following.

\par 
\begin{thm}{Theorem (1976)}

Let $x$ be a real number larger than $2\,010\,760$. Then the interval
$$
\Bigl] x \Bigl(1-\frac1{16\,597}\Bigr),x \Bigr]
$$
contains at least one prime.
\end{thm}


The main ingredient is the explicit formula and a numerical verification of
the Riemann hypothesis.

From a numerical point of view, the Riemann Hypothesis is known to hold up to
a very large height (and larger than in 1976).
\cite{Wedeniwski*02} and the Zeta grid
project verified this hypothesis till height $T_0=2.41\cdot 10^{11}$
and
\cite{Gourdon-Demichel*04}
till height $T_0=2.44 \cdot 10^{12}$
thus extending the work
\cite{Lune-Riele-Winter*86}
who had conducted such a verification in 1986 till
height $5.45\times10^8$.
This latter computations has appeared in a refereed journal, but this is not
the case so far concerning the other computations; section 4 of the paper
\cite{Saouter-Demichel*10}
casts some doubts on whether all the zeros where checked.
Discussions in 2012 with Dave Platt from the university of Bristol led me to
believe that the results of
\cite{Wedeniwski*02}
can be replicated in a very rigorous setting, but that it may be difficult to
do so with the results of 
\cite{Gourdon-Demichel*04}
with the
hardware at our disposal.

In
\cite{Ramare-Saouter*02},
we used
the value $T_0=3.3 \cdot 10^{9}$ and obtained the following.



\begin{thm}{Theorem (2002)}

Let $x$ be a real number larger than $10\,726\,905\,041$. Then the interval
$$
\Bigl] x \Bigl(1-\frac1{28\,314\,000}\Bigr),x \Bigr]
$$
contains at least one prime.
\end{thm}




If one is interested in somewhat larger value, the paper
\cite{Ramare-Saouter*02} also
contains the following.

\begin{thm}{Theorem (2002)}

Let $x$ be a real number larger than $\exp(53)$. Then the interval
$$
\Bigl] x \Bigl(1-\frac1{204\,879\,661}\Bigr),x \Bigr]
$$
contains at least one prime.
\end{thm}




Increasing the lower bound in $x$ only improves the constant by less than 5
percent. If we rely on
\cite{Gourdon-Demichel*04},
we can prove
that

\begin{thm}{Theorem (2004, conditional)}

Let $x$ be a real number larger than $\exp(60)$. Then the interval
$$
\Bigl] x \Bigl(1-\frac1{14\,500\,755\,538}\Bigr),x \Bigr]
$$
contains at least one prime.
\end{thm}


Note that all prime gaps have been computed up
to $10^{15}$ in
\cite{Nicely*99}, extending a
result of
\cite{Young-Potler*89}.

In
\cite{Trudgian*16},
we find
\begin{thm}{Theorem (2016)}

Let $x$ be a real number larger than $2\,898\,242$. The interval
$$
\Bigl[ x, x \Bigl(1+\frac1{111(\log x)^2}\Bigr) \Bigr]
$$
contains at least one prime.
\end{thm}

In
\cite{Dusart*16},
we find
\begin{thm}{Theorem (2016)}

Let $x$ be a real number larger than $468\,991\,632$. The interval
$$
\Bigl[ x, x \Bigl(1+\frac1{5000(\log x)^2}\Bigr),x \Bigr]
$$
contains at least one prime.
\par 

  Let $x$ be a real number larger than $89\,693$. The interval
$$
\Bigl[ x, x \Bigl(1+\frac1{\log^3 x}\Bigr) \Bigr]
$$
contains at least one prime.
\end{thm}


The proof of these latter results has an asymptotical part, for $x\ge 10^{20}$ where we used the
numerical verification of the Riemann hypothesis together with two other
arguments: a (very strong) smoothing argument and a use of the Brun-Titchmarsh
inequality.

The second part is of algorithmic nature and covers the range $10^{10} \le x \le
10^{20}$ and uses prime generation
techniques
\cite{Maurer*95}:
we only look at families of numbers whose primality can be established with
one or two Fermat-like or Pocklington's congruences. This kind of technique
has been already used in a quite similar problem in
\cite{Deshouillers-teRiele-Saouter*98}.
The generation technique we
use relies on a theorem proven in
\cite{Brillhart-Lehmer-Selfridge*75}
and enables us to generate dense
enough families for the upper part of the range to be investigated. For the
remaining (smaller) range, we use theorems of
\cite{Jaeschke*93}
that yield a fast primality test (for this limited range).


 

\par 

Let us recall here that a second line of approach following the original
work of \v Ceby\v sev is still under examination though it does not give
results as good
as analytical means (see
\cite{CostaPereira*89}
for the latest result).

\par 

For very large numbers
\cite{Dudek*16b}
proved the following.
\begin{thm}{Theorem (2014)}

The interval $(x,x+3 x^{2/3}]$ contains a prime for $x\ge \exp(\exp(34.32))$.
\end{thm}



This is improved in 
\cite{Cully*21} as follows.

\begin{thm}{Theorem (2021)}

The interval $(x,x+3 x^{2/3}]$ contains a prime for $x\ge \exp(\exp(33.99))$.
\end{thm}



\section{Interval with primes under RH, without any congruence condition}



\begin{thm}{Theorem (2002)}

Under the Riemann Hypothesis, the interval $\bigl]x-\tfrac85\sqrt{x}\log x,x\bigr]$
contains a prime for $x\ge2$.
\end{thm}

This is improved upon
in 
\cite{Dudek*15} 
into:
\begin{thm}{Theorem (2015)}

Under the Riemann Hypothesis, the interval $\bigl]x-\tfrac4{\pi}\sqrt{x}\log x,x\bigr]$
contains a prime for $x\ge2$.
\end{thm}


In \cite{Carneiro-Milinovich-Soundararajan*19}, the authors go one step further and prove the next result.
\begin{thm}{Theorem (2019)}

Under the Riemann Hypothesis, the interval $\bigl]x-\tfrac{22}{25}\sqrt{x}\log x,x\bigr]$
contains a prime for $x\ge4$.
\end{thm}



\par 







\section{Interval with primes, with congruence condition}




Collecting references:
\cite{McCurley*84-2},
\cite{McCurley*84-3},
\cite{Kadiri*05-2}.

























  
\begin{flushright}\small\sl{}   Last updated on September 1rst, 2021, by Charles Greathouse.
 \end{flushright}













