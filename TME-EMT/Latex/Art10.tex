\chapter{   Averages of non-negative multiplicative functions}

Corresponding html file: \texttt{../Articles/Art10.html}










 
 





\section{Asymptotic estimates}


When looking for averages of functions that look like 1 or like the divisor
function, Lemma 3.2 of
\cite{Ramare*95} offers an
efficient easy path. The technique of comparison of two arithmetical function
via their Dirichlet series is known as the Convolution method and is for
instance decribed at length in 
\cite{Berment-Ramare*12}, and in
the course that can be found 
here\footnote{\url{https://ramare-olivier.github.io/CoursNouakchott/index.html}}.


\begin{thm}{Theorem (1995)}

Let $(g_n)_{n\ge1}$, $(h_n)_{n\ge1}$ and $(k_n)_{n\ge1}$ be three
complex sequences. Let $H(s)=\sum_nh_nn^{-s}$, and
$\overline{H}(s)=\sum_n|h_n|n^{-s}$.
We assume that $g=h\star k$, that $\overline{H}(s)$ is convergent for
$\Re(s)\ge-1/3$ and further that
there exist four constants $A$, $B$, $C$ and $D$ such that
$$
\sum_{n\le t}k_n
=
A\log^2t+B\log t+C+\mathcal{O}^*(D t^{-1/3})
\text{ for $t>0$.}
$$
Then we have for all $t>0$ :
$$
\sum_{n\le t}g_n
=
u\log^2t+v\log t+w+\mathcal{O}^*(D t^{-1/3}\overline{H}(-1/3))
$$
with
$u=AH(0)$, $v=2AH^{\prime}(0)+BH(0)$ and $w=AH^{\prime\prime}(0)+BH^{\prime}(0)+CH(0)$.
We have also
$$
\sum_{n\le t}ng_n
=
Ut\log t+Vt+W+\mathcal{O}^*(2.5D t^{2/3}\overline{H}(-1/3))
$$
with
$$
\begin{aligned}
U=&2AH(0), V=-2AH(0)+2AH^{\prime}(0)+BH(0),\\
W=&A(H^{\prime\prime}(0)-2H^{\prime}(0)+2H(0))+B(H^{\prime}(0)-H(0))+CH(0).
\end{aligned}
$$
\end{thm}

This Lemma says that one derives information from $g_n$ from informations on
the model $k_n$. When this model is $k_n=1$, the values concerning $A$,
$B$ and $C$ are given by 
the first half of Lemma 3.3 of
\cite{Ramare*95}:

\par 
\begin{thm}{Lemma (1995)}

$
\sum_{n\le t}1/n=\log t+\gamma+\mathcal{O}^*(0.9105 t^{-1/3}).
$
\end{thm}


When this model is $k_n=\tau(n)$, the number of divisors of $n$, the values concerning $A$,
$B$ and $C$ are given by Corollary 2.2 of
\cite{Berkane-Bordelles-Ramare*11}. Please
note the "$\gamma^2-2\gamma_1$" which is wrongly typed as
"$\gamma^2-\gamma_1$" in the aforementioned paper (and thanks to Tim
Trudgian and David Platt for spotting this typo):

\par 
\begin{thm}{Lemma (2011)}

  $
    \sum_{n\le t}\tau(n)/n
    =\tfrac12\log^2t+2\gamma\log t
    +\gamma^2-2\gamma_1+ \mathcal{O}^*(1.16/t^{1/3})
  $
  where $\gamma_1$ is the second Laurent-Stieljes constant --
  for instance \cite{Kreminski*03}
 and \cite{Coffey*06}. In particular, we have
  $
    \gamma_1=
    -0.0728158454836767248605863758749013191377
    + \mathcal{O}^*(10^{-40}).
  $
\end{thm}

The constants $H(0)$, $H'(0)$ and $H''(0)$ are to be computed. In most cases,
the Dirichlet series has an Euler product, in which case,
(see section 3 of
\cite{Ramare*95})
 we have
\par 
$
H(0)=\prod_p(1+\sum_mh_{p^m}),
$
then
$\displaystyle
\frac{H^{\prime}(0)}{H(0)}=
\sum_p
\frac{\sum_mmh_{p^m}}{1+\sum_mh_{p^m}}(-\log p),
$
and also
\par 
$$
\frac{H^{\prime\prime}(0)}{H(0)}=
\left(
\frac{H^{\prime}(0)}{H(0)}
\right)^2+
\sum_p
\left\{
\frac{\sum_mm^2h_{p^m}}{1+\sum_mh_{p^m}}
-\left[\frac{\sum_mmh_{p^m}}{1+\sum_mh_{p^m}}\right]^2
\right\}
\log^2p.
$$
It is sometimes more expedient to use the same convolution method but
by comparing the function to the function $q\mapsto q$. In such a
case, the next lemma, Lemma 4.3 from
\cite{Ramare*13d},
is handy.
\begin{thm}{Theorem (2015)}

  We have, for any real number $x\ge0$ and any real number $c\in[1,2]$,
  $\displaystyle \sum_{q\le x}q=\tfrac12 x^2+O^*(x^c/2)$.
\end{thm}

This leads to the next theorem.
\begin{thm}{Theorem}

  Let $(h_n)_{n\ge1}$  be a 
complex sequences. Let $H(s)=\sum_nh_nn^{-s}$, and
$\overline{H}(s)=\sum_n|h_n|n^{-s}$.
We assume that  $\overline{H}(s)$ is convergent for
$\Re(s)\ge c$, for some $c\in[1,2]$. 
Then we have for all $t>0$ :
$$
\sum_{n\le t}\sum_{d|n}\frac{n}{d}h(d)
=
\frac{t^2}{2}H(2)+O^*(t^c\overline{H}(c)/2).
$$
\end{thm}

A typical usage is to evaluate $\sum_{n\le t}\phi(n)$, with
$h(d)=\mu(d)$.
\par 
  
The convolution method has been brought one step further in
\cite{Ramare*14-1}
where the following theorem is proved.
\par 
\begin{thm}{Theorem (2017)}

Let $(g(m))_{m\ge1}$ be a sequence of complex numbers such that both series
  $\sum_{m\ge1} g(m)/m$ and $\sum_{m\ge1} g(m)(\log m)/m$ converge. We define
  $G^\sharp(x)=\sum_{m> x} g(m)/m$ and assume that
  $\int_1^\infty |G^\sharp(t)|dt/t$ converges. Let $A_0\ge1$ be a real parameter.
  We have 
  \begin{equation*}
    \sum_{n\le D}\frac{(g\star{1\!\!1})(n)}{n}
    =\sum_{m\ge1}\frac{g(m)}{m}\Bigl(\log\frac{D}{m}+\gamma\Bigr)
    +\int_{D/A_0}^\infty G^\sharp(t)\frac{dt}{t}
    +O^*(\mathfrak{R})
  \end{equation*}
  where $\mathfrak{R}$ is defined by
  \begin{equation*}
    \mathfrak{R}
    =
    \left|\sum_{1\le a\le A_0}\frac{1}{a}G^\sharp\left(\frac{D}{a}\right)+
      G^\sharp\left(\frac{D}{A_0}\right)\left(\log\frac{A_0}{[A_0]}
        -R([A_0])\right)
    \right|
    +\frac{6/11}{D}\sum_{m\le D/A_0}|g(m)|
  \end{equation*}
  where $[A_0]$ is the integer part of $A_0$,
while the remainder $R$ is defined by
$R(X)=\sum_{n\le X}1/n-\log X-\gamma$.
\end{thm}

The remainder $R(X)$ is shown in Lemma to verify $|R(X)|\le \gamma/X$
for every $X > 0$, and $|R(X)|\le (6/11)/X$ when $X\ge1$.

\par 

Theorem 21.1 of
\cite{Ramare*06}
offers a fully explicit estimate for the average of a general non-negative
multiplicative function, but 
it is often numerically rather poor. It relies on the
technique developped by 
\cite{Levin-Fainleib*67}.

\par 
\begin{thm}{Theorem (2009)}

Let $g$ be a non-negative multiplicative function.
Let $A$ and $\kappa$ be three positive real parameters such that, for any
$Q\ge1$, one has
$$
\sum_{\substack{ p\ge2, \nu\ge1\\  p^{\nu}\le Q}}
g\bigl(p^{\nu}\bigr)\log\bigl(p^{\nu}\bigr)
=
\kappa\log Q+\mathcal{O}^*(L)
$$
and
$
\sum_{p\ge2}
\sum_{\nu,k\ge1}g\bigl(p^k\bigr)g\bigl(p^{\nu}\bigr)\log\bigl(p^{\nu}\bigr)
\le A.
$
Then, when $D\ge\exp(2(L+A))$, we have
$$
\sum_{d\le D}g(d)= C\left(\log D\right)^{\kappa}
\left(1+\mathcal{O}^*\bigl(B/\log D\bigr)\right)
$$
where $B=2(L+A)\bigl(1+2(\kappa+1)e^{\kappa+1}\bigr)$ and
$$
C=\frac{1}{\Gamma(\kappa+1)}
\prod_{p\ge2}\biggl\{
\biggl(\sum_{\nu\ge0}g\bigl(p^{\nu}\bigr)\biggr)
\biggl(1-\frac1p\biggr)^{\kappa}\biggr\}.
$$
\end{thm}



\section{Upper bounds}


When looking for an upper bound, it is common to compare sums to an Euler
product, via, 
$$
\sum_{n\le y}f(n)/n\le \prod_{p\le y}
\left(1+\sum_{1\le m\le \log y/\log p}f(p^m)\right)
$$
valid when $f$ is non-negative and multiplicative.
Lemma 4 of
\cite{Daboussi-Rivat*01}
extends this. Let $z$ be a parameter and $v_z(n)$ be the characteristic
function of those integers that have all their prime factors $p\le z$.

\begin{thm}{Theorem (2000)}

Let $z\ge2$, $f$ a multiplicative function with $f\ge0$ and 
$S=\sum_{p\le z}\frac{f(p)}{1+f(p)}\log p$. We assume that $S>0$ and write
$K(t)=\log t-1-(1/t)$. For any $y$ such that $\log y\ge S$, we have
$$
\sum_{n > y}v_z(n)\mu^2(n)f(n)
\le \prod_{p\le z}(1+f(p))\exp\left(-\frac{\log y}{\log z}
K\left(\frac{\log y}{S}\right)\right)
$$
$$
\sum_{n \le y}v_z(n)\mu^2(n)f(n)
\ge \prod_{p\le z}(1+f(p))\left\{1-\exp\left(-\frac{\log y}{\log z}
K\left(\frac{\log y}{S}\right)\right)\right\}
$$
and in particular, when $\log y\ge 7S$, we have
$$
\sum_{n > y}v_z(n)\mu^2(n)f(n)
\le \prod_{p\le z}(1+f(p))\exp\left(-\frac{\log y}{\log z}\right)
$$
$$
\sum_{n \le y}v_z(n)\mu^2(n)f(n)
\ge \prod_{p\le z}(1+f(p))\left\{1-\exp\left(-\frac{\log y}{\log z}\right)\right\}.
$$
\end{thm}


It is sometimes required to compare a function close to $1$ (or more generally
to the divisor
function $\tau_k$) to a function 
close to $1/n$ or $\tau_k(n)/n$. Theorem 01 of 
\cite{Hall-Tenenbaum*88} offers a
fast way of doing so.
\par 
\begin{thm}{Theorem (1988)}

Let $f$ be a non-negative multiplicative function such that, for some $A$ and
$B$,
$$
\sum_{p\le y} f(p)\log p\le Ay \quad(y\ge 0),\quad
\sum_p\sum_{\nu\ge2} \frac{f(p^\nu)}{p^\nu}\log p^{\nu}\le B.
$$
Then, for $x > 1$,
$$
\sum_{n\le x}f(n)\le (A+B+1)\frac{x}{\log x}\sum_{n\le x}\frac{f(n)}{n}
$$
\end{thm}

See also Section 4.6, and for instance Theorem 4.22, of
\cite{Bordelles*12}.
In particular,
in case a further condition is assumed, we have Theorem 4.28 of
\cite{Bordelles*12}
at our disposal.
\par 
\begin{thm}{Theorem (2012)}

Let $f$ be a non-negative multiplicative function such that, for every
prime $p$ and every non-negative power $a$ the condition
$f(p^{a+1})\ge f(p^a)$ holds, we have
for $x \ge 1$
$$
\sum_{n\le x}f(n)\le x\prod_{p\le x}
\Bigl(1-\frac{1}{p}\Bigr)\Bigl(
1+\sum_{a\ge 1}\frac{f(p^a)}{p^a}\Bigr).
$$
\end{thm}

  
The next lemma is handy to remove coprimality conditions.
It originates from
\cite{van-Lint-Richert*65}.
\par 
\begin{thm}{Theorem (1965)}

Let $f$ be a non-negative multiplicative function and let $d$ be a
positive integer. We have
for $x \ge 0$
$$
\sum_{n\le x}\mu^2(n)f(n)\le \prod_{p|d}(1+f(p))
\sum_{\substack{n\le x,\\ (n,d)=1}}\mu^2(n)f(n)
\le
\sum_{n\le xd}\mu^2(n)f(n).
$$
\end{thm}

Though it is somewhat difficult to get, this lemma has been further generalized in Lemma 4.1 of
\cite{Ramare*10}.

\section{Estimates of some special functions}


\cite{Cohen-Dress*88} contains the
following Theorem.

\par 
\begin{thm}{Theorem (1988)}

  Let $R(x)=\sum_{n\le x}\mu^2(n)-6x/\pi^2$. We have
  $ |R(x+y)-R(x)|\le 1.6749\sqrt{y}+0.6212 x/y$ and 
  $ |R(x+y)-R(x)|\le 0.7343y/x^{1/3}+1.4327 x^{1/3}$ for $x,y\ge1$.
\end{thm}


See also \cite{CostaPereira*89}.
\cite{Moser-MacLeod*66} and 
\cite{Cohen-Dress-ElMarraki*07}
contains:
\par 
\begin{thm}{Theorem (2008)}

  We have
  $ |\sum_{n\le x}\mu^2(n)-6x/\pi^2|\le 0.02767\sqrt{x}$ for $x\ge 438653$.
  One can replace $(0.02767, 438653)$ by $(0.036438, 82005)$, by
  $(0.1333, 1004)$, by $(1/2, 8)$ or by $(1,1)$.
\end{thm}




\par 
Lemma 3.4
of  \cite{Ramare*13d}
gives us:
\par 
\begin{thm}{Theorem (2013)}

We have
$\frac{6}{\pi^2}\log x+0.578\le \sum_{n\le x}\mu^2(n)/n\le \frac{6}{\pi^2}\log x+1.166$ for $x\ge1$

When $x\ge1000$, one can also replace the couple $(0.578, 1.166)$ by $(1.040, 1.048)$.
\end{thm}

In fact, in the same paper, the asymptotic
$$
\sum_{n\le x}\frac{\mu^2(n)}{n}
=\frac{6}{\pi^2}\Bigl(\log x+2\sum_{p\ge2}\frac{\log
p}{p^2-1}+\gamma\Bigr)
+O^*(3/x^{1/3})
$$
valid for $x\ge1$ is proved. A script using SAGE and another one using GP/PARI are then
displayed to explain how to cover the initial range in $x$.
See also Lemma 1
of \cite{Schoenfeld*69} for an
earlier version.



\par 
The main result
\cite{Berkane-Bordelles-Ramare*11}
reads as follows.
\par 
\begin{thm}{Theorem (2012)}

We define $\Delta(x)=\sum_{n\le x}\tau(n)-x(\log x+2\gamma-1)$. We have
\begin{itemize}
\item  When $x\ge 1$, we have $|\Delta(x)|\le 0.961\, {x^{1/2}}$.

\item  When $x\ge 1\,981$, we have $|\Delta(x)|\le 0.482\, {x^{1/2}}$.

\item  When $x\ge 5\,560$, we have $|\Delta(x)|\le 0.397\, {x^{1/2}}$.

\item  When $x\ge 5$, we have $|\Delta(x)|\le 0.764\, {x^{1/3}\log x}$.

\end{itemize}
\end{thm}


For evaluation of the average of the divisor function on integers belonging to
a fixed residue class modulo 6, see Corollary to Proposition 3.2 of 
\cite{Deshouillers-Dress*88}.



For more complicated sums and when $x$ is large with respect to $k$, one can use
\cite{Mardjanichvili*39}.
\par 
\begin{thm}{Theorem (1939)}

  Let $k$ and $\ell$ be two positive integers. We have for any real number
$x\ge1$ 
$$
 \sum_{m\le x}\tau_k^\ell(m) \le
x\frac{k^\ell}{(k!)^{\frac{k^\ell-1}{k-1}}}(\log x+k^\ell-1)^{k^\ell-1}.  
$$
\end{thm}

See \cite{Deshouillers-Dress*88}
for some upper bounds linked with $\tau_3$.

\par 
\cite{Bordelles*02}
contains the following bounds, better than the above when $x$ is small with
respect to $k$.
\par 
\begin{thm}{Theorem (2002)}

  Let $k\ge1$ be a positive integer.
\begin{itemize}
\item  When  $x\ge1$ is a real number, we have
  $\sum_{m\le x}\tau_k(m)\le x(\log x+\gamma+(1/x))^{k-1}$.

\item  When $x\ge6$ is a real number, we have
  $\sum_{m\le x}\tau_k(m)\le 2x(\log x)^{k-1}$.

\end{itemize}
\end{thm}

In 
\cite{Cully-Trudgian*21}, 
we find the next result.
\par 
\begin{thm}{Theorem (2021)}

  For $x\ge 2$, we have
$$
\sum_{n\le x} d_4(n) = C_1x(\log x)^3
+ C_2x(\log x)^2
+C_3x(\log x
+C_4x
+\mathcal{O}^*(4.48\,x^{3/4}\log x)
$$
where the constants $C_1=1/6$, $C_2=2\gamma-1/2$, and $C_3$ and $C_4$
  are the expected ones and may be expressed in terms of the Stieltjes
  constants $\gamma_i$.
\end{thm}

They deduce for instance from this that $\sum_{n\le
x}d_4(n)\le(1/3)x(\log x)^3$ when $x\ge 193$.

In the same paper, these authors also establish the next estimate.
\par 
\begin{thm}{Theorem (2021)}

  For $x\ge 2$, we have
$$
\sum_{n\le x} d(n)^2 = D_1x(\log x)^3
+ D_2x(\log x)^2
+D_3x(\log x
+D_4x
+\mathcal{O}^*(9.73\,4.48\,x^{3/4}\log x+0.73\,\sqrt{x})
$$
where the constants $D_1=1/\pi^2$, and $D_2$, $D_3$ and $D_4$
  are the expected ones and may be expressed in terms of usual constants.
\end{thm}

They deduce for instance from this that $\sum_{n\le
x}d(n)^2\le(1/4)x(\log x)^3$ when $x\ge 433$ and that 
$\sum_{n\le
x}d(n)^2\le x(\log x)^3$ when $x\ge 7$.

In
\cite{Lapkova*16}, we find
the next result.
\par 
\begin{thm}{Theorem (2015)}

  Let $b$ and $c$ be two integers such that $\delta=b^2-c$\, is non-zero,
  square-free and not congruent to 1 modulo 4. Assume further that the
  function $n^2+2bn+c$ is positive and non-decreasing when
  $n\ge1$.Then, for $N\ge1$, we have
  $$
  \sum_{n\le N}\tau(n^2+2bn+c)\le C_1 N\log N+C_2+C_3
  $$
  where the constants $C_1$, $C_2$\, and $C_3$\, are defined
  as follows. We first define $\xi=\sqrt{1+2|b|+|c|}$ and
  $\kappa=\frac{4}{\pi^2}\sqrt{4|\delta|}(\log(4|\delta|)+0.648)$. Then
  $$
  C_1=\frac{12}{\pi^2}(\log\kappa+1),
  C_2=2\biggl[\kappa+(\log\kappa+1)\Bigl(\frac{6}{\pi^2}\log\xi+1.166\Bigr)\biggr],
  C_3=2\kappa (\max(|b|,|c|^{1/2})+1).
  $$
\end{thm}

See
\cite{Lapkova*16-2} for the
number of divisors of a reducible quadratic polynomial.

\par 
Evaluations of Lemma 4.3  of
\cite{Cipu*15-1}
are improved in Lemma 12 of
\cite{Trudgian*15-2}.
Only upper bounds are given, but the proof given there gives the lower
bounds as well. This gives the first two estimates, while the third
one comes from 
Lemma 4.3  of
\cite{Cipu*15-1}.
\par 
\begin{thm}{Theorem (2015)}

  Let $x\ge1$ be a real number. We have
  \begin{itemize}
\item  
  $\displaystyle 0.786x-0.3761-8.14x^{2/3}
  \le \sum_{n\le x}2^{\omega(n)}-\frac{6}{\pi^2}x\log x
    \le 0.787x-0.3762+8.14x^{2/3}$

\item  
 $\displaystyle 1.3947\log x+0.4106-3.253x^{-1/3}
  \le \sum_{n\le x}\frac{2^{\omega(n)}}{n}-\frac{3}{\pi^2}(\log x)^2
    \le 1.3948\log x+0.4107+3.253x^{-1/3}$,

\item  
  $\displaystyle \sum_{n\le x}\frac{2^{\omega(2n-1)}}{2n-1}
  \le \frac{3}{2\pi^2}(\log x)^2+3.123\log x+3.569+\frac{0.525}{x}$.

\end{itemize}
\end{thm}



We take the next lemma from
\cite{Trevino*15-2}, Lemma 2.
\par 
\begin{thm}{Theorem (2015)}

  Let $x\ge1$ be a real number. We have
  $\sum_{n\le x}\phi(n)/n\le \frac{6}{\pi^2}x+\log x +1$.
\end{thm}

  

Lemma 3 of the same paper is as follows.
\par 
\begin{thm}{Theorem (2015)}

  Let $x\ge1$ be a real number. We have
  $\sum_{n\le x}n\phi(n)\le \frac{2}{\pi^2}x^3+\frac12x^2\log x +x^2$.
\end{thm}


Several estimates are proved in
\cite{Ramare*14-1}.
For instance Theorem 1.2 gives the following.
\par 
\begin{thm}{Theorem (2017)}

  Let $x\ge1$ be a real number. We have
  $\sum_{n\le x}\mu^2(n)/\phi(n)= \log x
  +c_0+O^*(3.95/\sqrt{x})$ where $\displaystyle c_0=\gamma+\sum_{p\ge2}\frac{\log
  p}{p(p-1)}$.
  When $x > 1$, this $O^*$ can be replaced
  by $O^*(21/\sqrt{x\log x})$. 
\end{thm}

The function $\sum_{n\le x}\mu^2(n)/\phi(n)$ has been the subject of
several estimates,
see for instance Lemma 7 of
\cite{Montgomery-Vaughan*73},
Lemma 3.4-3.5 of
\cite{Ramare*95},
the earlier paper
\cite{Ward*27}
and Lemma 4.5 of
\cite{Buethe*14}
where the error term $O^*(58/\sqrt{x})$ is achieved.
The constant $c_0$ is evaluated precisely in (2.11) of
\cite{Rosser-Schoenfeld*62}.



\section{Euler products}



\cite{Rosser-Schoenfeld*62}
contains estimates regarding $\prod_{p}(1-1/p)$ and its inverse. In
particular we find the next results therein.
\par 
\begin{thm}{Theorem (1962)}

  \begin{itemize}
    \item When $x > 1$, we have $\displaystyle 1-\frac{1}{\log^2x}\le
      e^\gamma(\log x)\prod_{p\le x}\Bigl(1-\frac{1}{p}\Bigr)\le
      1+\frac{1}{2\log^2x}$.
    

    \item When $x > 1$, we have $\displaystyle 1-\frac{1}{2\log^2x}\le
      e^{-\gamma}\prod_{p\le x}\Bigl(1-\frac{1}{p}\Bigr)^{-1}/\log x\le
      1+\frac{1}{\log^2x}$.
    

    \end{itemize}
\end{thm}

Several other estimates are proven. In
\cite{Dusart*16}, it is
proved that
\par 
\begin{thm}{Theorem (2016)}

  \begin{itemize}
    \item When $x > 2\,278\,382$, we have $\displaystyle 1-\frac{1}{5\log^3x}\le
      e^\gamma(\log x)\prod_{p\le x}\Bigl(1-\frac{1}{p}\Bigr)\le
      1+\frac{1}{5\log^3x}$.
    

    \item When $x > 2\,278\,382$, we have $\displaystyle 1-\frac{1}{5\log^3x}\le
      e^{-\gamma}\prod_{p\le x}\Bigl(1-\frac{1}{p}\Bigr)^{-1}/\log x\le
      1+\frac{1}{5\log^3x}$.
    

    \end{itemize}
\end{thm}


In \cite{Bordelles*05},
the reader will find explicit upper bounds for $\displaystyle
\prod_{\substack{p\le x,\\ p\equiv a[q]}}\Bigl(1-\frac{1}{p}\Bigr)^{-1}$

Theorem 5 of
\cite{Vanlalnagaia*15-1}
contains the next result.
\par 
\begin{thm}{Theorem (2017)}

  Let $\epsilon$ be a complex number such that $|\epsilon|\le 2$. When
  $x\ge\exp(22)$, we have
  $\displaystyle\prod_{p\le x}\Bigl(1+\frac{\epsilon}{p}\Bigr)
  =e^{\gamma(\epsilon)+\epsilon B}(\log x)^\epsilon
  \biggl\{1+O^*\Bigl(\frac{0.841}{\log^3x}\Bigr)\biggr\}$
  where
  $\displaystyle\gamma(\epsilon)=\sum_{p\ge2}\sum_{n\ge2}(-1)^{n+1}\frac{\epsilon^n}{np^n}$
  and
  $\displaystyle B=\gamma+\sum_{p\ge2}\bigl(\log(1-1/p)+(1/p)\bigr)$.
\end{thm}

Equation (2.2) of
\cite{Rosser-Schoenfeld*62}
gives an approximate value for $B$.










  
\begin{flushright}\small\sl{}   Last updated on June 10th, 2019, by Olivier Ramar\'e
 \end{flushright}















