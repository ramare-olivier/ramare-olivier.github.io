\chapter{   Integer Points near Smooth Plane Curves}

Corresponding html file: \texttt{../Articles/Art11.html}










 
 


In what follows, $N \geqslant 1$ is an arbitrary large integer, $\delta \in
\left( 0,\frac{1}{2} \right)$ and if $f : [N,2N] \longrightarrow \mathbb {R}$
is any positive function, then let $\mathcal {R}(f,N,\delta)$ be the number of
integers $n \in [N,2N]$ such that $\lVert f(n) \rVert < \delta$, where as
usual $\lVert x \rVert$ denotes the distance from $x \in \mathbb {R}$ to its
nearest integer. Note that, since $\delta$ is very small, $\mathcal
{R}(f,N,\delta)$ roughly counts the number of integer points very close to the
arc $y = f(x)$ with $N \leqslant x \leqslant 2N$. Hence the trivial estimate
is given by $\mathcal {R}(f,N,\delta) \leqslant N+1$.  


The number $\mathcal {R}(f,N,\delta)$ arises fairly naturally in a large
collection of problems in number theory, e.g.
\cite{Filaseta*90},
\cite{Filaseta-Trifonov*96},
\cite{Huxley*96},
\cite{Huxley-Sargos*95},
\cite{Huxley-Sargos*06},
\cite{Huxley-Trifonov*96} and
bibref("Huxley*07"").
We deal with either getting an
asymptotic formula of the shape 

$$\mathcal {R}(f,N,\delta) = N \delta + \textrm{Error terms}$$

where the remainder terms depend on the derivatives of $f$ but not on
$\delta$, or finding an upper bound for $\mathcal {R}(f,N,\delta)$ as accurate
as possible.  



\section{Bounds using elementary methods}


The basic result of the theory is well-known and may be found in
\cite{Vinogradov*54}.
The proof follows from a clever use of the mean-value theorem (see
Theorem~5.6 of 
\cite{Bordelles*12} for instance).
\begin{thm}{Theorem (First derivative test)}

Let $f \in C^1 [N,2N]$ such that there exist $\lambda_1 >0$ and $c_1 \geqslant 1$ such that, for all $x \in [N,2N]$, we have
$$
   \lambda_1 \leqslant \bigl | f'(x) \bigr | \leqslant c_1 \lambda_1.
$$
Then
$$
  \mathcal {R}(f,N,\delta) \leqslant 2 c_1 N \lambda_1 + 4c_1 N \delta + \frac{2 \delta}{\lambda_1} + 1.$$
\end{thm}



This result is useful when $\lambda_1$ is very small, so that the condition is
in general too restrictive in the applications. Using a rather neat
combinatorial trick,
\cite{Huxley*96}
succeeded in passing from the first
derivative to the second derivative. This reduction step enables him to apply
this theorem to a function being approximatively of the same order of
magnitude as $f'$. This provides the following useful result.   

\begin{thm}{Theorem (Second derivative test)}

Let $f \in C^2[N,2N]$ such that there exist $\lambda_2 >0$ and $c_2 \geqslant
  1$ such that, for all $x \in [N,2N]$, we have 
$$
   \lambda_2 \leqslant \bigl | f''(x) \bigr | \leqslant c_2 \lambda_2
  \quad\text{and}\quad N \lambda_2 \geqslant c_2^{-1}.
$$
Then
$$\mathcal {R}(f,N,\delta) \leqslant 6 \left\lbrace \left( 3 c_2 \right)^{1/3} N \lambda_2^{1/3} + \left(12 c_2 \right)^{1/2} N \delta^{1/2} + 1 \right\rbrace.$$
\end{thm}


Both hypotheses above are often satisfied in practice, so that this result may
be considered as the first useful tool of the theory. A proof of
this Theorem may be found in Theorem 5.8 of
\cite{Bordelles*12}.


Using a $k$th version of Huxley's reduction principle may allow us to
generalize the above results. A better way is to split the integer points into
two classes, namely the major arcs in which the points belong to a
same algebraic curve of degree $\le k-1$, and the minor arcs. The points
coming from the minor arcs are treated by divided differences techniques,
generalizing the proof of both theorems above and, by a careful analysis of the
points belonging to major arcs,
\cite{Huxley-Sargos*95}
and
\cite{Huxley-Sargos*06}
succeeded in
proving the following fundamental result. A proof of an explicit version may
be found in Theorem 5.12 of
\cite{Bordelles*12}.

\begin{thm}{Theorem ($k$th derivative test)}

Let $k \geqslant 3$ be an integer and $f \in C^k [N,2N]$ such that there exist
  $\lambda_k >0$ and $c_k \geqslant 1$ such that, for all $x \in [N,2N]$, we
  have 
$$
   \lambda_k \leqslant \bigl | f^{(k)}(x) \bigr | \leqslant c_k \lambda_k. \label{e4}
$$
Let $\delta \in \left( 0,\frac{1}{4} \right)$. Then
$$
  \mathcal {R}(f,N,\delta) \leqslant \alpha_k N \lambda_k^{\frac{2}{k(k+1)}} +
  \beta_k N \delta^{\frac{2}{k(k-1)}} + 8k^3 \left( \frac{\delta}{\lambda_k}
  \right)^{1/k} + 2 k^2 \left( 5e^3 + 1 \right)
  $$
where
$$
  \alpha_k = 2k^2 c_k^{\frac{2}{k(k+1)}} \quad \text{and} \quad \beta_k = 4
  k^2 \left( 5 e^3 c_k^{\frac{2}{k(k-1)}} + 1 \right).
  $$
\end{thm}





\section{Bounds using exponential sums techniques}


The next result leads us to estimate $\mathcal {R}(f,N,\delta)$ with the help
of exponential sums (see 
\cite{Graham-Kolesnik*91}
for instance), which have been extensively
studied in the 20th century by many specialists, such as van der Corput,
Weyl or Vinogradov. Nevertheless, even using the finest exponent pairs to
date, the result generally does not significantly improve on the previous
estimates seen above. A simple proof of the following inequality may be found
in 
\cite{Filaseta*90}.

\begin{thm}{Theorem ($k$th derivative test)}

Let $f : [N,2N] \longrightarrow \mathbb {R}$ be any function and $\delta \in
  \left( 0,\frac{1}{4} \right)$. Set $K = \left \lfloor (8 \delta)^{-1} \right
  \rfloor +1$. Then, for any positive integer $H \leqslant K$, we have 
$$\mathcal {R}(f,N,\delta) \leqslant \frac{4N}{H} + \frac{4}{H} \sum_{h=1}^{H}
  \left | \sum_{N \leqslant n \leqslant 2N} e(hf(n)) \right |.$$ 
\end{thm}




\section{Integer points on curves}



This last part is somewhat out of the scope of the TME-EMT project, but may
help the reader in orienting him/herself in the litterature.

\par 
When $\delta \longrightarrow 0$, we are led to counting the number of integer
points lying on curves, and we denote this number by $\mathcal
{R}(f,N,0)$. This problem goes back to Jarn\'ik
\cite{Jarnik*25}
who proved that a
strictly convex arc $y=f(x)$ with length $L$ has at most 
$$\leqslant \frac{3}{(2 \pi)^{1/3}} \, L^{2/3} + O \left( L^{1/3} \right)$$ 
integer points and this is a nearly best possible result under the sole
hypothesis of convexity. However, 
\cite{Swinnerton-Dyer*74}
proved that if $f
\in C^3[0,N]$ is such that $|f(x)| \leqslant N$ and $f'''(x) \neq 0$ for all
$x \in [0,N]$, then the number of integer points on the arc $y=f(x)$ with $0
\leqslant x \leqslant N$ is $\ll N^{3/5+\varepsilon}$. This result was later
generalized by
\cite{Bombieri-Pila*89}
who showed among
other things the following estimate.  

\begin{thm}{Theorem (1989)}

Let $N \geqslant 1$, $k \geqslant 4$ be integers and define $K = \binom{k+2}{2}$. Let $\mathcal{I}$ be an interval with length $N$ and $f \in C^K (\mathcal{I})$ satisfying $|f'(x)| \leqslant 1$, $f''(x) >0$ and such that the number of solutions of the equation $f^{(K)} (x) = 0$ is $\leqslant m$. Then there exists a constant $c_0 = c_0(k) >0$ such that
$$\mathcal {R}(f,N,0) \leqslant c_0 (m+1) N^{1/2+3/(k+3)}.$$
\end{thm}











  
\begin{flushright}\small\sl{}   Last updated on July 23rd, 2012, by Olivier Bordell\`es
 \end{flushright}















