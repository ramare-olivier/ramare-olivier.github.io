\chapter{  Explicit upper bounds for some special arithmetic functions}

Corresponding html file: \texttt{../Articles/Art12.html}










 
 

The following bounds may be useful is applications.

From
\cite{Robin*83-1}:

\begin{thm}{Theorem (1983)}

For any integer $n\ge3$, the number of prime divisors $\omega(n)$ of $n$ satisfies:
  $$\omega(n)\le 1.3841\frac{\log n}{\log\log n}.$$
\end{thm}


From
\cite{Nicolas-Robin*83}:

\begin{thm}{Theorem (1983)}

For any integer $n\ge3$, the number $\tau(n)$  of divisors of $n$ satisfies:
  $$\tau(n)\le n^{1.538\,\log 2/\log\log n}.$$
\end{thm}


From page 51 of \cite{Robin*83-0}:

\begin{thm}{Theorem (1983)}

For any integer $n\ge3$, we have
  $$\tau_3(n)\le n^{1.59141\,\log 3/\log\log n}$$
  where $\tau_3(n)$ is the number of triples $(d_1,d_2,d_3)$ such that $d_1d_2d_3=n$.
\end{thm}


The PhD
memoir
\cite{Duras*93}
contains result concerning the maximum
of $\tau_k(n)$, i.e. the number of $k$-tuples $(d_1,d_2,\ldots, d_k)$
such that $d_1d_2\cdots d_k=n$, when $3\le k\le 25$.

\par 
  From
  \cite{Duras-Nicolas-Robin*99}:

\begin{thm}{Theorem (1999)}

For any integer $n\ge1$, any real number $s>1$ and any integer $k\ge1$, we have
  $$\tau_k(n)\le n^s\zeta(s)^{k-1}$$
  where $\tau_k(n)$ is the number of $k$-tuples $(d_1,d_2,\cdots,d_k)$ such
  that $d_1d_2\cdots d_k=n$.
\end{thm}

The same paper also announces the bound for $n\ge3$ and $k\ge2$
$$
\tau_k(n)\le n^{a_k\log k/\log\log k}
$$
where $a_k=1.53797\log k / \log 2$ but the proof never appeared.

From \cite{Nicolas*08}:

\begin{thm}{Theorem (2008)}

For any integer $n\ge3$, we have
  $$\sigma(n)\le 2.59791\, n\log\log(3\tau(n)),$$
  $$\sigma(n)\le n\{ e^\gamma\log\log(e\tau(n))+\log\log\log(e^e\tau(n))+0.9415\}.$$
\end{thm}


The first estimate above is a slight improvement of the bound
  $$\sigma(n)\le 2.59 n\log\log n\quad(n\ge7)$$
obtained in
\cite{Ivic*77}.
In this same paper,
  the author proves that
$$\sigma^*(n)\le \frac{28}{15} n\log\log n\quad(n\ge31)$$
where $\sigma^*(n)$ is the sum of the unitary divisors of $n$, i.e. divisors
  $d$ of $n$ that are such that $d$ and $n/d$ are coprime.

\par 
\par 
\par 
On this subject, the readers may consult the web site

Computation about the paper The sum of divisors function and the
Riemann hypothesis
.





  
\begin{flushright}\small\sl{}   Last updated on September 2nd, 2021, by Olivier Ramar\'e
 \end{flushright}














