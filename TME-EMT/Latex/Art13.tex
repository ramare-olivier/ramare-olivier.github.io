\chapter{  Explicit bounds for class numbers}

Corresponding html file: \texttt{../Articles/Art13.html}









Let $K$ be a number field of degree $n\ge2$, signature $(r_1,r_2)$, absolute
value of discriminant $d_K$, class number $h_K$, regulator $\mathcal{R}_K$ and
$w_K$ the number of roots of unity in $K$. We further denote by $\kappa_K$ the
residue at $s=1$ of the Dedekind zeta-function $\zeta_K(s)$ attached to $K$.
\par \par 
Estimating $h_K$ is a long-standing problem in algebraic number theory.


\section{Majorising $h_K\mathcal{R}_K$}




One of
the classic way is the use of the so-called analytic class number
formula stating that
$$
h_K\mathcal{R}_K=\frac{w_K \sqrt{d_K}}{2^{r_1}(2\pi)^{r_2}}\kappa_K
$$
and to use Hecke's integral representation of the Dedekind zeta function to
bound $\kappa_K$. This is done in
\cite{Louboutin*00} and in
\cite{Louboutin*01} with additional
properties of log-convexity of some functions related to $\zeta_K$ and enabled
Louboutin to reach the following bound:
$$
h_K\mathcal{R}_K
\le\frac{w_K}{2}\left(\frac{2}{\pi}\right)^{r_2}
\left(\frac{e\log d_K}{4n-4}\right)^{n-1}\sqrt{d_K}.
$$
Furthermore, if $\zeta_K(\beta)=0$ for some $\tfrac12\le \beta< 1$, then we have 
$$
h_K\mathcal{R}_K
\le(1-\beta)w_K\left(\frac{2}{\pi}\right)^{r_2}
\left(\frac{e\log d_K}{4n}\right)^{n}\sqrt{d_K}.
$$
When $K$ is abelian, then the residue $\kappa_K$ may be expressed as a
product of values at $s=1$ of $L$-functions associated to primitive Dirichlet
characters attached to $K$. On using estimates for such $L$-functions from
\cite{Ramare*01}, we get for instance
$$
h_K\mathcal{R}_K
\le
\frac{w_K}{2}\left(\frac{2}{\pi}\right)^{r_2}
\left(\frac{\log d_K}{4n-4}+\frac{5-\log 36}{4}\right)^{n-1}\sqrt{d_K}.
$$
Note that the constant $\frac14(5-\log 36)=0.354\cdots$ can be improved upon
in many cases. For instance, when $K$ is abelian and totally real (i.e. $r_2=0$), a result
from
\cite{Ramare*01} implies that the
constant may be replaced by 0, so that
$$
h_K\mathcal{R}_K
\le
\left(\frac{\log d_K}{4n-4}\right)^{n-1}\sqrt{d_K}.
$$


\section{Majorising $h_K$}



One may also estimate $h_K$ alone, without any contamination by the regulator
since this contamination is often difficult to control,
see \cite{Pohst-Zassenhaus*89}.

In this case, one
rather uses explicit bounds for the Piltz-Dirichlet divisor functions $\tau_n$
(see
\cite{Bordelles*02}
and
\cite{Bordelles*06})
and get
$$
h_K\le \frac{M_K}{(n-1)!}
\left(\frac{\log\bigl(M^2_Kd_K\bigr)}{2}+n-2\right)^{n-1}
\sqrt{d_K}
$$
as soon as
$$
n\ge 3,\quad d_K\ge 139 M_K^{-2}
\quad\text{where}\quad
M_K=(4/\pi)^{r_2}n!/n^n.
$$
The constant $M_K$ is known as the Minkowski constant of K.

In
\cite{Cully-Trudgian*21}
we find the following.
\begin{thm}{Theorem (2021)}

  Ler $K$ be a quartic number field with class number $h_K$ and
  Minkowski bound $b$. Then if $b\ge 193$, we have
  $h_K\le (1/3) x(\log x)^3$.
\end{thm}


\section{Using the influence of small primes}



It is explained in
\cite{Louboutin*05} how the
behavior of certain small primes may subtantially improve on the previous
bounds. To make things more significant, define, for a rational prime $p$,
$$
\Pi_K(p)=\prod_{\mathfrak{p}|p}\left(1-\frac{1}{\mathcal{N}_K(\mathfrak{p})}\right)^{-1}.
$$
From \cite{Louboutin*05}, we have
among other things
$$
h_K\mathcal{R}_K
\le\frac{w_K}{2}
\left(\frac{2}{\pi}\right)^{r_2}
\frac{\Pi_K(2)}{\Pi_{\mathbb{Q}}(2)^n}
\left(\frac{e\log d_K}{4n-4}\times
e^{n\log 4/\log d_K}
\right)^{n-1}\sqrt{d_K}
$$
where $K$ is any number field of degree $n\ge3$. In particular, when $2$ is
inert in $K$, then
$$
h_K\mathcal{R}_K
\le\frac{w_K}{2(2^n-1)}
\left(\frac{2}{\pi}\right)^{r_2}
\left(\frac{e\log d_K}{4n-4}\times
e^{n\log 4/\log d_K}
\right)^{n-1}\sqrt{d_K}.
$$




\section{The $h^-_K$ of CM-fields}



Let $K$ be here a CM-field of degree $2n > 2$, i.e. a totally complex
quadratic extension $K$ of its maximal totally real subfield $K^+$. it is well
known that $h_{K^+}$ divides $h_K$. The quotient is denoted by $h^-_K$ and is
called the relative class number of $K$. The analytic class number
formula yields
$$
h^-_K=\frac{Q_Kw_K}{(2\pi)^n}
\left(\frac{d_K}{d_{K^+}}\right)^{1/2}
\frac{\kappa_K}{\kappa_{K^+}}
=
\frac{Q_Kw_K}{(2\pi)^n}
\left(\frac{d_K}{d_{K^+}}\right)^{1/2}
L(1,\chi)
$$
where $\chi$ is the quadratic character of degree 1 attached to the extension
$K/K^+$ and $Q_K\in\{1,2\}$ is the Hasse unit index of $K$.
Here are three results originating in this formula.

From \cite{Louboutin*00}:
\begin{thm}{Theorem (2000)}

  We have
  $$
  h^-_K
  \le
  2Q_Kw_K\left(\frac{d_K}{d_{K^+}}\right)^{1/2}
  \left(
  \frac{e\log(d_K/d_{K^+})}{4\pi n}
  \right)^n.
  $$
\end{thm}


From \cite{Louboutin*03}:
\begin{thm}{Theorem (2003)}

  Assume that $(\zeta_K/\zeta_{K^+})(\sigma)\ge0$ whenever $0 < \sigma <
  1$. Then we have
  $$
  h^-_K
  \ge
  \frac{Q_Kw_K}{\pi e \log d_K}
  \left(\frac{d_K}{d_{K^+}}\right)^{1/2}
  \left(
  \frac{n-1}{\pi e\log d_K}
  \right)^{n-1}.
  $$
\end{thm}


Again from \cite{Louboutin*03}:
\begin{thm}{Theorem (2003)}

  Let $c=6-4\sqrt{2}=0.3431\cdots$. Assume that $d_K\ge 2800^n$ and that
  either $K$ does not contain any imaginary quadratic subfield, or that the
  real zeros in the range $1-\frac{c}{\log d_N}\le \sigma < 1$ of the Dedekind
  zeta-functions of the imaginary quadratic subfields of $K$ are nor zeros of
  $\zeta_K(s)$, where $N$ is the normal closure of $K$. Then we have
  $$
  h^-_K
  \ge
  \frac{cQ_Kw_K}{4ne^{c/2}[N:\mathbb{Q}]}
  \left(\frac{d_K}{d_{K^+}}\right)^{1/2}
  \left(
  \frac{n}{\pi e\log d_K}
  \right)^{n}.
  $$
\end{thm}


And a third result from \cite{Louboutin*03}:
\begin{thm}{Theorem (2003)}

  Assume $n > 2$, $d_K > 2800^n$ and that
  $K$ contains an imaginary quadratic subfield $F$ such that
  $\zeta_F(\beta)=\zeta_K(\beta)=0$ for some $\beta$ satisfying
  $1-\frac{2}{\log d_K}\le \beta < 1$.
  Then we have
  $$
  h^-_K
  \ge
  \frac{6}{(\pi e)^2}
  \left(\frac{d_K}{d_{K^+}}\right)^{1/2-1/n}
  \left(
  \frac{n}{\pi e\log d_K}
  \right)^{n-1}.
  $$
\end{thm}






 
 







  
\begin{flushright}\small\sl{}   Last updated on August 23rd, 2012, by Olivier Bordell\`es
 \end{flushright}














