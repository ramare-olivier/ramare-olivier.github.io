\chapter{  Bounds on the Dedekind zeta-function}

Corresponding html file: \texttt{../Articles/Art18.html}










 
 


\par 
\section{Size }


The knowledge on the general Dedekind zeta is less accomplished than
the one of the Riemann zeta-function, but we still have interesting
results. Theorem 4 of \cite{Rademacher*59} gives
the convexity bound. See also section 4.1 of
\cite{Trudgian*13}.
\par 
\begin{thm}{Theorem (1959)}

In the strip $-\eta\le \sigma\le 1+\eta$, $0 < \eta\le 1/2$, the Dedekind zeta
function $\zeta_K(s)$ belonging to the algebraic number field $K$ of degree
$n$ and discriminant $d$ satisfies the inequality
$$
|\zeta_K(s)|\le 3 \left|\frac{1+s}{1-s}\right|
\left(\frac{|d||1+s|}{2\pi}\right)^{\frac{1+\eta-\sigma}{2}}
\zeta(1+\eta)^n.
$$
\end{thm}


\par 

\section{Zeroes and zero-free regions }


We denote by $N_K(T)$ the number of zeros $\rho$, of the Dedekind
zeta-function of the number field $K$ of degree $n$ and discriminant
$d_K$,
zeros that lie in the critical strip
$0 < \Re \rho = \sigma < 1$ and which verify $|\Im \rho|\le T$.
After a first result in
\cite{Kadiri-Ng*12},
we find in
\cite{Trudgian*14-1}
the following result.

\par 
\begin{thm}{Theorem (2014)}

 When $T\ge1$, we have
 $N_K(T)=\frac{T}{\pi}\log\Bigl(|d_K|\Big(\frac{T}{2\pi e}\Bigr)^n\Bigr)
 +O^*\bigl(0.316(\log |d_K|+n\log T)+5.872 n+3.655\bigr)$ .
\end{thm}


This is improved in 
\cite{Hasanalizade-Shen-Wong*21}
into:

\par 
\begin{thm}{Theorem (2021)}

 When $T\ge1$, we have
 $N_K(T)=\frac{T}{\pi}\log\Bigl(|d_K|\Big(\frac{T}{2\pi e}\Bigr)^n\Bigr)
 +O^*\bigl(0.228(\log |d_K|+n\log T)+23.108 n+4.520\bigr)$ .
\end{thm}


In
\cite{Kadiri*12},
a zero-free region is proved.

\par 
\begin{thm}{Theorem (1959)}

Let $K$ be a number field of degree $n$ over $\mathbb{Q}$ and of
discriminant $d \ge 2$. The associated Dedekind
zeta-function $\zeta_K$ has no zeros in the region
$$
\sigma\ge 1-\frac{1}{12.55\log|d_K|+n(9.69\log|t|+3.03)+58.63}, |t|\ge1
$$
and at most one zero in the region
$$
\sigma\ge 1-\frac{1}{12.74\log|d_K|}, |t|\le 1.
$$
The exceptional zero, if it exists, is simple and real.
\end{thm}

See
\cite{Ahn-Kwon*14}
for a result for Hecke $L$-series.









  
\begin{flushright}\small\sl{}   Last updated on April 29th, 2022, by Olivier Ramar\'e
 \end{flushright}














