\documentclass[10pt,twoside, svgnames]{book}
\usepackage{amsfonts, amsmath}
\usepackage{amssymb, url}
\usepackage{latexsym}
\usepackage{mathrsfs}
\newenvironment{thm}[1]{\begin{trivlist}
\item[{\bfseries #1}]}{\end{trivlist}}

\usepackage{csquotes}
\usepackage[%
    backend=biber,
    style=numeric-comp,
    citetracker=true,
    pagetracker=true,
    hyperref=true,
    backref=true,
    firstinits=true,
    bibencoding=utf8
    ]{biblatex}

%\newif\iffootnote
%\let\Footnote\footnote
%\renewcommand\footnote[1]{\begingroup\footnotetrue\Footnote{#1}\endgroup}

\makeatletter
\renewbibmacro*{cite:comp}{%
  \xdef\cbx@citekey{\thefield{entrykey}}%
  \addtocounter{cbx@tempcntb}{1}%
  \ifciteseen
    {}
    {\csnumgdef{cbx@instcount\cbx@citekey}{-100}}%  
  \ifsamepage{\value{instcount}}{\number\csuse{cbx@instcount\cbx@citekey}}%
    {\usebibmacro{author}, }
    {\renewcommand{\@makefntext}[1]{\noindent\normalfont##1}%
     \usebibmacro{author},
     \footnotetext{
       \printtext[labelnumberwidth]{%
         \printfield{labelprefix}%
         \printfield{labelnumber}}%
       \addspace
       \renewcommand{\newunitpunct}{\addcomma\addspace}
       \bibfootnotewrapper{%\fullcite{\thefield{entrykey}}%
         \printfield{labelname}\usebibmacro{author}\newunit\printfield{year}\newunit\printfield{title}\addperiod}}}%
   \csnumgdef{cbx@instcount\cbx@citekey}{\value{instcount}}%
   \iffieldundef{shorthand}
    {\ifbool{bbx:subentry}
       {\iffieldundef{entrysetcount}
          {\usebibmacro{cite:comp:comp}}
          {\usebibmacro{cite:comp:inset}}}
       {\usebibmacro{cite:comp:comp}}}
    {\usebibmacro{cite:comp:shand}}}
\makeatletter

\addbibresource{Local-TME-EMT.bib}

\usepackage{placeins}
\usepackage{epic}
\usepackage{graphicx}
\usepackage{epstopdf}
\usepackage{pstricks}
\usepackage{pst-plot}

\usepackage[colorlinks=true, pdftex]{hyperref}
\hypersetup{citecolor=blue, linkcolor=blue}
\renewcommand{\sectionmark}[1]{\markright{\thesection\ #1}}%rappel du titre de la section

%%%%%%%%%%%%%%%%%%%%%%%%%%%%%%%%%%%%%%%%%%%%%%%%%%%%%%%
%%%%%%%%%%%%%%%%%%%%%%%%%%%%%%%%%%%%%%%%%%%%%%%%%%%%%%%
%%%%%%%%%%%%%%%%%%%%%%%%%%%%%%%%%%%%%%%%%%%%%%%%%%%%%%%
\usepackage{fancyhdr}
\pagestyle{fancy} 
 % ceci permet d'avoir les noms de chapitre et section en minuscules
\renewcommand{\sectionmark}[1]{\markright{\thesection\ #1}}
  %rappel du titre de la section
\renewcommand{\subsectionmark}[1]{\markright{\thesubsection\ #1}}
  %rappel du titre de la section

\fancyhf{} \fancyhead[LE,RO]{\bfseries\thepage}
  %met en gras les entetes definies gauches paires et droite impaires
\fancyhead[LO]{\bfseries{\sc \rightmark}}%definit l'entete gauche impaire
\fancyhead[RE]{\bfseries{\small \sc \leftmark}}%definit l'entete droite paire
%%%% Les trois suivantes sont dans le fichier tex !!
\fancyfoot[LE,RO]{{\small \today}}%\copyright\ 2005}}
      %definit le pied de page gauche pair et droit impair 
%%%%%
\renewcommand{\headrulewidth}{0.5pt} %definit une ligne a la suite de l'entete
\renewcommand{\footrulewidth}{0.5pt} 
  %definit une ligne a la suite du pied de page
\addtolength{\headheight}{0.5pt} % espace pour le filet
\fancypagestyle{plain}{ %pages de tetes de chapitre :
  % redefinition du style plain pour avoir un debut de chapitre different
    \fancyhead{} %supprime l'entete
    \fancyfoot{} %supprime le pied de page
    \renewcommand{\headrulewidth}{0pt} %le filet haut
    \renewcommand{\footrulewidth}{0pt}} %le filet bas
%%%%%%%%%%%%%%%%%%%%%%%%%%%%%
\fancyfoot[LO]{{\sl The TME-EMT project, 2021}}
      %definit le pied de page gauche impair
\fancyfoot[RE]{\date}%definit le pied de page droit pair
%%%%%%%%%%%%%%%%%%%%%%%%%%%%%%%%%%%%%%%%%%%%%%%%%%%%%%%
%%%%%%%%%%%%%%%%%%%%%%%%%%%%%%%%%%%%%%%%%%%%%%%%%%%%%%%
%%%%%%%%%%%%%%%%%%%%%%%%%%%%%%%%%%%%%%%%%%%%%%%%%%%%%%%


\nonstopmode

\title{{\textsc{\Huge The TME-EMT Project
         \\---\\[1ex]
      The PDF Notes\\[2ex]
       ---}}\\[1ex]
       %Version 2\\
       {\huge Th\'eorie Multiplicative Explicite des nombres
         \\ ---\\[1ex]
         Explicit Multiplicative number Theory
         }
     \\[1ex]
       ---}

\author{Contributors:\\ {\sl Olivier Bordell\`es, Pierre Dusart,
    Charles Greathouse}\\ {\sl Harald Helfgott, Pieter
  Moree, Akhilesh P.,  Olivier Ramar\'e} \\ {\sl \& Enrique Trevi\~no}}

\begin{document}
\frontmatter
\maketitle
%\relax{}
%\relax

%\hypersetup{pageanchor=true}

\mainmatter
\chapter*{Introduction}

Fully explicit results in multiplicative number theory are often
scattered through the litterature. The aim of this site is to present
an annoted bibliography in order to keep track of the current
knowledge. By the way, the acronym TME-EMT stands for 
\begin{center}
    Th\'eorie Multiplicative Explicite des nombres  

    ---

    Explicit
    Multiplicative number Theory 
\end{center}

Being up-to date is a difficulty, so please do not consider this site
as presenting the best available, but as a first line on which to
strengthen more specialized inquiries.

  {\small \tableofcontents\par}

\vfill
\section*{Notation}

Notation is standard, except maybe for the following one: we write
$f=O^*(g)$ to say that $|f|\le g$. This is simply a Landau-bigO symbol
with an implied constant equal to one. Furthermore, the letter p
always denotes a prime variable.

\part{Averages of arithmetical functions}

\chapter{   Explicit bounds on primes}

Corresponding html file: \texttt{../Articles/Art01.html}










 
 


Collecting references:
\cite{Dusart*98},
\cite{Dusart*07},

\section{Bounds on primes, in special ranges}

The paper
\cite{Rosser-Schoenfeld*62},
contains several bounds valid only when the variable is small enough.

In 
\cite{Buthe*16},  the
author proves the next theorem.
\par 
\begin{thm}{Theorem (2016)}

    Assume the Riemann Hypothesis has been checked up to height
    $H_0$. Then when $x$ satisfies $\sqrt{x/\log x}\le H_0/4.92$, we have
  \begin{itemize}
    \item $|\psi(x)-x|\le \frac{\sqrt{x}}{8\pi}\log^2x$ when $x > 59$,

    \item $|\theta(x)-x|\le \frac{\sqrt{x}}{8\pi}\log^2x$ when $x > 599$,

    \item $|\pi(x)-\text{li}(x)|\le \frac{\sqrt{x}}{8\pi}\log x$ when
    $x > 2657$.

					\end{itemize}
\end{thm}

If we use the value $H_0=30\,610\,046\,000$ obtained by D. Platt
in
\cite{Platt*17}, these
bounds are thus valid for $x\le 1.8\cdot 10^{21}$.

In
\cite{Buthe*18},
the following bounds are also obtained.
\par 
\begin{thm}{Theorem (2018)}

    We have
  \begin{itemize}
    \item $|\psi(x)-x|\le 0.94\sqrt{x}$ when $11 < x\le 10^{19}$,

   \item $0<\text{li}(x)-\pi(x)\le\frac{\sqrt{x}}{\log
   x}\Bigl(1.95+\frac{3.9}{\log x}+\frac{19.5}{\log^2x}\Bigr)$  when
   $2\le x\le 10^{19}$.

\end{itemize}
\end{thm}



\section{Bounds on primes, without any congruence condition}


The subject really started with the four papers
\cite{Rosser*41},
\cite{Rosser-Schoenfeld*62},
\cite{Rosser-Schoenfeld*75}
and
\cite{Schoenfeld*76}.
We recall the usual notation: $\pi(x)$ is the number of primes up to
$x$ (so that $\pi(3)=2$), the function $\psi(x)$ is the summatory
function of the van Mangold function $\Lambda$,
i.e. $\psi(x)=\sum_{n\le x}\Lambda(n)$, while we also define
$\vartheta(x)=\sum_{p\le x}\log p$.
Here are some elegant bounds that one can find in these papers.
\par 
\begin{thm}{Theorem (1962)}

  \begin{itemize}
    \item For $x > 0$, we have
      $\psi(x)\le 1.03883 x$ and the maximum of $\psi(x)/x$ is
      attained at $x=113$.

    \item When $x\ge17$, we have $\pi(x) > x/\log x$.

    \item When $x > 1$, we have $\displaystyle \sum_{p\le x}1/p >  \log\log x$.

    \item When $x > 1$, we have $\displaystyle \sum_{p\le x}(\log p)/p <
									\log x$.

									\end{itemize}
\end{thm}

There are many other results in these papers.
In
\cite{Dusart*99-1},
on can find among other things the inequality
$$
\text{When $x\ge17$, we have } \pi(x) > \frac{x}{\log x -1}.
        $$
        And also
\par 
\begin{thm}{Theorem (1999)}

  \begin{itemize}
    \item When $x\ge e^{22}$, we have
  $\displaystyle\psi(x)=x+O^*\Bigl(0.006409\frac{x}{\log
      x}\Bigr)$.

    \item  When $x\ge 10\,544\,111$, we have $\displaystyle\vartheta(x)=x+O^*\Bigl(0.006788\frac{x}{\log
      x}\Bigr)$.

    \item  When $x\ge 3\,594\,641$, we have $\displaystyle\vartheta(x)=x+O^*\Bigl(0.2\frac{x}{\log^2
      x}\Bigr)$.

    \item  When $x > 1$, we have $\displaystyle\vartheta(x)=x+O^*\Bigl(515\frac{x}{\log^3
      x}\Bigr)$.

    \end{itemize}
\end{thm}

This is improved in
\cite{Dusart*16},
and in particular, it is shown that the 515 above can be replaced by
20.83 and also that
$$
\text{When $x\ge 89\,967\,803$, we have } \vartheta(x)=x+O^*\Bigl(\frac{x}{\log^3
x}\Bigr).
$$

Bounds of the shape $|\psi(x)-x|\le \epsilon x$ have started appearing
in
\cite{Rosser-Schoenfeld*62}.
The latest paper is
\cite{Kadiri-Faber*13}
with its corrigendum
\cite{Kadiri-Faber*18},
where the explicit density estimate from
\cite{Kadiri*13}
is put to contribution, even for moderate
values of the variable. In particular
$$
\text{When $x\ge 485\,165\,196$, we have } \psi(x)=x+O^*(0.00053699\,x).
$$

In
\cite{Platt-Trudgian*21b},
one find the following estimate
\par 
\begin{thm}{Theorem (2021)}

  When $x\ge e^{2000}$, we have
  $\biggl|\frac{\psi(x)-x}{x}\biggr|\le 235\,(\log
  x)^{0.52}\exp-\sqrt{\frac{\log x}{5.573412}}\;.$



Refined bounds for $\pi(x)$ are to be found in
\cite{Panaitopol*00}
and in
\cite{Axler*16}.



By comparing in an efficient manner with $\psi(x)-x$,
\cite{Ramare*12-1},
obtained the next two results.

\par 
Theorem (2013)


  For $x > 1$, we have
  $\sum_{n\le x}\Lambda(n)/n=\log x-\gamma+O^*(1.833/\log^2x)$.
  When $x\ge 23$, we can replace the error term by $O^*(0.0067/\log
  x)$.
  Furthermore, when $1\le x\le 10^{10}$, this error term can be
  replaced by $O^*(1.31/\sqrt{x})$. 
\end{thm}


\par 
\begin{thm}{Theorem (2013)}

  For $x\ge 8950$, we have
  $$
  \sum_{n\le x}\Lambda(n)/n=\log x-\gamma
  +\frac{\psi(x)-x}{x}+O^*\Bigl(\frac{1}{2\sqrt{x}}+1.75\cdot 10^{-12}\Bigl)
  $$.
\end{thm}



\cite{Vanlalnagaia*15-1}
developed the method to incorporate more functions (and corrected the
initial work), extending results of
\cite{Rosser-Schoenfeld*62}.

Here are some of his results.

\par 
\begin{thm}{Theorem (2017)}

  For $x\ge 2$, we have
  $$
  \sum_{p\le x}\frac1p=\log\log  x+B+O^*\Bigl(\frac{4}{\log^3x}\Bigr).
  $$
  When $x\ge 1000$, one can replace the 4 in the error term by 2.3,
  and when $x\ge24284$, by 1. The constant $B$ is the expected one.
\end{thm}



\par 
\begin{thm}{Theorem (2017)}

  For $\log x\ge 4635$, we have
  $$
  \sum_{p\le x}\frac1p=\log\log
  x+B+O^*\Bigl(1.1\frac{\exp(-\sqrt{0.175\log x})}{(\log x)^{3/4}}\Bigr).
  $$
\end{thm}






\par 
When truncating sums over primes, Lemma 3.2 of
\cite{Ramare*13d}
is handy.
\par 
\begin{thm}{Theorem (2016)}

  Let $f$ be a C${}^1$ non-negative, non-increasing function over
  $[P,\infty[$, where $P\ge 3\,600\,000$ is a real number and such
  that $\lim_{t\rightarrow\infty}tf(t)=0$. 
  We have
  \begin{equation*}
    \sum_{p\ge P} f(p)\log p
    \le (1+\epsilon) \int_P^\infty f(t) dt  +  \epsilon P f(P)  +  P
  f(P) / (5 \log^2 P) 
  \end{equation*}
  with $\epsilon=1/914$. When we can only ensure $P\ge2$, then a similar
  inequality holds, simply replacing the last $1/5$ by a 4.
\end{thm}


The above result  relies on (5.1*) of
\cite{Schoenfeld*76}
because it is easily accessible. However on using
Proposition 5.1 of
\cite{Dusart*07},
one has access to $\epsilon=1/36260$.

\par 
  \par 
Here is a result due to 
\cite{Trevino*12}.
\par 
\begin{thm}{Theorem (2012)}

For $x$ a positive real number. If $x \geq x_0$ then there exist $c_1$
and $c_2$ depending on $x_0$ such that
$$
\frac{x^2}{2\log{x}} +
\frac{c_1 x^2}{\log^2{x}} \leq \sum_{p \leq x} p \leq
\frac{x^2}{2\log{x}} + \frac{c_2 x^2}{\log^2{x}}.
$$
The constants
$c_1$ and $c_2$ are given for various values of $x_0$ in the next
table.

  
  
    
      $x_0$
      $c_1$
      $c_2$
    
  
  
    315437
    0.205448
    0.330479
  
  
    468577
    0.211359
    0.32593
  
  
    486377
    0.212904
    0.325537
  
  
    644123
    0.21429
    0.322609
  
  
    678407
    0.214931
    0.322326
  
  
    758231
    0.215541
    0.321504
  
  
    758711
    0.215939
    0.321489
  
  
    10544111
    0.239818
    0.29251
  
  

\end{thm}

\par 


\section{Bounds containing $n$-th prime}



Denote by $p_n$  the $n$-th prime. Thus $p_1=2,\;p_2=3,\; p_4=5,\cdots$.

The classical form of prime number theorem yields easily
$p_n \sim n \log n.$ 

\cite{Rosser*38}
shows that this equivalence does not oscillate
by proving that $p_n$ is greater than $n\log n$ for $n\geq 2$.  


The asymptotic formula for $p_n$ can be developped as shown in
\cite{Cipolla*02}:
$$
p_n\sim n\left(\log n+\log\log n -1+\frac{\log\log n-2}{\log n}
-\frac{(\ln\ln n)^2-6\log\log n +11}{2\log^2 n}+\cdots\right).
$$

This asymptotic expansion is the inverse of the logarithmic integral
$\mbox{Li}(x)$ obtained by series reversion. 


 But
\cite{Rosser*38}
also proved  that for every $n> 1$:
$$
n (\log n + \log \log n - 10) < p_n < n (\log n + \log\log n +8).
$$
He improves these results in
\cite{Rosser*41}
:  for every $n\geq 55$,
$$
n (\log n + \log \log n - 4) < p_n < n (\log n + \log\log n +2).
$$

 This result was subsequently improved by Rosser and Schoenfeld
\cite{Rosser-Schoenfeld*62}
 in 1962 to
$$
n (\log n + \log \log n - 3/2) < p_n < n (\log n + \log\log n -1/2),
$$
for $n > 1$ and $n > 19$  respectively.

The constants  were subsequently reduced by
\cite{Robin*83-1}.
In particular, the lower bound 
$$
n (\log n + \log \log n - 1.0072629) < p_n
$$
is valid for $n>1$ and the constant $1.0072629$ can be replaced by 1 for
$p_k\leq 10^{11}$.
Then 
\cite{Massias-Robin*96}
  showed that the lower bound constant equals to 1 was admissible for
$p_k < e^{598}$
and $p_k > e^{1800}$. Finally,  
\cite{Dusart*99-2}
showed
that
$$
  n(\log n - \log \log n - 1) < p_n
$$ for all $n > 1$, and also that
$$
p_n < n (\log n + \log\log n - 0.9484)
$$ for $n > 39017$ i.e. $p_n > 467\,473$.

In
\cite{Carneiro-Milinovich-Soundararajan*19},
the authors  prove the next result.
\begin{thm}{Theorem (2019)}

Under the Riemann Hypothesis we have $p_{n+1}-p_n\le\frac{22}{25}\sqrt{p_n}\log p_n$.
\end{thm}


\par 
\par 
In 
\cite{DeKoninck-Letendre*20},
we find in passing (Lemma 4.8) the next result.
\par 
\begin{thm}{Theorem (2020)}

  We have
  $$
  \sum_{i\le k}\log p_i\le
  k\bigl(\log k+\log\log -3/4\Bigr)\quad(\text{for $k\ge 4$}),
  $$
  as well as
  $$
  \sum_{i\le k}\log\log p_i\ge
  k\biggl(\log\log k+\frac{\log\log -5/4}{\log k}\biggr)
  \quad(\text{for $k\ge319$}).
  $$
\end{thm}









\section{Bounds on primes in arithmetic progressions}


Collecting references:
\cite{McCurley*84-2},
\cite{McCurley*84-3},
\cite{Ramare-Rumely*96},
\cite{Dusart*01},
Lemma 10 of \cite{Moree*04},
section 4 of
\cite{Moree-teRiele*04}.

A consequence of Theorem 1.1 and 1.2 of
\cite{Bennett-Martin-OBryant-Rechnitzer*18}
states that
\begin{thm}{Theorem (2018)}

  We have
  $\displaystyle
  \max_{3\le q\le 10^4}\max_{ x\ge 8\cdot 10^9}\max_{\substack{1\le a\le q,\\
  (a,q)=1}}
  \frac{\log x}{x}\Bigl|
  \sum_{\substack{n\le x,\\ n\equiv a[q]}}\Lambda(n)
  -\frac{x}{\varphi(q)}\Bigr|\le 1/840.
  $
  \par 
  When $q\in(10^4, 10^5]$, we may replace $1/840$ by $1/160$ and when
  $q\ge 10^5$, we may replace $1/840$ by $\exp(0.03\sqrt{q}\log^3q)$.
  \par 
    Furthermore, we may replace
  $\sum_{\substack{n\le x,\\ n\equiv a[q]}}\Lambda(n)$ by
  $\sum_{\substack{p\le x,\\ p\equiv a[q]}}\log p$ with no changes in
  the constants. 
\end{thm}

Similarly, as a consequence of Theorem 1.3 of
\cite{Bennett-Martin-OBryant-Rechnitzer*18}
states that
\begin{thm}{Theorem (2018)}

  We have
  $\displaystyle
  \max_{3\le q\le 10^4}\max_{ x\ge 8\cdot 10^9}\max_{\substack{1\le a\le q,\\
  (a,q)=1}}
  \frac{\log^2 x}{x}\Bigl|
  \sum_{\substack{p\le x,\\ p\equiv a[q]}}1
  -\frac{\textrm{Li}(x)}{\varphi(q)}\Bigr|\le 1/840.
  $
  \par 
  When $q\in(10^4, 10^5]$, we may replace $1/840$ by $1/160$ and when
  $q\ge 10^5$, we may replace $1/840$ by $\exp(0.03\sqrt{q}\log^3q)$.
\end{thm}





\par 
Concerning estimates with a logarithmic density, in
\cite{Ramare*02}
and in
\cite{Ramare*12-0},
estimates for the functions
$\displaystyle\sum_{\substack{n\le x,\\ n\equiv a[q]}}\Lambda(n)/n$
are considered.
Extending computations from the former, the latter paper Theorem 8.1
reads as follows.
\begin{thm}{Theorem (2016)}

  We have
  $\displaystyle
  \max_{q\le 1000}\max_{q\le x\le 10^5}\max_{\substack{1\le a\le q,\\
  (a,q)=1}}
  \sqrt{x}\Bigl|
  \sum_{\substack{n\le x,\\ n\equiv a[q]}}\frac{\Lambda(n)}{n}
  -\frac{\log x}{\varphi(q)}-C(q,a)\Bigr|\in(0.8533,0.8534)
  $
  and the maximum is attained with $q=17$, $x=606$ and $a=2$.
\end{thm}

The constant $C(q,a)$ is the one expected, i.e. such that
$\sum_{\substack{n\le x,\\ n\equiv a[q]}}\frac{\Lambda(n)}{n}
-\frac{\log x}{\varphi(q)}-C(q,a)$ goes to
zero when $x$ goes to infinity.

\par 
  When $q$ belongs to "Rumely's list", i.e. in one of the
following set:
\begin{itemize}
    \item \,\,$\{k\le 72\}$

    \item \,\,$\{k\le 112, \text{$k$ non premier}\}$

    \item \,$\begin{aligned}\{116, 117, &120, 121, 124, 125, 128, 132, 140,
     143, 144, 156, 163, \\ &169, 180, 216, 243, 256, 360, 420, 432\}\end{aligned}$

\end{itemize}
Theorem 2 of
\cite{Ramare*02}
gives the following.
\begin{thm}{Theorem (2002)}

  When $q$ belongs to Rumely's list and $a$ is prime to $q$, we have
  $\displaystyle
  \sum_{\substack{n\le x,\\ n\equiv a[q]}}\frac{\Lambda(n)}{n}
  =\frac{\log x}{\varphi(q)}+C(q,a)+O^*(1)
  $
  as soon as $x\ge1$.
\end{thm}

More precise bounds are given.




\section{Least prime verifying a condition}


\cite{Bach-Sorenson*96},
\cite{Kadiri*05-2},







  
\begin{flushright}\small\sl{}   Last updated on January 22nd, 2022, by Olivier Ramar\'e
 \end{flushright}


















\chapter{   Explicit bounds on the Moebius function}

Corresponding html file: \texttt{../Articles/Art02.html}










 
 


Collecting references:
\cite{Diamond-Erdos*80},
\cite{Deleglise-Rivat*96-2},
\cite{Borwein-Ferguson-Mossinghoff*08}.


\section{Bounds on $M(D)=\sum_{d\le D}\mu(d)$}


The first explicit estimate for $M(D)$ is due to
\cite{VonSterneck*98}
where the author proved that $|M(D)|\le \tfrac19 D+8$ for any $D\ge0$. 
A popular estimate is the one of
\cite{MacLeod*69}.

\begin{thm}{Theorem (1967)}

When $D\ge 0$, we have $|M(D)|\le \tfrac1{80} D+5$. When $D\ge 1119$, we have $|M(D)|\le D/80$.
\end{thm}

We mention at this level the annoted bibliography contained at the end of 
\cite{Dress*83}.
\cite{CostaPereira*89} shows that

\begin{thm}{Theorem (1993)}

When $D\ge 120\,727$, we have $|M(D)|\le D/1036$.
\end{thm}


On elaborating on this method,
\cite{Dress-ElMarraki*93} showed that

\begin{thm}{Theorem (1993)}

When $D\ge 617\,973$, we have $|M(D)|\le D/2360$.
\end{thm}


One of the argument is the estimate from 
\cite{Dress*93}

\begin{thm}{Theorem (1993)}

When $33\le D\le 10^{12}$, we have $|M(D)|\le 0.571\sqrt{D}$.
\end{thm}


This has been extended by
\cite{Kotnik-VanDeLune*04}
to $10^{14}$ and recently in
\cite{Hurst*18}
to $10^{16}$, i.e.

\begin{thm}{Theorem (2018)}

When $33\le D\le 10^{16}$, we have $|M(D)|\le 0.571\sqrt{D}$.
\end{thm}



Another tool is 
\cite{Cohen-Dress*88}
where refined explicit estimates for the remainder term of the counting
functions of the squarefree numbers in intervals are obtained.

\par 
The latest best estimate of this shape comes from
\cite{Cohen-Dress-ElMarraki*96}.
This preprint being difficult to get, it has been republished in
\cite{Cohen-Dress-ElMarraki*07}.
\begin{thm}{Theorem (1996)}

When $D\ge 2\,160\,535$, we have $|M(D)|\le D/4345$.
\end{thm}

These results are used in
\cite{Dress*99}
to study the discrepancy of the Farey series.

\par \par 
Concerning upper bounds that tend to $0$, 
\cite{Schoenfeld*69} is the pioneer
and shows among other estimates that 
\begin{thm}{Theorem (1969)}

When $D>0$, we have $|M(D)|/D\le 2.9/\log D$.
\end{thm}

\cite{ElMarraki*95} improves that
into
\begin{thm}{Theorem (1995)}

When $D\ge 685$, we have $|M(D)|/D\le 0.10917/\log D$.
\end{thm}

The latest bound coming from
\cite{Ramare*12-2} improves that:
\begin{thm}{Theorem (2012)}

When $D\ge 1\,100\,000$, we have $|M(D)|/D\le 0.013/\log D$.
\end{thm}


In
\cite{Ramare*12-5},
bounds including coprimality conditions are proved and here is a
typical example.
\begin{thm}{Theorem (2013)}

  When $1\le q < D$, we have
                $\Bigl|\sum_{\substack{ d\le D, \\
                (d,q)=1}}\mu(d)\Bigr|/D\le
                \frac{q}{\varphi(q)}/(1+\log (D/q))$.
\end{thm}








\section{Bounds on $m(D)=\sum_{d\le D}\mu(d)/d$}


\cite{MacLeod*69} shows that the sum
$m(D)$ takes its minimal value at $D=13$.
A folklore result is generalized in 
\cite{Granville-Ramare*96} and reads
\begin{thm}{Theorem (1996)}

When $D\ge 0$ and for any integer $r\ge1$, we have $\Bigl|\sum_{\substack{d\le D,\\
(d,r)=1}}\mu(d)/d\Bigr|\le 1$.
\end{thm}

In fact, Lemma 1 of \cite{Davenport*37-1} already
contains the requisite material.

The next result is proved in
\cite{Ramare*12-5}.
\begin{thm}{Theorem (2013)}

When $D\ge 7$, we have $|\sum_{d\le D}\mu(d)/d|\le 1/10$. We can
replace the couple (7, 1/10) by (41, 1/20) or (694, 1/100).
\end{thm}


This is further extended in
\cite{Ramare*12-4} where it is shown
that
\begin{thm}{Theorem (2012)}

When $D\ge 0$ and for any integer $r\ge1$ and any real number $\varepsilon\ge0$, we have $\Bigl|\sum_{\substack{d\le D,\\
(d,r)=1}}\mu(d)/d^{1+\varepsilon}\Bigr|\le 1+\varepsilon$.
\end{thm}



Concerning upper bounds that tend to $0$, 
\cite{ElMarraki*96}
is the first to get such an estimate.
\begin{thm}{Theorem (1996)}

When $D\ge33$ we have $|m(D)|\le 0.2185/\log D$.

\par 
When $D > 1$ we have $|m(D)|\le 726/(\log D)^2$.
\end{thm}


This second bound is improved in
\cite{Bordelles*15}.
\begin{thm}{Theorem (2015)}

When $D > 1$ we have $|m(D)|\le 546/(\log D)^2$.
\end{thm}



\cite{Ramare*12-2} proves several
bounds of the shape $m(D)\ll 1/\log D$.
This is improved in
\cite{Ramare*12-5} by using
\cite{Balazard*12}.
which provides us with a better manner to convert bounds on $M(D)$
into bounds for $m(D)$. Here is one result obtained.

\begin{thm}{Theorem (2015)}

When $D\ge 463\,421$ we have $|m(D)|\le 0.0144/\log D$.
\par 
  We can for instance replace the couple  (463 421, 0.0144)by
any of  (96 955, 1/69), (60 298, 1/65), (1426, 1/20)
or (687, 1/12).
\end{thm}



In
\cite{Ramare*12-3} and
\cite{Ramare*12-5}, the
problem of adding coprimality conditions is further addressed.
Here is one of the results obtained.
\begin{thm}{Theorem (2015)}

  When $1\le q < D$ we have
  $\Bigl|\sum_{\substack{d\le D,\\ (d,q)=1}}\mu(d)/d\Bigr|\le
   \frac{q}{\varphi(q)}0.78/\log(D/q)$. When $D/q\ge 24233$, we can
  replace 0.78 by 17/125. 
\end{thm}




\section{Bounds on $\check{m}(D)=\sum_{d\le D}\mu(d)\log(D/d)/d$}


The initial investigations on this function go back to 
 \cite{vonSterneck*02}.
In \cite{Ramare*12-5} it is
proved that
\begin{thm}{Theorem (2015)}

  When $3846 \le D$ we have
  $|\check{m}(D)-1|\le 0.00257/\log D$.
  When $D > 1$, we have
    $|\check{m}(D)-1|\le 0.213/\log D$.
\end{thm}


This implies in particular that
\begin{thm}{Theorem (2015)}

  When $222 \le D$ we have
  $|\check{m}(D)-1|\le 1/1250$.
  When $D > 1$, the optimal bound 1 holds.
\end{thm}


These bounds are a consequence of the identity:
$$
|\check{m}(D)-1|\le \frac{\frac74-\gamma}{D^2}\int_1^D|M(t)|dt+\frac{2}{D}.
$$
It is also proved that, for any $D\ge1$, we have
$$
0\le \sum_{\substack{d\le D,\\ (d,q)=1}}\mu(d)\log(D/d)/d
\le 1.00303 q/\varphi(q).
$$


\section{Miscellanae}


Here is an elegant wide ranging estimate, taken from Claim 3.1 of
\cite{Trevino*15}.

\begin{thm}{Theorem (2015)}

When $D\ge1$ we have $|\sum_{d>D}\mu(d)/d^2|\le 1/D$.
\end{thm}



\section{The Moebius function and arithmetic progressions}


The results in this section are scarce. We mention a Theorem of
\cite{Bordelles*15}.

\begin{thm}{Theorem (2015)}

  Let $\chi$\, be a non-principal Dirichlet character mmodulo
  $q\ge37$ and let $k\ge1$ be some integer. Then
  $$
  \biggl|\sum_{\substack{n\le x,\\ (n,k)=1}}\frac{\mu(n)\chi(n)}{n}\biggr|
  \le\frac{k}{\varphi(k)}\frac{2\sqrt{q}\log q}{L(1,\chi)}.
  $$
\end{thm}






  
\begin{flushright}\small\sl{}   Last updated on February 18th, 2018, by Olivier Ramar\'e
 \end{flushright}
















\chapter{   Averages of non-negative multiplicative functions}

Corresponding html file: \texttt{../Articles/Art10.html}










 
 





\section{Asymptotic estimates}


When looking for averages of functions that look like 1 or like the divisor
function, Lemma 3.2 of
\cite{Ramare*95} offers an
efficient easy path. The technique of comparison of two arithmetical function
via their Dirichlet series is known as the Convolution method and is for
instance decribed at length in 
\cite{Berment-Ramare*12}, and in
the course that can be found 
here\footnote{\url{https://ramare-olivier.github.io/CoursNouakchott/index.html}}.


\begin{thm}{Theorem (1995)}

Let $(g_n)_{n\ge1}$, $(h_n)_{n\ge1}$ and $(k_n)_{n\ge1}$ be three
complex sequences. Let $H(s)=\sum_nh_nn^{-s}$, and
$\overline{H}(s)=\sum_n|h_n|n^{-s}$.
We assume that $g=h\star k$, that $\overline{H}(s)$ is convergent for
$\Re(s)\ge-1/3$ and further that
there exist four constants $A$, $B$, $C$ and $D$ such that
$$
\sum_{n\le t}k_n
=
A\log^2t+B\log t+C+\mathcal{O}^*(D t^{-1/3})
\text{ for $t>0$.}
$$
Then we have for all $t>0$ :
$$
\sum_{n\le t}g_n
=
u\log^2t+v\log t+w+\mathcal{O}^*(D t^{-1/3}\overline{H}(-1/3))
$$
with
$u=AH(0)$, $v=2AH^{\prime}(0)+BH(0)$ and $w=AH^{\prime\prime}(0)+BH^{\prime}(0)+CH(0)$.
We have also
$$
\sum_{n\le t}ng_n
=
Ut\log t+Vt+W+\mathcal{O}^*(2.5D t^{2/3}\overline{H}(-1/3))
$$
with
$$
\begin{aligned}
U=&2AH(0), V=-2AH(0)+2AH^{\prime}(0)+BH(0),\\
W=&A(H^{\prime\prime}(0)-2H^{\prime}(0)+2H(0))+B(H^{\prime}(0)-H(0))+CH(0).
\end{aligned}
$$
\end{thm}

This Lemma says that one derives information from $g_n$ from informations on
the model $k_n$. When this model is $k_n=1$, the values concerning $A$,
$B$ and $C$ are given by 
the first half of Lemma 3.3 of
\cite{Ramare*95}:

\par 
\begin{thm}{Lemma (1995)}

$
\sum_{n\le t}1/n=\log t+\gamma+\mathcal{O}^*(0.9105 t^{-1/3}).
$
\end{thm}


When this model is $k_n=\tau(n)$, the number of divisors of $n$, the values concerning $A$,
$B$ and $C$ are given by Corollary 2.2 of
\cite{Berkane-Bordelles-Ramare*11}. Please
note the "$\gamma^2-2\gamma_1$" which is wrongly typed as
"$\gamma^2-\gamma_1$" in the aforementioned paper (and thanks to Tim
Trudgian and David Platt for spotting this typo):

\par 
\begin{thm}{Lemma (2011)}

  $
    \sum_{n\le t}\tau(n)/n
    =\tfrac12\log^2t+2\gamma\log t
    +\gamma^2-2\gamma_1+ \mathcal{O}^*(1.16/t^{1/3})
  $
  where $\gamma_1$ is the second Laurent-Stieljes constant --
  for instance \cite{Kreminski*03}
 and \cite{Coffey*06}. In particular, we have
  $
    \gamma_1=
    -0.0728158454836767248605863758749013191377
    + \mathcal{O}^*(10^{-40}).
  $
\end{thm}

The constants $H(0)$, $H'(0)$ and $H''(0)$ are to be computed. In most cases,
the Dirichlet series has an Euler product, in which case,
(see section 3 of
\cite{Ramare*95})
 we have
\par 
$
H(0)=\prod_p(1+\sum_mh_{p^m}),
$
then
$\displaystyle
\frac{H^{\prime}(0)}{H(0)}=
\sum_p
\frac{\sum_mmh_{p^m}}{1+\sum_mh_{p^m}}(-\log p),
$
and also
\par 
$$
\frac{H^{\prime\prime}(0)}{H(0)}=
\left(
\frac{H^{\prime}(0)}{H(0)}
\right)^2+
\sum_p
\left\{
\frac{\sum_mm^2h_{p^m}}{1+\sum_mh_{p^m}}
-\left[\frac{\sum_mmh_{p^m}}{1+\sum_mh_{p^m}}\right]^2
\right\}
\log^2p.
$$
It is sometimes more expedient to use the same convolution method but
by comparing the function to the function $q\mapsto q$. In such a
case, the next lemma, Lemma 4.3 from
\cite{Ramare*13d},
is handy.
\begin{thm}{Theorem (2015)}

  We have, for any real number $x\ge0$ and any real number $c\in[1,2]$,
  $\displaystyle \sum_{q\le x}q=\tfrac12 x^2+O^*(x^c/2)$.
\end{thm}

This leads to the next theorem.
\begin{thm}{Theorem}

  Let $(h_n)_{n\ge1}$  be a 
complex sequences. Let $H(s)=\sum_nh_nn^{-s}$, and
$\overline{H}(s)=\sum_n|h_n|n^{-s}$.
We assume that  $\overline{H}(s)$ is convergent for
$\Re(s)\ge c$, for some $c\in[1,2]$. 
Then we have for all $t>0$ :
$$
\sum_{n\le t}\sum_{d|n}\frac{n}{d}h(d)
=
\frac{t^2}{2}H(2)+O^*(t^c\overline{H}(c)/2).
$$
\end{thm}

A typical usage is to evaluate $\sum_{n\le t}\phi(n)$, with
$h(d)=\mu(d)$.
\par 
  
The convolution method has been brought one step further in
\cite{Ramare*14-1}
where the following theorem is proved.
\par 
\begin{thm}{Theorem (2017)}

Let $(g(m))_{m\ge1}$ be a sequence of complex numbers such that both series
  $\sum_{m\ge1} g(m)/m$ and $\sum_{m\ge1} g(m)(\log m)/m$ converge. We define
  $G^\sharp(x)=\sum_{m> x} g(m)/m$ and assume that
  $\int_1^\infty |G^\sharp(t)|dt/t$ converges. Let $A_0\ge1$ be a real parameter.
  We have 
  \begin{equation*}
    \sum_{n\le D}\frac{(g\star{1\!\!1})(n)}{n}
    =\sum_{m\ge1}\frac{g(m)}{m}\Bigl(\log\frac{D}{m}+\gamma\Bigr)
    +\int_{D/A_0}^\infty G^\sharp(t)\frac{dt}{t}
    +O^*(\mathfrak{R})
  \end{equation*}
  where $\mathfrak{R}$ is defined by
  \begin{equation*}
    \mathfrak{R}
    =
    \left|\sum_{1\le a\le A_0}\frac{1}{a}G^\sharp\left(\frac{D}{a}\right)+
      G^\sharp\left(\frac{D}{A_0}\right)\left(\log\frac{A_0}{[A_0]}
        -R([A_0])\right)
    \right|
    +\frac{6/11}{D}\sum_{m\le D/A_0}|g(m)|
  \end{equation*}
  where $[A_0]$ is the integer part of $A_0$,
while the remainder $R$ is defined by
$R(X)=\sum_{n\le X}1/n-\log X-\gamma$.
\end{thm}

The remainder $R(X)$ is shown in Lemma to verify $|R(X)|\le \gamma/X$
for every $X > 0$, and $|R(X)|\le (6/11)/X$ when $X\ge1$.

\par 

Theorem 21.1 of
\cite{Ramare*06}
offers a fully explicit estimate for the average of a general non-negative
multiplicative function, but 
it is often numerically rather poor. It relies on the
technique developped by 
\cite{Levin-Fainleib*67}.

\par 
\begin{thm}{Theorem (2009)}

Let $g$ be a non-negative multiplicative function.
Let $A$ and $\kappa$ be three positive real parameters such that, for any
$Q\ge1$, one has
$$
\sum_{\substack{ p\ge2, \nu\ge1\\  p^{\nu}\le Q}}
g\bigl(p^{\nu}\bigr)\log\bigl(p^{\nu}\bigr)
=
\kappa\log Q+\mathcal{O}^*(L)
$$
and
$
\sum_{p\ge2}
\sum_{\nu,k\ge1}g\bigl(p^k\bigr)g\bigl(p^{\nu}\bigr)\log\bigl(p^{\nu}\bigr)
\le A.
$
Then, when $D\ge\exp(2(L+A))$, we have
$$
\sum_{d\le D}g(d)= C\left(\log D\right)^{\kappa}
\left(1+\mathcal{O}^*\bigl(B/\log D\bigr)\right)
$$
where $B=2(L+A)\bigl(1+2(\kappa+1)e^{\kappa+1}\bigr)$ and
$$
C=\frac{1}{\Gamma(\kappa+1)}
\prod_{p\ge2}\biggl\{
\biggl(\sum_{\nu\ge0}g\bigl(p^{\nu}\bigr)\biggr)
\biggl(1-\frac1p\biggr)^{\kappa}\biggr\}.
$$
\end{thm}



\section{Upper bounds}


When looking for an upper bound, it is common to compare sums to an Euler
product, via, 
$$
\sum_{n\le y}f(n)/n\le \prod_{p\le y}
\left(1+\sum_{1\le m\le \log y/\log p}f(p^m)\right)
$$
valid when $f$ is non-negative and multiplicative.
Lemma 4 of
\cite{Daboussi-Rivat*01}
extends this. Let $z$ be a parameter and $v_z(n)$ be the characteristic
function of those integers that have all their prime factors $p\le z$.

\begin{thm}{Theorem (2000)}

Let $z\ge2$, $f$ a multiplicative function with $f\ge0$ and 
$S=\sum_{p\le z}\frac{f(p)}{1+f(p)}\log p$. We assume that $S>0$ and write
$K(t)=\log t-1-(1/t)$. For any $y$ such that $\log y\ge S$, we have
$$
\sum_{n > y}v_z(n)\mu^2(n)f(n)
\le \prod_{p\le z}(1+f(p))\exp\left(-\frac{\log y}{\log z}
K\left(\frac{\log y}{S}\right)\right)
$$
$$
\sum_{n \le y}v_z(n)\mu^2(n)f(n)
\ge \prod_{p\le z}(1+f(p))\left\{1-\exp\left(-\frac{\log y}{\log z}
K\left(\frac{\log y}{S}\right)\right)\right\}
$$
and in particular, when $\log y\ge 7S$, we have
$$
\sum_{n > y}v_z(n)\mu^2(n)f(n)
\le \prod_{p\le z}(1+f(p))\exp\left(-\frac{\log y}{\log z}\right)
$$
$$
\sum_{n \le y}v_z(n)\mu^2(n)f(n)
\ge \prod_{p\le z}(1+f(p))\left\{1-\exp\left(-\frac{\log y}{\log z}\right)\right\}.
$$
\end{thm}


It is sometimes required to compare a function close to $1$ (or more generally
to the divisor
function $\tau_k$) to a function 
close to $1/n$ or $\tau_k(n)/n$. Theorem 01 of 
\cite{Hall-Tenenbaum*88} offers a
fast way of doing so.
\par 
\begin{thm}{Theorem (1988)}

Let $f$ be a non-negative multiplicative function such that, for some $A$ and
$B$,
$$
\sum_{p\le y} f(p)\log p\le Ay \quad(y\ge 0),\quad
\sum_p\sum_{\nu\ge2} \frac{f(p^\nu)}{p^\nu}\log p^{\nu}\le B.
$$
Then, for $x > 1$,
$$
\sum_{n\le x}f(n)\le (A+B+1)\frac{x}{\log x}\sum_{n\le x}\frac{f(n)}{n}
$$
\end{thm}

See also Section 4.6, and for instance Theorem 4.22, of
\cite{Bordelles*12}.
In particular,
in case a further condition is assumed, we have Theorem 4.28 of
\cite{Bordelles*12}
at our disposal.
\par 
\begin{thm}{Theorem (2012)}

Let $f$ be a non-negative multiplicative function such that, for every
prime $p$ and every non-negative power $a$ the condition
$f(p^{a+1})\ge f(p^a)$ holds, we have
for $x \ge 1$
$$
\sum_{n\le x}f(n)\le x\prod_{p\le x}
\Bigl(1-\frac{1}{p}\Bigr)\Bigl(
1+\sum_{a\ge 1}\frac{f(p^a)}{p^a}\Bigr).
$$
\end{thm}

  
The next lemma is handy to remove coprimality conditions.
It originates from
\cite{van-Lint-Richert*65}.
\par 
\begin{thm}{Theorem (1965)}

Let $f$ be a non-negative multiplicative function and let $d$ be a
positive integer. We have
for $x \ge 0$
$$
\sum_{n\le x}\mu^2(n)f(n)\le \prod_{p|d}(1+f(p))
\sum_{\substack{n\le x,\\ (n,d)=1}}\mu^2(n)f(n)
\le
\sum_{n\le xd}\mu^2(n)f(n).
$$
\end{thm}

Though it is somewhat difficult to get, this lemma has been further generalized in Lemma 4.1 of
\cite{Ramare*10}.

\section{Estimates of some special functions}


\cite{Cohen-Dress*88} contains the
following Theorem.

\par 
\begin{thm}{Theorem (1988)}

  Let $R(x)=\sum_{n\le x}\mu^2(n)-6x/\pi^2$. We have
  $ |R(x+y)-R(x)|\le 1.6749\sqrt{y}+0.6212 x/y$ and 
  $ |R(x+y)-R(x)|\le 0.7343y/x^{1/3}+1.4327 x^{1/3}$ for $x,y\ge1$.
\end{thm}


See also \cite{CostaPereira*89}.
\cite{Moser-MacLeod*66} and 
\cite{Cohen-Dress-ElMarraki*07}
contains:
\par 
\begin{thm}{Theorem (2008)}

  We have
  $ |\sum_{n\le x}\mu^2(n)-6x/\pi^2|\le 0.02767\sqrt{x}$ for $x\ge 438653$.
  One can replace $(0.02767, 438653)$ by $(0.036438, 82005)$, by
  $(0.1333, 1004)$, by $(1/2, 8)$ or by $(1,1)$.
\end{thm}




\par 
Lemma 3.4
of  \cite{Ramare*13d}
gives us:
\par 
\begin{thm}{Theorem (2013)}

We have
$\frac{6}{\pi^2}\log x+0.578\le \sum_{n\le x}\mu^2(n)/n\le \frac{6}{\pi^2}\log x+1.166$ for $x\ge1$

When $x\ge1000$, one can also replace the couple $(0.578, 1.166)$ by $(1.040, 1.048)$.
\end{thm}

In fact, in the same paper, the asymptotic
$$
\sum_{n\le x}\frac{\mu^2(n)}{n}
=\frac{6}{\pi^2}\Bigl(\log x+2\sum_{p\ge2}\frac{\log
p}{p^2-1}+\gamma\Bigr)
+O^*(3/x^{1/3})
$$
valid for $x\ge1$ is proved. A script using SAGE and another one using GP/PARI are then
displayed to explain how to cover the initial range in $x$.
See also Lemma 1
of \cite{Schoenfeld*69} for an
earlier version.



\par 
The main result
\cite{Berkane-Bordelles-Ramare*11}
reads as follows.
\par 
\begin{thm}{Theorem (2012)}

We define $\Delta(x)=\sum_{n\le x}\tau(n)-x(\log x+2\gamma-1)$. We have
\begin{itemize}
\item  When $x\ge 1$, we have $|\Delta(x)|\le 0.961\, {x^{1/2}}$.

\item  When $x\ge 1\,981$, we have $|\Delta(x)|\le 0.482\, {x^{1/2}}$.

\item  When $x\ge 5\,560$, we have $|\Delta(x)|\le 0.397\, {x^{1/2}}$.

\item  When $x\ge 5$, we have $|\Delta(x)|\le 0.764\, {x^{1/3}\log x}$.

\end{itemize}
\end{thm}


For evaluation of the average of the divisor function on integers belonging to
a fixed residue class modulo 6, see Corollary to Proposition 3.2 of 
\cite{Deshouillers-Dress*88}.



For more complicated sums and when $x$ is large with respect to $k$, one can use
\cite{Mardjanichvili*39}.
\par 
\begin{thm}{Theorem (1939)}

  Let $k$ and $\ell$ be two positive integers. We have for any real number
$x\ge1$ 
$$
 \sum_{m\le x}\tau_k^\ell(m) \le
x\frac{k^\ell}{(k!)^{\frac{k^\ell-1}{k-1}}}(\log x+k^\ell-1)^{k^\ell-1}.  
$$
\end{thm}

See \cite{Deshouillers-Dress*88}
for some upper bounds linked with $\tau_3$.

\par 
\cite{Bordelles*02}
contains the following bounds, better than the above when $x$ is small with
respect to $k$.
\par 
\begin{thm}{Theorem (2002)}

  Let $k\ge1$ be a positive integer.
\begin{itemize}
\item  When  $x\ge1$ is a real number, we have
  $\sum_{m\le x}\tau_k(m)\le x(\log x+\gamma+(1/x))^{k-1}$.

\item  When $x\ge6$ is a real number, we have
  $\sum_{m\le x}\tau_k(m)\le 2x(\log x)^{k-1}$.

\end{itemize}
\end{thm}


In
\cite{Lapkova*16}, we find
the next result.
\par 
\begin{thm}{Theorem (2015)}

  Let $b$ and $c$ be two integers such that $\delta=b^2-c$\, is non-zero,
  square-free and not congruent to 1 modulo 4. Assume further that the
  function $n^2+2bn+c$ is positive and non-decreasing when
  $n\ge1$.Then, for $N\ge1$, we have
  $$
  \sum_{n\le N}\tau(n^2+2bn+c)\le C_1 N\log N+C_2+C_3
  $$
  where the constants $C_1$, $C_2$\, and $C_3$\, are defined
  as follows. We first define $\xi=\sqrt{1+2|b|+|c|}$ and
  $\kappa=\frac{4}{\pi^2}\sqrt{4|\delta|}(\log(4|\delta|)+0.648)$. Then
  $$
  C_1=\frac{12}{\pi^2}(\log\kappa+1),
  C_2=2\biggl[\kappa+(\log\kappa+1)\Bigl(\frac{6}{\pi^2}\log\xi+1.166\Bigr)\biggr],
  C_3=2\kappa (\max(|b|,|c|^{1/2})+1).
  $$
\end{thm}

See
\cite{Lapkova*16-2} for the
number of divisors of a reducible quadratic polynomial.

\par 
Evaluations of Lemma 4.3  of
\cite{Cipu*15-1}
are improved in Lemma 12 of
\cite{Trudgian*15-2}.
Only upper bounds are given, but the proof given there gives the lower
bounds as well. This gives the first two estimates, while the third
one comes from 
Lemma 4.3  of
\cite{Cipu*15-1}.
\par 
\begin{thm}{Theorem (2015)}

  Let $x\ge1$ be a real number. We have
  \begin{itemize}
\item  
  $\displaystyle 0.786x-0.3761-8.14x^{2/3}
  \le \sum_{n\le x}2^{\omega(n)}-\frac{6}{\pi^2}x\log x
    \le 0.787x-0.3762+8.14x^{2/3}$

\item  
 $\displaystyle 1.3947\log x+0.4106-3.253x^{-1/3}
  \le \sum_{n\le x}\frac{2^{\omega(n)}}{n}-\frac{3}{\pi^2}(\log x)^2
    \le 1.3948\log x+0.4107+3.253x^{-1/3}$,

\item  
  $\displaystyle \sum_{n\le x}\frac{2^{\omega(2n-1)}}{2n-1}
  \le \frac{3}{2\pi^2}(\log x)^2+3.123\log x+3.569+\frac{0.525}{x}$.

\end{itemize}
\end{thm}



We take the next lemma from
\cite{Trevino*15-2}, Lemma 2.
\par 
\begin{thm}{Theorem (2015)}

  Let $x\ge1$ be a real number. We have
  $\sum_{n\le x}\phi(n)/n\le \frac{6}{\pi^2}x+\log x +1$.
\end{thm}

  

Lemma 3 of the same paper is as follows.
\par 
\begin{thm}{Theorem (2015)}

  Let $x\ge1$ be a real number. We have
  $\sum_{n\le x}n\phi(n)\le \frac{2}{\pi^2}x^3+\frac12x^2\log x +x^2$.
\end{thm}


Several estimates are proved in
\cite{Ramare*14-1}.
For instance Theorem 1.2 gives the following.
\par 
\begin{thm}{Theorem (2017)}

  Let $x\ge1$ be a real number. We have
  $\sum_{n\le x}\mu^2(n)/\phi(n)= \log x
  +c_0+O^*(3.95/\sqrt{x})$ where $\displaystyle c_0=\gamma+\sum_{p\ge2}\frac{\log
  p}{p(p-1)}$.
  When $x > 1$, this $O^*$ can be replaced
  by $O^*(21/\sqrt{x\log x})$. 
\end{thm}

The function $\sum_{n\le x}\mu^2(n)/\phi(n)$ has been the subject of
several estimates,
see for instance Lemma 7 of
\cite{Montgomery-Vaughan*73},
Lemma 3.4-3.5 of
\cite{Ramare*95},
the earlier paper
\cite{Ward*27}
and Lemma 4.5 of
\cite{Buethe*14}
where the error term $O^*(58/\sqrt{x})$ is achieved.
The constant $c_0$ is evaluated precisely in (2.11) of
\cite{Rosser-Schoenfeld*62}.



\section{Euler products}



\cite{Rosser-Schoenfeld*62}
contains estimates regarding $\prod_{p}(1-1/p)$ and its inverse. In
particular we find the next results therein.
\par 
\begin{thm}{Theorem (1962)}

  \begin{itemize}
    \item When $x > 1$, we have $\displaystyle 1-\frac{1}{\log^2x}\le
      e^\gamma(\log x)\prod_{p\le x}\Bigl(1-\frac{1}{p}\Bigr)\le
      1+\frac{1}{2\log^2x}$.
    

    \item When $x > 1$, we have $\displaystyle 1-\frac{1}{2\log^2x}\le
      e^{-\gamma}\prod_{p\le x}\Bigl(1-\frac{1}{p}\Bigr)^{-1}/\log x\le
      1+\frac{1}{\log^2x}$.
    

    \end{itemize}
\end{thm}

Several other estimates are proven. In
\cite{Dusart*16}, it is
proved that
\par 
\begin{thm}{Theorem (2016)}

  \begin{itemize}
    \item When $x > 2\,278\,382$, we have $\displaystyle 1-\frac{1}{5\log^3x}\le
      e^\gamma(\log x)\prod_{p\le x}\Bigl(1-\frac{1}{p}\Bigr)\le
      1+\frac{1}{5\log^3x}$.
    

    \item When $x > 2\,278\,382$, we have $\displaystyle 1-\frac{1}{5\log^3x}\le
      e^{-\gamma}\prod_{p\le x}\Bigl(1-\frac{1}{p}\Bigr)^{-1}/\log x\le
      1+\frac{1}{5\log^3x}$.
    

    \end{itemize}
\end{thm}


In \cite{Bordelles*05},
the reader will find explicit upper bounds for $\displaystyle
\prod_{\substack{p\le x,\\ p\equiv a[q]}}\Bigl(1-\frac{1}{p}\Bigr)^{-1}$

Theorem 5 of
\cite{Vanlalnagaia*15-1}
contains the next result.
\par 
\begin{thm}{Theorem (2017)}

  Let $\epsilon$ be a complex number such that $|\epsilon|\le 2$. When
  $x\ge\exp(22)$, we have
  $\displaystyle\prod_{p\le x}\Bigl(1+\frac{\epsilon}{p}\Bigr)
  =e^{\gamma(\epsilon)+\epsilon B}(\log x)^\epsilon
  \biggl\{1+O^*\Bigl(\frac{0.841}{\log^3x}\Bigr)\biggr\}$
  where
  $\displaystyle\gamma(\epsilon)=\sum_{p\ge2}\sum_{n\ge2}(-1)^{n+1}\frac{\epsilon^n}{np^n}$
  and
  $\displaystyle B=\gamma+\sum_{p\ge2}\bigl(\log(1-1/p)+(1/p)\bigr)$.
\end{thm}

Equation (2.2) of
\cite{Rosser-Schoenfeld*62}
gives an approximate value for $B$.










  
\begin{flushright}\small\sl{}   Last updated on June 10th, 2019, by Olivier Ramar\'e
 \end{flushright}
















\chapter{  Explicit upper bounds for some special arithmetic functions}

Corresponding html file: \texttt{../Articles/Art12.html}










 
 

The following bounds may be useful is applications.

From
\cite{Robin*83-1}:

\begin{thm}{Theorem (1983)}

For any integer $n\ge3$, the number of prime divisors $\omega(n)$ of $n$ satisfies:
  $$\omega(n)\le 1.3841\frac{\log n}{\log\log n}.$$
\end{thm}


From
\cite{Nicolas-Robin*83}:

\begin{thm}{Theorem (1983)}

For any integer $n\ge3$, the number $\tau(n)$  of divisors of $n$ satisfies:
  $$\tau(n)\le n^{1.538\,\log 2/\log\log n}.$$
\end{thm}


From page 51 of \cite{Robin*83-0}:

\begin{thm}{Theorem (1983)}

For any integer $n\ge3$, we have
  $$\tau_3(n)\le n^{1.59141\,\log 3/\log\log n}$$
  where $\tau_3(n)$ is the number of triples $(d_1,d_2,d_3)$ such that $d_1d_2d_3=n$.
\end{thm}


The PhD
memoir
\cite{Duras*93}
contains result concerning the maximum
of $\tau_k(n)$, i.e. the number of $k$-tuples $(d_1,d_2,\ldots, d_k)$
such that $d_1d_2\cdots d_k=n$, when $3\le k\le 25$.

\par 
  From
  \cite{Duras-Nicolas-Robin*99}:

\begin{thm}{Theorem (1999)}

For any integer $n\ge1$, any real number $s>1$ and any integer $k\ge1$, we have
  $$\tau_k(n)\le n^s\zeta(s)^{k-1}$$
  where $\tau_k(n)$ is the number of $k$-tuples $(d_1,d_2,\cdots,d_k)$ such
  that $d_1d_2\cdots d_k=n$.
\end{thm}

The same paper also announces the bound for $n\ge3$ and $k\ge2$
$$
\tau_k(n)\le n^{a_k\log k/\log\log k}
$$
where $a_k=1.53797\log k / \log 2$ but the proof never appeared.

From \cite{Nicolas*08}:

\begin{thm}{Theorem (2008)}

For any integer $n\ge3$, we have
  $$\sigma(n)\le 2.59791\, n\log\log(3\tau(n)),$$
  $$\sigma(n)\le n\{ e^\gamma\log\log(e\tau(n))+\log\log\log(e^e\tau(n))+0.9415\}.$$
\end{thm}


The first estimate above is a slight improvement of the bound
  $$\sigma(n)\le 2.59 n\log\log n\quad(n\ge7)$$
obtained in
\cite{Ivic*77}.
In this same paper,
  the author proves that
$$\sigma^*(n)\le \frac{28}{15} n\log\log n\quad(n\ge31)$$
where $\sigma^*(n)$ is the sum of the unitary divisors of $n$, i.e. divisors
  $d$ of $n$ that are such that $d$ and $n/d$ are coprime.

\par 
\par 
\par 
On this subject, the readers may consult the web site

Computation about the paper The sum of divisors function and the
Riemann hypothesis
.





  
\begin{flushright}\small\sl{}   Last updated on September 2nd, 2021, by Olivier Ramar\'e
 \end{flushright}
















\part{Exact computations}

\chapter{   Exact computations of the number of primes}

Corresponding html file: \texttt{../Articles/Art03.html}









Collecting references:
\cite{Deleglise-Rivat*96-1},
\cite{Deleglise-Rivat*98},
\cite{Platt*11}.




 
 







  
\begin{flushright}\small\sl{}   Last updated on July 14th, 2012, by Olivier Ramar\'e
 \end{flushright}

















\chapter{   Computations of arithmetical constants}

Corresponding html file: \texttt{../Articles/Art04.html}









Collecting references:
\cite{Cazaran-Moree*99}.

 
 


\section{Euler products of rational functions}


The computation of Euler product of rational function is dealt with in
\cite{Moree*12}. The reader may
also consult the following
web page\footnote{\url{http://guests.mpim-bonn.mpg.de/moree/Moree.en.html}}.


\section{Some special sums over prime values that are derivatives}









  
\begin{flushright}\small\sl{}   Last updated on July 14th, 2012, by Olivier Ramar\'e
 \end{flushright}

















\part{General analytical tools}

\chapter{  Tools on Fourier transforms}

Corresponding html file: \texttt{../Articles/Art16.html}










\section{The large sieve inequality}


The best version of the large sieve inequality from
\cite{Montgomery-Vaughan*74}
and
\cite{Montgomery-Vaughan*73}
(obtained at the same time by A. Selberg) is as follows.
\par 
\begin{thm}{Theorem (1974)}

Let $M$ and $N\ge 1$ be two real numbers. Let $X$ be a set of points of
$[0,1)$ such that 
$$
\min_{x,y\in X}\min_{k\in\mathbb{Z}}|x-y+k|\ge \delta>0.
$$
Then, for any sequence of complex numbers $(a_n)_{M < n\le M+N}$, we have
$$
\sum_{x\in X}\left|
\sum_{M < n\le M+N} a_n \exp(2i\pi nx)
\right|^2
\le \sum_{M < n\le M+N}|a_n|^2 (N-1+\delta^{-1}).
$$
\end{thm}


It is very often used with part of the Farey dissection.
\par 
\begin{thm}{Theorem (1974)}

Let $M$ and $N\ge 1$ be two real numbers. Let $Q\ge1$ be a real parameter.
For any sequence of complex numbers $(a_n)_{M < n\le M+N}$, we have
$$
\sum_{q\in Q}\sum_{\substack{a\mod q,\\ (a,q)=1}}\left|
\sum_{M < n\le M+N} a_n \exp(2i\pi na/q)
\right|^2
\le \sum_{M < n\le M+N}|a_n|^2 (N-1+Q^2).
$$
\end{thm}

The summation over $a$ runs over all invertible classes $a$ modulo $q$.


 
 








  
\begin{flushright}\small\sl{}   Last updated on July 14th, 2013, by Olivier Ramar\'e
 \end{flushright}

















\chapter{  Tools on Mellin transforms}

Corresponding html file: \texttt{../Articles/Art17.html}










\section{Explicit truncated Perron formula}



Here is Theorem 7.1 of
\cite{Ramare*07a}.

\par 
\begin{thm}{Theorem (2007)}

Let $F(z)=\sum_{n}a_n/n^z$ be a Dirichlet series that converges absolutely
  for $\Re z>\kappa_a$, and let $\kappa>0$ be strictly larger than 
  $\kappa_a$. For $x\ge1$ and $T\ge1$, we have
$$
    \sum_{n\le x}a_n
    =\frac1{2i\pi}\int_{\kappa-iT}^{\kappa+iT}F(z)\frac{x^zdz}z
    +\mathcal{O}^*\left(
      \int_{1/T}^{\infty}
      \sum_{|\log(x/n)|\le u}\frac{|a_n|}{n^\kappa}
      \frac{2x^\kappa du}{T u^2}
    \right).
$$
\end{thm}

See
\cite{Ramare*14-6}
for different versions.




\section{L${}^2$-means}


We start with a majorant principle taken for instance from
\cite{Montgomery*94},
chapter 7, Theorem 3.
\par 
\begin{thm}{Theorem}

  Let $\lambda_1,\cdots,\lambda_N$ be $N$ real numbers, and suppose
  that $|a_n|\le A_n$ for all $n$. Then
  $$
  \int_{-T}^T\Bigl|\sum_{1\le n\le N}a_n e(\lambda_n t)\Bigr|^2dt
  \le 3
  \int_{-T}^T\Bigl|\sum_{1\le n\le N}A_n e(\lambda_n t)\Bigr|^2dt
  $$.
\end{thm}

The constant 3 has furthermore been shown to be optimal in
\cite{Logan*88}
where the reader will find an intensive discussion on this
question. The next lower estimate is also proved there:
\par 
\begin{thm}{Theorem}

  Let $\lambda_1,\cdots,\lambda_N$ be $N$ be real numbers, and suppose
  that $a_n\ge 0$ for all $n$. Then
  $$
  \int_{-T}^T\Bigl|\sum_{1\le n\le N}a_n e(\lambda_n t)\Bigr|^2dt
  \ge
  T \sum_{n\le N}a_n^2.
  $$.
\end{thm}



We follow the idea of Corollary 3 of
\cite{Montgomery-Vaughan*74}
but rely on
\cite{Preissmann*84} to get the following.
\par 
\begin{thm}{Theorem (2013)}

  Let $(a_n)_{n\ge1}$ be a series of complex numbers that are such that
  $\sum_n n|a_n|^2 < \infty$ and $\sum_n |a_n| < \infty$. We have, for $T\ge0$,
  \begin{equation*}
    \int_0^T\Bigl|
    \sum_{n\ge1} a_{n}n^{it}
    \Bigr|^{2}dt = 
    \sum_{n\le N}|a_n|^2 \bigl(T+\mathcal{O}^*(2\pi c_0(n+1))\bigr),
  \end{equation*}
  where $c_0=\sqrt{1+\frac23\sqrt{\frac{6}{5}}}$. Moreover, when $a_n$ is
  real-valued, the constant $2\pi c_0$ may be reduced to $\pi c_0$.
\end{thm}

This is Lemma 6.2 from \cite{Ramare*13d}.

\par 
Corollary 6.3 and 6.4 of
\cite{Ramare*13d}
contain explicit versions of a Theorem of
\cite{Gallagher*70}

\par 
\begin{thm}{Theorem (2013)}

  Let $(a_n)_{n\ge1}$ be a series of complex numbers that are such that
  $\sum_n n|a_n|^2 < \infty$ and $\sum_n |a_n| < \infty$. We have, for $T\ge0$,
  $$
  \sum_{q\le Q}\frac{q}{\varphi(q)}
  \sum_{\substack{\chi\mod q,\\ \text{$\chi$ primitive}}}
  \int_{-T}^T
    \biggl|\sum_{n}a_n \chi(n)n^{it}\biggr|^2dt
    \le
    7
    \sum_{n}|a_n|^2( n+ Q^2\max(T, 3) ).
  $$
\end{thm}


\par 
\begin{thm}{Theorem (2013)}

  Let $(a_n)_{n\ge1}$ be a series of complex numbers that are such that
  $\sum_n n|a_n|^2 < \infty$ and $\sum_n |a_n| < \infty$. We have, for $T\ge0$,
  $$
  \sum_{q\le Q}\frac{q}{\varphi(q)}
  \sum_{\substack{\chi\mod q,\\ \text{$\chi$ primitive}}}
  \int_{-T}^T
    \biggl|\sum_{n}a_n \chi(n)n^{it}\biggr|^2dt
    \le
    \sum_{n}|a_n|^2( 43n+ \tfrac{33}{8} Q^2\max(T, 70) ).
  $$
\end{thm}




 
 








  
\begin{flushright}\small\sl{}   Last updated on July 14th, 2013, by Olivier Ramar\'e
 \end{flushright}

















\part{Exponential sums / points close to curves}

\chapter{   Explicit results on exponential sums}

Corresponding html file: \texttt{../Articles/Art05.html}










Collecting references:
\cite{Granville-Ramare*96},
\cite{Daboussi-Rivat*01}.



 
 







  
\begin{flushright}\small\sl{}   Last updated on July 14th, 2012, by Olivier Ramar\'e
 \end{flushright}

















\chapter{   Integer Points near Smooth Plane Curves}

Corresponding html file: \texttt{../Articles/Art11.html}










 
 


In what follows, $N \geqslant 1$ is an arbitrary large integer, $\delta \in
\left( 0,\frac{1}{2} \right)$ and if $f : [N,2N] \longrightarrow \mathbb {R}$
is any positive function, then let $\mathcal {R}(f,N,\delta)$ be the number of
integers $n \in [N,2N]$ such that $\lVert f(n) \rVert < \delta$, where as
usual $\lVert x \rVert$ denotes the distance from $x \in \mathbb {R}$ to its
nearest integer. Note that, since $\delta$ is very small, $\mathcal
{R}(f,N,\delta)$ roughly counts the number of integer points very close to the
arc $y = f(x)$ with $N \leqslant x \leqslant 2N$. Hence the trivial estimate
is given by $\mathcal {R}(f,N,\delta) \leqslant N+1$.  


The number $\mathcal {R}(f,N,\delta)$ arises fairly naturally in a large
collection of problems in number theory, e.g.
\cite{Filaseta*90},
\cite{Filaseta-Trifonov*96},
\cite{Huxley*96},
\cite{Huxley-Sargos*95},
\cite{Huxley-Sargos*06},
\cite{Huxley-Trifonov*96} and
bibref("Huxley*07"").
We deal with either getting an
asymptotic formula of the shape 

$$\mathcal {R}(f,N,\delta) = N \delta + \textrm{Error terms}$$

where the remainder terms depend on the derivatives of $f$ but not on
$\delta$, or finding an upper bound for $\mathcal {R}(f,N,\delta)$ as accurate
as possible.  



\section{Bounds using elementary methods}


The basic result of the theory is well-known and may be found in
\cite{Vinogradov*54}.
The proof follows from a clever use of the mean-value theorem (see
Theorem~5.6 of 
\cite{Bordelles*12} for instance).
\begin{thm}{Theorem (First derivative test)}

Let $f \in C^1 [N,2N]$ such that there exist $\lambda_1 >0$ and $c_1 \geqslant 1$ such that, for all $x \in [N,2N]$, we have
$$
   \lambda_1 \leqslant \bigl | f'(x) \bigr | \leqslant c_1 \lambda_1.
$$
Then
$$
  \mathcal {R}(f,N,\delta) \leqslant 2 c_1 N \lambda_1 + 4c_1 N \delta + \frac{2 \delta}{\lambda_1} + 1.$$
\end{thm}



This result is useful when $\lambda_1$ is very small, so that the condition is
in general too restrictive in the applications. Using a rather neat
combinatorial trick,
\cite{Huxley*96}
succeeded in passing from the first
derivative to the second derivative. This reduction step enables him to apply
this theorem to a function being approximatively of the same order of
magnitude as $f'$. This provides the following useful result.   

\begin{thm}{Theorem (Second derivative test)}

Let $f \in C^2[N,2N]$ such that there exist $\lambda_2 >0$ and $c_2 \geqslant
  1$ such that, for all $x \in [N,2N]$, we have 
$$
   \lambda_2 \leqslant \bigl | f''(x) \bigr | \leqslant c_2 \lambda_2
  \quad\text{and}\quad N \lambda_2 \geqslant c_2^{-1}.
$$
Then
$$\mathcal {R}(f,N,\delta) \leqslant 6 \left\lbrace \left( 3 c_2 \right)^{1/3} N \lambda_2^{1/3} + \left(12 c_2 \right)^{1/2} N \delta^{1/2} + 1 \right\rbrace.$$
\end{thm}


Both hypotheses above are often satisfied in practice, so that this result may
be considered as the first useful tool of the theory. A proof of
this Theorem may be found in Theorem 5.8 of
\cite{Bordelles*12}.


Using a $k$th version of Huxley's reduction principle may allow us to
generalize the above results. A better way is to split the integer points into
two classes, namely the major arcs in which the points belong to a
same algebraic curve of degree $\le k-1$, and the minor arcs. The points
coming from the minor arcs are treated by divided differences techniques,
generalizing the proof of both theorems above and, by a careful analysis of the
points belonging to major arcs,
\cite{Huxley-Sargos*95}
and
\cite{Huxley-Sargos*06}
succeeded in
proving the following fundamental result. A proof of an explicit version may
be found in Theorem 5.12 of
\cite{Bordelles*12}.

\begin{thm}{Theorem ($k$th derivative test)}

Let $k \geqslant 3$ be an integer and $f \in C^k [N,2N]$ such that there exist
  $\lambda_k >0$ and $c_k \geqslant 1$ such that, for all $x \in [N,2N]$, we
  have 
$$
   \lambda_k \leqslant \bigl | f^{(k)}(x) \bigr | \leqslant c_k \lambda_k. \label{e4}
$$
Let $\delta \in \left( 0,\frac{1}{4} \right)$. Then
$$
  \mathcal {R}(f,N,\delta) \leqslant \alpha_k N \lambda_k^{\frac{2}{k(k+1)}} +
  \beta_k N \delta^{\frac{2}{k(k-1)}} + 8k^3 \left( \frac{\delta}{\lambda_k}
  \right)^{1/k} + 2 k^2 \left( 5e^3 + 1 \right)
  $$
where
$$
  \alpha_k = 2k^2 c_k^{\frac{2}{k(k+1)}} \quad \text{and} \quad \beta_k = 4
  k^2 \left( 5 e^3 c_k^{\frac{2}{k(k-1)}} + 1 \right).
  $$
\end{thm}





\section{Bounds using exponential sums techniques}


The next result leads us to estimate $\mathcal {R}(f,N,\delta)$ with the help
of exponential sums (see 
\cite{Graham-Kolesnik*91}
for instance), which have been extensively
studied in the 20th century by many specialists, such as van der Corput,
Weyl or Vinogradov. Nevertheless, even using the finest exponent pairs to
date, the result generally does not significantly improve on the previous
estimates seen above. A simple proof of the following inequality may be found
in 
\cite{Filaseta*90}.

\begin{thm}{Theorem ($k$th derivative test)}

Let $f : [N,2N] \longrightarrow \mathbb {R}$ be any function and $\delta \in
  \left( 0,\frac{1}{4} \right)$. Set $K = \left \lfloor (8 \delta)^{-1} \right
  \rfloor +1$. Then, for any positive integer $H \leqslant K$, we have 
$$\mathcal {R}(f,N,\delta) \leqslant \frac{4N}{H} + \frac{4}{H} \sum_{h=1}^{H}
  \left | \sum_{N \leqslant n \leqslant 2N} e(hf(n)) \right |.$$ 
\end{thm}




\section{Integer points on curves}



This last part is somewhat out of the scope of the TME-EMT project, but may
help the reader in orienting him/herself in the litterature.

\par 
When $\delta \longrightarrow 0$, we are led to counting the number of integer
points lying on curves, and we denote this number by $\mathcal
{R}(f,N,0)$. This problem goes back to Jarn\'ik
\cite{Jarnik*25}
who proved that a
strictly convex arc $y=f(x)$ with length $L$ has at most 
$$\leqslant \frac{3}{(2 \pi)^{1/3}} \, L^{2/3} + O \left( L^{1/3} \right)$$ 
integer points and this is a nearly best possible result under the sole
hypothesis of convexity. However, 
\cite{Swinnerton-Dyer*74}
proved that if $f
\in C^3[0,N]$ is such that $|f(x)| \leqslant N$ and $f'''(x) \neq 0$ for all
$x \in [0,N]$, then the number of integer points on the arc $y=f(x)$ with $0
\leqslant x \leqslant N$ is $\ll N^{3/5+\varepsilon}$. This result was later
generalized by
\cite{Bombieri-Pila*89}
who showed among
other things the following estimate.  

\begin{thm}{Theorem (1989)}

Let $N \geqslant 1$, $k \geqslant 4$ be integers and define $K = \binom{k+2}{2}$. Let $\mathcal{I}$ be an interval with length $N$ and $f \in C^K (\mathcal{I})$ satisfying $|f'(x)| \leqslant 1$, $f''(x) >0$ and such that the number of solutions of the equation $f^{(K)} (x) = 0$ is $\leqslant m$. Then there exists a constant $c_0 = c_0(k) >0$ such that
$$\mathcal {R}(f,N,0) \leqslant c_0 (m+1) N^{1/2+3/(k+3)}.$$
\end{thm}











  
\begin{flushright}\small\sl{}   Last updated on July 23rd, 2012, by Olivier Bordell\`es
 \end{flushright}


















\part{Size of $L(1,\chi)$ and character sums }

\chapter{   Size of $L(1,\chi)$}

Corresponding html file: \texttt{../Articles/Art07.html}









Collecting references:
\cite{Louboutin*93},


\section{Upper bounds for $|L(1,\chi)|$}


\cite{Louboutin*96},
\cite{Granville-Soundararajan*02},
\cite{Granville-Soundararajan*04}.
\cite{Ramare*01},
\cite{Ramare*02-??},
\cite{Louboutin*98},

\section{Lower bounds for $|L(1,\chi)|$}

\cite{Louboutin*13}
announces the following lower bound proved in
\cite{Louboutin*15}
.

\begin{thm}{Theorem (2013)}

For any non-quadratic primitive Dirichlet character $\chi$ of conductor $f$,
we have $|L(1,\chi)|\ge 1/ ( 10\log(f/\pi))$.
\end{thm}





 
 








  
\begin{flushright}\small\sl{}   Last updated on September 11th, 2014, by Olivier Ramar\'e
 \end{flushright}
















%\end{document}
\chapter{  Character sums}

Corresponding html file: \texttt{../Articles/Art15.html}










\section{Explicit Polya-Vinogradov inequalities}



The main Theorem of \cite{Qiu*91}
implies the following result.

\par 
\begin{thm}{Theorem (1991)}

  For $\chi$ a primitive character to the modulus $q > 1$, we have
$
\left|\sum\limits_{a=M+1}^{M+N}\chi(a)\right|
\le
\frac{4}{\pi^2}\sqrt{q}\log q+0.38\sqrt{q}+\frac{0.637}{\sqrt{q}}
$.
\par 
 When $\chi$ is not especially primitive, but is still non-principal, we
  have
$
\left|\sum\limits_{a=M+1}^{M+N}\chi(a)\right|
\le
\frac{8\sqrt{6}}{3\pi^2}\sqrt{q}\log q+0.63\sqrt{q}+\frac{1.05}{\sqrt{q}}
$.
\end{thm}

This was improved later by
\cite{Bachman-Rachakonda*01} into
the following.
\par 
\begin{thm}{Theorem (2001)}

  For $\chi$ a non-principal character to the modulus $q > 1$, we have
$
\left|\sum\limits_{a=M+1}^{M+N}\chi(a)\right|
\le
\frac{1}{3\log 3}\sqrt{q}\log q+6.5\sqrt{q}
$. 
\end{thm}


These results are superseded by
\cite{Frolenkov*11} and more
recently by
\cite{Frolenkov-Soundararajan*13} into
the following.
\par 
\begin{thm}{Theorem (2013)}

  For $\chi$ a non-principal character to the modulus $q\ge 1000$, we have
$
\left|\sum\limits_{a=M+1}^{M+N}\chi(a)\right|
\le
\frac{1}{\pi\sqrt{2}}\sqrt{q}(\log q+6)+\sqrt{q}
$. 
\end{thm}


In the same paper they improve upon estimates of
\cite{Pomerance*11} and get the following.
\par 
\begin{thm}{Theorem (2013)}

  For $\chi$ a primitive character to the modulus $q \ge 1200$, we have
$$
\max_{M,N}\left|\sum_{a=M+1}^{M+N}\chi(a)\right|
\le
\begin{cases}
\frac{2}{\pi^2}\sqrt{q}\log q+\sqrt{q},&
  \text{$\chi$ even,}\\
\frac{1}{2\pi}\sqrt{q}\log q+\sqrt{q},&
  \text{$\chi$ odd}.
\end{cases}
$$
This latter estimates holds as soon as $q\ge40$.
\end{thm}


In case $\chi$ odd, the constant $1/(2\pi)$ has already
been asymptotically obtained in
\cite{Landau*18-3}.
When $\chi$ is odd and $M=1$, the best asymptotical constant before 2020 was
$1/(3\pi)$ from Theorem 7 of
\cite{Granville-Soundararajan*07},
In case $\chi$ even, we have
$$
\max_{M,N}\left|\sum_{a=M}^N\chi(a)\right|
=2\max_{N}\left|\sum_{a=1}^N\chi(a)\right|.
$$
(The LHS is always less than the RHS. Equality is then easily proved).
The asymptotical best constant in 2007
was $23/(35\pi\sqrt{3})$ from Theorem 7 of
\cite{Granville-Soundararajan*07}.

These results are improved upon for large values squarefree values of $q$ in
\cite{Bordignon-Kerr*20}
by a different method into the following.
\par 
\begin{thm}{Theorem (2020)}

  For $\chi$ a primitive character to the squarefree modulus $q \ge \exp(1088\ell^2)$, we have
$$
\max_{N}\left|\sum_{a=1}^{N}\chi(a)\right|
\le
\begin{cases}
  \frac{2}{\pi^2}\sqrt{q}\bigl(\frac14+\frac{1}{4\ell}\bigr)\log q
  +\bigl(49+\frac{1}{1088\ell}\bigr)\sqrt{q},&
  \text{$\chi$ even,}\\
  \frac{1}{2\pi}\bigl(\frac12+\frac{1}{2\ell}\bigr)\sqrt{q}\log q
  +\bigl(49+\frac{1}{1088\ell}\bigr)\sqrt{q},&
  \text{$\chi$ odd}.
\end{cases}
$$
This latter estimates holds as soon as $q\ge40$.
\end{thm}


Corresponding estimates when $q$ is not squarefree are proved in
\cite{Bordignon*21}, the
saving $1/4$ being slightly degraded to $3/8$.



\section{Burgess type estimates}


The following from 
\cite{Trevino*15-2}
is an explicit version of Burgess with the only restriction being
$p\ge 10^7$.
\par 
\begin{thm}{Theorem (2015)}

Let $p$ be a prime such that $p \ge 10^7$. Let $\chi$ be a non-principal character $\bmod{\,p}$. Let $r$ be a positive integer, and let $M$ and $N$ be non-negative integers with $N\ge 1$. Then
$$
\left|\sum_{a=M+1}^{M+N}\chi(a)\right|
\le 2.74 N^{1-\frac{1}{r}}
p^{\frac{r+1}{4r^2}}(\log{p})^{\frac{1}{r}}.
$$
\end{thm}

\par 
  From the same paper, we get the following more specific result.
\par 
\begin{thm}{Theorem (2015)}

Let $p$ be a prime. Let $\chi$ be a non-principal character
$\bmod{\,p}$. Let $M$ and $N$ be non-negative integers with $N\ge 1$,
let $2\le r\le 10$ be a positive integer, and let $p_0$ be a positive
real number. Then for $p \ge p_0$, there exists $c_1(r)$, a constant
depending on $r$ and $p_0$ such that 
$$
\left|\sum_{a=M+1}^{M+N}\chi(a)\right|
\le
c_1(r) N^{1-\frac{1}{r}} p^{\frac{r+1}{4r^2}}(\log{p})^{\frac{1}{r}}
$$
where $c_1(r)$ is given by

  
  
    
      $r$
      $p_0=10^7$
      $p_0=10^{10}$
      $p_0=10^{20}$
    
  
  
    2
    2.7381
    2.5173
    2.3549
  
  
    3
    2.0197
    1.7385
    1.3695
  
  
    4
    1.7308
    1.5151
    1.3104
  
  
    5
    1.6107
    1.4572
    1.2987
  
  
    6
    1.5482
    1.4274
    1.2901
  
  
    7
    1.5052
    1.4042
    1.2813
  
  
    8
    1.4703
    1.3846
    1.2729
  
  
    9
    1.4411
    1.3662
    1.2641
  
  
    10
    1.4160
    1.3495
    1.2562
  


\end{thm}

\par 

 We can get  a smaller exponent on $\log$ if we restrict the range of
 $N$ or if we have $r\ge 3$.
\par 
\begin{thm}{Theorem (2015)}

Let $p$ be a prime. Let $\chi$ be a non-principal character
$\bmod{\,p}$. Let $M$ and $N$ be non-negative integers with $1\le N\le
2 p^{\frac{1}{2} + \frac{1}{4r}}$ or $r\ge 3$. Let $r\le 10$ be a
positive integer, and let $p_0$ be a positive real number. Then for $p
\ge p_0$, there exists $c_2(r)$, a constant depending on $r$ and $p_0$
such that 
$$
\left|\sum_{a=M+1}^{M+N}\chi(a)\right|
\le
 c_2(r) N^{1-\frac{1}{r}} p^{\frac{r+1}{4r^2}}(\log{p})^{\frac{1}{2r}},
$$
where $c_2(r)$ is given by
  
  
  
    
      $r$
      $p_0=10^7$
      $p_0=10^{10}$
      $p_0=10^{20}$
    
  
  
    2
    3.7451
    3.5700
    3.5341
  
  
    3
    2.7436
    2.5814
    2.4936
  
  
    4
    2.3200
    2.1901
    2.1071
  
  
    5
    2.0881
    1.9831
    1.9037
  
  
    6
    1.9373
    1.8504
    1.7748
  
  
    7
    1.8293
    1.7559
    1.6843
  
  
    8
    1.7461
    1.6836
    1.6167
  
  
    9
    1.6802
    1.6262
    1.5638
  
  
    10
    1.6260
    1.5786
    1.5210
  
 

\end{thm}

\par 

Kevin McGown  in
\cite{McGown*12}
has slightly worse constants in a slightly larger range of $N$ for
smaller values of $p$.
\par 
\begin{thm}{Theorem (2012)}

Let $p\ge 2\cdot 10^{4}$ be a prime number. Let $M$ and $N$ be
non-negative integers with $1\le N\le 4 p^{\frac{1}{2} +
\frac{1}{4r}}$. Suppose $\chi$ is a non-principal character
$\bmod{\,p}$. Then there exists a computable constant $C(r)$ such that
$$
\left|\sum_{a=M+1}^{M+N}\chi(a)\right|
\le
C(r) N^{1-\frac{1}{r}} p^{\frac{r+1}{4r^2}}(\log{p})^{\frac{1}{2r}},
$$
where $C(r)$ is given by

  
    
      
	$r$
	$C(r)$
	$r$
	$C(r)$
      
    
    
      2
      10.0366
      9
      2.1467
    
    
      3
       4.9539
      10
      2.0492
    
    
      4
      3.6493
      11
      1.9712
    
    
      5
      3.0356
      12
      1.9073
    
    
      6
      2.6765
      13
      1.8540
    
    
      7
      2.4400
      14
      1.8088
    
    
      8
      2.2721
      15
      1.7700
    
  

\end{thm}

\par 


If the character is quadratic (and with a more restrictive
range), we have slightly stronger results due to Booker in  
\cite{Booker*06}.
\par 
\begin{thm}{Theorem (2006)}

Let $p > 10^{20}$ be a prime number with $p \equiv 1 \pmod{4}$. Let
$r\in \{2,3,4,\ldots,15\}$. Let $M$ and $N$ be real numbers such that
$0 < M , N \le 2\sqrt{p}$. Let $\chi$ be a non-principal quadratic
character $\bmod{\,p}$.  Then
$$
\left|\sum_{a=M+1}^{M+N}\chi(a)\right|
\le \alpha(r) N^{1-\frac{1}{r}} p^{\frac{r+1}{4r^2}}\left(\log{p} +
     \beta(r)\right)^{\frac{1}{2r}},
$$
     where $\alpha(r)$ and $\beta(r)$ are given by

  
  
    
      $r$
      $\alpha(r)$
      $\beta(r)$
      $r$
      $\alpha(r)$
      $\beta(r)$
    
  
  
    2
    1.8221
    8.9077
    9
    1.4548
    0.0085
  
  
    3
    1.8000
    5.3948
    10
    1.4231
    -0.4106
  
  
    4
    1.7263
    3.6658
    11
    1.3958
    -0.7848
  
  
    5
    1.6526
    2.5405
    12
    1.3721
    -1.1232
  
  
    6
    1.5892
    1.7059
    13
    1.3512
     -1.4323
  
  
    7
    1.5363
    1.0405
    14
    1.3328
    -1.7169
  
  
    8
     1.4921
    0.4856
    15
    1.3164
    -1.9808
  
  

\end{thm}


Concerning composite moduli, we have the next result in
\cite{Jain-Sharma-Khale-Liu*21}
.
\par 
\begin{thm}{Theorem (2021)}

  Let $\chi$ be a primitive character with modulus $q\ge e^{e^{9.594}}$.
  Then for $N\le q^{5/8}$, we have
$$
\left|\sum_{a=M+1}^{M+N}\chi(a)\right|
  \le 9.07 \sqrt{N}q^{3/16}(\log q)^{1/4}
  \bigl(2^{\omega(q)}d(q)\bigr)^{3/4}
  \sqrt{\frac{q}{\varphi(q)}}.
$$
\end{thm}

\par 


 
 








  
\begin{flushright}\small\sl{}   Last updated on December 11th, 2021, by Olivier Ramar\'e
 \end{flushright}

















\part{Zeros and zero-free regions}

\chapter{   Bounds for $|\zeta(s)|$, $|L(s,\chi)|$ and related questions}

Corresponding html file: \texttt{../Articles/Art06.html}









Collecting references:
\cite{Trudgian*11},
\cite{Kadiri-Ng*12},

 
 

\par 
\section{Size of $|\zeta(s)|$ and of $L$-series}



Theorem 4 of \cite{Rademacher*59} gives
the convexity bound. See also section 4.1 of \cite{Trudgian*13}.
\par 
\begin{thm}{Theorem (1959)}

In the strip $-\eta\le \sigma\le 1+\eta$, $0 < \eta\le 1/2$, the Dedekind zeta
function $\zeta_K(s)$ belonging to the algebraic number field $K$ of degree
$n$ and discriminant $d$ satisfies the inequality
$$
|\zeta_K(s)|\le 3 \left|\frac{1+s}{1-s}\right|
\left(\frac{|d||1+s|}{2\pi}\right)^{\frac{1+\eta-\sigma}{2}}
\zeta(1+\eta)^n.
$$
\end{thm}


On the line $\Re s=1/2$, Lemma 2 of
\cite{Lehman*70} gives a better
result, namely
\par 
\begin{thm}{Theorem (1970)}

If $t\ge 1/5$, we have
$
|\zeta(\tfrac12+it)|\le 4 (t/(2\pi))^{1/4}
$.
\end{thm}

In fact, Lehman states this Theorem for $t\ge 64/(2\pi)$, but modern means of
computations makes it easy to check that it holds as soon as $t\ge 0.2$.
See also equation (56)
of \cite{Backlund*18} reproduced below.

For Dirichlet $L$-series, Theorem 3
of \cite{Rademacher*59} gives 
the corresponding convexity bound.
\par 
\begin{thm}{Theorem (1959)}

In the strip $-\eta\le \sigma\le 1+\eta$, $0 < \eta\le 1/2$, for all moduli $q
> 1$ and all primitive
characters $\chi$ modulo $q$, the inequality 
$$
|L(s,\chi)|\le  
\left(q\frac{|1+s|}{2\pi}\right)^{\frac{1+\eta-\sigma}{2}}
\zeta(1+\eta)
$$
holds.
\end{thm}

This paper contains other similar convexity bounds.


Corollary to Theorem 3
of \cite{Cheng-Graham*01} goes beyond convexity. 

\par 
\begin{thm}{Theorem (2001)}

  For $0\le t\le e$, we have $|\zeta(\tfrac12+it)|\le 2.657$. For $t\ge e$, we
  have $|\zeta(\tfrac12+it)|\le 3t^{1/6}\log t$.
  Section 5 of \cite{Trudgian*13}
  shows that one can replace the constant 3 by 2.38.
\end{thm}


This is improved in
\cite{Hiary*16}.
\par 
\begin{thm}{Theorem (2016)}

  When $t\ge 3$, we
  have $|\zeta(\tfrac12+it)|\le 0.63t^{1/6}\log t$.
\end{thm}


Concerning $L$-series, the situation is more difficult but
\cite{Hiary*16b}
manages, among other and more precise results, to prove the following.
\par 
\begin{thm}{Theorem (2016)}

  Assume $\chi$ is a primitive Dirichlet character modulo $q>1$. Assume
  further that $q$ is a sixth power. Then, when $|t|\ge 200$, we
    have
    $$|L(\tfrac12+it,\chi)|\le 9.05d(q) (q|t|)^{1/6}(\log
    q|t|)^{3/2}$$
    where $d(q)$ is the number of divisors of $q$.
\end{thm}


It is often useful to have a representation of the Riemann zeta function
or of $L$-series inside the critical strip. In the case of $L$-series,
\cite{Spira*69}
and
\cite{Rumely*93}
proceed via decomposition in Hurwitz zeta function which they compute through
an Euler-MacLaurin development. We have a more efficient approximation of the
Riemann zeta function provided by the Riemann Siegel formula, see
for instance equations (3-2)--(3.3)
of \cite{Odlyzko*87}. This
expression is due to 
\cite{Gabcke*79}.
See also 
equations (2.4)-(2.5) of
\cite{Lehman*66}, a corrected
version of Theorem 2 of \cite{Titchmarsh*47}.

\par 
In general, we have the following estimate taken from equations
(53)-(54), (56) and (76)
of  \cite{Backlund*18}
(see also \cite{Backlund*14}).
\par 
\begin{thm}{Theorem (1918)}

  \begin{itemize}
  \item When $t\ge 50$ and $\sigma\ge1$, we have $|\zeta(\sigma+it)|\le \log
  t-0.048$.

  \item  When $t\ge 50$ and $0\le \sigma\le1$, we have $|\zeta(\sigma+it)|\le
  \frac{t^2}{t^2-4}\left(\frac{t}{2\pi}\right)^{\frac{1-\sigma}{2}}\log t$.
  

  \item  When $t\ge 50$ and $-1/2\le \sigma\le0$, we have $|\zeta(\sigma+it)|\le
  \left(\frac{t}{2\pi}\right)^{\frac{1}{2}-\sigma}\log t$.
  

  \end{itemize}
\end{thm}



On the line $\Re s=1$, one can rely on
\cite{Trudgian*12b}.
\par 
\begin{thm}{Theorem (2012)}

  When $t\ge 3$, we have $|\zeta(1+it)|\le\tfrac34 \log t$.
\end{thm}


Asymptotically better bounds are available since the huge work of
\cite{Ford*02}.
\par 
\begin{thm}{Theorem (2002)}

  When $t\ge 3$ and $1/2\le \sigma\le 1$, we have $|\zeta(\sigma+it)|\le 76.2
  t^{4.45(1-\sigma)^{3/2} } (\log t)^{2/3}$.
\end{thm}

The constants are still too large for this result to be of use in any decent
region. See \cite{Kulas*94} for an
earlier estimate.





\par 
\section{On the total number of zeroes}


The first explicit estimate for the number of zeros of the Riemann
$\zeta$-function goes back to
\cite{Backlund*14}.
An elegant consequence of the result of Backlund is the following easy
estimate taken from Lemma 1 of
\cite{Lehman*66a}.

\begin{thm}{Theorem (1966)}

If $\varphi$ is a continuous function which is positive and monotone
decreasing for $2\pi e\le T_1\le t\le T_2$, then
$$
\sum_{T_1 < \gamma\le T_2} \varphi(\gamma)
            =\frac{1}{2\pi}\int_{T_1}^{T_2}\varphi(t)\log\frac{t}{2\pi}dt
            +O^*\biggl(4\varphi(T_1)\log
            T_1+2\int_{T_1}^{T_2}\frac{\varphi(t)}{t}
            dt\biggr)
            $$
             where the summation is over all zeros of the Riemann
            $\zeta$-function of
            imaginary part between $T_1$ and $T_2$, with multiplicity.
\end{thm}


Theorem 19 of
\cite{Rosser*41}
gives a bound for the total number of zeroes.

\begin{thm}{Theorem (1941)}

For $T\ge2$, we have
$$
N(T)=\sum_{\substack{\rho,\\ 0 < \gamma\le T}} 1=
            \frac{T}{2\pi}\log\frac{T}{2\pi}-\frac{T}{2\pi}+\frac{7}{8}
            +O^*\Bigl(0.137\log T+0.443\log\log T+1.588
            \Bigr)
            $$
            where the summation is over all zeros of the Riemann
            $\zeta$-function of
            imaginary part between 0 and $T$, with multiplicity.
\end{thm}


It is noted in Lemma 1 of
\cite{Ramare-Saouter*02}
that the $O$-term can be replaced by the simpler
$O^*(0.67\log\frac{T}{2\pi})$ when $T\ge 10^3$.

This is improved in Corollary 1 of
\cite{Trudgian*13}
into
\begin{thm}{Theorem (2014)}

For $T\ge e$, we have
$$
N(T)=\sum_{\substack{\rho,\\ 0 < \gamma\le T}} 1=
            \frac{T}{2\pi}\log\frac{T}{2\pi}-\frac{T}{2\pi}+\frac{7}{8}
            +O^*\bigl(0.112\log T+0.278\log\log T+2.510+\frac{1}{5T}
            \bigr)
            $$
            where the summation is over all zeros of the Riemann
            $\zeta$-function of
            imaginary part between 0 and $T$, with multiplicity.
\end{thm}



Corollary 1.4 of the main theorem of
\cite{Hasanalizade-Shen-Wong*22}
reads
\begin{thm}{Theorem (2022)}

For $T\ge e$, we have
$$
N(T)=\sum_{\substack{\rho,\\ 0 < \gamma\le T}} 1=
            \frac{T}{2\pi}\log\frac{T}{2\pi}-\frac{T}{2\pi}+\frac{7}{8}
            +O^*\bigl(0.1038\log T+0.2573\log\log T+9.3675
            \bigr)
            $$
            where the summation is over all zeros of the Riemann
            $\zeta$-function of
				 imaginary part between 0 and $T$, with multiplicity.
				 We may also replace $0.1038\log
				 T+0.2573\log\log T+9.3675$ by $0.1095\log T+0.2042\log\log T+3.0305$.
\end{thm}



Concerning Dirichlet $L$-functions, the paper
\cite{Bennett-Martin-OBryant-Rechnitzer*21}
contains the next result.

\begin{thm}{Theorem (2021)}

    Let $\chi$ be a Dirichlet character of conductor $q > 1$.
For $T\ge 5/7$ and $\ell= \log\frac{q(T+2)}{2\pi} > 1.567, we have
$$
N(T,\chi)=\sum_{\substack{\rho,\\ 0 < \gamma\le T}} 1=
            \frac{T}{\pi}\log\frac{qT}{2\pi}-\frac{T}{\pi}+\frac{\chi(-1)}{4}
            +O^*\bigl(0.22737\ell+2\log(1+\ell)-0.5
            \bigr)
            $$
            where the summation is over all zeros of the Dirichlet
            function $L(\cdot,\chi)$ of
				 imaginary part between $-T$ and $T$, with multiplicity.
				 
\end{thm}





\par 
\section{L${}^2$-averages}


We can find in
\cite{Helgott*17u} the proof
of the following estimate. Though it is unpublished yet, the full proof
is available.
\begin{thm}{Theorem (2019)}

Let $0 < \sigma\le1$ and $T \ge 3$. Then
  $$
    \frac{1}{2\pi}\biggl(
    \int_{\sigma-i\infty}^{\sigma-iT}
    +
    \int^{\sigma+i\infty}_{\sigma+iT}
    \biggr)
    \frac{|\zeta(s)|^2}{|s|^2}ds\le
    \kappa_{\sigma,T}
    \begin{cases}
    \frac{c_{1,\sigma}}{T}+\frac{c^\flat_{1,\sigma}}{T^{2\sigma}}
    &\text{when $\sigma > 1/2$,}\\
    \frac{\log T}{2T}+\frac{c^\flat_{2,\sigma}}{T}
    &\text{when $\sigma=1/2$,}\\
    c_{3,\sigma}/T^{2\sigma}&\text{when $\sigma < 1/2$.}
    \end{cases}
  $$
 where
$$
c_{1,\sigma}=\zeta(2\sigma)/2,
 c_{1,\sigma}^\flat=c^2 \frac{3^{2\sigma}}{2\sigma},
  c_{2,\sigma}^\flat=3c^2+\frac{1-\log 3}{2},
c=9/16,						  
 $$
 $$
 c_{3,\sigma}=
\Bigl(\frac{c^2}{2\sigma}+\frac{1/6}{1-2\sigma}\Bigr)
\Bigl(1+\frac{1}{\sigma}\Bigr)^{2\sigma},
\kappa_{\sigma,T}=
\begin{cases}
\frac{9/4}{\left(1-\frac{9/2}{T^2}\right)^2}
&\text{when $1/2\le \sigma\le 1$,}\\
\frac{(1+\sigma)^2}{\left(1-\frac{(1+\sigma)^2}{\sigma T^2}\right)^2}
&\text{when $0 < \sigma < 1/2$.}
 \end{cases}
 $$
						  
\end{thm}





\par 
\section{Bounds on the real line}


After some estimates
from \cite{Bastion-Rogalaski*02}, 
Lemma 5.1 of \cite{Ramare*13d} shows
the following.
\par 
\begin{thm}{Theorem (2013)}

  When $\sigma> 1$ and $t$ is any real number, we have $|\zeta(\sigma+it)|\le   e^{\gamma(\sigma-1) }/(\sigma-1)$.
\end{thm}


Here is the Theorem of
\cite{Delange*87}.
See also Lemma 2.3 of
\cite{Ford*01} for a
slightly weaker version.
\par 
\begin{thm}{Theorem (1987)}

  When $\sigma> 1$ and $t$ is any real number, we have
  $$
  -\Re\frac{\zeta'}{\zeta}(\sigma+it)\le
  \frac{1}{\sigma-1}-\frac{1}{2\sigma^2}.
  $$ 
\end{thm}








  
\begin{flushright}\small\sl{}   Last updated on April 29th, 2022, by Olivier Ramar\'e
 \end{flushright}
















\chapter{  Explicit zero-free regions for the $\zeta$ and $L$ functions}

Corresponding html file: \texttt{../Articles/Art08.html}









\section{Numerical verifications of the Generalized Riemann Hypothesis}


Numerical verifications of the Riemann hypothesis for the Riemann
$\zeta$-function have been pushed extremely far. B. Riemann himself computed the
first zeros. Concerning more recent published papers, we mention
\cite{Lune-Riele-Winter*86}
who proved that
\begin{thm}{Theorem (1986)}

  Every zero $\rho$ of $\zeta$ that have a real part between 0 and 1 and
  an imaginary part not more, in absolute value, than $\le T_0=545\,439\,823$
  are in fact on the critical line, i.e. satisfy $\Re \rho=1/2$.
\end{thm}

The bound $545\,439\,823$ is increased to $1\,000\,000\,000$ in
\cite{Platt*11}.
In
\cite{Platt*17},
this bound is further increased to
$30\,610\,046\,000$.
Between these results, 
\cite{Wedeniwski*02}
announced that, he and many collaborators proved, using a network method:
\begin{thm}{Theorem (2002)}

  $T_0=29\,538\,618\,432$ is admissible in the theorem above.
\end{thm}


\cite{Gourdon-Demichel*04}
went one step further
\par 
\begin{thm}{Theorem (2004)}

  $T_0=2.445\cdot 10^{12}$ is admissible in the theorem above.
\end{thm}

These two last announcements have not been subject to any academic papers.

We now have
\cite{Platt-Trudgian*21a}
\par 
\begin{thm}{Theorem (2021)}

  $T_0=3\cdot 10^{12}$ is admissible in the theorem above.
\end{thm}





One of the key ingredient is an explicit Riemann-Siegel formula due to
\cite{Gabcke*79}
(the preprint of Gourdon mentionned above gives a version of Gabcke's result)
and such a formula is missing in the case of Dirichlet $L$-function.

Let us introduce some terminology. We say that a modulus $q\ge1$ (i.e. an
integer!) satisfies $GRH(H)$ for some numerical value $H$ when
every zero $\rho$ of the $L$-function associated to a primitive Dirichlet
character of conductor $q$ and whose real part lies within the critical line (i.e. has a
real part lying inside the open interval $(0,1)$) and whose imaginary part is
below, in absolute value, $H$, in fact satisfies $\Re\rho=1/2$.

By employing an Euler-McLaurin formula,
\cite{Rumely*93}
has proved that

\begin{thm}{Theorem (1993)}

  \begin{itemize}
  \item Every $q\le 13$ satisfies $GRH(10\,000)$.

  \item Every $q$ belonging to one of the sets
  \begin{itemize}
    \item \,\,$\{k\le 72\}$

    \item \,\,$\{k\le 112, \text{$k$ non premier}\}$

    \item \,$\begin{aligned}\{116, 117, &120, 121, 124, 125, 128, 132, 140,
     143, 144, 156, 163, \\ &169, 180, 216, 243, 256, 360, 420, 432\}\end{aligned}$

    \end{itemize}
    satisfies $GRH(2\,500)$.
    

    \end{itemize}
\end{thm}

These computations have been extended by 
\cite{Bennett*01}
by using Rumely's programm. All these computations have been
superseded by the work of D. Platt.
\cite{Platt*11} and
\cite{Platt*13}
use two fast Fourier transforms, one in the $t$-aspect and one in the
$q$-aspect, as well as an approximate functionnal equation to prove via
extremely rigorous computations that
\begin{thm}{Theorem (2011-2013)}

  Every modulus $q\le 400\,000$ satisfies
    $GRH(100\,000\,000/q)$.
\end{thm}


We mention here the algorithm of
\cite{Omar*01}
that enables one to prove efficiently that some $L$-functions have no zero
    within the rectangle
$1/2\le \sigma\le1$ et $2\sigma-|t|=1$ though this algorithm has not been put
    in practice.

There are much better results concerning real zeros of Dirichlet $L$-functions
    associated to real characters.


\section{Asymptotical zero-free regions}


The first fully explicit zero free region for the Riemann zeta-function is due
to \cite{Rosser*38} in Lemma 19 (essentially
with $R_0=19$ in the notations below). This is next imporved upon in Theorem 1
of \cite{Rosser-Schoenfeld*75}
by using a device of
\cite{Stechkin*70} (getting
essentially $R_0=9.646$).
The next step is in
\cite{Ramare-Rumely*96} 
where the second order term is improved upon, relying on
\cite{Stechkin*89}.

Next, in
\cite{Kadiri*02}
and later in
\cite{Kadiri*05},
the following result is proven.

\begin{thm}{Theorem (2002)}

  The Riemann $\zeta$-function has no zeros in the region
  $$
    \Re s \ge 1- \frac1{R_0 \log (| \Im s|+2)}\quad\text{with}\  R_0=5.70175.
  $$
\end{thm}



\cite{Jang-Kwon*14}
improved the value of $R_0$ by showing that $R_0=5.68371$ is admissible.
By plugging a better trigonometric polynomial in the same method,
it is proved in
\cite{Mossinghoff-Trudgian*15}
that

\begin{thm}{Theorem (2015)}

  The Riemann $\zeta$-function has no zeros in the region
  $$
    \Re s \ge 1- \frac1{R_0 \log (| \Im s|+2)}\quad\text{with}\  R_0=5.573412.
  $$
\end{thm}


Concerning Dirichlet $L$-function, the first explicit zero-free region has been obtained in
\cite{McCurley*84-1} by adaptating
\cite{Rosser-Schoenfeld*75}.
\cite{Kadiri*02} (cf also
\cite{Kadiri*02-2})
improves that into:

\begin{thm}{Theorem (2002)}

  The Dirichlet $L$-functions associated to a character of conductor $q$ has
  no zero in the region:
  $$
    \Re s \ge 1- \frac1{R_1 \log(q \max(1,| \Im s|))}  \quad\text{with}\
    R_1=6.4355, 
  $$
  to the exception of at most one of them which would hence be attached to a
  real-valued character. This exceptional one would have at most one zero
  inside the forbidden region (and which is loosely called a "Siegel zero").
\end{thm}

In
\cite{Kadiri*18}, the next
theorem is proved.
\begin{thm}{Theorem (2016)}

  The Dirichlet $L$-functions associated to a character of conductor $q\in[3,400\,000]$ has
  no zero in the region:
  $$
    \Re s \ge 1- \frac1{R_2 \log(q \max(1,| \Im s|))}  \quad\text{with}\
    R_1=5.60. 
  $$
\end{thm}



Concerning the Vinogradov-Korobov zero-free region,
\cite{Ford*01}
shows that

\begin{thm}{Theorem (2001)}

  The Riemann $\zeta$-function has no zeros in the region
  $$
    \Re s\ge  1-\frac{1}{58(\log |\Im s|)^{2/3}(\log\log |\Im s|)^{1/3}}
    \quad(|\Im s|\ge 3).
  $$
\end{thm}


Concerning the Dedekind $\zeta$-function, see
\cite{Kadiri*12}.



\section{Real zeros}


\cite{Rosser*49},
\cite{Rosser*50},
\cite{Chua*05},
\cite{Watkins*00-1},

\section{Density estimates}


After initial work of
\cite{Chen-Wang*89-2}
and
\cite{Liu-Wang*02-1},
here are the latest two best results. We first define
$$
  N(\sigma,T,\chi)=\sum_{\substack{\rho=\beta+i\gamma,\\ L(\rho,\chi)=0,\\
      \sigma\le \beta, |\gamma|\le T}}1
$$
which thus counts the number of zeroes $\rho$ of $L(s,\chi)$, zeroes
whose real part is denoted by $\beta$ (and assumed to be larger than
$\sigma$), and whose imaginary part in absolute value $\gamma$ is assumed to be
not more than $T$. For the Riemann $\zeta$-function (i.e. when
$\chi=\chi_0$ the principal character modulo~1), it is customary
to count only the zeroes with positive imaginary part. The relevant
number is usually denoted by $N(\sigma,T)$. We have $2N(\sigma,T)=N(\sigma,T,\chi_0)$.

For low values, we start with the main Theorem of
\cite{Kadiri-Ng*12}.
We reproduce only a special case.

\par 
\begin{thm}{Theorem (2013)}

  Let $T\ge3.061\cdot10^{10}$. We have
  $
    2N(17/20,T,\chi_0)\le 0.5561T+0.7586\log T-268 658
  $
  where $\chi_0$ is the principal character modulo 1.
\end{thm}


Otherwise, here is the result of
\cite{Ramare*13d}.
\par 

\begin{thm}{Theorem (2016)}

  For $T\ge2\,000$ and $T\ge Q\ge10$, as well as $\sigma\ge0.52$, we have 
  $$
    \sum_{q\le Q}\frac{q}{\varphi(q)}
    \sum_{\chi\mod^* q}N(\sigma,T,\chi)
    \le 
    20\bigl(56\,Q^{5}T^3\bigr)^{1-\sigma}\log^{5-2\sigma}(Q^2T)
    +32\,Q^2\log^2(Q^2T)
  $$
  where $\chi\mod^* q$ denotes a sum over all primitive Dirichlet character
  $\chi$ to the modulus $q$. Furthermore, we have
  $$
    N(\sigma,T,\chi_0)\le 6T\log T
    \log\biggl(1+\frac{6.87}{2T}(3T)^{8(1-\sigma)/{3}}\log^{4-2\sigma}(T)\biggr)
    +103(\log T)^2
  $$
  where $\chi_0$ is the principal character modulo 1.
\end{thm}

In
\cite{Kadiri-Lumley-Ng*18}.
this result is improved upon, we refer to their paper for their result
by quote a corollary.


  For $T\ge1$, we have 
  $
    N(0.9,T)
    \le 
    11.5\, T^{4/14}\log^{16/5}(T)
    +3.2\,\log^2(T)
  $
  where $N(\sigma,T)=N(\sigma,T,\chi_0)$ and $\chi_0$ is the principal character modulo 1.


\section{Miscellanae}



The LMFDB\footnote{\url{http://www.lmfdb.org}} database contains the first zeros
of many $L$-functions. A part of Andrew Odlyzko's 
website\footnote{\url{http://www.dtc.umn.edu/~odlyzko/zeta_tables/index.html}}
contains extensive tables concerning zeroes of the Riemann zeta function.




 
 







  
\begin{flushright}\small\sl{}   Last updated on September 19th, 2021, by Olivier Ramar\'e
 \end{flushright}
















\part{Sieve and short interval results }

\chapter{   Short intervals containing primes}

Corresponding html file: \texttt{../Articles/Art09.html}










 
 







\section{Interval with primes, without any congruence condition}




The story seems to start in 1845 when Bertrand conjectured after
numerical trials
that the interval $]n,2n-3]$ contains a prime as soon as $n\ge4$. This was proved
by \v Ceby\v sev in 1852 in a famous work where he got the first good
quantitative estimates for the number of primes less than a given bound,
say $x$. By now, analytical means combined with sieve methods
(see
\cite{Baker-Harman-Pintz*01}
)
ensures us that each of the
intervals $[x,x+x^{0.525}]$ for $x \geq x_0$
contains at least one prime.
This statement concerns only for the (very) large integers.

It falls very close to what we can get under the assumption
of the Riemann
Hypothesis: the interval $[x-K\sqrt{x}\log x,x]$ contains a prime, where
$K$ is an effective large constant and $x$ is sufficiently large
(cf
\cite{Wolke*83}
for an account on this subject). A theorem
of Schoenfeld
\cite{Schoenfeld*76}
also tells us that the interval
\begin{equation*}
  [x-\sqrt{x}\log^2x/(4\pi),x]
\end{equation*}
contains a prime for $x\geq 599$ under the Riemann Hypothesis. These results
are still far from the conjecture in
\cite{Cramer*36}
on
probabilistic grounds: the interval $[x-K\log^2x,x]$ contains a prime for any
$K > 1$ and $x\geq x_0(K)$. Note that this statement has been proved for almost
all intervals in a quadratic average sense in
\cite{Selberg*43}
assuming the Riemann Hypothesis and replacing $K$ by a function $K(x)$ tending
arbitrarily slowly to infinity.


\cite{Schoenfeld*76}
proved the following.

\par 
\begin{thm}{Theorem (1976)}

Let $x$ be a real number larger than $2\,010\,760$. Then the interval
$$
\Bigl] x \Bigl(1-\frac1{16\,597}\Bigr),x \Bigr]
$$
contains at least one prime.
\end{thm}


The main ingredient is the explicit formula and a numerical verification of
the Riemann hypothesis.

From a numerical point of view, the Riemann Hypothesis is known to hold up to
a very large height (and larger than in 1976).
\cite{Wedeniwski*02} and the Zeta grid
project verified this hypothesis till height $T_0=2.41\cdot 10^{11}$
and
\cite{Gourdon-Demichel*04}
till height $T_0=2.44 \cdot 10^{12}$
thus extending the work
\cite{Lune-Riele-Winter*86}
who had conducted such a verification in 1986 till
height $5.45\times10^8$.
This latter computations has appeared in a refereed journal, but this is not
the case so far concerning the other computations; section 4 of the paper
\cite{Saouter-Demichel*10}
casts some doubts on whether all the zeros where checked.
Discussions in 2012 with Dave Platt from the university of Bristol led me to
believe that the results of
\cite{Wedeniwski*02}
can be replicated in a very rigorous setting, but that it may be difficult to
do so with the results of 
\cite{Gourdon-Demichel*04}
with the
hardware at our disposal.

In
\cite{Ramare-Saouter*02},
we used
the value $T_0=3.3 \cdot 10^{9}$ and obtained the following.



\begin{thm}{Theorem (2002)}

Let $x$ be a real number larger than $10\,726\,905\,041$. Then the interval
$$
\Bigl] x \Bigl(1-\frac1{28\,314\,000}\Bigr),x \Bigr]
$$
contains at least one prime.
\end{thm}




If one is interested in somewhat larger value, the paper
\cite{Ramare-Saouter*02} also
contains the following.

\begin{thm}{Theorem (2002)}

Let $x$ be a real number larger than $\exp(53)$. Then the interval
$$
\Bigl] x \Bigl(1-\frac1{204\,879\,661}\Bigr),x \Bigr]
$$
contains at least one prime.
\end{thm}




Increasing the lower bound in $x$ only improves the constant by less than 5
percent. If we rely on
\cite{Gourdon-Demichel*04},
we can prove
that

\begin{thm}{Theorem (2004, conditional)}

Let $x$ be a real number larger than $\exp(60)$. Then the interval
$$
\Bigl] x \Bigl(1-\frac1{14\,500\,755\,538}\Bigr),x \Bigr]
$$
contains at least one prime.
\end{thm}


Note that all prime gaps have been computed up
to $10^{15}$ in
\cite{Nicely*99}, extending a
result of
\cite{Young-Potler*89}.

In
\cite{Trudgian*16},
we find
\begin{thm}{Theorem (2016)}

Let $x$ be a real number larger than $2\,898\,242$. The interval
$$
\Bigl[ x, x \Bigl(1+\frac1{111(\log x)^2}\Bigr) \Bigr]
$$
contains at least one prime.
\end{thm}

In
\cite{Dusart*16},
we find
\begin{thm}{Theorem (2016)}

Let $x$ be a real number larger than $468\,991\,632$. The interval
$$
\Bigl[ x, x \Bigl(1+\frac1{5000(\log x)^2}\Bigr),x \Bigr]
$$
contains at least one prime.
\par 

  Let $x$ be a real number larger than $89\,693$. The interval
$$
\Bigl[ x, x \Bigl(1+\frac1{\log^3 x}\Bigr) \Bigr]
$$
contains at least one prime.
\end{thm}


The proof of these latter results has an asymptotical part, for $x\ge 10^{20}$ where we used the
numerical verification of the Riemann hypothesis together with two other
arguments: a (very strong) smoothing argument and a use of the Brun-Titchmarsh
inequality.

The second part is of algorithmic nature and covers the range $10^{10} \le x \le
10^{20}$ and uses prime generation
techniques
\cite{Maurer*95}:
we only look at families of numbers whose primality can be established with
one or two Fermat-like or Pocklington's congruences. This kind of technique
has been already used in a quite similar problem in
\cite{Deshouillers-teRiele-Saouter*98}.
The generation technique we
use relies on a theorem proven in
\cite{Brillhart-Lehmer-Selfridge*75}
and enables us to generate dense
enough families for the upper part of the range to be investigated. For the
remaining (smaller) range, we use theorems of
\cite{Jaeschke*93}
that yield a fast primality test (for this limited range).


 

\par 

Let us recall here that a second line of approach following the original
work of \v Ceby\v sev is still under examination though it does not give
results as good
as analytical means (see
\cite{CostaPereira*89}
for the latest result).

\par 

For very large numbers
\cite{Dudek*16b}
proved the following.
\begin{thm}{Theorem (2014)}

The interval $(x,x+3 x^{2/3}]$ contains a prime for $x\ge \exp(\exp(34.32))$.
\end{thm}



This is improved in 
\cite{Cully*21} as follows.

\begin{thm}{Theorem (2021)}

The interval $(x,x+3 x^{2/3}]$ contains a prime for $x\ge \exp(\exp(33.99))$.
\end{thm}



\section{Interval with primes under RH, without any congruence condition}



\begin{thm}{Theorem (2002)}

Under the Riemann Hypothesis, the interval $\bigl]x-\tfrac85\sqrt{x}\log x,x\bigr]$
contains a prime for $x\ge2$.
\end{thm}

This is improved upon
in 
\cite{Dudek*15} 
into:
\begin{thm}{Theorem (2015)}

Under the Riemann Hypothesis, the interval $\bigl]x-\tfrac4{\pi}\sqrt{x}\log x,x\bigr]$
contains a prime for $x\ge2$.
\end{thm}


In \cite{Carneiro-Milinovich-Soundararajan*19}, the authors go one step further and prove the next result.
\begin{thm}{Theorem (2019)}

Under the Riemann Hypothesis, the interval $\bigl]x-\tfrac{22}{25}\sqrt{x}\log x,x\bigr]$
contains a prime for $x\ge4$.
\end{thm}



\par 







\section{Interval with primes, with congruence condition}




Collecting references:
\cite{McCurley*84-2},
\cite{McCurley*84-3},
\cite{Kadiri*05-2}.

























  
\begin{flushright}\small\sl{}   Last updated on September 1rst, 2021, by Charles Greathouse.
 \end{flushright}














\chapter{  Sieve bounds}

Corresponding html file: \texttt{../Articles/Art14.html}










 
 

\par 
\section{Some upper bounds}


Theorem 2 of \cite{Montgomery-Vaughan*73} contains the
following explicit version of the Brun-Tichmarsh Theorem.
\par 
\begin{thm}{Theorem (1973)}

Let $x$ and $y$ be positive real numbers, and let $k$ and $\ell$ be relatively
prime positive integers. Then 
$
\pi(x+y;k,\ell)-\pi(x;k,\ell)
<  \frac{2y}{\phi(k)\log (y/k)}
$ provided only that $y>k$.
\end{thm}

Here as usual, we have used the notation
$$
\pi(z;k,\ell)=\sum_{\substack{p\le z,\\ p\equiv \ell [k]}}1,
$$
i.e. the number of primes up to $z$ that are coprime to $\ell$ modulo $k$.
See
\cite{Buethe*14}
for a generic weighted version of this inequality.

\par 
Here is a bound concerning a sieve of dimension 2 proved by
\cite{Siebert*76}.
\par 
\begin{thm}{Theorem (1976)}

Let $a$ and $b$ be coprime integers with $2|ab$. Then we have, for $x>1$,
$$
\sum_{\substack{p\le x,\\ \text{$ap+b$ prime}}}1
\le 16 \omega\frac{x}{(\log x)^2}\prod_{\substack{p|ab,\\ p >
2}}\frac{p-1}{p-2}
\qquad \omega=\prod_{p > 2}(1-(p-1)^{-2}).
$$
\end{thm}



\section{Combinatorial sieve estimates}


The combinatorial sieve is known to be difficult from an explicit
viewpoint. For the linear sieve, the reader may look at Chapter 9,
Theorem 9.7 and 9.8 from
\cite{Nathanson*96-2}.


\section{Integers free of small prime factors}


In
\cite{Fan*22}, K. Fan proved
the following neat estimate.
\par 
\begin{thm}{Theorem (2022)}

Let $\Phi(x,z)$ be the number of integers $\le x$ that do not have any
prime factors below $z$. We have
$\displaystyle
\Phi(x,z)\le \frac{x}{\log z},
\quad(1 < z\le x).
$
\end{thm}







  
\begin{flushright}\small\sl{}   Last updated on May 19th, 2022, by Olivier Ramar\'e
 \end{flushright}
















\part{Analytic Number Theory in Number Fields}

\chapter{  Bounds on the Dedekind zeta-function}

Corresponding html file: \texttt{../Articles/Art18.html}










 
 


\par 
\section{Size }


The knowledge on the general Dedekind zeta is less accomplished than
the one of the Riemann zeta-function, but we still have interesting
results. Theorem 4 of \cite{Rademacher*59} gives
the convexity bound. See also section 4.1 of
\cite{Trudgian*13}.
\par 
\begin{thm}{Theorem (1959)}

In the strip $-\eta\le \sigma\le 1+\eta$, $0 < \eta\le 1/2$, the Dedekind zeta
function $\zeta_K(s)$ belonging to the algebraic number field $K$ of degree
$n$ and discriminant $d$ satisfies the inequality
$$
|\zeta_K(s)|\le 3 \left|\frac{1+s}{1-s}\right|
\left(\frac{|d||1+s|}{2\pi}\right)^{\frac{1+\eta-\sigma}{2}}
\zeta(1+\eta)^n.
$$
\end{thm}


\par 

\section{Zeroes and zero-free regions }


We denote by $N_K(T)$ the number of zeros $\rho$, of the Dedekind
zeta-function of the number field $K$ of degree $n$ and discriminant
$d_K$,
zeros that lie in the critical strip
$0 < \Re \rho = \sigma < 1$ and which verify $|\Im \rho|\le T$.
After a first result in
\cite{Kadiri-Ng*12},
we find in
\cite{Trudgian*14-1}
the following result.

\par 
\begin{thm}{Theorem (2014)}

 When $T\ge1$, we have
 $N_K(T)=\frac{T}{\pi}\log\Bigl(|d_K|\Big(\frac{T}{2\pi e}\Bigr)^n\Bigr)
 +O^*\bigl(0.316(\log |d_K|+n\log T)+5.872 n+3.655\bigr)$ .
\end{thm}


This is improved in 
\cite{Hasanalizade-Shen-Wong*21}
into:

\par 
\begin{thm}{Theorem (2021)}

 When $T\ge1$, we have
 $N_K(T)=\frac{T}{\pi}\log\Bigl(|d_K|\Big(\frac{T}{2\pi e}\Bigr)^n\Bigr)
 +O^*\bigl(0.228(\log |d_K|+n\log T)+23.108 n+4.520\bigr)$ .
\end{thm}


In
\cite{Kadiri*12},
a zero-free region is proved.

\par 
\begin{thm}{Theorem (1959)}

Let $K$ be a number field of degree $n$ over $\mathbb{Q}$ and of
discriminant $d \ge 2$. The associated Dedekind
zeta-function $\zeta_K$ has no zeros in the region
$$
\sigma\ge 1-\frac{1}{12.55\log|d_K|+n(9.69\log|t|+3.03)+58.63}, |t|\ge1
$$
and at most one zero in the region
$$
\sigma\ge 1-\frac{1}{12.74\log|d_K|}, |t|\le 1.
$$
The exceptional zero, if it exists, is simple and real.
\end{thm}

See
\cite{Ahn-Kwon*14}
for a result for Hecke $L$-series.









  
\begin{flushright}\small\sl{}   Last updated on April 29th, 2022, by Olivier Ramar\'e
 \end{flushright}
















\part{Applications}

\chapter{  Explicit bounds for class numbers}

Corresponding html file: \texttt{../Articles/Art13.html}









Let $K$ be a number field of degree $n\ge2$, signature $(r_1,r_2)$, absolute
value of discriminant $d_K$, class number $h_K$, regulator $\mathcal{R}_K$ and
$w_K$ the number of roots of unity in $K$. We further denote by $\kappa_K$ the
residue at $s=1$ of the Dedekind zeta-function $\zeta_K(s)$ attached to $K$.
\par \par 
Estimating $h_K$ is a long-standing problem in algebraic number theory.


\section{Majorising $h_K\mathcal{R}_K$}




One of
the classic way is the use of the so-called analytic class number
formula stating that
$$
h_K\mathcal{R}_K=\frac{w_K \sqrt{d_K}}{2^{r_1}(2\pi)^{r_2}}\kappa_K
$$
and to use Hecke's integral representation of the Dedekind zeta function to
bound $\kappa_K$. This is done in
\cite{Louboutin*00} and in
\cite{Louboutin*01} with additional
properties of log-convexity of some functions related to $\zeta_K$ and enabled
Louboutin to reach the following bound:
$$
h_K\mathcal{R}_K
\le\frac{w_K}{2}\left(\frac{2}{\pi}\right)^{r_2}
\left(\frac{e\log d_K}{4n-4}\right)^{n-1}\sqrt{d_K}.
$$
Furthermore, if $\zeta_K(\beta)=0$ for some $\tfrac12\le \beta< 1$, then we have 
$$
h_K\mathcal{R}_K
\le(1-\beta)w_K\left(\frac{2}{\pi}\right)^{r_2}
\left(\frac{e\log d_K}{4n}\right)^{n}\sqrt{d_K}.
$$
When $K$ is abelian, then the residue $\kappa_K$ may be expressed as a
product of values at $s=1$ of $L$-functions associated to primitive Dirichlet
characters attached to $K$. On using estimates for such $L$-functions from
\cite{Ramare*01}, we get for instance
$$
h_K\mathcal{R}_K
\le
\frac{w_K}{2}\left(\frac{2}{\pi}\right)^{r_2}
\left(\frac{\log d_K}{4n-4}+\frac{5-\log 36}{4}\right)^{n-1}\sqrt{d_K}.
$$
Note that the constant $\frac14(5-\log 36)=0.354\cdots$ can be improved upon
in many cases. For instance, when $K$ is abelian and totally real (i.e. $r_2=0$), a result
from
\cite{Ramare*01} implies that the
constant may be replaced by 0, so that
$$
h_K\mathcal{R}_K
\le
\left(\frac{\log d_K}{4n-4}\right)^{n-1}\sqrt{d_K}.
$$


\section{Majorising $h_K$}



One may also estimate $h_K$ alone, without any contamination by the regulator
since this contamination is often difficult to control,
see \cite{Pohst-Zassenhaus*89}.

In this case, one
rather uses explicit bounds for the Piltz-Dirichlet divisor functions $\tau_n$
(see
\cite{Bordelles*02}
and
\cite{Bordelles*06})
and get
$$
h_K\le \frac{M_K}{(n-1)!}
\left(\frac{\log\bigl(M^2_Kd_K\bigr)}{2}+n-2\right)^{n-1}
\sqrt{d_K}
$$
as soon as
$$
n\ge 3,\quad d_K\ge 139 M_K^{-2}
\quad\text{where}\quad
M_K=(4/\pi)^{r_2}n!/n^n.
$$
The constant $M_K$ is known as the Minkowski constant of K.

In
\cite{Cully-Trudgian*21}
we find the following.
\begin{thm}{Theorem (2021)}

  Ler $K$ be a quartic number field with class number $h_K$ and
  Minkowski bound $b$. Then if $b\ge 193$, we have
  $h_K\le (1/3) x(\log x)^3$.
\end{thm}


\section{Using the influence of small primes}



It is explained in
\cite{Louboutin*05} how the
behavior of certain small primes may subtantially improve on the previous
bounds. To make things more significant, define, for a rational prime $p$,
$$
\Pi_K(p)=\prod_{\mathfrak{p}|p}\left(1-\frac{1}{\mathcal{N}_K(\mathfrak{p})}\right)^{-1}.
$$
From \cite{Louboutin*05}, we have
among other things
$$
h_K\mathcal{R}_K
\le\frac{w_K}{2}
\left(\frac{2}{\pi}\right)^{r_2}
\frac{\Pi_K(2)}{\Pi_{\mathbb{Q}}(2)^n}
\left(\frac{e\log d_K}{4n-4}\times
e^{n\log 4/\log d_K}
\right)^{n-1}\sqrt{d_K}
$$
where $K$ is any number field of degree $n\ge3$. In particular, when $2$ is
inert in $K$, then
$$
h_K\mathcal{R}_K
\le\frac{w_K}{2(2^n-1)}
\left(\frac{2}{\pi}\right)^{r_2}
\left(\frac{e\log d_K}{4n-4}\times
e^{n\log 4/\log d_K}
\right)^{n-1}\sqrt{d_K}.
$$




\section{The $h^-_K$ of CM-fields}



Let $K$ be here a CM-field of degree $2n > 2$, i.e. a totally complex
quadratic extension $K$ of its maximal totally real subfield $K^+$. it is well
known that $h_{K^+}$ divides $h_K$. The quotient is denoted by $h^-_K$ and is
called the relative class number of $K$. The analytic class number
formula yields
$$
h^-_K=\frac{Q_Kw_K}{(2\pi)^n}
\left(\frac{d_K}{d_{K^+}}\right)^{1/2}
\frac{\kappa_K}{\kappa_{K^+}}
=
\frac{Q_Kw_K}{(2\pi)^n}
\left(\frac{d_K}{d_{K^+}}\right)^{1/2}
L(1,\chi)
$$
where $\chi$ is the quadratic character of degree 1 attached to the extension
$K/K^+$ and $Q_K\in\{1,2\}$ is the Hasse unit index of $K$.
Here are three results originating in this formula.

From \cite{Louboutin*00}:
\begin{thm}{Theorem (2000)}

  We have
  $$
  h^-_K
  \le
  2Q_Kw_K\left(\frac{d_K}{d_{K^+}}\right)^{1/2}
  \left(
  \frac{e\log(d_K/d_{K^+})}{4\pi n}
  \right)^n.
  $$
\end{thm}


From \cite{Louboutin*03}:
\begin{thm}{Theorem (2003)}

  Assume that $(\zeta_K/\zeta_{K^+})(\sigma)\ge0$ whenever $0 < \sigma <
  1$. Then we have
  $$
  h^-_K
  \ge
  \frac{Q_Kw_K}{\pi e \log d_K}
  \left(\frac{d_K}{d_{K^+}}\right)^{1/2}
  \left(
  \frac{n-1}{\pi e\log d_K}
  \right)^{n-1}.
  $$
\end{thm}


Again from \cite{Louboutin*03}:
\begin{thm}{Theorem (2003)}

  Let $c=6-4\sqrt{2}=0.3431\cdots$. Assume that $d_K\ge 2800^n$ and that
  either $K$ does not contain any imaginary quadratic subfield, or that the
  real zeros in the range $1-\frac{c}{\log d_N}\le \sigma < 1$ of the Dedekind
  zeta-functions of the imaginary quadratic subfields of $K$ are nor zeros of
  $\zeta_K(s)$, where $N$ is the normal closure of $K$. Then we have
  $$
  h^-_K
  \ge
  \frac{cQ_Kw_K}{4ne^{c/2}[N:\mathbb{Q}]}
  \left(\frac{d_K}{d_{K^+}}\right)^{1/2}
  \left(
  \frac{n}{\pi e\log d_K}
  \right)^{n}.
  $$
\end{thm}


And a third result from \cite{Louboutin*03}:
\begin{thm}{Theorem (2003)}

  Assume $n > 2$, $d_K > 2800^n$ and that
  $K$ contains an imaginary quadratic subfield $F$ such that
  $\zeta_F(\beta)=\zeta_K(\beta)=0$ for some $\beta$ satisfying
  $1-\frac{2}{\log d_K}\le \beta < 1$.
  Then we have
  $$
  h^-_K
  \ge
  \frac{6}{(\pi e)^2}
  \left(\frac{d_K}{d_{K^+}}\right)^{1/2-1/n}
  \left(
  \frac{n}{\pi e\log d_K}
  \right)^{n-1}.
  $$
\end{thm}






 
 







  
\begin{flushright}\small\sl{}   Last updated on August 23rd, 2012, by Olivier Bordell\`es
 \end{flushright}















\chapter{  Primitive Roots}

Corresponding html file: \texttt{../Articles/Art19.html}










 
 







  
\begin{flushright}\small\sl{}   Last updated on July 14th, 2012, by Olivier Ramar\'e
 \end{flushright}
















\part{Development}

\chapter{README}

\section{How to write}

This part is technical and destined at the development team only.

\begin{enumerate}
\item Sections are coded via \texttt{<div
    class="section">1. Title</div>}. The numbering is done by hand.
\item Theorems, Lemmas, Propositions are coded via
  \begin{verbatim}
<span class="THM">Theorem (1986)</span>
<blockquote class="outer-thm">
<div class="thm">
    Every zero $\rho$ of $\zeta$ that have a
    real part between 0 and 1 and an imaginary part not more, 
    in absolute value, than $T_0=545439823$ are in fact on
    the critical line, i.e. satisfy $\Re \rho=1/2$.
</div></blockquote> 
\end{verbatim}
Note that the part \verb|</div></blockquote>|
should be on a single line.
  
\item Mathematics are entered latex style and processed via MathJax. Macros are to be avoided, of course.
\item A reference is introduced on one line in the form \newline
\texttt{<script language="javascript">bibref("Ramare*12")</script>}
\newline
 where Ramare*12 is the key of the bibtex entry, which has to be
 introduced in the Local-TME-EMT.bib file.  
\end{enumerate}

The file \texttt{Latex/booklet.tex} is the master file of the PDF
booklet and has to be edited by hand. The Perl script
\texttt{Biblio/UpDateBiblio.pl} will create tex-files that are to be
used as chapter for each html-file found in \texttt{Articles/}, save
the file \texttt{Template\_Article.html} of course.


\section{How to contribute}


Everyone is most welcome to help us keep track of the results. You can
do so by simply sending to the development team a mail with the proper
information (in bibtex format for the relevant part).  You may also
propose a new annoted bibliography, for instance for "Explicit results
in the combinatorial sieve", or any other missing entry.  There are
other ways to contribute, like modifying the CSS so that this site
would be readable under windows, a fact we do not guarantee. Or by
proposing a rewrite of some already present bibliography. Or anything
else we did not think about.


\section{Adding a part or a chapter}

Adding a chapter requires several steps.
\begin{itemize}
\item Modify the file \texttt{accueil.html}.
\item Modify in the file \texttt{../MiseEnPage.js} the variable
  \texttt{Architecture\_TME\_EMT}.
  This modification changes the numbers and each file
  \texttt{Articles/Art**.html} has to be subsequently modified at the
  level of the command \texttt{BandeauGeneral(2, "../../", [0, 2])}.
\item Modify accordingly the file \texttt{Latex/booklet.tex}.
\end{itemize}




%%%%%%%%%%%%%%%%%%%%%%%%%%%%%%%%%%%%%%%%%%%%%%
\backmatter

\printbibliography[heading=bibintoc]




\end{document}
%%% Local Variables:
%%% mode: latex
%%% TeX-master: t
%%% End:
